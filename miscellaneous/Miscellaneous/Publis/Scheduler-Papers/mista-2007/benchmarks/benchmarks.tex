\documentclass[11pt]{article}

% Set the page margins to 1 inch all around:
\marginparwidth 0pt\marginparsep 0pt
\topskip 0pt\headsep 0pt\headheight 0pt
\oddsidemargin 0pt\evensidemargin 0pt
\textwidth 6.5in \topmargin 0pt\textheight 9.0in

\newcommand{\bench}[1]{{\bf BENCH NAME:}{ #1}}

\begin{document}

\pagestyle{empty}

\begin{center}

% Title:

{\LARGE Benchmarks for MISTA 2007 paper}\\[12pt]

% Authors and addresses:

\footnotesize

\mbox{\large \underline{Daniel Godard}}\\
ILOG, 9 rue de Verdun, 94253 Gentilly, France, \mbox{dgodard@ilog.fr}\\[6pt]

\mbox{\large \underline{Philippe Laborie}}\\
ILOG, 9 rue de Verdun, 94253 Gentilly, France, \mbox{plaborie@ilog.fr}\\[6pt]

\normalsize
\end{center}

% Here is the abstract:

\noindent\hrulefill

\noindent This paper describes the set of benchmarks used in the companion MISTA-07 paper. Note that all the problems studied below are NP-complete.

\bigskip

% Here are the Keywords:
\noindent \textit{Keywords:} Constraint Logic Programming, Heuristic Search, Local Search.
% Agent Based Scheduling
% Algorithmics
% Applications
% Automated Reasoning
% Batch Scheduling
% Commercial Packages
% Complexity of Scheduling Problems
% Constraint Logic Programming
% Delivery Scheduling
% Evolutionary Algorithms
% Heuristic Search
% Knowledge-Based Systems
% Large Scale Scheduling
% Local Search
% Machine Scheduling
% Meta-heuristic Search
% Multi-processor Scheduling
% Process Scheduling
% Production Scheduling
% Real World Scheduling
% Real-Time Scheduling
% Rostering
% Rule-Based Expert Systems
% Shop-Floor Scheduling
% Sports Scheduling
% Theoretical Scheduling
% Timetabling
% Transport Scheduling
% Vehicle Routing



\noindent\hrulefill

% The body of the paper starts here:
\section{Introduction}

We give in this paper some references about the different benchmarks used in the computational experiments of our MISTA-07 paper.

\section{Job-shop scheduling}

\bench{JobShop}

\subsection{Benchmark description}

\begin{itemize}
\item Fisher-Thompson (ft*): \cite{Fisher1963}  
\item Carlier (car*): \cite{Carlier1978}
\item Lawrence (la*): \cite{Lawrence1984} 
\item Adams-Balas-Zawack (abz*): \cite{Adams1988}
\item Yamada-Nakano (yam*): \cite{Yamada1992}
\item Taillard (tail*): \cite{Taillard1993}
\item Applegate-Cook (orb*): \cite{Applegate1991}
\item Storer-Wu-Vaccari (swv*): \cite{Storer1992}
\end{itemize}

We selected the abz instances, yam instances, some of the swv instances (randomly 4 upon 10 in each different size), and some of the Taillard instances (randomly 4 upon 10 in each different size).

\section{Resource-constrained project scheduling}
\bench{RCPSP}

We selected the j120 instances.

\subsection{Benchmark description}

\begin{itemize}
\item PSP Lib: \cite{Kolisch1996}
\item Baptiste-LePape: \cite{Baptiste1997}
\item Patterson: \cite{Patterson1984}
\item Carlier-Neron: ?
\end{itemize}
 
\subsection{State-of-the-art}
 
\cite{Kolisch2006}
 
\section{Cumulative job-shop}
\bench{CumulativeJobShop}

\subsection{Benchmark description}

Use instances from \cite{Nuijten1994}. We selected duplicated open instances. We also performed a complete study to compare to the results we obtained and published for ICAPS 2005.

\begin{table}[htb]
	\centering
		\begin{tabular}{|l|l|l|l|} \hline
		Problem & Lower bound & Upper bound \\ \hline
		la03d & 593 & 593 \\ \hline
		la04d & 572 & 576 \\ \hline
		la16d & 892 & 925 \\ \hline
		la17d & 754 & 755 \\ \hline
		la18d & 803 & 811 \\ \hline
		la19d & 756 & 795 \\ \hline
		la20d & 849 & 859 \\ \hline
		la21d & 1017 & 1034 \\ \hline
		la22d & 913 & 926 \\ \hline
		la24d & 885 & 903 \\ \hline
		la25d & 907 & 952 \\ \hline
		la29d & 1117& 1131 \\ \hline
		la36d & 1229 & 1250 \\ \hline
		la37d & 1378 & 1397 \\ \hline
		la38d & 1092 & 1175 \\ \hline
		la39d & 1221 & 1226 \\ \hline
		la40d & 1180 & 1205 \\ \hline
		ft10d & 837 & 891 \\ \hline
		\end{tabular}
	\caption{Cumulative job-shop}
	\label{tab:CumulativeJobShop}
\end{table}


\subsection{State-of-the-art}
 
\cite{Godard2005}

\section{Semiconductor testing}
\bench{SemiconductorTesting}

\subsection{Benchmark description and state-of-the-art}

We randomly selected 18 instances among the biggest ones described in \cite{Ovacik1996}. The selected instances are described in the table below.

\begin{table}[htb]
	\centering
		\begin{tabular}{|l|l|l|} \hline
		Problem & Upper bound \\ \hline
		i305-306 & 2028 \\ \hline
		i305-307 & 2674 \\ \hline
		i305-317 & 2026 \\ \hline
		i315-302 & 4398 \\ \hline
		i315-313 & 2587 \\ \hline
		i315-320 & 2770 \\ \hline
		i325-304 & 3185 \\ \hline
		i325-307 & 2371 \\ \hline
		i325-319 & 1972 \\ \hline
		i605-306 & 2654 \\ \hline
		i605-312 & 4695 \\ \hline
		i605-316 & 5197 \\ \hline
		i615-308 & 1942 \\ \hline
		i615-309 & 4591 \\ \hline
		i615-317 & 2479 \\ \hline
		i625-301 & 2712 \\ \hline
		i625-310 & 1503 \\ \hline
		i625-311 & 3689 \\ \hline
		\end{tabular}
	\caption{Semiconductor testing}
	\label{tab:SemiconductorTesting}
\end{table}

\section{Flow-shop with earliness/tardiness costs}
\bench{FlowShopEarliTardi}

\subsection{Benchmark description}

Use instances from Morton and Pentico \cite{Morton1993}: 12 instances. Total running time is 8430 seconds. Room to apply 2 to 3 different seeds.

\begin{table}[htb]
	\centering
		\begin{tabular}{|l|l|l|l|l|} \hline
		Problem & Size & Lower bound & Upper bound \\ \hline
		jb1 & 10x3 & 0.191 & 0.191 \\ \hline
		jb2 & 10x3 & 0.137 & 0.137 \\ \hline
		jb4 & 10x5 & 0.568 & 0.568 \\ \hline
		jb9 & 15x3 & 0.333 & 0.333 \\ \hline
		jb11 & 15x5 & 0.213 & 0.213 \\ \hline
		jb12 & 15x5 & 0.190 & 0.190 \\ \hline
		ljb1 & 30x3 & 0.215 & 0.215 \\ \hline
		ljb2 & 30x3 & 0.230 & 0.508 \\ \hline
		ljb7 & 50x5 & 0.0.058 & 0.110 \\ \hline
		ljb9 & 50x5 & 0.124 & 0.739 \\ \hline
		ljb10 & 50x8 & 0.197 & 0.512 \\ \hline
		ljb12 & 50x8 & 0.053 & 0.399 \\ \hline
		\end{tabular}
	\caption{Flow-shop with earliness/tardiness costs}
	\label{tab:FlowShopEarliTardi}
\end{table}




\subsection{State-of-the-art}

\cite{Vazquez2000,Danna2003}

\section{Resource-constrained project scheduling with earliness/tardiness costs}
\bench{RCPSPEarliTardi}

\subsection{Benchmark description and state-of-the-art}

We take the 5400 instances of \cite{Vanhoucke2001}. 

We randomly select 30 instances for which the approach in \cite{Vanhoucke2001} finds the optimal value within 30s whereas and our approach does not and 30 instances for which the approach in \cite{Vanhoucke2001} did not prove optimality. 

We then run these 60 instances during 300s both with the approach in the paper and with our approach and compare the results.



\section{Air land scheduling}
\bench{AirLand}

\subsection{Benchmark description and state-of-the-art}

Take the first 8 instances (upon 13) from Beasley \cite{Beasley2000} as we do not have results for the other ones to compare to.
 

 
\section{Open-shop scheduling}
\bench{OpenShop}

\subsection{Benchmark description}

We use instances of:

\begin{itemize}
\item Brucker: \cite{Brucker1997}
\item Taillard: \cite{Taillard1993}
\item Gueret-Prins: \cite{Gueret1999}
\end{itemize}

For each type of instance, we select the biggest problems, thus, for Brucker: $j8*$, for Taillard: $tai\_20x20\_*$, for Gueret-Prins: $gp10-*$.

It results in 28 instances.

\subsection{state-of-the-art}

We compare to the best results between the ones mentioned in \cite{Blum2005} and the ones in \cite{Laborie2005}. We use a time-limit equal to half of the time limit used in \cite{Blum2005} to cope with the fact Blum worked on a 1100 MHz PC.

\section{Quality maximization resource-constrained project scheduling}
\bench{QMRCPSP}

\subsection{Benchmark description and state-of-the-art}
 
We use the 3600 instances described in \cite{Policella2005} and compare with their results using a time-limit of 30s for each instance.

\section{Trolley scheduling}
\bench{Trolley}

\subsection{Benchmark description}

We automatically generated some instances of the scheduling problem described in \cite{Hentenryck1999}. These instances are generated by extending some classical job-shop problems adding the trolley resource with some transition times for transporting the items from one machine to the other. They are listed on Table \ref{tab:TrolleyInstances}. We compare with OPL search results.

\begin{table}[htb]
	\centering
		\begin{tabular}{|l|l|} \hline
		Original problem & Size of trolley instance \\ \hline
		abz5	& 430 \\ \hline
		abz6	& 430 \\ \hline
		orb2	& 430 \\ \hline
		orb7	& 430 \\ \hline
		orb9	& 430 \\ \hline
		la04	& 230 \\ \hline
		la18	& 430 \\ \hline
		la19	& 430 \\ \hline
		la20	& 430 \\ \hline
		ft10	& 430 \\ \hline
		ft20	& 460 \\ \hline
		e0ddr2-1	& 230 \\ \hline
		car5	& 270 \\ \hline
		car6	& 312 \\ \hline
		car8	& 280 \\ \hline
		\end{tabular}
	\caption{Trolley instances}
	\label{tab:TrolleyInstances}
\end{table}

 
\section{Job-shop scheduling with earliness/tardiness costs}
\bench{JobShopEarliTardi}

Reference from Philippe Baptiste, Marta Flamini and Francis Sourd in 2005. Selected instances with 15 jobs and 20 jobs as we don't have UB for instance with 10 jobs.

\subsection{Benchmark description and state-of-the-art}

\cite{Baptiste2005}

\section{Single-machine scheduling with earliness/tardiness costs}
\bench{SingleMachineEarliTardi}

\subsection{Benchmark description}

\begin{itemize}
\item BKY: \cite{Bulbul2001}
\item SKS: \cite{Sourd2003}
\end{itemize}

For the benchmark SKS, we take only the biggest instances (50 activities) and randomly select among these instances: 10 instances for which the B\&B approach in \cite{Sourd2005} proved the optimality and 10 for which it could not prove the optimality, for these last instances, we compare with the results of the heuristic HEUR{\em n} described in the same paper.

For the benchmark BKY, we randomly selected 10 instances of the biggest problems with 200 activities and 10 instances with 100 activities and compare with the heuristics HEUR{\em n} of \cite{Sourd2005}. The selected instances are described on Table \ref{tab:SingleMachineEarliTardiInstances} together with the best known upper-bound and lower-bound when available.

\begin{table}[htb]
	\centering
		\begin{tabular}{|l|l|l|} \hline
		Problem & Lower bound & Upper bound \\ \hline
		
		sks524l	& 77519	& 77519\\ \hline
		sks524r	& 35160	& 35160\\ \hline
		sks525e	& 44885	& 44885\\ \hline
		sks527q	& 18706	& 18706\\ \hline
		sks538l	& 12489	& 12489\\ \hline
		sks544b	& 32682	& 32682\\ \hline
		sks545s	& 26683	& 26683\\ \hline
		sks555a	& 19781	& 19781\\ \hline
		sks576l	& 14163	& 14163\\ \hline
		sks587s	& 18588	& 18588\\ \hline \hline
		sks526u	& 		& 38051\\ \hline
		sks543k	& 		& 30849\\ \hline
		sks545z	& 		& 27455\\ \hline
		sks556u	& 		& 17090\\ \hline
		sks563a	& 		& 27590\\ \hline
		sks564g	& 		& 31302\\ \hline
		sks565a	& 		& 21073\\ \hline
		sks566f	& 		& 18058\\ \hline
		sks568x	& 		& 11650\\ \hline
		sks585x	& 		& 22765\\ \hline \hline
		bky100\_21	&	1215688.67 &	1241950 \\ \hline
		bky100\_33	&	825013.58  &	836286 \\ \hline
		bky100\_65	&	1630922.89 &	1659231\\ \hline
		bky100\_66	&	2720987.61 &	2738845\\ \hline
		bky100\_75	&	2703753.82 &	2737821\\ \hline
		bky100\_129	&	3333021.95 &	3407414\\ \hline
		bky100\_178	&	3633135.29 &	3651789\\ \hline
		bky100\_199	&	2932066.48 &	2974934\\ \hline
		bky100\_251	&	11094424   &	11131513 \\ \hline
		bky100\_298	&	6913028.19 &	6962141\\ \hline \hline
		bky200\_3	  &	4863520.4  &	4971667\\ \hline
		bky200\_4	  &	5317123.58 & 5378151 \\ \hline
		bky200\_64	&	5879030.06 & 5957791 \\ \hline
		bky200\_74	&	13485119.43 &	13716648 \\ \hline
		bky200\_107	&	5520158.49 &	5585045\\ \hline
		bky200\_218	&	16166389.17 &	16238878\\ \hline
		bky200\_219	&	12390622.3 &	12526464 \\ \hline
		bky200\_265	&	21171454.96 &	21345342 \\ \hline
		bky200\_269	&	32178497.8 &	32470446 \\ \hline
		bky200\_286	&	11757326.22 &	11870994 \\ \hline
		\end{tabular}
	\caption{Single-machine scheduling with earliness/tardiness cost instances}
	\label{tab:SingleMachineEarliTardiInstances}
\end{table}

\subsection{state-of-the-art}

\cite{Sourd2005}

\section{Aircraft assembly scheduling}
\bench{AircraftAssembly}

\subsection{Benchmark description}

\cite{Fox1995a}

\subsection{state-of-the-art}

\cite{Crawford1996}

\section{Air traffic flow management}
\bench{AirTrafficFlowManagement}

Only one instance. No reference. Compare with SetTimesForward.

\cite{ILOG2003a}

\section{Flow-shop scheduling}
\bench{FlowShop}

\subsection{Benchmark description}

\begin{itemize}
\item Heller: \cite{Heller1960}
\item Carlier: \cite{Carlier1978}
\item Reeves: \cite{Reeves1995}
\item Taillard: \cite{Taillard1993}
\end{itemize}

Use instances from Taillard only (no reference for other instances). 120 instances (20x5, 20x10, 20x20, 50x5, 50x10, 50x20, 100x5, 100x10, 100x20, 200x20, 500x20), some are open (50x20, 100x20, 200x20, 500x20). Choose randomly 2 instances of each group having less than 500 activities so total running time does not exceed 6 hours.

\section{Single-machine scheduling with common due-date}
\bench{CommonDueDate}

\subsection{Benchmark description}

We randomly selected 10 instances of the problems with 100 activities and 10 instances with 200 activities in the benchmark described in \cite{Biskup2001}. The selected instances are described on Table \ref{tab:SinglaMachineCommonDueDateInstances} together with the best known upper-bound and lower-bound when available.

\begin{table}[htb]
	\centering
		\begin{tabular}{|l|l|l|} \hline
		Problem & Upper bound \\ \hline
		sch100-1, h=0.8 & 72019\\ \hline
		sch100-3, h=0.2 & 137463\\ \hline
		sch100-3, h=0.4 & 85363\\ \hline
		sch100-3, h=0.6 & 68537\\ \hline
		sch100-4, h=0.4 & 87730\\ \hline
		sch100-4, h=0.6 & 69231\\ \hline
		sch100-5, h=0.2 & 136761\\ \hline
		sch100-8, h=0.4 & 95361\\ \hline
		sch100-9, h=0.6 & 58771\\ \hline
		sch100-10, h=0.2 & 124446\\ \hline\hline
		sch200-1, h=0.6 & 254268\\ \hline
		sch200-5, h=0.2 & 547953\\ \hline
		sch200-5, h=0.8 & 260455\\ \hline
		sch200-7, h=0.2 & 479651\\ \hline
		sch200-7, h=0.8 & 247555\\ \hline
		sch200-8, h=0.8 & 225572\\ \hline
		sch200-9, h=0.4 & 331107\\ \hline
		sch200-9, h=0.8 & 255029\\ \hline
		sch200-10, h=0.4 & 332808\\ \hline
		sch200-10, h=0.6 & 269236\\ \hline		
		\end{tabular}
	\caption{Single-machine scheduling with common due-date}
	\label{tab:SinglaMachineCommonDueDateInstances}
\end{table}	

Instances with 100 activities are run for 600s, instances with 200 activities are run for 1200s.
		
		
\section{Multi-stage hybrid flow-shop scheduling}
\bench{MultiStageHybridFlowShop}

\subsection{Benchmark description and state-of-the-art}

We randomly select 20 instances from the biggest ones described in \cite{Sivrikaya-Serifolu2004}. 10 instances with 20 jobs and 10 machines (200 operations), 5 instances with 50 jobs and 10 machines (500 operations) and 5 instances with 100 jobs and 10 machines (1000 operations). 

The selected instances are described on Table \ref{tab:MultiStageHybridFlowShop}.

\begin{table}[htb]
	\centering
		\begin{tabular}{|l|l|l|} \hline
		Problem & Lower bound & Upper bound \\ \hline
		2010HA1 & 1210 & 1565 \\ \hline 
		2010HA3 & 1468 & 1653 \\ \hline 
		2010HA4 & 1500 & 1660 \\ \hline 
		2010HA7 & 1295 & 1600 \\ \hline 
		2010HA10 & 1254 & 1603 \\ \hline 
		2010HB2 & 956 & 1437 \\ \hline 
		2010HB4 & 1116 & 1608 \\ \hline 
		2010HB6 & 1136 & 1538 \\ \hline 
		2010HB8 & 1017 & 1403 \\ \hline 
		2010HB10 & 912 & 1442 \\ \hline \hline 
		5010HA4 & 3073 & 3370 \\ \hline 
		5010HA5 & 2493 & 3080 \\ \hline 
		5010HA9 & 2646 & 3031 \\ \hline 
		5010HB1 & 1967 & 2815 \\ \hline 
		5010HB5 & 2209 & 3023 \\ \hline \hline 
		H10HA1 & 5327 & 5557 \\ \hline 
		H10HA3 & 5809 & 6112 \\ \hline 
		H10HB5 & 3321 & 4740 \\ \hline 
		H10HB8 & 3674 & 5238 \\ \hline 
		H10HB9 & 3645 & 4880 \\ \hline 
		\end{tabular}
	\caption{Multi-stage hybrid flow-shop instances}
	\label{tab:MultiStageHybridFlowShop}
\end{table}

\section{Single-processor scheduling with total tardiness cost}
\bench{MultiProcessorTotalTardiness}

\subsection{Benchmark description}

We randomly selected 20 instances among the ones described in \cite{Kara2002}: 10 among the biggest ones (500 activities), 10 for problems with 200 activities. The selected instances are described on Table \ref{tab:SingleProcessorTotalTardiness}.

\begin{table}[htb]
	\centering
		\begin{tabular}{|l|l|l|} \hline
		Problem & Upper bound \\ \hline
		inp200.14 & 69665 \\ \hline
		inp200.22 & 186170 \\ \hline
		inp200.43 & 6 \\ \hline
		inp200.44 & 24 \\ \hline
		inp200.50 & 12 \\ \hline
		inp200.86 & 0 \\ \hline
		inp200.119 & 376528 \\ \hline
		inp200.133 & 1683 \\ \hline
		inp200.168 & 0 \\ \hline
		inp200.199 & 403664 \\ \hline \hline
		inp500.2 & 47170 \\ \hline
		inp500.3 & 48193 \\ \hline
		inp500.43 & 45 \\ \hline
		inp500.49 & 26 \\ \hline
		inp500.72 & 2421810 \\ \hline
		inp500.146 & 819767 \\ \hline
		inp500.177 & 0 \\ \hline
		inp500.180 & 0 \\ \hline
		inp500.191 & 2591064 \\ \hline
		inp500.198 & 2607595 \\ \hline
		\end{tabular}
	\caption{Single-processor scheduling with total tardiness cost}
	\label{tab:SingleProcessorTotalTardiness}
\end{table}

\subsection{state-of-the-art}

\cite{Pavlin2003}

\section{Shop scheduling with set-up times}
\bench{UnaryAlternativeTransitionTime}

\subsection{Benchmark description}

Use instance from Brucker and Thielle: 15 jobshop instances with sequence dependent setup times (5x10x5, 5x15x5, 5x20x10). Other instances of type open-shop and generalized-shop have not been used as we don't have reference. Total runnning time is less than 2 hours. There is room to try up to 3 different seeds.

\cite{Brucker1996}

\subsection{state-of-the-art}

\cite{Artigues2006}

\section{Flow-shop with intermediate buffers}
\bench{FlowShopBuffers}


\subsection{Benchmark description}

We randomly selected 30 instances among the ones of size 20x5, 20x10, 20x20, 50x5, 50x10 and 100x5 from \cite{Taillard1993} and buffer sizes of $0$, $1$ and $2$. The selected instances are described on Table \ref{tab:FlowShopBuffers}. We use the classical time-limit depending on the size of the instance.

We compare to the results of \cite{Brucker2003}.


\begin{table}[htb]
	\centering
		\begin{tabular}{|l|l|l|l|} \hline
		Problem & Buffer size & Problem size & Upper bound \\ \hline
		ta002 & 0 & 100 & 1451\\ \hline
	  ta003 & 0 & 100 & 1353\\ \hline
	  ta015 & 0 & 200 & 1678\\ \hline
	  ta020 & 0 & 200 & 1806\\ \hline
	  ta037 & 0 & 250 & 3166\\ \hline
	  ta039 & 0 & 250 & 3045\\ \hline
	  ta021 & 0 & 400 & 2512\\ \hline
	  ta026 & 0 & 400 & 2414\\ \hline
	  ta043 & 0 & 500 & 3658\\ \hline
	  ta049 & 0 & 500 & 3771\\ \hline
	  ta001 & 1 & 100 & 1287\\ \hline
	  ta010 & 1 & 100 & 1134\\ \hline
	  ta013 & 1 & 200 & 1565\\ \hline
	  ta018 & 1 & 200 & 1582\\ \hline
	  ta034 & 1 & 250 & 2888\\ \hline
	  ta038 & 1 & 250 & 2769\\ \hline
	  ta021 & 1 & 400 & 2428\\ \hline
	  ta029 & 1 & 400 & 2310\\ \hline
	  ta067 & 1 & 500 & 5526\\ \hline
	  ta068 & 1 & 500 & 5393\\ \hline
	  ta007 & 2 & 100 & 1251\\ \hline
	  ta010 & 2 & 100 & 1115\\ \hline
	  ta017 & 2 & 200 & 1521\\ \hline
	  ta020 & 2 & 200 & 1642\\ \hline
	  ta034 & 2 & 250 & 2764\\ \hline
	  ta038 & 2 & 250 & 2697\\ \hline
	  ta024 & 2 & 400 & 2242\\ \hline
	  ta028 & 2 & 400 & 2249\\ \hline
	  ta047 & 2 & 500 & 3234\\ \hline
	  ta044 & 2 & 500 & 3129\\ \hline
	  \end{tabular}
	\caption{Flow-shop with intermediate buffers}
	\label{tab:FlowShopBuffers}
\end{table}

%\section{Shortest common super-sequence}
%\bench{SchedSCS}
%\subsection{Benchmark description and state-of-the-art}
%\cite{Cotta2005}

\bibliography{biblio}
%\bibliography{C:/D/biblio/biblio}
\bibliographystyle{plain}


\end{document}
