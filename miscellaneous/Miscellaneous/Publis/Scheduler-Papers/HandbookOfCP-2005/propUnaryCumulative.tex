%%%%%%%%%%%%%%%%%%%%%%%%%%%%%%%%%%%%%%%%%%%%%%%%%%%%%%%%%%%%%%%%%%%%%
%%%%%%%%%%%%%%%%%%%%%%%%%%%%%%%%%%%%%%%%%%%%%%%%%%%%%%%%%%%%%%%%%%%%%
\subsection{Unary resources (PBap)}
%%%%%%%%%%%%%%%%%%%%%%%%%%%%%%%%%%%%%%%%%%%%%%%%%%%%%%%%%%%%%%%%%%%%%
%%%%%%%%%%%%%%%%%%%%%%%%%%%%%%%%%%%%%%%%%%%%%%%%%%%%%%%%%%%%%%%%%%%%%
A set of $n$ activities $\{A_1, \ldots, A_n\}$ require the same
resource of capacity 1. As we will sse, the propagation of resource
constraints is a purely deductive process that allows to deduce
inconsistencies and to tighten the temporal characteristics of
activities and resources.

Several mechanisms have been developed to propagate unary resource
constraints. We restrict our study to the disjunctive constraint
propagation scheme and to the well-known edge-finding algorithm. Also
note that the time tabling mechanism described in Section
\ref{CUMULATIVE-RESOURCES} can be applied to unary resources. We
refer to \cite{BLN-BOOK} for a detailed introduction and comparison
of several other constraint propagation techniques.

The ideas underlying the disjunctive and the edge-finding propagation have been successfully applied to planning for propagating mutex sets, that is, sets of actions such that any two actions in the set are mutex \cite{vidal-geffner-05}. 

\subsubsection{Disjunctive Constraint Propagation} 
\label{DisjConstraintProp}
Two activities $A_i$ and $A_j$ requiring
the same machine cannot overlap in time. So, either $A_i$ precedes $A_j$
or $A_j$ precedes $A_i$. If $n$ activities require the resource, the
constraint can be implemented as $ n (n - 1) / 2$ (explicit or
implicit) disjunctive constraints. As for Time-Table constraints,
variants exist in the literature
\cite{Erschler76,Carlier84,Esquirol87,LePape88,SmithCheng93,VarnierEtAl93,BaptisteLePape95a}, but in most cases the propagation
consists of maintaining arc-B-consistency on the formula:
\begin{equation*}
[\evar(A_i) \le \svar(A_j)] \vee [\evar(A_j) \le \svar(A_i)]
\end{equation*}

Enforcing arc-B-consistency on this formula is done as follows:
Whenever the earliest end time of $A_i$ exceeds the latest start time of
$A_j$, $A_i$ cannot precede $A_j$; hence $A_j$ must precede $A_i$. The
time bounds of $A_i$ and $A_j$ are consequently updated with respect
to the new temporal constraint $\evar(A_j) \le \svar(A_i)$. Similarly,
when the earliest possible end time of $A_j$ exceeds the latest
possible start time of $A_i, A_j$ cannot precede $A_i$. When neither
of the two activities can precede the other, a contradiction is
detected.  Disjunctive constraints provide more precise time bounds
than the corresponding Time-Table constraints. Indeed, if an activity
$A_j$ is known to execute at some time $t$ between the earliest start
time $\estm_i$ and the earliest end time $\eetm_i$ of $A_i$, then the
first disjunct of the above formula is false. Thus, $A_j$ must
precede $A_i$ and the propagation of the disjunctive constraint
implies $\svar(A_i) \ge \eetm_j > t$.

\begin{center}
\begin{figure}[htbp]
\epsfxsize=12cm
\centering\mbox{\epsfbox{propDisj.eps}}
\caption{Propagation of the disjunctive constraint.
\label{PROP-DISJ-FIG}}
\end{figure}
\end{center}

Figure \ref{PROP-DISJ-FIG} shows that disjunctive constraints may
propagate more than Time-Table constraints. The earliest end time of
each activity does not exceed its latest start time, so the Time-Table
constraint cannot deduce anything. On the contrary, the propagation of
the disjunctive constraint imposes $\evar(A_1) \le \svar(A_2)$ which,
in turn, results in updating both $\letm_1$ and $\estm_2$.


\subsubsection{Edge-Finding} 

 
The Edge-Finding bounding technique consists of deducing that some
activities from a given set $\Omega$ must, can, or cannot, execute
first (or last) in $\Omega$. Such deductions lead to new ordering
relations (``edges'' in the graph representing the possible orderings
of activities) and new time bounds, \ie, strengthened earliest start
times and latest end times of activities.

In the following, $\estm_\Omega$ and $\eetmset_\Omega$ denote
respectively the smallest of the earliest start times and the smallest
of the earliest end times of the activities in $\Omega$.  Similarly,
$\letm_\Omega$ and $\lstmset_\Omega$ denote respectively the largest of
the latest end times and the largest of the latest start times of the
activities in $\Omega$. Finally, let $p_\Omega$ be the sum of the
minimal processing times of the activities in $\Omega$. Let $A_i \ll
A_j$ ($A_i \gg A_j$) mean that $A_i$ executes before (after) $A_j$ and
$A_i \ll \Omega$ ($A_i \gg \Omega$) mean that $A_i$ executes before
(after) all the activities in $\Omega$. Once again, variants exist
\cite{Pinson88,Carlier89,Carlier90,Carlier94,Caseau94,Nuijten94,BruckerThiele96,MartinShmoys96,Peridy96,XXXREFRECENTES} but the following rules
capture the ``essence'' of the Edge-Finding bounding technique:
\begin{eqnarray}
\label{EF-RULES-EQ1}
\forall \Omega, \forall A_i \notin \Omega, 
[\letm_{\Omega \cup \{A_i\}} - \estm_{\Omega} < p_{\Omega} + p_i] 
\Rightarrow [A_i \ll \Omega] \\
\label{EF-RULES-EQ2}
\forall \Omega, \forall A_i \notin \Omega, 
[\letm_{\Omega} - \estm_{\Omega \cup \{A_i\}} < p_{\Omega} + p_i] 
\Rightarrow [A_i \gg \Omega] \\
\label{EF-RULES-EQ3}
\forall \Omega, \forall A_i \notin \Omega, 
[A_i \ll \Omega] \Rightarrow 
[\evar(A_i) \le \min_{\emptyset \neq \Omega' \subseteq \Omega} (\letm_{\Omega'} - p_{\Omega'})] \\
\label{EF-RULES-EQ4}
\forall \Omega, \forall A_i \notin \Omega, 
[A_i \gg \Omega] \Rightarrow 
[\svar(A_i) \ge \max_{\emptyset \neq \Omega' \subseteq \Omega} (\estm_{\Omega'} + p_{\Omega'})] 
\end{eqnarray}

If $n$ activities require the resource, there are a priori $O(n *
2^n)$ pairs $(A_i, \Omega)$ to consider. Still, it is easy to see that
when a rule is applied for some set $\Omega$, the same rule provides
even better deductions for the superset $\Omega' = \{A_i: [\estm_i,
\letm_i) \subseteq [\estm_{\Omega}, \letm_{\omega})\}$. This enables
us to restrict the rules to sets $\Omega_I$ made of all activities
whose time window belongs to some interval $I$. As there are no more
that $O(n^2)$ such $I$, we have a straightforward polynomial algorithm, running in cubic time, to implement the edge-finding rules. 
Algorithms that perform
all the time-bound adjustments in $O(n^2)$ are presented in
\cite{Carlier90, Nuijten93a, Nuijten94}. Another variant of the Edge-Finding technique is presented in
\cite{Carlier94}. It runs in $O(n \log n)$ but requires much more
complex data structures.

An interesting property of the Edge-Finding technique is established
in \cite{Baptiste95} and in \cite{MartinShmoys96}: 

\begin{lemma}
Considering only the resource constraint and the current time bounds
of activities, the Edge-Finding algorithm computes the smallest
earliest start time at which each activity $A_i$ could start if all
the other activities were interruptible.
\end{lemma}

%%%%%%%%%%%%%%%%%%%%%%%%%%%%%%%%%%%%%%%%%%%%%%%%%%%%%%%%%%%%%%%%%%%%%
%%%%%%%%%%%%%%%%%%%%%%%%%%%%%%%%%%%%%%%%%%%%%%%%%%%%%%%%%%%%%%%%%%%%%
\subsection{Cumulative resources (PBap)} \label{CUMULATIVE-RESOURCES}
%%%%%%%%%%%%%%%%%%%%%%%%%%%%%%%%%%%%%%%%%%%%%%%%%%%%%%%%%%%%%%%%%%%%%
%%%%%%%%%%%%%%%%%%%%%%%%%%%%%%%%%%%%%%%%%%%%%%%%%%%%%%%%%%%%%%%%%%%%%
Cumulative resource constraints represent the fact that activities
$A_i$ use some amount $\capvar(A_i)$ of resource throughout their
execution. 
Many algorithms have been proposed 
for the propagation of the non-preemptive
cumulative constraint. A limited  subset of these algorithms is
presented in this paper. 

%====================================================================
\subsection{Time-Table Constraint}
\label{TimeTableConstraint}
%====================================================================

First we consider the simple Time-Table mechanism, widely used in
Constraint-Based Scheduling tools, that allows to propagate the
resource constraint in an incremental fashion. The ``Time-Table'' is
used to maintain information about resource utilization and resource
availability over time. Resource constraints are propagated in two
directions: from resources to activities, to update the time bounds of
activities (earliest start times and latest end times) according to
the availability of resources; and from activities to resources, to
update the Time-Tables according to the time bounds of
activities. Although several variants exist \cite{LePape88, Fox90,
LePape94, Smith94, Lock96} the propagation mainly consists of
maintaining for any time $t$ arc-B-consistency on the formula:
$$
\forall t 
\sum_{\svar(A_i) \le t < \evar(A_i)} \capvar(A_i) \le C
$$

%====================================================================
\subsection{Disjunctive Constraint}

%====================================================================
Two activities $A_i$ and $A_j$ such that $c_i + c_j > C$ cannot
overlap in time. Hence either $A_i$ precedes $A_j$ or $A_j$ precedes
$A_i$, \ie, the disjunctive constraint holds between these
activities. Arc-B-consistency is then achieved on the formula
\begin{eqnarray*}
&~    & [\capvar(A_i) + \capvar(A_j) \le \capvar] \\
&\vee & [\evar(A_i) \le \svar(A_j)]               \\
&\vee & [\evar(A_j) \le \svar(A_i)]
\end{eqnarray*}



%====================================================================
\subsection{Energetic Reasoning}
%====================================================================

XXX
\cite{Erschler91} and \cite{LopezErschler92}
propose a sharper definition of the required energy consumption that
takes into account the fact that activities cannot be
interrupted. 
Given an
activity $A_i$ and a time interval $[t_1, t_2), \Wsh(A_i, t_1, t_2)$,
the ``Left-Shift / Right-Shift'' required energy consumption of $A_i$
over $[t_1, t_2)$ is $c_i$ times the minimum of the three following
durations.
\begin{itemize}
\item
$t_2 - t_1$, the length of the interval;
\item
$p_i^{+}(t_1) = \max(0, p_i - \max(0, t_1 - \estm_i))$, the number of time
units during which $A_i$ executes after time $t_1$ if $A_i$ is
left-shifted, \ie, scheduled as soon as possible; 
\item
$p_i^{-}(t_2) = \max(0, p_i - \max(0, \letm_i - t_2))$, the number of time
units during which $A_i$ executes before time $t_2$ if $A_i$ is
right-shifted, \ie, scheduled as late as possible.  
\end{itemize}
This leads to
$\Wsh(A_i, t_1, t_2) = c_i \min(t_2 - t_1, p_i^{+}(t_1),
p_i^{-}(t_2))$. The required energy consumption of $A_1$ over $[2, 7)$
on Figure \ref{REQ-LSRS-FIG} is 8. Indeed, at least 4 time units of $A_1$
have to be executed in $[2, 7)$; \ie, $\Wsh(A, 2, 7) = 2 \min(5, 5,
4) = 8$.
\begin{center}
\begin{figure}[htbp]
\epsfxsize=12cm
\centering\mbox{\epsfbox{reqEneygyLSRS.eps}}
\caption{Left-Shift / Right-Shift. \label{REQ-LSRS-FIG}}
\end{figure}
\end{center}

We can now define the Left-Shift / Right-Shift overall required energy
consumption $\Wsh(t_1, t_2)$ over an interval $[t_1, t_2)$ as the
sum over all activities $A_i$ of $\Wsh(A_i, t_1, t_2)$. We can also
define the Left-Shift / Right-Shift slack over $[t_1, t_2)$:
$\Ssh(t_1, t_2) = C (t_2 - t_1) - \Wsh(t_1, t_2)$. It is obvious
that if there is a feasible schedule then
$\forall t_1, \forall t_2 \ge t_1, \Ssh(t_1, t_2) \ge 0$.

It is shown in \cite{BLNbook} that the set of intervals, for which the slack needs to be
calculated to guarantee that no interval with negative slack exists, is not verylarge.
Let us define the sets $O_1$, $O_2$ and $O(t)$.
\begin{eqnarray*}
O_1 &=& \{\estm_i, 1 \le i \le n\} \cup 
     \{\letm_i - p_i, 1 \le i \le n\} \cup
     \{\estm_i + p_i, 1 \le i \le n\} \\
O_2 &=& \{\letm_i, 1 \le i \le n\} \cup
     \{\estm_i + p_i, 1 \le i \le n\} \cup 
     \{\letm_i - p_i, 1 \le i \le n\} \\
O(t) &=& \{\estm_i + \letm_i - t, 1 \le i \le n\} \\
\end{eqnarray*}

\begin{lemma} \label{RELEVANT-INTERVALS-LSRS}
$$\forall t_1, \forall  t_2 \ge  t_1, \Ssh(t_1, t_2) \ge 0
\Leftrightarrow 
\left\{
\begin{array}{l}
\forall  s \in  O_1, \forall  e \in  O_2, e \ge  s , \Ssh(s, e) \ge 0 \\
\forall  s \in  O_1, \forall  e \in  O(s), e \ge  s, \Ssh(s, e) \ge 0 \\
\forall  e \in  O_2, \forall  s \in  O(e), e \ge  s, \Ssh(s, e) \ge 0 \\
\end{array}
\right.
$$
\end{lemma}

Lemma \ref{RELEVANT-INTERVALS-LSRS} provides a characterization
of interesting intervals over which the slack must be computed to
ensure it is always non-negative over any interval. This
characterization is weaker than the one proposed for the partially
elastic case where the interesting time intervals $[t_1, t_2)$ are in
the Cartesian product of the set of the earliest start times and of the set
of latest end times. However, there are still only $O(n^2)$ relevant pairs
$(t_1, t_2)$ to consider. Some of these pairs belong to the Cartesian
product $O_1 * O_2$. The example of Figure
\ref{RELEVANT-POINTS-LSRS-FIG} proves that some pairs do not. In this
example, (resource of capacity 2 and 5 activities $A_1, A_2, A_3, A_4,
A_5$), the pair (1, 9) corresponds to the minimal slack and does not
belong to $\{0, 1, 4, 5, 6\} * \{4, 5, 6, 8, 10\}$. In this interval,
the slack is negative, which proves that there is no feasible
schedule. Notice that neither the fully elastic nor the partially
elastic relaxation can trigger a contradiction.
\begin{center}
\begin{figure}[htbp]
\epsfxsize=12cm
\centering\mbox{\epsfbox{relevPoints.eps}}
\caption{Some relevant time intervals are outside  $O_1 * O_2$
\label{RELEVANT-POINTS-LSRS-FIG}}
\end{figure}  
\end{center}
Since there are $O(n^2)$ relevant time intervals, the minimum slack
computation can be obviously achieved in cubic time. This time
complexity can be improved to $O(n^2)$ \cite{BLNbook}.

As for partially elastic adjustments, the values of $\Wsh$ can be
used to {\em adjust time bounds}. Given an activity $A_i$ and a time
interval $[t_1, t_2)$ with $t_2 < \letm_i$, we examine whether $A_i$ can
end before $t_2$.
\begin{lemma}
\label{TIME-BOUND-ADJUSTMENTS-LSRS-CUSP-PROP}
If there is a time interval $[t_1, t_2)$ such that 
$$
\Wsh(t_1, t_2) - \Wsh(A_i, t_1, t_2) + c_i p_i^{+}(t_1) > C (t_2 - t_1)
$$
then a valid lower bound of the end time of $A_i$ is:
$$
t_2 + \frac{1}{c_i}(\Wsh(t_1, t_2) - \Wsh(A_i, t_1, t_2) + c_i p_i^{+}(t_1) - C (t_2 - t_1))
$$
\end{lemma}

Similarly, when
$$
\Wsh(t_1, t_2) - \Wsh(A_i, t_1, t_2) + c_i \min(t_2 - t_1, p_i^{+}(t_1)) > C (t_2 - t_1),
$$
$A_i$ cannot start before $t_1$ and a valid
lower bound of the start time of $A_i$ is
$$
t_2 - \frac{1}{c_i}(C (t_2 - t_1) - \Wsh(t_1, t_2) + \Wsh(A_i, t_1,t_2)).
$$

There is an obvious $O(n^3)$ algorithm to compute all the adjustments
corresponding to Proposition \ref{TIME-BOUND-ADJUSTMENTS-LSRS-CUSP-PROP}
which can be obtained on the intervals $[t_1, t_2)$ which correspond
to potential local minima of the slack function. There are $O(n^2)$
intervals of interest and $n$ activities which can be adjusted. Given
an interval and an activity, the adjustment procedure runs in
$O(1)$. The overall complexity of the algorithm is then $O(n^3)$. An
interesting open question is whether there is a quadratic algorithm to
compute all the adjustments on the $O(n^2)$ intervals under
consideration.
\begin{algorithm}[htbp]
\caption{An algorithm for Left-Shift / Right-Shift adjustments}
\label{TIME-BOUND-ADJUSTMENTS-LSRS-CUSP-ALGO}
\begin{algorithmic}[1]
\FOR{all relevant time-intervals $[t_1, t_2)$ (\cf, Proposition \ref{RELEVANT-INTERVALS-LSRS})}
  \STATE $W \la 0$
  \FOR{$i \in \{1, \ldots, n\}$}
    \STATE{$W \la W + c_i \min(t_2 - t_1, p_i^{+}(t_1), p_i^{-}(t_2))$}
  \ENDFOR
  \IF{$W > C (t_2 - t_1)$}
    \STATE{there is no feasible schedule, exit}
  \ELSE
    \FOR{$i \in \{1, \ldots, n\}$}
      \STATE $\slk \la C (t_2 - t_1) - W + c_i \min(t_2 - t_1, p_i^{+}(t_1), p_i^{-}(t_2))$
      \IF{$\slk < c_i p_i^{+}(t_1)$}
        \STATE{$\eetm_i \la \max(\eetm_i, t_2 + \lceil(c_i p_i^{+}(t_1) - \slk) / c_i\rceil)$}
      \ENDIF
      \IF{$\slk < c_i p_i^{-}(t_2)$}
        \STATE{$\lstm_i \la \min(\lstm_i, t_1 - \lceil (c_i p_i^{-}(t_2) - \slk)/c_i \rceil)$}
      \ENDIF
    \ENDFOR
  \ENDIF
\ENDFOR
\end{algorithmic}
\end{algorithm}

Several questions are still open
at this point.
\begin{itemize}
\item
First, for the Left-Shift / Right-Shift technique, we have shown that
the energetic tests can be limited to $O(n^2)$ time intervals. We have
also provided a precise characterization of these intervals. However,
it could be that this characterization can be sharpened in order to
eliminate some intervals and reduce the practical complexity of the
corresponding algorithm.  
\item
Second, it seems reasonable to think that our time-bound adjustments
could be sharpened. Even though the energetic tests can be limited
(without any loss) to a given set of intervals, it could be that the
corresponding adjustment rules cannot.
\end{itemize} 


XXX others ???? I dont think so





























