%\tracingcommands=2\tracingmacros=2

%\chaptertitleauthors[This is the first chapter]
%  {This is the first chapter and we use a long title for it}
%  {Simon Pepping, Andy Deelen, Joyce Happee, Bestsy Lightfoot, Hans
%  Oosterom and George Burlage}
%\authormark{Simon Pepping et al.}


%\section{This is the second section and again we use a long title for it}
% ??? \sectionmark{This is the second section}

%\footnote{A book}: 
% the front matter\index{front matter}

%    See \citet{bib:2}.
% books have all the elements described below \citep[e.g.][Ch. 2]{bib:3}.

% \section*{Problems}
% \sectionmarknonum{Problems}
% \addcontentsline{toc}{section}{Problems}
% \begin{small}


%%%%%%%%%%%%%%%%%%%%%%%%%%%%%%%%%%%%%%%%%%%%%%%%%%%%%%%%%%%%%%
%%%%%%%%%%%%%%%%%%%%%%%%%%%%%%%%%%%%%%%%%%%%%%%%%%%%%%%%%%%%%%
%%%%%%%%%%%%%%%%%%%%%%%%%%%%%%%%%%%%%%%%%%%%%%%%%%%%%%%%%%%%%%


%?\usepackage{fullpage,latexsym,amsmath,amsfonts,amsthm}
%?\usepackage{color,epsfig,algorithm,algorithmic}

%already done: \usepackage[latin1]{inputenc} % because there are accents in the bibtex file

%\usepackage[pdftex]{graphicx} % instructions for generating the pdf file.


\input{macroshcp.tex}

\newcommand{\SetFigFont}[6]{#6}


\chaptertitleauthors[Constraint-Based Scheduling and Planning]
  {Constraint-Based Scheduling and Planning}
  {Philippe Baptiste, Philippe Laborie, Claude Le Pape and Wim Nuijten}
\authormark{Baptiste et al.}


% Alternative titles: Constraint Programming for Scheduling and Planning
% Constraint-Based Scheduling and Planning
% \chaptertitleauthors[Applying Constraint Programming to Scheduling and Planning]

%\thanks{CNRS LIX, Ecole Polytechnique,
%91128 Palaiseau, France.
%baptiste@lix.polytechnique.fr.}

% The \institute command lets you specify the your affiliation and
% you address. Seperate two or more different affiliations by the
% \and command. 
%\institute{
%  ILOG S.A., Gentilly, France. \\
%  e-mail: {\tt  {nuijten,tbousonville,focacci,dgodard,clepape@ilog.fr}}\\
%  url: {\tt http://www.ilog.com}\\
%}

%\maketitle

%\begin{abstract}
%\end{abstract}

\index{scheduling|(}
\index{planning|(}

%%%%%%%%%%%%%%%%%%%%%%%%%%%%%%%%%%%%%%%%%%%%%%%%%%%%%%%%%%%%%%%%%%%%%
%%%%%%%%%%%%%%%%%%%%%%%%%%%%%%%%%%%%%%%%%%%%%%%%%%%%%%%%%%%%%%%%%%%%%

Solving a scheduling problem generally consists in allocating scarce
\index{resource}{resources} to a given set of
\index{activity}{activities} over time. Planning can be seen as a
generalization of scheduling where the set of activities to be
scheduled is not known in advance. The additional complexity of
planning thus lies in the fact one also has to decide on the set of
activities that will be scheduled. \index{scheduling!constraint
based}{Constraint-Based Scheduling} is the discipline that studies how
to solve scheduling problems by using Constraint Programming
(CP). \index{planning!constraint based}{Constraint-Based Planning} in
turn is the discipline that studies how to solve planning problems
by CP. As the use of CP in scheduling is more mature, we first turn to
Constraint-Based Scheduling after which we will come back to
Constraint-Based Planning. 

Constraint-Based Scheduling has over the years grown into one of the
most successful application areas of CP. One of the key factors of
this success lies in the fact that a combination was found of the best
of two fields of research that pay attention to scheduling, namely
\index{operations research}{{\em Operations Research}} (OR) and \index{artificial intelligence}{{\em Artificial Intelligence}} (AI).
Traditionally, a lot of the attention in OR has been paid to rather
``pure'' scheduling problems that are based on relatively simple
mathematical models. For solving the problem at hand, the
combinatorial structure of the problem is heavily exploited, leading
to improved performance characteristics. We could say that an OR
approach often aims at achieving a high level of {\em efficiency} in
its algorithms. However, when modeling a practical scheduling problem
using these classical models, one is often forced to discard degrees
of freedom and side constraints that exist in the practical scheduling
situation.  Discarding degrees of freedom may result in the
elimination of interesting solutions, regardless of the solution
method used. Discarding side constraints gives a simplified problem
and solving this simplified problem may result in impractical
solutions for the original problem.

In contrast, AI research tends to investigate more general scheduling
models and tries to solve the problems by using general
problem-solving paradigms. We could say an AI approach tends to focus
more on the {\em generality of application} of its algorithms. This,
however, implies that AI algorithms may perform poorly on specific
cases, compared to OR algorithms.

So, on one hand we have \index{operations research}{OR} which offers us {\em efficient}
algorithms to solve problems that in comparison have a more limited
application area. On the other hand we have AI that offers us
algorithms that are more {\em generally applicable}, but that might
suffer from somewhat poor performance in the specific cases an
efficient OR algorithm exists. An important way to combine the two was
found by incorporating OR algorithms inside \index{constraint!global}{{\em global}
constraints}. Such algorithms are able to take into account a set of
constraints from a global point of view in an efficient way. The
typical scheduling example of a global constraint is the constraint
that propagates on the combination of all activities requiring
capacity from a shared resource. The basics of many of the algorithms
inside global constraints, certainly in the early stages of the field,
can be found in OR. By applying the locality principle
\cite{Steele80}, such specialized algorithms can work side by side
with general propagation algorithms that take care of the rest of the
constraints. In this way one can preserve the general modeling and
problem-solving paradigm of CP while the integration of efficient
propagation algorithms improves the overall performance of the
approach. Stated in another way, efficient OR algorithms integrated in
a CP approach allow the user to benefit from the efficiency of OR
techniques in a flexible framework. Translated to the area of
\index{scheduling!constraint based}{Constraint-Based Scheduling} two strengths emerge: i) natural and flexible
modeling of scheduling problems as Constraint Satisfaction Problems
(CSPs) \cite{Tsang93} and ii) powerful propagation of
temporal and resource constraints. All this said, we want to remark
that over the years the distinction between AI and OR is often
becoming less and less clear and is also deemed less and less
important.

%% Planning comes into play

Let us now step from Constraint-Based Scheduling to Constraint-Based
Planning. There are basically two approaches for applying CP to planning:
\begin{itemize}
\item The first approach (see Section \ref{PlanSpaceSearch}) dates
back to the first attempts to build non-linear plans in the 70s
\cite{Sacerdoti75,Chapman87,McAllesterRosenblitt91} and consists in
refining a partial plan made of a temporal network of
activities. Similarly to Constraint-Based Scheduling, constraint
propagation can be used in this temporal network to propagate
temporal, state, and resource constraints. These approaches are
flexible enough to handle {\em complex and realistic} planning
problems \cite{SmithFrank00}. Although in these approaches the idea
of constraint propagation has been present since the beginning, it is only in
recent years that efficient \index{constraint!global}{global propagation algorithms},
partially inspired from the ones available in Constraint-Based
Scheduling have been designed \cite{VidalGeffner05}.

\item 
The second approach (see Section \ref{csp-graphplan}) consists in
compiling the planning problem into a CSP and use CSP or SAT solvers
as a blackbox to solve the problem \cite{BlumFurst97,VanBeekChen99,
DoKambhampati01, LopezBacchus03}. These approaches work on a
simplification of the real planning problem expressed in a STRIPS-like
formalism and tend to focus on {\em efficiency} rather than on
generality of application. Several ideas stemming from this approach
can be used to provide efficient global constraint propagation
algorithms and heuristics to guide the search \cite{VidalGeffner05}.

\end{itemize}

Although the use of CP in planning is, due to the problem complexity,
less mature than its use in scheduling, \index{planning!constraint
based}{Constraint-Based Planning} thus follows the same pattern as
Constraint-Based Scheduling where CP is used as a framework for
integrating efficient special purpose algorithms into a flexible and
expressive paradigm.

Besides the strengths mentioned above, we want to mention two more
reasons that contributed to the success of
\index{scheduling!constraint based}{Constraint-Based Scheduling}: a
natural fit of expressing scheduling specific heuristics with CP tree
search, and a proven good potential of combining the CP approach with
solution techniques as Local Search, Large Neighborhood Search, and
Mixed Integer Programming. The first allows to both exploit all the
work done on scheduling heuristics in the past as to write new
scheduling heuristics with easy to understand scheduling
semantics. The second allows to get improved performance in the cases
where that is needed. The same strengths should also benefit
\index{planning!constraint based}{Constraint-Based Planning}.

The remainder of this chapter is organized as
follows. Section~\ref{SCPModelsSched} presents CP models for
scheduling together with descriptions of propagation techniques for
constraints used in these models. Section~\ref{SCPModelsPlan} does the
same for planning problems. As over the years dedicated global
constraint propagation techniques for resource constraints have
received a lot of attention, they are discussed in more detail in
Section~\ref{SPropRct}. In Section~\ref{SPropOptCrit} constraint
propagation techniques on optimization criteria are discussed, after
which Section~\ref{SHeuristics} pays attention to the search
procedures that are used in \index{planning!constraint
based}{Constraint-Based Planning} and \index{scheduling!constraint
based}{Scheduling} to solve resource constraints. Finally in 
Section~\ref{SConclusion} conclusions
are presented together with identification of potential future
research directions.



%%%%%%%%%%%%%%%%%%%%%%%%%%%%%%%%%%%%%%%%%%%%%%%%%%%%%%%%%%%%%%%%%%%%%
%%%%%%%%%%%%%%%%%%%%%%%%%%%%%%%%%%%%%%%%%%%%%%%%%%%%%%%%%%%%%%%%%%%%%

%%%%%%%%%%%%%%%%%%%%%%%%%%%%%%%%%%%%%%%%%%%%%%%%%%%%%%%%%%%%%%%%%%%%%
%%%%%%%%%%%%%%%%%%%%%%%%%%%%%%%%%%%%%%%%%%%%%%%%%%%%%%%%%%%%%%%%%%%%%
\section{Constraint Programming Models for Scheduling} \label{SCPModelsSched}
%%%%%%%%%%%%%%%%%%%%%%%%%%%%%%%%%%%%%%%%%%%%%%%%%%%%%%%%%%%%%%%%%%%%%
%%%%%%%%%%%%%%%%%%%%%%%%%%%%%%%%%%%%%%%%%%%%%%%%%%%%%%%%%%%%%%%%%%%%%

In this section we give an overview of the kind of scheduling problems that are studied by the \index{scheduling!constraint based}{Constraint-Based Scheduling} community. We present the different components together with a description of how these components can be modeled as a part of a CSP. 

Throughout this chapter we use the following notation. Let $\{A_1,
\ldots, A_n\}$ be a set of $n$ activities and $\{R_1, \ldots, R_m\}$ a
set of $m$ resources. Let's for now consider a basic scheduling
problem where each of the activities has a processing time and
requires a certain capacity from one or several resources. The
resources have a given capacity that can not be exceeded at any point
in time. There may furthermore be a set of temporal constraints
between activities and an objective function. The problem to be solved
is to decide when to execute each activity to optimize the objective
function, while respecting both temporal and resource
constraints. Later on in this section several extensions are
discussed, but for now this basic scheduling problem suffices for the
discussion.



%=====================================================================
\subsection{Activities} 
%=====================================================================

\index{activity|(}
When looking at the type of activities in a
problem, we distinguish {\em non-preemptive scheduling}, {\em
preemptive scheduling}, and {\em elastic scheduling}. In
\index{scheduling!non-preemptive}{non-preemptive scheduling}, activities cannot be interrupted. Each
activity must execute without interruption from its start time to its
end time. In \index{scheduling!preemptive}{preemptive scheduling}, activities can be interrupted at
any time, \eg, to let some other activities execute. In \index{scheduling!elastic}{elastic
scheduling} the amount of resource assigned to an activity $A_i$ can,
at any time $t$, assume any value between $0$ and the resource
capacity, provided that the sum over time of the assigned capacity
equals a given value called {\em energy}. The equivalent notion of
energy in the case of a non-preemptive activity is the product of its
processing time and the capacity required. 

A \index{scheduling!non-preemptive}{non-preemptive scheduling} problem can be efficiently encoded as a
CSP in the following way. For each activity three variables are
introduced, $\svar(A_i)$, $\evar(A_i)$, and $\pvar(A_i)$. They
represent the start time, the end time, and the processing time of
$A_i$, respectively. We are not aware of CP approaches not using this encoding.

With $\estm_i$ the release date and $\letm_i$ the deadline of activity
$A_i$ as defined in the initial data of the scheduling problem,
$[\estm_i, \letm_i]$ is the time window in which $A_i$ has to execute.
Based on that the initial domains of $\svar(A_i)$ and $\evar(A_i)$ are
$[\estm_i, \lstm_i]$ and $[\eetm_i, \letm_i]$, respectively. Here
$\lstm_i$ and $\eetm_i$ stand for the latest start time and the
earliest end time of $A_i$. For convenience, we also use this notation
to denote the current domains of $\svar(A_i)$ and $\evar(A_i)$, \ie,
the domains when we are in the process of propagating constraints. Of
course in that case instead of the initial release date and deadline,
$\estm_i$ and $\letm_i$ denote the current earliest start time and
latest end time

The processing time of the activity is defined as the difference
between the end time and the start time of the activity: $\pvar(A_i) =
\evar(A_i) - \svar(A_i)$.  $p_i$ denotes the smallest value in the
domain of $\pvar(A_i)$. All data related to an activity are summarized
in Figure \ref{ACTIVITY-DATA-FIG}. Light gray is used to depict the
time window $[\estm_i, \letm_i]$ of an activity and dark gray is used
to represent the processing time of the activity.

\begin{center}
\begin{figure}[htbp]
\epsfxsize=10cm
\centering\mbox{\epsfbox{figures/activity.eps}}
\caption{Data related to an activity \label{ACTIVITY-DATA-FIG}}
\end{figure}
\end{center}

\index{scheduling!preemptive}{Preemptive scheduling} problems are more
difficult to represent since a schedule is more complex than a simple
set of start and end times of activities. We discuss two
possibilities. One can either associate a set variable (\ie, a
variable the value of which will be a set) $\setvar(A_i)$ with each
activity $A_i$, or alternatively define a 0-1 variable $X(A_i, t)$ for
each activity $A_i$ and time $t$.  $\setvar(A_i)$ represents the set
of times at which $A_i$ executes, while $X(A_i, t)$ takes value 1 if
and only if $A_i$ executes at time $t$. The processing time
$\pvar(A_i)$ of $A_i$ is defined as the number of time points $t$ at
which $A_i$ executes, \ie, as $|\setvar(A_i)|$. In practice, the
$X(A_i, t)$ variables are not represented explicitly as the value of
$X(A_i, t)$ is 1 if and only if $t$ belongs to $\setvar(A_i)$.

Assuming time is discretized, $\svar(A_i)$ and $\evar(A_i)$ can be
defined by $\svar(A_i) = \min_{t \in \setvar(A_i)} t$ and $\evar(A_i)
= \max_{t \in \setvar(A_i)} t + 1$. Notice that in the non-preemptive
case, $\setvar(A_i) = [\svar(A_i), \evar(A_i))$, with the interval
$[\svar(A_i), \evar(A_i))$ closed on the left and open on the right so
that $|\setvar(A_i)| = \evar(A_i) - \svar(A_i) = \pvar(A_i)$.

These constraints are easily propagated by maintaining a lower bound
and an upper bound for the set variable $\setvar(A_i)$. The lower
bound $\lb(\setvar(A_i))$ is a series of disjoint intervals ${\emph
ILB}_i^u$ such that each ${\emph ILB}_i^u$ is constrained to be included
in $\setvar(A_i)$.  The upper bound $\ub(\setvar(A_i))$ is a series of
disjoint intervals ${\emph IUB}_i^v$ such that $\setvar(A_i)$ is
constrained to be included in the union of the ${\emph IUB}_i^v$. If the
size of the lower bound (\ie, the sum of the sizes of the ${\emph
ILB}_i^u$) becomes larger than the upper bound of $\pvar(A_i)$ or if
the size of the upper bound (\ie, the sum of the sizes of the ${\emph
IUB}_i^v$) becomes smaller than the lower bound of $\pvar(A_i)$, a
contradiction is detected. If the size of the lower bound (or of the
upper bound) becomes equal to the upper bound (respectively, lower
bound) of $\pvar(A_i)$, $\setvar(A_i)$ receives the lower bound
(respectively, the upper bound) as its final value. Minimal and
maximal values of $\svar(A_i)$ and $\evar(A_i)$, \ie, earliest and
latest start and end times, are also maintained. Each of the following
rules, considered independently one from another, is used to update
the bounds of $\setvar(A_i)$, $\svar(A_i)$ and $\evar(A_i)$. Let $t$
be any point in time, then
\begin{eqnarray*}
&~&t < \estm_i \Rightarrow t \notin \setvar(A_i)  \\
&~&t \in \lb(\setvar(A_i)) \Rightarrow \svar(A_i) \leq t \\
&~&\letm_i \leq t \Rightarrow t \notin \setvar(A_i) \\
&~&t \in \lb(\setvar(A_i)) \Rightarrow t < \evar(A_i)  \\
&~&[\forall_{u < t} \; u \notin \ub(\setvar(A_i))]\Rightarrow {t} \leq \svar(A_i) \\
&~&[\forall_{u \ge t} \; u \notin \ub(\setvar(A_i))]\Rightarrow \evar(A_i) \leq t \\
&&\svar(A_i) \le \max\{u \mid \exists_{S \subseteq \ub(\setvar(A_i))} \;
|S| = p_i \wedge \min(S) = u\} \\
&&\evar(A_i) \ge \min\{u \mid \exists_{S \subseteq \ub(\setvar(A_i))} \;
|S|= p_i \wedge \max(S) = u-1\}
\end{eqnarray*}
Needless to say, whenever any of these rules leads to a situation
where the lower bound of a variable is larger than its upper bound, a
contradiction is detected.

In the following, we may occasionally use the notations $X(A_i, t)$ and
$\setvar(A_i)$ for an activity $A_i$ that cannot be interrupted. In such
a case, the following rules are also applied:
\begin{eqnarray*}
&&X(A_i, t) = 0 \wedge t < \eetm_i \Rightarrow \svar(A_i) > t \\
&&X(A_i, t) = 0 \wedge \lstm_i \le t \Rightarrow \evar(A_i) \le t 
\end{eqnarray*}

\noindent Elastic activities are discussed in the following section.

\index{activity|)}

%=====================================================================
\subsection{Resource Constraints} 
%=====================================================================

\index{constraint!resource|(}

When looking at the type of resources found in a problem, we
distinguish {\em disjunctive scheduling} and {\em cumulative
scheduling}. In a \index{resource!unary}{disjunctive} scheduling
problem, all resources are of capacity 1 (such resources are often
called {\em machines}) and thus can execute at most one activity at a
time. In a
\index{resource!cumulative}{cumulative} scheduling problem, resources exist that can execute
several activities in parallel, of course provided that the resource
capacity is not exceeded.

Resource constraints represent the fact that activities require some
amount of resource throughout their execution. Given an activity $A_i$
and a resource $R$ whose capacity is $\capvar(R)$, generally a
variable $\capvar(A_i, R)$ is introduced that represents the amount of
resource $R$ required by activity $A_i$. Where no confusion is
possible we will often omit ``$R$'' and use $\capvar(A_i)$ to denote
$\capvar(A_i, R)$. For fully elastic activities, the $\capvar(A_i, R)$
variable is not meaningful and a variable $E(A_i, R)$ that represents
the {\em energy} required by the activity on the resource $R$ is
introduced. Note that for non-elastic activities, we have $E(A_i, R) =
\capvar(A_i, R) \pvar(A_i)$. To represent a schedule, a set of
variables $E(A_i, t, R)$ is required that denote the number of units
of the resource $R$ used by activity $A_i$ at time $t$. In all cases,
we have the constraint stating that enough resource capacity must be
allocated to activities to cover the energy requirement:
%\begin{equation*} \label{ELASTIC-GEN}
$$
E(A_i, R) = \sum_t E(A_i, t, R)
$$
%\end{equation*}


\noindent If $A_i$ is not an elastic activity, there are some strong
relations between $E(A_i, t, R)$ and $X(A_i, t)$:
%\begin{equation*} 
$$
E(A_i, t, R) =  X(A_i, t) \capvar(A_i, R)
$$
%\end{equation*}
For \index{scheduling!elastic}{elastic} activities, we have a weaker relation between the
variables:
%\begin{equation*} 
$$
[E(A_i, t, R) > 0] \Leftrightarrow [X(A_i, t) > 0]
$$
%\end{equation*}
Generally speaking, the resource constraint can be written
as follows. For each point in time $t$
%\begin{equation} \label{RES-CTS-GENERAL}
\begin{eqnarray} \label{RES-CTS-GENERAL}
\sum_{i = 1}^n E(A_i, t, R) \le \capvar(R)
\end{eqnarray}
%\end{equation}
Depending on the scheduling situation, (\ref{RES-CTS-GENERAL}) can be
rewritten. In the \index{scheduling!non-preemptive}{non-preemptive case}, (\ref{RES-CTS-GENERAL}) leads
for all times $t$ to
%\begin{equation*} % \label{RES-CTS-NP} 
$$
\sum_{A_i \mid \svar(A_i) \le t < \evar(A_i)} \capvar(A_i, R) \le
\capvar(R)
$$
%\end{equation*}
In the \index{scheduling!preemptive}{preemptive case}, (\ref{RES-CTS-GENERAL}) leads for all
times $t$ to
%\begin{equation*} %\label{RES-CTS-P}
$$
\sum_{A_i \mid \svar(A_i) \le t < \evar(A_i)} X(A_i, t) \capvar(A_i,
R) \le \capvar(R)
$$
%\end{equation*}

\index{constraint!resource|)}

%=====================================================================
\subsection{Temporal Constraints} 
%=====================================================================
\index{constraint!temporal|(} Temporal relations between activities
can be expressed by \index{constraint!linear}{linear constraints}
between start and end variables of activities. For instance, a
standard precedence constraint between two activities $A_1$ and $A_2$
stating that $A_2$ is to be started after $A_1$ is ended can be
modeled by the linear constraint $\evar(A_1) \le \svar(A_2)$. In
general, with both $x$ and $y$ a start or end variable and $d$ an
integer, temporal relations can be expressed by constraints of the
type $x - y \leq d$.

When the temporal constraint network is sparse, as it is usually the
case in scheduling, such constraints can be easily propagated using a
standard \index{consistency!arc consistency}{arc-B-consistency}
algorithm \cite{Lhomme93}. In addition, a variant of Ford's algorithm
(see for instance \cite{Gondran84}) proposed by Cesta and Oddi
\cite{CestaOddi96} can be used to detect any inconsistency between
such constraints, in time polynomial in the number of constraints and
independent of the domain sizes.

When the temporal network is dense or when it is useful to compute and
maintain the minimal and maximal delay between any pair of time points
in the schedule, \index{consistency!path consistency}{path
consistency} can be enforced on the network \cite{DechterMeiri91} for
example by applying Floyd-Warshall's All-Pairs-Shortest-Path algorithm
\cite{Floyd62}.  \index{constraint!temporal|)}

%=====================================================================
\subsection{Extensions of the Basic Model} 
\label{SWNREFINED-MODEL-SEC}
%=====================================================================

Although the model presented until now covers quite a number of
scheduling problems, in this section we pay attention to extensions that are
frequently found in industrial applications. 

\subsubsection{Alternative Resources}
\index{resource!alternative|(}

In some scheduling situations an activity $A_i$ can be scheduled on
any one resource from a set $S$ of resources. We say that $S$ is the
set of {\em alternative resources} for $A_i$. A common way to model
this is to for each activity $A_i$ introduce a variable $\altvar(A_i)$
representing the chosen resource among the resource alternatives. To
simplify notation, we assume that resources are numbered from $1$ to
$m$ and that $\altvar(A_i)$ denotes the variable whose value
represents the index of the resource on which $A_i$ is executed. We
remark that quite commonly the processing time of the activity depends
on the resource on which the given activity is executed, \ie, the
resources are unrelated. The same goes for the cost of executing the
activity, \ie, different alternatives can have different
costs. Another commonly found type of constraints reasons on
interdependencies of resource allocations, \eg, a constraint like ``if
$A_1$ is scheduled on resource $R_1$ then $A_3$ has to be scheduled on
resource $R_2$''. These constraints are used to model things like
alternative production lines.

%As seen in the scheduling model, each activity $A_i$ may be
%processed on a resource $R_k$ chosen within a given set of
%possible alternative resources $R_h=\{R_{h,1},...,R_{h,p}\}$.

\index{constraint!resource}{Alternative resource constraints} are
propagated as if the activity $A_i$ were split into $|\domain(\altvar(A_i))|$
fictive activities $A_i^u$ where each activity $A_i^u$ requires
resource $R_u$ \cite{LePapeSmith88}. Following this notation $r_i^u$
denotes the earliest start time of $A_i^u$, \etc~The alternative
resource constraint maintains the constructive disjunction
between the alternative activities $A_i^u$ for $u
\in \domain(\altvar(A_i))$, \ie, it ensures that:
\begin{eqnarray*}
\estm_i &=& \min\{\estm_i^u \mid u \in \domain(\altvar(A_i))\}  \\
\lstm_i &=& \max\{\lstm_i^u \mid u \in \domain(\altvar(A_i))\}  \\
\eetm_i &=& \min\{\eetm_i^u \mid u \in \domain(\altvar(A_i))\}  \\
\letm_i &=& \max\{\letm_i^u \mid u \in \domain(\altvar(A_i))\}  \\
\lb(\pvar(A_i)) &=& \min\{\lb(\pvar(A_i^u)) \mid u \in \domain(\altvar(A_i))\}  \\
\ub(\pvar(A_i)) &=& \max\{\ub(\pvar(A_i^u)) \mid u \in \domain(\altvar(A_i))\}  \\
\end{eqnarray*}

Constraint propagation will deduce new bounds for alternative
activities $A_i^u$ on the alternative resource $R_u$. Whenever the
bounds of an activity $A_i^u$ turn out to be incoherent, the resource
$R_u$ is simply removed from the set of possible alternative resources
for activity $A_i$, \ie, $\domain(\altvar(A_i))$ becomes
$\domain(\altvar(A_i)) - \{u\}$.

In some approaches the fictive activities $A_i^u$ are actually
generated together with a way to express that only one of them per
original activity $A_i$ will really require one of the alternative
resources. In this context the generated activities $A_i^u$ are often
referred to as {\em optional} activities. See below for a discussion
on optional activities.

%WN: decided to not talk about redundant resources.
\index{resource!alternative|)}

\subsubsection{Setup Times and Setup Costs}
%\subsubsection{Transition Times and Transition Costs}

Setup  times and  setup costs  are of  great importance  in industrial
applications. They  are found abundantly and the  correct treatment of
them is often crucial, both  because they are a mandatory component to
express the  problem in the  required detail as  it is needed  to find
good solutions  with respect to  them as they represent  a substantial
part of  the real-life cost. The  \index{setup!time}{setup time} (also
transition time) $\setupt(A_1, A_2)$  between two activities $A_1$ and
$A_2$ is  defined as the amount  of time that must  elapse between the
end of  $A_1$ and  the start  of $A_2$, when  $A_1$ immediately
precedes  $A_2$ on a given resource. A
\index{setup!cost}{setup  cost}   $\setupc(A_1,  A_2)$  can   also  be
associated to the transition between $A_1$ and $A_2$. The objective of
the scheduling  problem can be to  find a schedule  that minimizes the
sum of the setup costs.

In a vast majority of problems activities subjected to setups are to
be scheduled on the same machine (the semantics of setups is much more
complex on resources of capacity greater than 1).  Setup
considerations can be combined with alternative resources. In such a
case, two parameters are associated to each tuple $(A_i, A_j, R_u)$:
the setup time $\setupt(A_i, A_j, R_u)$ and the setup cost
$\setupc(A_i, A_j, R_u)$ between activities $A_i$ and $A_j$ if $A_i$
and $A_j$ are scheduled sequentially on the same machine $R_u$. The
attached constraint is that $\svar(A_j^u) \geq \evar(A_i^u) +
\setupt(A_i, A_j, R_u)$. There may furthermore exist a setup time
$\setupt(-, A_i, R_u)$ (with corresponding cost $\setupc(-, A_i,
R_u)$) that has to elapse before the start of $A_i$ when $A_i$ is the
first activity on $R_u$ and, similarly, a teardown time $\setupt(A_i,
-, R_u)$ (with corresponding cost $\setupc(A_i, -, R_u)$) that has to
elapse after the end of $A_i$ when $A_i$ is the last activity on
$R_u$. Section~\ref{PROP-SUMTT-SEC} pays attention to constraint
propagation methods for setup times and setup costs constraints.

\subsubsection{Breakable Activities and Calendars}
\index{resource!calendar|(}
\index{activity!breakable|(}

In \cite{NuijtenBousonville04} a problem is described that has a lot
of properties frequently found in industrial applications. One such
property is the fact resources can be governed by a {\em calendar}
under which activities scheduled on the resource are executed. Such a
calendar thus defines the execution conditions for activities and
consists of a list of breaks and a {\em productivity profile}. An
activity scheduled on the resource at hand can be interrupted by
breaks smaller than the maximal break duration $mBD$. An activity is
thus {\em breakable} when $mBD > 0$ and {\em not breakable} when $mBD
= 0$. The productivity profile defines the efficiency of the activity
execution. If an activity is scheduled in a time interval with
productivity $p \%$, $p \%$ of a processing time unit is executed per
time unit. A productivity below $100 \%$ thus means that per time unit
less than one processing time unit is executed which in turn implies
that the duration of an activity will exceed its processing time. For
productivities exceeding $100 \%$ obviously the inverse holds. The
processing time of an activity is therefore equal to the integral,
from start to end, of the productivity. We refer to \cite{Schwindt05}
for a more extensive discussion on breakable activities and
productivity profiles.

A CP model for this consists in introducing a {\em duration variable}
per activity and redefining the meaning of the processing time
variable. The duration variable $\dvar(A_i)$ of an activity is then
defined as the difference between the end time and the start time of
the activity: $\dvar(A_i) = \evar(A_i) - \svar(A_i)$. The processing
time variable is defined as the time it will take to execute the
activity when it is not interrupted by breaks and all along the
execution the productivity is exactly $100 \%$. All four activity
variables $\svar$, $\evar$, $\dvar$, and $\pvar$ are governed by the
break constraint and the productivity profile constraint of the
resource on which the activity is executed. For sake of clarity we do
not use this meaning for the processing time variable in the rest of
this chapter, \ie, the processing time is defined as the difference
between the end time and the start time of an activity.
\index{activity!breakable|)}
\index{resource!calendar|)}


\subsubsection{Optional Activities/Leaving Activities Unperformed}

\index{activity!optional|(}
Whether coming from alternative resources (see above) or directly
present in the model, in many scheduling problems activities exist for
which it is not yet decided whether they will be executed on a
resource or not. Such activities are often referred to as {\em
optional} activities.

Modeling an optional activity is often obtained by allowing the
processing time variable to take on the value $0$. Care is to be taken
then that non-standard precedence constraints are not still active. If
for instance an optional activity $A_1$ is part of a chain of
activities and a precedence constraint of the type $\evar(A_1) + d
\leq \svar(A_2)$ is defined, setting the processing time variable to
become 0 and keeping the precedence constraint active will still
induce a possibly unwanted delay $d$ between the predecessors of $A_1$
and $A_2$. Another way to model optional activities is to introduce a
variable per activity expressing whether the activity really exists or
not. This is obviously a more direct way of modeling but requires the
adaptation of propagation algorithms to deal with this additional
variable and concept \cite{VilimBartak04}.

In industrial scheduling problems one often finds the possibility of
subcontracting an activity $A_i$, this against a certain incurred cost
$cost_i$. This means that the activity does not use resource capacity
but does take time. A way to model this is to allow the capacity
variable $\capvar(A_i)$ to take on the value $0$ and to introduce a
variable $cost$ representing the cost together with the constraint
$\capvar(A_i) = 0 \Rightarrow cost = cost_i$. Another way to model
this is to introduce an alternative resource corresponding to the
subcontracting alternative. This can be interesting if subcontracting
is subject to a different calendar than in-house production, which of
course is not uncommon. The cost of the alternative would correspond
to $cost_i$. On the ``subcontracting'' resource one would in this case
not enforce the capacity constraint, even though this model can easily
be extended to enforce the capacity constraint to model restricted
subcontracting possibilities.

In \cite{NuijtenBousonville04} a model is described where one can
decide that an activity will be left {\em unperformed}, meaning that the
activity will not require capacity, but will obey potential temporal
constraints, etc.,  and will also obey the calendar of the chosen
resource. This corresponds to the situation where one wants to include
the possibility to temporarily increase the production capacity in its
own production facility, this of course against a certain cost. Other
useful constraints included are {\em performance compatibility constraint}
between two activities expressing that either they are both performed
or they are both unperformed.
\index{activity!optional|)}

\subsubsection{State resources}
\label{StateResources}

A \index{resource!state}{state resource} represents a resource of infinite capacity, the state of which can vary over time. Each activity may, throughout its execution, require a state resource to be in a given state (or in any of a given set of states). Consequently, two activities may not overlap if they require incompatible states of a state resource during their execution. Adaptations of the Timetable Constraint (see Section \ref{TimetableConstraint}) and Disjunctive Constraint (see Section \ref{DisjConstraintProp}) are used as basic propagation algorithms on those resources. 

\subsubsection{Reservoirs}

A \index{resource!reservoir}{reservoir resource} is a multi-capacity
resource that can be consumed and/or produced by activities. A
reservoir has an integer maximal capacity and may have an initial
level. As an example of a reservoir you can think of a fuel tank. Note
that a cumulative resource can be seen as a special case of a
reservoir that is consumed at the beginning of the activity and
produced in the same quantity at the end of the activity when the
activity releases the resource.

The Timetable Constraint presented in
Section~\ref{TimetableConstraint} can be generalized to the case of
reservoirs and is classically used as the basic propagation algorithm
on those resources. However, we will see in
Section~\ref{BalanceConstraint} that other techniques are available to
provide additional propagation.


%=====================================================================
\subsection{Objective Function} 
\label{OBJECTIVE-FUNCTION-SEC}
%=====================================================================

Finally, {\em decision} problems and {\em optimization} problems are
distinguished. In decision problems, one has only to determine whether
a schedule exists that meets all constraints. In optimization
problems, an objective function has to be optimized. In this chapter we
concentrate on problems where there is one objective function, \ie, we
do not consider cases where multiple objective functions are
defined.

The commonly used way of modeling an objective function is simply by
introducing a variable $\critvar$ that is constrained to be equal to
the value of the objective function. Although the minimization of the
makespan, \ie, the end time of the schedule, is commonly used, other
criteria are of great practical interest \eg, the sum of
\index{setup!time}{setup times} or \index{setup!cost}{costs}, the
number of \index{activity!late}{late activities}, the maximal or
average tardiness or earliness, storage costs, alternative costs, the peak or average resource
utilization, etc. We will come back to these criteria later.

Many of the classical scheduling criteria take into account
a due date $\delta_i$ that one would like to meet for each
activity. In contrast to a deadline $d_i$ which is mandatory, a due
date $\delta_i$ can be seen as a preference. In the following, $C_i$
denotes the completion time of activity $A_i$. Lateness $L_i$ of $A_i$
is defined as the difference between the completion time and the due
date of $A_i$, \ie, $L_i = C_i-\delta_i$. The tardiness $T_i$ of $A_i$
is defined as $\max(0,L_i)$, while earliness of $A_i$ is defined as
$\max(0, -L_i)$. The notation $U_i$ is used to denote a unit penalty
per late job, \ie, $U_i$ equals 0 when $C_i \le \delta_i$ and equals 1
otherwise. See also Figure~\ref{OBJ-F-FIG}.
\begin{center}
\begin{figure}[htbp]
\epsfxsize=12cm
\centering\mbox{\epsfbox{figures/objFunctions.eps}}
\caption{Some scheduling related objective functions \label{OBJ-F-FIG}}
\end{figure}
\end{center}
The commonly studied criteria $F$ are either formulated as a sum or as
a maximum. A weight per activity $w_i$ may be used to give more
importance to some activities. We mention the following well-known
optimization criteria:
\begin{itemize}
\item
Makespan: $F = C_{\max} = \max C_i$ 
\item
Total weighted flow (or completion) time: $F = \sum w_i C_i$ 
\item
Maximum tardiness: $F = T_{\max} = \max T_i$ 
\item
Total weighted tardiness: $F = \sum w_i T_i$ 
\item
Total weighted number of \index{activity!late}{late jobs}: $F = \sum w_i U_i$ 
\end{itemize}

For these simple cases which are the most often studied in the
literature, the objective function is thus a function of
the end variables of the activities. 
%\begin{equation} \label{OBJ-CT}
$$ 
\critvar =  F(\evar(A_1), \ldots, \evar(A_n))
$$
In that case, the objective constraint is a simple arithmetic expression on
which \index{consistency!arc consistency}{arc-B-consistency} can be
easily achieved.

Considering the objective constraint and the resource constraints
independently is not a problem when $F$ is a maximum such as
$C_{\max}$ or $T_{\max}$.  Indeed, the upper bound on $\critvar$ is
directly propagated on the completion time of each activity, \ie,
latest end times are tightened efficiently. The situation is much more
complex for sum functions such as $\sum w_i C_i, \sum w_i T_i$, or
$\sum w_i U_i$. For these functions, efficient constraint propagation
techniques must take into account the resource constraints and the
objective constraint simultaneously. We pay attention to propagation
for sum functions in Section~\ref{SPropOptCrit}. There we will also
pay attention to objective functions that are not a function of the end
variables of the activities like sum of setup times and sum of setup
costs.


Once all constraints of the problem are added, the most commonly used technique to
look for an optimal solution is to solve successive decision variants
of the problem. Several strategies can be considered to minimize the
value of $\critvar$. One way is to iterate on the possible values,
either from the lower bound of its domain up to the upper bound until
one solution is found, or from the upper bound down to the lower
bound determining each time whether there still is a
solution. Another way is to use a dichotomizing algorithm, where one
starts by computing an initial upper bound $\ub(\critvar)$ and an
initial lower bound $\lb(\critvar)$ for $\critvar$. Then

\begin{enumerate}
\item
Set $\begin{displaystyle} D = \left\lfloor \frac{\lb(\critvar)
 + \ub(\critvar)}{2} \right\rfloor \end{displaystyle}$
\item
Constrain $\critvar$ to be at most $D$. Then solve the resulting CSP,
\ie, determine a solution with $\critvar \le D$ or prove that no such
solution exists. If a solution is found, set $\ub(\critvar)$ to the
value of $\critvar$ in the solution; otherwise, set $\lb(\critvar)$ to
$D + 1$.
\item
Iterate steps 1 and 2 until $\ub(\critvar) = \lb(\critvar)$.  
\end{enumerate}




\section{Constraint Programming Models for Planning} \label{SCPModelsPlan}

While the previous section presented CP models for scheduling, in this
section we pay attention to CP models for planning as studied by the
\index{planning!constraint based}{Constraint-Based Planning}
community. In this section and in general throughout this chapter we
assume that the readers are familiar with basic planning techniques
and terminology. For more information on such planning techniques and
terminology, we refer to \cite{GhallabNau04}.

A general formulation of the planning problem defines three inputs in
some formal language:
\begin{enumerate}
\item A description of the initial state and expected changes of the world,
\item A description of the agent's goal (\ie, what behavior is desired), and
\item A description of the possible actions that can be performed. This last description is often called a domain theory.
\end{enumerate}

The planner's output is a feasible sequence of actions referred to as a {\em plan} which, when executed in any
world satisfying the initial state and undergoing the expected changes, will achieve the agent's goal. 

\subsection{CSPs for Planning-Graph Techniques}
\label{csp-graphplan}

The classical STRIPS \cite{FikesNilsson71} representation describes
the initial state of world with a complete set of ground
propositions. The representation is restricted to goals of attainment,
and these goals are defined as a conjunction of propositions; all
world states satisfying the goal formula are considered equally
good. A domain theory completes a planning problem. In the STRIPS
representation, each \index{planning!operator}{operator} is described
with a conjunctive precondition and conjunctive effect that define a
transition function from states to states. An
\index{planning!action}{action} is a fully grounded operator. An
action can be executed in any state $s$ satisfying the
\index{planning!operator!precondition}{precondition} formula. The
\index{planning!operator!effect}{effect} of executing an action in a
state $s$ is described by eliminating from $s$ each proposition of the
action delete-list and adding to $s$ each proposition from the action
add-list. The add-list and delete-list of the action are called the
effect of the action. This defines the so called classical planning
problem.

For instance in a blocks world domain, the propositions for describing the state of the world are shown in Table \ref{tab:propositions}. The operators of this planning domain move a clear block onto another clear block or the table as shown in Table \ref{tab:Operators}.

\begin{table}[htbp]
\centering
\small
\begin{tabular}{|ll|}\hline
$clear(X)$: & \textbf{There is no block on block X.}\\\hline
$onT(X)$:   & \textbf{Block X is on the table.}\\\hline
$on(X,Y)$:  & \textbf{Block X is on block Y.}\\\hline
\end{tabular}
\caption{Propositions for the blocks world domain}
\label{tab:propositions}
\end{table}

\begin{table}[htbp]
\centering
\small
\begin{tabular}{|rl|}\hline
BB(X,Y,Z) & \textbf{Move X from atop Y to atop Z} \\ preconditions: &
$clear(X) \land clear(Z)\land on(X,Y)$ \\ add-list: & $clear(Y)\land
on(X,Z)$ \\ delete-list: & $clear(Z) \land on(X,Y)$ \\ \hline TB(X,Y)
& \textbf{Move X from the table to atop Y} \\ preconditions: &
$clear(X) \land clear(Y) \land onT(X)$ \\ add-list: & $on(X,Y)$ \\
delete-list: & $clear(Y) \land onT(X)$ \\ \hline BT(X,Y) &
\textbf{Move X from atop Y to the table} \\ preconditions: & $clear(X)
\land on(X,Y)$ \\ add-list: & $onT(X) \land clear(Y)$ \\ delete-list:
& $on(X,Y)$ \\ \hline
\end{tabular}
\caption{Operators for the blocks world domain}
\label{tab:Operators}
\end{table}

In this section, we will consider the planning problem with initial state: $[on(C,A) \land onT(A) \land onT(B)]$ and goal state: $[on(A,B)\land on(B,C)]$ depicted in Figure \ref{fig:sussman}.

\begin{figure}[ht]
	\centering
   \includegraphics[angle=270,width=4cm,keepaspectratio,clip]{figures/sussman.prn.epsi}
	\caption{A planning problem}
	\label{fig:sussman}
\end{figure}

In recent years, researchers have investigated the reformulation of
planning problems as constraint satisfaction problems (CSPs) in an
attempt to use powerful algorithms for constraint satisfaction to find
plans more efficiently \cite{VanBeekChen99, DoKambhampati01,
LopezBacchus03}. In these approaches, each CSP typically represents
the problem of finding a plan with a fixed number of steps. A solution
to the CSP can be mapped back to a plan; if no solution exists, the
number of steps permitted in the plan is increased and a new CSP is
generated.

%% Van Beek creates encoding by hand, using hand-coded domain axioms to improve performance

Graphplan \cite{BlumFurst97} works on STRIPS domains by creating a \index{planning!graph}
{planning graph} which represents the set of propositions which can be achieved after a number of steps along with mutual exclusion (mutex) relationships between propositions and actions. Mutually exclusive actions are actions that cannot be executed in the same step. Two actions in the same step are mutex if either of the actions deletes a precondition or add-effect of the other. Two propositions $p$ and $q$ in the same step are mutex if all possible actions for establishing proposition $p$ are exclusive with all possible actions for establishing proposition $q$. If two actions $a$ and $b$ have some mutex precondition then they are mutex. Note that the persistence of a proposition between two steps is considered as a particular type of action called a {\em persistence} action.

This planning graph is then searched for a plan which achieves the goals from the initial state and that is mutex-free in each step. The planning graph corresponding to our blocks world example with 3 developed steps is shown in Figure \ref{fig:planninggraph}. Note that this planning graph is not complete, some mutex are missing in all steps and some actions are missing in step~2.

\vspace*{1.5mm}
\begin{figure}[htbp]
\centering
   \includegraphics[angle=270,width=12.5cm,keepaspectratio,clip]{figures/planning-graph.prn.epsi}
   \caption{Planning graph with 3 steps}
\label{fig:planninggraph}
\end{figure}

While the original algorithm performed backward search, the plan graph can also be transformed into a CSP which can be solved by any CSP solver. 

In \cite{DoKambhampati01}, variables of this CSP are propositions distinguished by step. Values are possible actions for establishing those propositions with a special dummy value ($\bot$) stating that the proposition is inactive. Constraints say that preconditions of actions that are used to establish active propositions cannot be inactive and mutex action/proposition pairs must be satisfied. For instance, a constraint $on\_A\_B\_G = TB\_A\_B  \Rightarrow (onT\_A_3 \neq \bot \land clear\_A_3 \neq \bot \land clear\_B_3 \neq \bot)$ means that if proposition $on\_A\_B$ in the goal state is to be established by action $TB\_A\_B$, then, it means that propositions $onT\_A$, $clear\_A$ and $clear\_B$ must hold in step $3$, that is, they must have been established by some action. A fragment of the CSP corresponding to the planning graph of Figure \ref{fig:planninggraph} is shown in Table \ref{tab:CSPFormulation}. 

\begin{table}[htbp]
\centering
\scriptsize
\begin{tabular}{|rl|}\hline
Variables: & $on\_C\_A_1$, $onT\_A_1$, $clear\_B_1$, $clear\_C_1$, $onT\_B_1$, \\
           & $on\_C\_B_2$, $clear\_A_2$, $onT\_C_2$, $on\_C\_A_2$, $onT\_A_2$, $clear\_B_2$, $clear\_C_2$, $onT\_B_2$, $on\_B\_C_2$, \\
           & $on\_A\_C_3$, $on\_A\_B_3$, $clear\_A_3$, $on\_B\_A_3$, $onT\_A_3$, $clear\_B_3$, $clear\_C_3$, $onT\_B_3$, $on\_B\_C_3$, \\
           & $on\_A\_B_G$, $on\_B\_C_G$,... \\\hline

Domains: & $on\_C\_A_1:\{Init\}$, $onT\_A_1:\{Init\}$, $clear\_B_1:\{Init\}$, $clear\_C_1:\{Init\}$, $onT\_B_1:\{Init\}$, \\
         & $on\_C\_B_2:\{BB\_C\_A\_B,\bot\}$, $clear\_A_2:\{BB\_C\_A\_B,BT\_C\_A,\bot\}$, $onT\_C_2:\{BT\_C\_A,\bot\}$, \\
         & $on\_C\_A_2:\{Persist,\bot\}$, $onT\_A_2:\{Persist,\bot\}$, $clear\_B_2:\{Persist,\bot\}$, \\
         & $clear\_C_2:\{Persist,\bot\}$, $on\_B\_C_2:\{TB\_B\_C,\bot\}$, $onT\_B_2:\{Persist,\bot\}$, ... \\
         & $on\_A\_B_G:\{BB\_A\_C\_B,TB\_A\_B,Persist,\bot\}$, $on\_B\_C_G:\{TB\_B\_C,BB\_B\_A\_C,\bot\}$\\\hline
         
Constraints: & \textbf{activity preconditions} \\
	           & $on\_A\_B\_G = TB\_A\_B  \Rightarrow (onT\_A_3 \neq \bot \land clear\_A_3 \neq \bot \land clear\_B_3 \neq \bot)$ \\
	           & $on\_A\_B\_G = BB\_A\_C\_B  \Rightarrow (on\_A\_C_3 \neq \bot \land clear\_A_3 \neq \bot \land clear\_B_3 \neq \bot)$ \\
             & ...\\\hline
             
Constraints: & \textbf{mutexes}\\
             & $on\_C\_B_2 = BB\_C\_A\_B \Rightarrow onT\_C_2 \neq BT\_C\_A$ \\
             & ...\\\hline
                       
Constraints: & \textbf{goal state}\\
             & $on\_C\_B_G \neq \bot$, $on\_A\_B_G \neq \bot$ \\ \hline
\end{tabular}
\caption{Fragment of the generated CSP problem for 3 steps}
\label{tab:CSPFormulation}
\end{table}

If no solution is found, the planning graph is extended by adding an
additional step and the CSP is extended accordingly until it is
feasible. In the example, the CSP with 3 steps is feasible and a
solution is: step 1: $BT\_C\_A$, step 2: $TB\_B\_C$, step 3:
$TB\_A\_B$.

%I (WN) removed the \tt's as that font command is not supported
%in ``memoir'' 
% step 1: {\tt BT\_C\_A}, step 2: {\tt TB\_B\_C}, step 3: {\tt TB\_A\_B}.

In \cite{LopezBacchus03}, a slightly different CSP model is used with only boolean variables, one variable for each proposition in each step and one variable for each action in each step. Some redundant constraints (other than mutex) are identified and added to the CSP model. For instance the fact that an action cannot be immediately followed by its opposite action in a plan of optimal length.

In general, the reformulated problem is solved using classical CSP
techniques (arc-consistency, dynamic variable ordering, memoization)
and does not require specific constraint propagation techniques this
is the reason why we will not focus on this family of approaches in
the sequel of this chapter.


\subsection{CSPs in Plan-Space Search}
\label{PlanSpaceSearch}

A second growing trend in planning is the extension of planning systems to reason about both time and resources. STRIPS is simply not expressive enough to represent more realistic planning problems. This demand for increased sophistication has led to the need for more powerful techniques to reason about time and resources during planning. 

In \index{planning!constraint based}{Constraint-Based Planning}, each search node represents a partial plan and consists of a set of time intervals, connected by constraints. The partial plan may be incomplete, in that constraints are not necessarily satisfied and pending choices have not been made. The planning process then involves modifying a partial plan until it has been turned into a complete and valid plan. Traditional search-based methods accomplish this by trying different options for completing partial plans, and backtracking when constraints are found to be violated. Constraint reasoning methods, such as propagation and consistency checks can be used to help out in that process. 

The scheduling community has used constraint satisfaction techniques to perform this sort of reasoning. The main difference between constraint-based planning and scheduling is that, in planning, the set of activities of the plan is not completely known a-priori and must be determined during the search. The rules that govern the insertion of new activities in the plan are expressed as implicit or explicit constraints and, although they slightly differ from one planning system to the other \cite{GhallabLaruelle94,LaborieGhallab95,ChienRabideau00,JonssonMorris00,FrankJonsson03}, they all have the same flavor.

To fix the ideas, we will use in this section the IxTeT formalism
\cite{GhallabLaruelle94, LaborieGhallab95}. In this planning language
the state of the world is described by a set of multi-valued state
attributes together with a set of resource attributes. Each state
attribute describes a particular feature of the world, for instance
the position of a block. A \index{resource}{resource} is an object (or
a set of objects) that can be simultaneously shared by several actions
provided its maximal capacity is not
exceeded. \index{planning!operator} {Operators} (or tasks) are
\index{constraint!temporal}{temporal structures} composed of a set of
events describing the changes of the world induced by the task
(event), a set of assertions on state attributes describing conditions
that must remain true during some time intervals (hold) and a set of
resource usage (use, produce, consume) describing how the task uses
some resources. All the above statements refer to time points that can
be constrained with respect to the time interval [start,end] of the
task. An example of model for the blocks world domain is shown in
Table \ref{tab:ixtet-model}. Note that the domain is represented here
with a finer grain than the STRIPS definition in Section
\ref{csp-graphplan}: the operator is defined with internal
time points, delays between these time points (duration) are expressed
together with resource requirements (here, the task requires one hand
from the two hands that are available to move the blocks). Beside
resource usage, note that state attributes are very similar to state
resources used in scheduling (see section \ref{StateResources}) in
that no task requiring a given state attribute to take different value
can overlap in time.

\begin{table}[htbp]
\centering
\small
\begin{tabular}{|rl|}\hline
constant  & blocks  = \{a, b, c\}; \\
          & positions  = \{a, b, c, table, hand\}; \\ \hline
attribute & clear(?x) \{ ?x in blocks; \textbf{?value in} \{yes, no\}; \}\\
          & on(?x) \{ ?x in blocks; \textbf{?value in} positions; \}\\ \hline
resource  & hands() \{ \textbf{capacity} = 2; \}\\ \hline
task & TB(?x, ?y)(\textbf{start}, \textbf{end}) \{\\
  & $\ $\textbf{// Move ?x from the table to atop ?y} \\
	& $\ $\textbf{timepoint} t1, t2;\\
	& $\ $\textbf{event}(clear(?x):(yes,no), \textbf{start});\\
	& $\ $\textbf{event}(on(?x):(table,hand), t1);\\
	& $\ $\textbf{event}(on(?x):(hand,?z), t2);\\
	& $\ $\textbf{event}(clear(?z):(yes,no), t2);\\
	& $\ $\textbf{hold}(on(?x):hand, (t1, t2));	\\
	& $\ $\textbf{event}(clear(?x):(no,yes), \textbf{end});\\
	& $\ $\textbf{hold}(clear(?x):no, (\textbf{start}, \textbf{end}));\\
	& $\ $\textbf{use}(hands(): 1, (\textbf{start}, \textbf{end}));\\
	& $\ $(t1 - \textbf{start}) \textbf{in} [00:10, 00:20];\\
	& $\ $(t2 - t1)    \textbf{in} [00:15, 00:25];\\
	& $\ $(\textbf{end} - t2)   \textbf{in} [00:10, 00:20];\\ 
  & \} \\ \hline
\end{tabular}
\caption{Part of the blocks world domain in IxTeT}
\label{tab:ixtet-model}
\end{table}

A partial plan is a set of tasks together with a set of constraints
between the tasks variables (including temporal constraints between
the task time points). The initial plan only contains a fake start
task that asserts all the state attributes of the initial state and a
fake end task that has the goal as (non-established) preconditions. A
partial plan is complete and valid if and only if each instantiation
of all the variables of the partial plan lead to a feasible
fully-grounded plan where (1) the change of state attributes over time
is non-ambiguously defined by and consistent with the events and
assertions of the tasks of the plan and (2) resource constraints are
satisfied. Figure \ref{fig:ixtetpartialplan} shows an example of
a partial plan.

\begin{figure}[ht]
\centering
   \includegraphics[angle=270,width=12.5cm,keepaspectratio,clip]{figures/ixtet-partial-plan.prn.epsi}
   \caption{Partial plan}
\label{fig:ixtetpartialplan}
\end{figure}

Partial plans are iteratively refined at each search node until the partial plan is complete and valid. Three types of plan refinements are considered:
\begin{itemize}
\item Non-established conditions are those events or assertions that
are still not established by a task or the initial state. For
instance, in the partial plan of Figure \ref{fig:ixtetpartialplan},
the condition $clear(A):yes$ related with the event at the start
time point of task $TB(A,B)$ is still not established. Non-established
conditions can be established by an existing event in the plan or by
inserting a new task. In the case of the example, a new task $BT(C,A)$
could be added before the start of task $TB(A,B)$ to establish the
condition.
\item Possible conflicts between unordered events/assertions are pairs of statements that may require the same state attribute to take different values at the same moment. These incompatibility constraints can be solved by posting precedence constraints to order the conflicting events/assertions.
\item Possible resource conflicts are subsets of resource requirements that may overlap in time and would in this case over-consume the resource. This would be the case of the two tasks in the example if the maximal capacity of the resource $hands$ is 1. These resource conflicts can be solved by ordering the tasks or, in case the resource can be produced, by inserting a task that produces the resource.
\end{itemize} 

In \index{planning!constraint based}{Constraint-Based Planning}, the
partial plan is usually represented as a CSP with variables
representing the task time points together with non-temporal variables
appearing in the task definition as well as special variables
representing the tasks that can be used to establish conditions in the
partial plan. Specialized constraint propagation algorithms can be
used to reduce the domain of variables or to deduce new temporal
constraints. As far as time and resources are concerned, these
propagation algorithms are essentially the same as the ones developed
for pure scheduling problems, which will be described in Section
\ref{SPropRct}. Indeed, unless stated otherwise, all the algorithms
described in Section \ref{SPropRct} can be implemented so as to accept
other tasks and variables as the search evolves. We also sketch in
that section how some of these algorithms can be adapted to propagate
on state attributes. See \cite{VidalGeffner05} for recent advances in
the field of using Constraint Programming in Plan-Space Search.

\index{scheduling!non-preemptive|(}

%%%%%%%%%%%%%%%%%%%%%%%%%%%%%%%%%%%%%%%%%%%%%%%%%%%%%%%%%%%%%%%%%%%%%
%%%%%%%%%%%%%%%%%%%%%%%%%%%%%%%%%%%%%%%%%%%%%%%%%%%%%%%%%%%%%%%%%%%%%
\section{Constraint Propagation for Resource Constraints}\label{SPropRct}
%%%%%%%%%%%%%%%%%%%%%%%%%%%%%%%%%%%%%%%%%%%%%%%%%%%%%%%%%%%%%%%%%%%%%
%%%%%%%%%%%%%%%%%%%%%%%%%%%%%%%%%%%%%%%%%%%%%%%%%%%%%%%%%%%%%%%%%%%%%
\index{constraint!global|(}
\index{constraint!resource|(}

{\em Resource constraints} represent the fact that activities require
some amount of resource throughout their execution. In this section
propagation for resource constraints is described for unary resources
(machines), cumulative resources, and reservoirs. As we will see, the
propagation of resource constraints is a purely deductive process that
allows to deduce inconsistencies and to tighten the temporal
characteristics of activities and resources. Throughout this section,
we will concentrate on non-preemptive scheduling. We refer the reader
to \cite{BaptisteLePape01} for the generalization of the described 
constraint propagation techniques to preemptive and elastic scheduling.


%...making that activities requiring the resource cannot overlap in
%time), cumulative resources (the capacity of the resource is fixed
%and,  at any time point $t$, the sum of the resource requirements of
%activities that are in process at $t$ does not exceed the resource
%capacity), and reservoirs.



\subsection{Unary resources}
%%%%%%%%%%%%%%%%%%%%%%%%%%%%%%%%%%%%%%%%%%%%%%%%%%%%%%%%%%%%%%%%%%%%%
%%%%%%%%%%%%%%%%%%%%%%%%%%%%%%%%%%%%%%%%%%%%%%%%%%%%%%%%%%%%%%%%%%%%%
\index{resource!unary|(}

Several mechanisms have been developed to propagate unary resource
constraints. Here we restrict the discussion to the disjunctive constraint
propagation scheme and to the well-known edge-finding algorithm. Also
note that the time tabling mechanism described in Section
\ref{CUMULATIVE-RESOURCES} can be applied to unary resources. We
refer to \cite{BaptisteLePape01} for a detailed introduction and comparison
of several other constraint propagation techniques.

\subsubsection{Disjunctive Constraint Propagation} 
\label{DisjConstraintProp}
\index{constraint!disjunctive|(} Two activities $A_i$ and $A_j$
requiring the same unary resource cannot overlap in time. So, either
$A_i$ precedes $A_j$ or $A_j$ precedes $A_i$. If $n$ activities
require the resource, one thus has $ n (n - 1) / 2$ (explicit or
implicit) of such disjunctive constraints. Variants exist in the
literature \cite{Erschler76, Carlier84, LePape88, SmithCheng93,
VarnierBaptiste93, BaptisteLePape95a}, but in most cases the
propagation consists of maintaining \index{consistency!arc
consistency}{arc-B-consistency} on the formula:
$$
[\evar(A_i) \le \svar(A_j)] \vee [\evar(A_j) \le \svar(A_i)]
$$ Enforcing arc-B-consistency on this formula is done as follows.
Whenever the earliest end time of $A_i$ exceeds the latest start time
of $A_j$, $A_i$ cannot precede $A_j$; hence $A_j$ must precede
$A_i$. The time bounds of $A_i$ and $A_j$ are consequently updated
with respect to the new temporal constraint $\evar(A_j) \le
\svar(A_i)$. Similarly, when the earliest end time of $A_j$ exceeds
the latest start time of $A_i, A_j$ cannot precede $A_i$. When neither
of the two activities can precede the other, a contradiction is
detected.

In \index{planning!constraint based}{Constraint-Based Planning}, the
disjunctive constraint can easily be adapted to propagate conflicts
between mutually exclusive statements \cite{VidalGeffner05}.
\index{constraint!disjunctive|)}

\subsubsection{Edge-Finding} 
\index{constraint!edge-finding|(}

The edge-finding constraint propagation technique consists of deducing
that some activities from a given set $\Omega$ must, can, or cannot,
execute first (or last) in $\Omega$. Such deductions lead to new
ordering relations (``edges'' in the graph representing the possible
orderings of activities) and new time bounds, \ie, strengthened
earliest start times and latest end times of activities.

In the following, $\estm_\Omega$, $\letm_\Omega$, and $p_\Omega$
denote the smallest of the earliest start times, the largest of the
latest end times, and the sum of the minimal processing times of the
activities in $\Omega$, respectively.  Let $A_i \ll A_j$ ($A_i \gg
A_j$) mean that $A_i$ executes before (after) $A_j$ and $A_i \ll
\Omega$ ($A_i \gg \Omega$) mean that $A_i$ executes before (after) all
the activities in $\Omega$. Once again, variants exist
\cite{CarlierPinson89, CarlierPinson90, CarlierPinson94,
CaseauLaburthe94, Nuijten94, BruckerThiele96, MartinShmoys96,
Peridy96, Vilim04} but the following rules capture the ``essence'' of
the edge-finding propagation technique:
\begin{eqnarray*}
\label{EF-RULES-EQ1}
\forall_\Omega \; \forall_{A_i \notin \Omega} \; 
[\letm_{\Omega \cup \{A_i\}} - \estm_{\Omega} < p_{\Omega} + p_i] 
\Rightarrow [A_i \ll \Omega] \\
\label{EF-RULES-EQ2}
\forall_\Omega \; \forall_{A_i \notin \Omega} \;
[\letm_{\Omega} - \estm_{\Omega \cup \{A_i\}} < p_{\Omega} + p_i] 
\Rightarrow [A_i \gg \Omega] \\
\label{EF-RULES-EQ3}
\forall_\Omega \; \forall_{A_i \notin \Omega} \;
[A_i \ll \Omega] \Rightarrow 
[\evar(A_i) \le \min_{\emptyset \neq \Omega' \subseteq \Omega} (\letm_{\Omega'} - p_{\Omega'})] \\
\label{EF-RULES-EQ4}
\forall_\Omega \; \forall_{A_i \notin \Omega} \;
[A_i \gg \Omega] \Rightarrow 
[\svar(A_i) \ge \max_{\emptyset \neq \Omega' \subseteq \Omega} (\estm_{\Omega'} + p_{\Omega'})] 
\end{eqnarray*}
If $n$ activities require the resource, there are a priori $O(n *
2^n)$ pairs $(A_i, \Omega)$ to consider. Still, it is easy to see that
when a rule is applied for some set $\Omega$, the same rule provides
even better deductions for the superset $\Omega' = \{A_i \mid [\estm_i,
\letm_i) \subseteq [\estm_{\Omega}, \letm_{\Omega})\}$. This makes
that one can restrict the rules to sets $\Omega_I$ made of all
activities whose time window belongs to some interval $I$. As there
are no more that $O(n^2)$ such intervals $I$, we have a straightforward
polynomial algorithm, running in cubic time, to implement the
edge-finding rules.  Algorithms that perform all the time bound
adjustments in $O(n^2)$ are presented in \cite{CarlierPinson90,
NuijtenAarts93a, Nuijten94}. Another variant of the edge-finding
technique is presented in \cite{CarlierPinson94}. It runs in $O(n \log
n)$ but requires much more complex data structures. \cite{Vilim04}
presents another variant running in $O(n \log n)$ that requires less
complex data structures than the ones used in \cite{CarlierPinson94}.

%An interesting property of the edge-finding technique is established
%in \cite{Baptiste95} and in \cite{MartinShmoys96}: 

%\begin{lemma}
%Considering only the resource constraint and the current time bounds
%of activities, the edge-finding algorithm computes the smallest
%earliest start time at which each activity $A_i$ could start if all
%the other activities were interruptible.
%\end{lemma}

Techniques similar to edge-finding have been proposed to propagate
groups of mutex relations in planning \cite{VidalGeffner05}.
\index{constraint!edge-finding|)}
\index{resource!unary|)}

\subsubsection{``Not-first'' and ``not-last'' rules} 

\index{constraint!not-first not-last}{``Not-first'' and ``not-last''} propagation rules have also been developed 
as a ``negative'' counterpart to edge-finding. These rules deduce that 
an activity $A_i$ cannot be the first (or the last) to execute in
$\Omega \cup \{ A_i \}$.
\begin{eqnarray*}
\label{NFNL-RULES-EQ1}
\forall_\Omega \; \forall_{A_i \notin \Omega} \;
[\letm_{A_i} - \estm_{\Omega} < p_{\Omega} + p_i] 
\Rightarrow [\evar(A_i) \le \max_{B \in \Omega} \lstm_{B}] \\
\label{NFNL-RULES-EQ2}
\forall_\Omega \; \forall_{A_i \notin \Omega} \;
[\letm_{\Omega} - \estm_{A_i} < p_{\Omega} + p_i] 
\Rightarrow [\svar(A_i) \ge \min_{B \in \Omega} \eetm_{B}] \\
\end{eqnarray*}
The corresponding time bound adjustments can be made in $O(n^2)$
\cite{BaptisteLePape96, TorresLopez00}.

\subsubsection{Conjunctive reasoning between temporal and resource constraints}

The above given propagation techniques reason on the time bounds of
activities on one unary resource. In \cite{NuijtenSourd00,
SourdNuijten00} propagation techniques are presented that reason on
the combination of time bounds of activities on multiple unary
resources and the temporal constraints linking these activities. Even
though these techniques have led to good computational results, they
have not yet been studied much. Propagation techniques that reason on
the combination of activity time bounds and temporal constraints on
one cumulative resource or reservoir have been studied. We discuss
these techniques in the Section~\ref{SWNConjReasCumul}.

%%%%%%%%%%%%%%%%%%%%%%%%%%%%%%%%%%%%%%%%%%%%%%%%%%%%%%%%%%%%%%%%%%%%%
%%%%%%%%%%%%%%%%%%%%%%%%%%%%%%%%%%%%%%%%%%%%%%%%%%%%%%%%%%%%%%%%%%%%%
\subsection{Cumulative resources} \label{CUMULATIVE-RESOURCES}
%%%%%%%%%%%%%%%%%%%%%%%%%%%%%%%%%%%%%%%%%%%%%%%%%%%%%%%%%%%%%%%%%%%%%
%%%%%%%%%%%%%%%%%%%%%%%%%%%%%%%%%%%%%%%%%%%%%%%%%%%%%%%%%%%%%%%%%%%%%

\index{resource!cumulative|(}

Cumulative resource constraints represent the fact that activities
$A_i$ use some amount $\capvar(A_i)$ of resource throughout their
execution. Many algorithms have been proposed for the propagation of
the \index{scheduling!non-preemptive}{non-preemptive} cumulative
constraint. A limited subset of these algorithms is presented in this
section.  In the remainder of this chapter, $c_i$ denotes the minimal
value of $\capvar(A_i)$, \ie, the minimal capacity required by $A_i$.

%====================================================================
\subsubsection{Timetable Constraint}
\label{TimetableConstraint}

%====================================================================
\index{constraint!timetable|(} First we consider the timetable
mechanism, widely used in \index{scheduling!constraint
based}{Constraint-Based Scheduling} tools, that allows to propagate
the resource constraint in an incremental fashion. The ``timetable''
is used to maintain information about resource utilization and
resource availability over time. Resource constraints are propagated
in two directions. From resources to activities, to update the time
bounds of activities (earliest start times and latest end times)
according to the availability of resources; and from activities to
resources, to update the timetables according to the time bounds of
activities. Although several variants exist \cite{LePape88, Fox90,
LePape94, Smith94, Lock96} the propagation mainly consists of
maintaining for any time $t$ \index{consistency!arc
consistency}{arc-B-consistency} on the formula:
$$
\sum_{A_i \mid \svar(A_i) \le t < \evar(A_i)} \capvar(A_i) \le \capvar(R)
$$
\index{constraint!timetable|)}

%====================================================================
\subsubsection{Disjunctive Constraint}

%====================================================================
\index{constraint!disjunctive|(}
Let $A_i$ and $A_j$ be two activities such that $c_i
+ c_j > \capvar(R)$. As such they cannot overlap in time and thus
either $A_i$ precedes $A_j$ or $A_j$ precedes $A_i$, \ie, the
disjunctive constraint holds between these
activities. In general the disjunctive constraint achieves
\index{consistency!arc consistency}{arc-B-consistency} on the formula
$$
[\capvar(A_i) + \capvar(A_j) \le \capvar(R)] \vee [\evar(A_i) \le
  \svar(A_j)] \vee [\evar(A_j) \le \svar(A_i)]
$$
\index{constraint!disjunctive|)}
%====================================================================
%\subsubsection{Energetic Reasoning}
\subsubsection{Energy Reasoning}
%====================================================================
\index{energy reasoning|(}
Energy based constraint propagation algorithms compare the amount of
energy provided by a resource over some interval $[t_1, t_2)$ to the
amount of energy required by activities that have to be processed over this
interval.
\cite{ErschlerLopez91}
proposes the following  definition of the required energy consumption that
takes into account the fact that activities cannot be
interrupted. 
Given an
activity $A_i$ and a time interval $[t_1, t_2), \Wsh(A_i, t_1, t_2)$,
the ``Left-Shift / Right-Shift'' required energy consumption of $A_i$
over $[t_1, t_2)$ is $c_i$ times the minimum of the three following
durations.
\begin{itemize}
\item
$t_2 - t_1$, the length of the interval;
\item
$p_i^{+}(t_1) = \max(0, p_i - \max(0, t_1 - \estm_i))$, the number of time
units during which $A_i$ executes after time $t_1$ if $A_i$ is
left-shifted, \ie, scheduled as soon as possible; 
\item
$p_i^{-}(t_2) = \max(0, p_i - \max(0, \letm_i - t_2))$, the number of time
units during which $A_i$ executes before time $t_2$ if $A_i$ is
right-shifted, \ie, scheduled as late as possible.  
\end{itemize}
This leads to
$\Wsh(A_i, t_1, t_2) = c_i \min(t_2 - t_1, p_i^{+}(t_1),
p_i^{-}(t_2))$. In Figure \ref{REQ-LSRS-FIG} an example is given where
the required energy consumption of $A_1$ over $[2, 7)$
is 8. Indeed, at least 4 time units of $A_1$
have to be executed in $[2, 7)$; \ie, $\Wsh(A, 2, 7) = 2 \min(5, 5,
4) = 8$.
\begin{center}
\begin{figure}[htbp]
\epsfxsize=12cm
\centering\mbox{\epsfbox{figures/reqEnergyLSRS.eps}}
\caption{Left-Shift / Right-Shift. \label{REQ-LSRS-FIG}}
\end{figure}
\end{center}

The Left-Shift / Right-Shift overall required energy consumption
$\Wsh(t_1, t_2)$ over an interval $[t_1, t_2)$ is then defined as the
sum over all activities $A_i$ of $\Wsh(A_i, t_1, t_2)$. The Left-Shift
/ Right-Shift slack $\Ssh(t_1, t_2)$ over $[t_1, t_2)$ is defined as
$C (t_2 - t_1) - \Wsh(t_1, t_2)$. It is obvious that if there is a
feasible schedule then $\Ssh(t_1, t_2) \ge 0$ for all $t_1$ and $t_2$
such that $t_2 \ge t_1$.

It is shown in \cite{BaptisteLePape01} how the values of $\Wsh$ can be
used to adjust activity time bounds. Given an activity $A_i$ and a
time interval $[t_1, t_2)$ with $t_2 < \letm_i$, it is examined
whether $A_i$ can end before $t_2$.  If there is a time interval
$[t_1, t_2)$ such that
$$
\Wsh(t_1, t_2) - \Wsh(A_i, t_1, t_2) + c_i p_i^{+}(t_1) > C (t_2 - t_1)
$$
then a valid lower bound of the end time of $A_i$ is
$$
t_2 + \frac{1}{c_i}(\Wsh(t_1, t_2) - \Wsh(A_i, t_1, t_2) + c_i p_i^{+}(t_1) - C (t_2 - t_1))
$$
Similarly, when
$$
\Wsh(t_1, t_2) - \Wsh(A_i, t_1, t_2) + c_i \min(t_2 - t_1, p_i^{+}(t_1)) > C (t_2 - t_1),
$$
$A_i$ cannot start before $t_1$ and a valid
lower bound of the start time of $A_i$ is
$$
t_2 - \frac{1}{c_i}(C (t_2 - t_1) - \Wsh(t_1, t_2) + \Wsh(A_i, t_1,t_2)).
$$

\cite{BaptisteLePape01} presents a $O(n^3)$ algorithm to compute 
these time bound adjustments for all $n$ activities. It is first shown
there are $O(n^2)$ intervals $[t_1, t_2)$ of interest. Given an
interval and an activity, the adjustment procedure runs in $O(1)$. As
such the obvious overall complexity of the algorithm is thus
$O(n^3)$. An interesting open question is whether there is a quadratic
algorithm to compute all the adjustments on the $O(n^2)$ intervals
under consideration. Another open question at this point is whether
the characterization of the $O(n^2)$ time intervals in
\cite{BaptisteLePape01} can be sharpened in order to eliminate some
intervals and reduce the practical complexity of the corresponding
algorithm. Finally, it seems reasonable to think that the time bound
adjustments could be sharpened. Even though the energy tests can be
limited (without any loss) to a given set of intervals, it could be
that the corresponding adjustment rules cannot.


%\begin{algorithm}[htbp]
%\caption{An algorithm for Left-Shift / Right-Shift adjustments}
%\label{TIME-BOUND-ADJUSTMENTS-LSRS-CUSP-ALGO}
%\For{all relevant time intervals $[t_1, t_2)$ } {
%  $E \la 0$\;
%  \For{$i \in \{1, \ldots, n\}$} {
%    $E \la E + c_i \min(t_2 - t_1, p_i^{+}(t_1), p_i^{-}(t_2))$\;
%  }
%  \eIf {$E > C (t_2 - t_1)$}{
%    there is no feasible schedule, exit
%  }{
%    \For{$i \in \{1, \ldots, n\}$} {
%      $\slk \la C (t_2 - t_1) - E + c_i \min(t_2 - t_1, p_i^{+}(t_1), p_i^{-}(t_2))$\;
%      \If{$\slk < c_i p_i^{+}(t_1)$} {
%        $\eetm_i \la \max(\eetm_i, t_2 + \lceil(c_i p_i^{+}(t_1) - \slk) / c_i\rceil)$\;
%      }
%      \If{$\slk < c_i p_i^{-}(t_2)$} {
%        $\lstm_i \la \min(\lstm_i, t_1 - \lceil (c_i p_i^{-}(t_2) - \slk)/c_i \rceil)$\;
%      }
%    }
%  }
%}
%\end{algorithm}

\index{energy reasoning|)}
\index{resource!cumulative|)}



%%%%%%%%%%%%%%%%%%%%%%%%%%%%%%%%%%%%%%%%%%%%%%%%%%%%%%%%%%%%%%%%%%%%%
%%%%%%%%%%%%%%%%%%%%%%%%%%%%%%%%%%%%%%%%%%%%%%%%%%%%%%%%%%%%%%%%%%%%%
\subsection{Conjunctive reasoning between temporal and resource
  constraints}\label{SWNConjReasCumul}
%%%%%%%%%%%%%%%%%%%%%%%%%%%%%%%%%%%%%%%%%%%%%%%%%%%%%%%%%%%%%%%%%%%%%
%%%%%%%%%%%%%%%%%%%%%%%%%%%%%%%%%%%%%%%%%%%%%%%%%%%%%%%%%%%%%%%%%%%%%

The propagation algorithms described in the previous section reason on
the time bounds of activities ($r_i$, $\lstm_i$, $\eetm_i$,$d_i$) and
do not directly take into account the
\index{constraint!temporal}{precedence constraints} that may exist
between them. We describe in this section two recently proposed
propagation algorithms respectively on cumulative resources and
reservoirs. Like previous propagation algorithms, both of them are
used to discover new time bounds and/or new precedence relations. The
main originality lies in the fact that they analyze the relative
position of activities (precedence relations in the precedence graph)
rather than their absolute position only, as was the case for the
previously discussed propagation techniques. As a consequence, they
allow a much stronger propagation when the time windows of activities
are large and when the current schedule contains a lot of precedence
relations, which is typically the case when integrating planning and
scheduling.

\subsubsection{Precedence Graph}

%%%%%%%%%%%%%%%%%%%%%%%%%%%%%%%%%%%%%
%% Precedence Graph
%%%%%%%%%%%%%%%%%%%%%%%%%%%%%%%%%%%%%
\index{constraint!precedence graph|(} The algorithms presented in this
section require that a temporal network representing the relations
between the time points of all activities (start and end times) using the
Point Algebra \cite{VilainKautz86} is maintained during search. We
denote the set of qualitative relations between time points by
$\{\emptyset, \prec, \preceq, =, \succ, \succeq, \neq, ? \}$. The
temporal network is in charge of maintaining the transitive closure of
those relations. 

If $s_i$ and $e_i$ denote the start and end time-point of activity
$A_i$, the initial set of relations consist of the precedences $s_i
\prec e_i$ for each activity $A_i$ and $x_i
\preceq x_j$ or $x_i \prec x_j$ for each precedence constraint $x_i \leq x_j + d_{ij}$ depending on the value of $d_{ij}$. For instance, for a precedence constraint $\evar(A_i) \le
\svar(A_j)$  between two activities $A_i$ and $A_j$, the initial precedence graph contains the relation $e_i
\preceq s_j$.

During search additional precedence relations can be added as
decisions or as the result of constraint propagation.
\index{constraint!precedence graph|)}

\subsubsection{Energy Precedence Constraint}

\index{constraint!energy precedence|(}

%%%%%%%%%%%%%%%%%%%%%%%%%%%%%%%%%%%%%
%% Energy Precedence Constraint
%%%%%%%%%%%%%%%%%%%%%%%%%%%%%%%%%%%%%

\index{resource!cumulative|(}

The {\em energy precedence} propagation \cite{Laborie03a} for an
activity $A_i$ on a cumulative resource $R_k$ ensures that for each
subset $\phi$ of predecessor activities of activity $A_i$ the resource
provides enough energy to execute all activities in $\phi$ between
$\estm_{\phi}$ and $s_i$. More formally, it performs the following
deduction rule
\[\forall_{\phi \subseteq \{ A_j \mid e_j \preceq s_i \}} \; \svar(A_i) >=
\estm_{\phi}+ \lceil E_{\phi} / cap(R_k) \rceil  \]
where $E_{\phi}$ is the sum of the minimal value of $E(A_j, R_k)$
(minimal energy) over all $A_j$ in $\phi$.  The propagation of the
energy precedence constraint can be performed for all the activities
on a resource and for all the subsets $\phi$ with a total worst-case
time complexity of $O(n (p+log(n))$ where $n$ is the number of
activities on the resource and $p$ the maximal number of predecessors
of a given activity in the temporal network ($p < n$).

\index{resource!cumulative|)}
\index{constraint!energy precedence|)}

\subsubsection{Balance Constraint}
\label{BalanceConstraint}
%%%%%%%%%%%%%%%%%%%%%%%%%%%%%%%%%%%%%
%% Balance Constraint
%%%%%%%%%%%%%%%%%%%%%%%%%%%%%%%%%%%%%
\index{constraint!balance|(}
\index{resource!reservoir|(}

On a reservoir resource, if $x$ is the start or end time of an
activity that changes the reservoir level, we denote by $q(x)$ the
integer variable representing the relative change of the reservoir
level due to the activity. By convention, $0<q(x)$ represents a
production event and $q(x)<0$ a consumption event. The basic idea of
the {\em balance constraint} \cite{Laborie03a} is to for each
event $x$ compute an upper and lower bound on the reservoir
level at the time of $x$. Using the temporal network, an upper bound on
the reservoir level at date $x - \epsilon$ just before $x$ can be
computed assuming that:
\begin{itemize}
\item All the production events $y$ that {\em may} be executed
      strictly before $x$ are executed strictly before $x$ and produce
      as much as possible, \ie, produce $\ub(q(y))$ (denoted by
      $q_{max}(y)$). Let $Poss^\preceq(x) = \{y \mid \neg(x \prec
      y)\}$ denote the set of events that may be executed before or at the
      same time as $x$.
\item All the consumption events $y$ that {\em need} to be executed
      strictly before $x$ are executed strictly before $x$ and consume
      as little as possible, \ie, consume $q_{max}(y)$\footnote{For a
      consumption event, $q<0$ and thus, $q_{max}$ really corresponds
      to the smallest consumption of the event.}. Let $Nec^\prec(x) =
      \{y \mid (y \prec x)\}$ denote the set of events that
      necessarily execute strictly before $x$.
\item All the consumption events that {\em may} execute simultaneously
      or after $x$ are executed simultaneously or after $x$.
\end{itemize}
More formally, if $P$ is the set of  production events and $C$ the set
of consumption events, the upper bound can be computed as follows:
\begin{eqnarray}
\label{formula1}
L^{<}_{max}(x) =  \sum_{y \in P \cap Poss^\preceq(x)} q_{max}(y) + \sum_{y \in C \cap Nec^\prec(x)}  q_{max}(y)
\end{eqnarray}
For symmetry reasons, we describe only the propagation based on
$L^{<}_{max}(x)$. Using this bound, the balance constraint is able to
discover four types of information: dead ends, new bounds for required
capacity variables, new bounds for time variables, and new precedence
relations.

\begin{itemize}

\item{Discovering  dead  ends.}   This  is the  most  trivial
propagation. Whenever $L^{<}_{max}(x)<0$,  we  know that the  level of
the reservoir  will surely be negative  just  before event $x$  so the
search has reached a dead end.

\item{Discovering new bounds on required capacity variables.}  Suppose
there exists a consumption event $y \in Nec^\prec(x)$ such that
$q_{max}(y)-q_{min}(y)>L^{<}_{max}(x)$.  If $y$ would consume a
quantity $q$ such that $q_{max}(y)-q>L^{<}_{max}(x)$ then, simply by
replacing $q_{max}(y)$ by $q$ in (\ref{formula1}), we see
that the level of the reservoir would be negative just before
$x$. Thus, we can deduce that $q(y) \geq q_{max}(y)-L^{<}_{max}(x)$.

\item{Discovering new bounds on time variables.} 

$\setminus Nec^\prec(x)$ consists of the set of events that may but
need not necessarily execute strictly before $x$, \ie, they can also
execute at the same time as or after $x$. Let $P(x)$ denote the set
of production events in $Poss^\preceq(x) \setminus Nec^\prec(x)$. The
reasoning behind the deduction here is that if the maximal reservoir
level of all events necessarily executed strictly before $x$ is
negative, then some production events in $P(x)$ need to be scheduled
before $x$, more specifically $x$ needs to be scheduled after a
sufficient number of these production events to not have a negative
reservoir level. More formally, let's rewrite (\ref{formula1}) as follows:

\begin{eqnarray*}
\label{formula2}
L^{<}_{max}(x) = \hspace{-2mm} \sum_{y \in Nec^\prec(x)} \hspace{-2mm}
q_{max}(y) +  \hspace{-5mm}\sum_{y \in P \cap (Poss^\preceq(x) \setminus Nec^\prec(x))}
\hspace{-10mm} q_{max}(y)
\end{eqnarray*}
The first term of this equation is the sum of the maximal production
and minimal consumption of the events that necessarily execute
strictly before $x$, thus giving the maximal reservoir level. As
$L^{<}_{max}(x) \geq 0$, we know that if this term is negative, it
means that some production events in $P(x)$ will have to be executed
strictly before $x$ in order to produce at least:
\begin{eqnarray*}
\Delta^{<}_{min}(x) = - \sum_{y \in Nec^\prec(x)}\hspace{-2mm} q_{max}(y)
\end{eqnarray*}
We suppose the production events
$(y_1,\cdots,y_i,\cdots,y_p)$ in $P(x)$ are ordered by non-decreasing
minimal time $t_{min}(y)$. $t_{min}(y)$ is either the earliest start
time or earliest end time, depending on whether $y$ is a start or end
time. Let $k$ be the index in $[1,p]$ such that:
\begin{eqnarray*}
\sum_{i=1}^{k-1}q_{max}(y_i) < \Delta^{<}_{min}(x) \le \sum_{i=1}^{k}q_{max}(y_i)
\end{eqnarray*}
If event $x$ is executed at a date $t(x) \le t_{min}(y_k)$, not enough
producers will be   able to execute  strictly  before $x$  in order to
ensure  a positive level just before  $x$. Thus, $t_{min}(y_k)+1$ is a
valid lower bound of $t(x)$. 

\item{Discovering new precedence relations.\label{dnpr}} There are
cases where one can perform an even stronger propagation.  $P(x)$ is again
the set of production events in $Poss^\preceq(x) \setminus
Nec^\prec(x)$. If there is one production event $y$ in $P(x)$ that is
needed to produce before $x$ to not get a negative reservoir level, a
precedence relation can be deduced between $y$ and $x$. So, suppose there
exists a production event $y$ in $P(x)$ such that:
\begin{eqnarray*}
\sum_{z \in P(x) \cap Poss^\preceq(y)} q_{max}(z) < \Delta^{<}_{min}(x) 
\end{eqnarray*}

Then, if we  had $t(x)\le t(y)$,  we would see  that again there is no
way to produce $\Delta^{<}_{min}(x)$ before  event $x$ as the only events
that  could produce strictly  before event $x$ are  the  ones in $P(x)
\cap Poss^\preceq(y)$.   Thus, we can deduce the necessary
precedence relation:  $t(y)<t(x)$.  Note that  a weaker version of this
propagation has been proposed in  \cite{CestaStella97} that runs in $O(n^2)$
and does  not analyze the precedence relations  between the  events of
$P(x)$.
\end{itemize}
The balance algorithm can be executed for all the events $x$ on a
reservoir with a global worst-case complexity in $O(n^2)$ if the
propagation that discovers new precedence relations is not turned on,
and in $O(n^3)$ for a full propagation. In practice, there are many
ways to shortcut this worst case and in particular, it has been
noticed that the algorithmic cost of the extra propagation that
discovers new precedence relations was in general negligible. Unlike all the other propagation algorithms we have seen in this section \ref{SPropRct}, the balance constraint cannot directly be applied in planning problems because it assumes that the set or producer and consumer events is completely known. The extension of the balance constraint to planning problems is discussed in \cite{Laborie03a}.

\index{resource!reservoir|)}
\index{constraint!balance|)}
\index{constraint!resource|)}
\index{constraint!global|)}


%%%%%%%%%%%%%%%%%%%%%%%%%%%%%%%%%%%%%%%%%%%%%%%%%%%%%%%%%%%%%%%%%%%%%
%%%%%%%%%%%%%%%%%%%%%%%%%%%%%%%%%%%%%%%%%%%%%%%%%%%%%%%%%%%%%%%%%%%%%
\section{Constraint Propagation on Optimization Criteria}\label{SPropOptCrit}
%%%%%%%%%%%%%%%%%%%%%%%%%%%%%%%%%%%%%%%%%%%%%%%%%%%%%%%%%%%%%%%%%%%%%
%%%%%%%%%%%%%%%%%%%%%%%%%%%%%%%%%%%%%%%%%%%%%%%%%%%%%%%%%%%%%%%%%%%%%

As said in Section~\ref{OBJECTIVE-FUNCTION-SEC}, the commonly used way
of modeling an objective function is by introducing a variable
$\critvar$ that is constrained to be equal to the value of the
objective function. In cases where the objective function $F$ is a
function of the end variables of the activities and $F$ is a maximum
such as $C_{\max}$ or $T_{\max}$, considering the objective constraint
and the resource constraints independently is not a problem. Indeed,
the upper bound on $\critvar$ is directly propagated on the end time
of each activity, \ie, latest end times are tightened efficiently.

The situation is more complex for sum functions such as $\sum w_i C_i,
\sum w_i T_i$, or $\sum w_i U_i$, and for objective functions that are
not a function of the end variables of the activities like sum of
setup times and sum of setup costs. For several of these cases, dedicated 
constraint propagation techniques have been developed often taking the resource
constraints and the objective function simultaneously into account.

In this section we describe such dedicated constraint propagation
techniques for two objective functions in more detail: weighted number
of late activities ($\sum w_i U_i$) and sum of setup times and setup
costs. For more general considerations on cost-based
constraint propagation we refer to \cite{Focacci01}.


%=====================================================================
\subsection{Weighted Number of Late Activities}
%=====================================================================
\index{activity!late|(}

In this section we pay attention to constraint propagation for the
objective function $\sum w_i U_i$, i.e., minimizing the weighted
number of late activities, as described in \cite{BaptisteLePape01}.

The basis for this constraint propagation is formed by calculating a
good lower bound on the weighted number of late activities. Such a
lower bound is obviously also a lower bound for the variable
$\critvar$. Relaxing non-preemption is a well-known technique to
obtain good lower bounds in scheduling. Unfortunately, the preemptive
problem remains NP-Hard. \index{scheduling!preemptive}{A ``relaxed
preemptive lower bound''}, \ie, a slightly stronger relaxation than
the preemptive relaxation, can be used. As explained below, it can be
computed in $O(n^2 \log n)$.

Let us recall a well-known result for the
\index{resource!unary}{One-Machine Scheduling Problem} (\ie, the
problem of scheduling activities on a unary resource).  Its preemptive
relaxation is polynomial and has the very interesting property that
there exists a feasible preemptive schedule if and only if over any
interval $[t_1, t_2)$, the sum of the processing times of the
activities in $\{A_i \mid [t_1 \le \estm_i] \wedge [\letm_i \le
t_2]\}$ is at most $t_2 - t_1$. It is well-known that
relevant values for $t_1$ and $t_2$ are respectively the release dates
and the deadlines \cite{Carlier84}.

A decision variable $x_i$ per activity is introduced that equals $1$
when the activity is on-time and 0 otherwise. Notice that if $\letm_i
\le \delta_i$, $A_i$ is on-time in any solution, \ie, $x_i = 1$. In
such a case we adjust the value of $\delta_i$ to $\letm_i$ (this has
no impact on solutions) so that due dates are always smaller than or
equal to deadlines. We also assume that there is a preemptive schedule
that meets all deadlines (if not, the resource constraint does not
hold and a backtrack occurs). The following \index{linear
programming!mixed integer programming}{Mixed Integer Program} (MIP)
\cite{Schrijver03} computes the minimum weighted number of late
activities in the preemptive case:
\begin{equation} \label{P-EQ}
\begin{array}{l}
\begin{displaystyle} \min \sum_1^n w_i(1-x_i) \end{displaystyle} \\
u.c.~
\left\{
\begin{array}{l}
\begin{displaystyle}
\forall_{ t_1} \; \forall_{t_2 > t_1} \; \sum_{S(t_1,t_2)} p_i
+\sum_{P(t_1,t_2)} p_i x_i \le t_2 - t_1 \end{displaystyle} \\
\forall_{i \in \{1, \ldots, n\}} \; x_i \in \{0,1\}
\end{array}
\right.
\end{array}
\end{equation}
where $S(t_1, t_2)$ is the set of activities that are sure to execute
between $t_1$ and $t_2$ and where $P(t_1, t_2)$ is the set of
activities that are preferred to execute between $t_1$ and $t_2$.
\begin{eqnarray*} \label{S-AND-P-DEF-EQ}
S(t_1,t_2) &=& \{A_i \mid \estm_i \ge t_1 \wedge \letm_i \le t_2\} \\
P(t_1,t_2) &=& \{A_i \mid \estm_i \ge t_1 \wedge \letm_i > t_2\ \wedge \delta_i \le t_2\}
\end{eqnarray*}
Actually, it is easy to see that the relevant values of $t_1$
correspond to the release dates and those of $t_2$ to the due dates
and deadlines. Hence, there are $O(n^2)$ constraints in the MIP.  We
now focus on the {\em continuous relaxation} of (\ref{P-EQ}) in which
for any activity $A_i$ such that $\estm_i + p_i > \delta_i$, \ie, for
any late activity, the constraint $x_i = 0$ is added.
\begin{equation} \label{CP-EQ}
\begin{array}{l}
\begin{displaystyle} \min \sum_1^n w_i(1-x_i) \end{displaystyle} \\
u.c.~
\left\{
\begin{array}{l}
\begin{displaystyle}
\forall_{t_1 \in \{\estm_i\}} \; 
\forall_{t_2 \in \{\letm_i\} \cup \{\delta_i\} \; \mid \; t_2 > t_1}
\;  \sum_{S(t_1,t_2)} p_i +\sum_{P(t_1,t_2)} p_i x_i \le t_2 - t_1
\end{displaystyle} \\
\forall_{i} \; \estm_i + p_i > \delta_i \Rightarrow x_i=0 \\
\forall_{i \in \{1, \ldots, n\}} \; x_i \in [0,1]
\end{array}
\right.
\end{array}
\end{equation}

The \index{linear programming}{linear program} (\ref{CP-EQ}) can be
solved with an LP solver and we can use reduced costs to prove that
some activities can, must or cannot end before their due date. In
\cite{BaptisteLePape01} an $O(n^2 \log n)$ algorithm is described
solving the same problem.
\index{activity!late|)}



%=====================================================================
%=====================================================================
\subsection{Sum of Setup Times and Sum of Setup Costs } \label{PROP-SUMTT-SEC} 
%=====================================================================
%=====================================================================
\index{setup!time|(}
\index{setup!cost|(}

In this section we pay attention to constraint propagation of setup
time and setup cost constraints. We discuss the constraint propagation
as described in \cite{FocacciLaborie00}.  We also refer to
\cite{FocacciNuijten00, Focacci01} that extend work of Brucker and
Thiele \cite{BruckerThiele96} in the context of CP. To simplify the
presentation, we only consider the case where there are no cumulative
resources. We do include the possible presence of alternative
resources (see Section~\ref{SWNREFINED-MODEL-SEC}).

The basis for the constraint propagation of setup times and
setup costs described in this section is formed by using a {\em
routing problem} as a relaxation of the scheduling problem. In this
problem, one has a set of {\em start} nodes, a set of {\em internal}
nodes, and a set of {\em end} nodes. Each internal node $i$ represents
an activity $A_i$. When having $m$ alternative machines, one is
looking for $m$ disjoint {\em routes} or {\em paths} in the graph
defined by these three sets. Each route corresponds to a different
machine, starting in the start node of the machine, traversing a
sequence of internal nodes, and ending in the end node of the
machine. More precisely, let $I = \{1, \ldots, n\}$ be a set of $n$
nodes, and $E = \{n+1, \dots, n + m\}$ and $S = \{n+m+1, \dots, n + 2
* m\}$ two sets of $m$ nodes. Nodes in $I$ represent internal nodes,
nodes in $S$ represent start nodes, and nodes in $E$ represent end
nodes. A \index{constraint!global}{global constraint} is defined ensuring that $m$ different
routes $\rho_1, \dots, \rho_m$ exist such that all internal nodes are
visited exactly once by a route starting from a node in $S$ and ending
in a node in $E$. Start nodes $n+m+1, \dots, n + 2 * m$ belong to
routes $\rho_1, \dots, \rho_m$, respectively. End nodes $n+1, \dots, n
+ m$ belong to routes $\rho_1, \dots, \rho_m$, respectively. Moreover,
sets of possible routes can be associated to each internal node.

In the CP model three variables per node are defined. Variables
$\Next_i$ and $\Prev_i$ identify the nodes visited directly after and
directly before node $i$, respectively. Variables $\Path_i$ identify
the route node $i$ belongs to. Variables $\Next_i$ and $\Prev_i$ take
their values in $\{1, \ldots, n + 2 m\}$. Variables $\Path_i$ take
their values in $\{1, \ldots, m\}$. Each start and end node has its
route variable bound, \ie, $\Path_{n+1} = 1, \ldots, \Path_{n+m} = m$,
$\Path_{n + m+1} = 1$, $\dots, \Path_{n+2 m} = m$.  In order to have a
uniform treatment of all nodes inside the constraint, each start node
$n+m+u$ has its $\Prev_{n+m+u}$ variable bound to the corresponding
end node ($\Prev_{n+m+u} = n+u$), and each end node $n+u$ has its
$\Next_{n+u}$ variable bound to the corresponding start node
($\Next_{n+u} = n+m+u$). There furthermore exists a setup cost
$c_{ij}^u$ that expresses that if node $j$ is visited directly after
node $i$ on a route $u$ ($\Next_i = j, \Path_i = \Path_j = u$), a cost
$c_{ij}^u$ is induced. A feasible solution is defined as an assignment
of distinct values to each next variable, while avoiding sub-tours
(tours containing only internal nodes), and respecting the constraints
\begin{eqnarray*}
\Next_i = j & \Leftrightarrow & \Prev_j = i \\
\Next_i = j & \Rightarrow     & \Path_i = \Path_j
\end{eqnarray*}

The problem is then to find an optimal feasible solution, \ie, a
feasible solution that minimizes
\begin{equation} \label{TOTAL-TRANS-COST-EQ}
\sum_{i = 1}^{n} c_{i \: next_i}^u
\end{equation}

As said, the routing problem described constitutes a relaxation of the
global scheduling problem. If an internal node $i$ has its next
variable assigned to another internal node $j$, activity $A_i$
directly precedes activity $A_j$. If an internal node $i$ has its next
variable assigned to an ending node $n+u$, activity $A_i$ is the last
activity scheduled on machine $R_u$. The setup cost function
$c_{ij}^u$ of the routing problem corresponds to the setup times
$\setupt(A_i, A_j, R_u)$ or setup costs $\setupc(A_i, A_j, R_u)$
between activities (see Section~\ref{SWNREFINED-MODEL-SEC}). As such the
minimization of the total setup cost (\ref{TOTAL-TRANS-COST-EQ}) in
the routing problem corresponds to the minimization of the sum of
setup times or setup costs in the scheduling problem.




%=====================================================================
\subsubsection{Route Optimization Constraint}
%=====================================================================

One of the basic ideas of the constraint propagation in
\cite{FocacciLaborie00} is to create a \index{constraint!global}{global constraint} having a
propagation algorithm aimed at removing those assignments from
variable domains which do not improve the best solution found so
far. Domain reduction is achieved by optimally solving an Assignment
Problem \cite{DellAmicoMartello97} which is a relaxation of the
routing problem described and thus also of the global scheduling
problem. The \index{assignment problem}{Assignment Problem} is the graph theory
problem of finding a set of disjoint sub-tours such that all the
vertices in a graph are visited and the overall cost is minimized.

In the routing problem we look for a set of $m$ disjoint routes each
of them starting from a start node and ending in the corresponding end
node covering all nodes in a graph, \ie, considering that each end
node is connected to the corresponding start node, we look for a set
of $m$ disjoint tours each of them containing a start node. This
problem can be formulated as an Assignment Problem on the graph
defined by the set of
nodes in the routing problem and the set of arcs $(i,j)$ such that $j
\in \domain(\Next_i)$. The cost on arc $(i,j)$ is the minimal setup cost (or
time), \ie,
$$
\min_{u \in \domain(\Path_i) \cap \domain(\Path_j)} \setupc(A_i, A_j,
R_u).$$ The value of the optimal solution of the Assignment Problem is
obviously a lower bound on the value of the optimal solution of the routing
problem. The {\em primal-dual} algorithm described in
\cite{CarpanetoMartello88} provides an optimal integer solution for the
Assignment Problem. Besides this optimal assignment
with the corresponding lower bound $LB$ on the original problem, a
reduced cost matrix $\bar c$ is obtained. Each $\bar c_{ij}$ estimates
the additional cost to be added to $LB$ if variable $\Next_i$ takes
the value $j$. These results can be used both in constraint
propagation as in the definition of \index{heuristic!branching}{search heuristics}. The lower bound
$LB$ is trivially linked to the $\critvar$ variable representing the
objective function through the constraint $LB \leq \critvar$. More
interesting is the propagation based on reduced costs. Given the
reduced cost matrix $\bar c$, it is known that $LB_{\Next_{i} = j} =
LB + \bar c_{ij}$ is a valid lower bound for the problem where
$\Next_{i}$ takes the value $j$. Therefore we can impose
\begin{eqnarray*}
LB_{\Next_{i} = j} > \ub(\critvar) \Rightarrow \ \Next_{i} \neq j
\end{eqnarray*}

For more details on the use of reduced costs for setup constraints we
refer to \cite{FocacciLaborie00}. We remark that reduced cost fixing
appears to be particularly suited for CP. In fact, while reduced cost
fixing is extensively used in OR frameworks, it is usually not
exploited to trigger other constraints, but only in the following
lower bound computation, \ie, the following node in the search
tree. When embedded in a CP framework, the reduced cost fixing
produces domain reductions which usually trigger propagation from
other constraints in the problem through shared variables.

%=====================================================================
\subsubsection{Precedence Graph Constraint}
\label{pgraph} 
%=====================================================================
\index{constraint!precedence graph|(}
Linking the routing model and the scheduling model is done thanks to a
{\em precedence graph constraint}. This constraint maintains for each
machine $R_u$ an extended precedence graph $G_u$ that allows to
represent and propagate \index{constraint!temporal}{temporal relations} between pairs of activities
on the machine as well as to dynamically compute the transitive
closure of those relations. More precisely, $G_u$ is a graph whose
vertices are the alternative activities $A_i^u$ that may execute on
machine $R_u$ (see Section~\ref{SWNREFINED-MODEL-SEC}). A node $A_i^u$ is
said to {\em surely contribute} if machine $R_u$ is the only possible
machine on which $A_i$ can be processed. Otherwise, if activity $A_i$
can also be processed on other machines, the node $A_i^u$ is said to
{\em possibly contribute}. Two kinds of edges are represented on
$G_u$:
\begin{itemize}
\item A {\em precedence edge} between two alternative activities $A_i^u
\rightarrow A_j^u$ means that if machine $R_u$ is chosen for both
activities $A_i$ and $A_j$, then $A_j$ will have to be processed after
$A_i$ on $R_u$.
\item A {\em next edge} between two alternative activities $A_i^u
\Rightarrow A_j^u$ means that if machine $R_u$ is chosen for both
activities $A_i$ and $A_j$ then $A_j$ will have to be processed
directly after $A_i$ on $R_u$. No activity may be processed on $R_u$
between $A_i$ and $A_j$.
\end{itemize}

The first role of the precedence graph is to incrementally maintain
the closure of this graph when new edges or vertices are inserted,
\ie, to deduce new edges given the ones already present in the
graph. The following five rules \cite{FocacciLaborie00} are used by the
precedence graph:
\begin{enumerate}
\item If $A_i^u \rightarrow A_j^u$, $A_j^u \rightarrow A_i^u$, and
$A_i^u$ surely contributes then $A_j^u$ does not contribute
(Incompatibility rule).
\item If $A_i^u \rightarrow A_l^u$, $A_l^u \rightarrow A_j^u$, and $A_l^u$
surely contributes then $A_i^u \rightarrow A_j^u$ (Transitive closure
through contributor).
\item If $A_l^u \Rightarrow A_i^u$, $A_l^u \rightarrow A_j^u$, and
$A_l^u$ surely contributes then $A_i^u \rightarrow A_j^u$
(Next-edge closure on the left).
\item If $A_j^u \Rightarrow A_l^u$, $A_i^u \rightarrow A_l^u$, and $A_l^u$
surely contributes then $A_i^u \rightarrow A_j^u$ (Next-edge closure
on the right).
\item If for all $A_l^u$ either $A_l^u \rightarrow A_i^u$ or $A_j^u
\rightarrow A_l^u$ then $A_i^u \Rightarrow A_j^u$ (Next-edge finding).
\end{enumerate}
New edges are added on the precedence graph $G_u$ by the scheduling
constraints (precedence and resource constraints) and by the route
optimization constraint (whenever a variable $\Next_i$ is bound a new
next-edge is added). Besides computing the incremental closure, the
precedence graph also incrementally maintains the set of activities
that are possibly next to a given activity $A_i^u$ given the current
topology of $G_u$. As such it allows to effectively reduce the domain
of the variables $\Next_i$ and $\Prev_i$. Furthermore, the precedence
graph constraint propagates the current set of precedence relations
expressed on $G_u$ on the start and end variables of activities.
\index{constraint!precedence graph|)}
\index{setup!cost|)}
\index{setup!time|)}

\index{scheduling!non-preemptive|)}


%%%%%%%%%%%%%%%%%%%%%%%%%%%%%%%%%%%%%%%%%%%%%%%%%%%%%%%%%%%%%%%%%%%%%
%%%%%%%%%%%%%%%%%%%%%%%%%%%%%%%%%%%%%%%%%%%%%%%%%%%%%%%%%%%%%%%%%%%%%

\section{Heuristic Search}\label{SHeuristics}

%\subsection{Branching Strategies}
\index{heuristic!branching|(}

The general principles around search in CP apply to both the
planning and scheduling domain:
\begin{itemize}
\item
Since for complexity reasons constraint propagation cannot remove 
all impossible values from the domains of variables, heuristic search
is required to generate a solution to the problem instance under
consideration.
\item
Once a solution with a given cost is found, this heuristic search
can be either continued or restarted with an additional constraint
stating that only solutions with a lower cost are searched for.
In the case of multiple criteria, this additional constraint can
be replaced by a set of constraints authorizing the solution to
deteriorate for some criteria if it improves for others.
\item
Some variables are more constrained than others, depending on the
problem instance: some activities lie on a critical path of the
precedence graph, some resources are more heavily loaded than
others, etc. Focusing on the more constrained variables first is
more likely to quickly lead to a solution.
\end{itemize}

However, the significance of temporal and resource constraints makes
it possible to use these principles in domain-specific manners. Let us
first consider the case of the pure Job Shop Scheduling Problem
\cite{French82}. The variables of the problem are basically just the
start and end times of activities and the $\critvar$ variable representing
the makespan ($C_{\max})$. \index{constraint!temporal}{Temporal constraints} relating these
variables are propagated in a perfect manner, \ie, the earliest and
latest start and end times resulting from constraint propagation
guarantee that the temporal constraints are satisfied. The only
remaining constraints are the \index{constraint!resource}{resource constraints}. As there are only
\index{resource!unary}{unary resources}, no two activities $A_i$ and $A_j$ requiring the same
resource can overlap in time, \ie, either $A_i$ precedes $A_j$ or
$A_j$ precedes $A_i$ (see Section~\ref{DisjConstraintProp}). Following
this basic observation, rather than attempting to instantiate the
start and end variables, an appealing and often much more efficient
strategy consists in deciding in which order activities shall execute,
\ie, whether $A_i$ shall execute before $A_j$ or $A_j$ before $A_i$.

Although it is less immediate, the same type of branching strategy can
also be considered for \index{resource!cumulative}{cumulative resources}. Indeed, whenever $n$
\index{scheduling!non-preemptive}{non-preemptible activities} are such that the sum of the capacities
required exceeds the available capacity (for a given resource), at
least two of these activities cannot overlap in time, and hence must
be ordered. An alternative but equivalent view consists in considering
a cumulative resource $R$ of capacity $\capvar(R)$ as a set of
$\capvar(R)$ ``lines'' of capacity 1, on which activities cannot
overlap.  Hence, if the activities can be organized along at most $c$
sequences such that (i) an activity $A_i$ requiring capacity $\capvar(A_i)$
appears in $\capvar(A_i)$ sequences and (ii) activities in each sequence are
totally ordered by temporal constraints, then the satisfaction of the
temporal constraints guarantees the satisfaction of the resource
constraint.

In practice, the right branching strategy also depends on the optimization
criterion (or multiple criteria) to optimize:
\begin{itemize}
\item
An optimization criterion to minimize is called ``regular'' if it 
increases with the end times of the activities. In other terms, 
a solution $S$ cannot be strictly better than another solution 
$S'$ if no activity $A_i$ finishes earlier in $S$ than in $S'$.
Examples of regular criteria include the makespan, the average completion
time of the activities, the maximal or weighted tardiness
of activities, the weighted number of late activities.
When the optimization criterion is regular, it is particularly appropriate
to solve the resource constraints by ordering activities: on any given 
branch of the search tree, the value of the criterion obtained by replacing
each end time variable by its lower bound is a lower bound for the
optimization function. In addition, if at a given node the earliest start
and end times satisfy all the constraints of the problem (which is
the case for resource constraints if they have been replaced by
appropriate temporal constraints and these temporal constraints have
been propagated), then these earliest start and end times provide the best
solution attainable from this node. ``Dominance properties'' can also be
applied to prune some nodes: whenever it can be shown that for any schedule
attainable from a node, an equivalent or better schedule is attainable from
another node, the first node can be discarded. For example, 
if a partial schedule contains a hole
on a resource (an interval of time over which it can be shown that no
activity requiring the resource can execute), and an activity is scheduled
after the hole for no good reason, then the node can be discarded since
another branch will lead to a schedule in which this activity (or another)
occupies the hole \cite{LePapeCouronne95}.
\item
An optimization criterion is called ``sequence-dependent'' if it
depends only on the relative order in which activities are
executed. Typical example are of course the sum of \index{setup!time}{setup times} and the
sum of \index{setup!cost}{setup costs}. When optimizing sequence-dependent criteria, it is
once again particularly appropriate to solve the resource constraints
by ordering activities: once activities are sequenced, the earliest
start and end times that result from constraint propagation can be
used as a solution. Note however that the dominance properties that
exist for regular criteria cannot be applied to sequence-dependent
criteria: for example, it might be worth leaving a hole in a schedule
by executing a specific activity later if it enables to save a costly
setup.
\item
Other optimization criteria are more difficult to optimize.  For
example, work in process time, \ie, the average difference between
the end time of the last activity composing a given job and the start
time of the first activity of the job, is an irregular criterion which
is difficult to optimize as the first activity of each job shall be
executed as late as possible while the last activity of each job shall
be executed as early as possible. Storage costs, in particular the
cost of storing intermediate products, are difficult to optimize for
the same reason. In such cases, it is not sufficient to sequence the
activities. It is however often the case that once the resource
constraints have been solved by sequencing activities, a linear
program can be used to determine the optimal solution for the chosen
sequences. \index{hybrid methods}{Hybrid algorithms} based on both CP and Mixed Integer
Programming (MIP) \cite{Schrijver03} can be used for this 
purpose \cite{BaptisteDemassey04}.

\end{itemize}
\index{heuristic!branching|)}

\subsection{The Use of Local Search}
\index{search!local|(}

Even when search can be simplified by looking for good sequences and
using dominance properties, search spaces for planning or scheduling
problems tend to be very large. In practice, it is often impossible to
explore a search space completely and guarantee the delivery of an
optimal solution. For an industrial planning or scheduling application
it however generally suffices to provide ``good'' solutions within
reasonable time. It is for such applications more important to be
robust with respect to variations in the problem instances like
variations in problem size, variations in numerical characteristics,
and addition of side constraints.  This is often achieved by mixing
constraint-based tree search with Local Search (LS) or by actually
implementing LS with constraints. Local search is taken as an
alternative way to explore the search space. Explored neighborhoods
vary a lot from an application to another, so it is difficult to
establish a general taxonomy of the approaches reported in the
literature. We will use two examples to convey the basic ideas.

Caseau and Laburthe \cite{CaseauLaburthe95} describe an
algorithm for the Job Shop Scheduling Problem which combines CP and
LS. The overall algorithm finds an approximate solution to start with,
makes local changes and repairs on it to quickly decrease the makespan
and, finally, performs an exhaustive search for decreasing
makespans. Given a schedule, a critical path is defined as a
sequence of activities where i) for each activity $A_i$ that appears before
activity $A_j$ in the sequence $A_i$ indeed precedes $A_j$ in the
schedule and ii) the sum of the processing times of the activities in
the sequence equals the makespan of the schedule.  Two types of local moves
are considered:
\begin{itemize}
\item
``Repair'' moves swap two activities scheduled on the same machine
to shrink or reduce the number of critical paths.
\item
``Shuffle'' moves \cite{ApplegateCook91} keep part of the solution and
search through the rest of the solution space to complete it.
Each shuffle move is implemented as
a constraint-based search algorithm with a limited number of backtracks
(typically 10, progressively increased to 100 or 1000), under the constraint
that the makespan of the solution must be improved (with a given
improvement step, typically 1\% of the makespan, progressively decreased
to one time unit).
\end{itemize}
Excellent computational results have been obtained with this approach
\cite{CaseauLaburthe95, CaseauEtAl01} as well as with other
constraint-based implementations of shuffle moves, as reported in
\cite{BaptisteLePape95a, NuijtenLePape98}.

In the same spirit, the best algorithm used by Le Pape and Baptiste
\cite{LePapeBaptiste99} for the \index{scheduling!preemptive}{Preemptive Job Shop Scheduling Problem}
relies on the combination of:
\begin{itemize}
\item
a strong constraint propagation algorithm (edge-finding);
\item
a local optimization operator called ``Jackson derivation'';
\item
limited discrepancy search \cite{HarveyGinsberg95} around the best schedule
found so far.
\end{itemize}
Limited discrepancy search is an alternative to depth-first search,
which relies on the assumption that a heuristic makes few mistakes
throughout the search.  Thus, considering the path from the root
node of the tree to the first solution found by a depth-first search
algorithm, there should be few ``wrong turns'' (\ie, few nodes which
were not immediately selected by the heuristic).  The basic idea is to
restrict the search to paths that do not diverge more than $w$ times
from the choices recommended by the heuristic.  Each time this limited
search fails to improve on the best current schedule, $w$ is
incremented and the process is iterated, until either a better
solution is found or it is proven that there is no better solution. It
is easy to prove that when $w$ gets large enough, limited discrepancy
search is complete. Yet it can be seen as a form of LS around the
recommendation of the heuristic.  On ten well-known problem instances,
each with 100 activities, experimental results show that each of the
three techniques mentioned above brings improvements in efficiency,
the average deviation to optimal solutions after 10 minutes of CPU
time falling from 13.72\% when none of these techniques is used to
0.23\% when they are all employed.

Globally, the integration of LS and CP is
promising whenever LS operators provide a good basis for the
exploration of the search space and either side constraints or effective
constraint propagation algorithms can be used to prune the search
space. The examples presented in the literature represent a
significant step toward the understanding of the possible combinations
of LS and CP.
Yet the definition of a general approach and methodology for
integrating LS and CP remains an important area of research.
\index{search!local|)}

\subsection{The Use of Mixed Integer Programming}

\index{linear programming!mixed integer programming|(}

In industrial applications, scheduling issues are often mixed with
resource allocation, capacity planning, or inventory management issues
for which MIP is a method of choice. Several examples have been
reported where a \index{hybrid methods}{hybrid} combination of CP and MIP was shown to be more
efficient than pure CP or MIP models (cf., for example,
\cite{RodosekWallace98, RodosekEtAl99, ElSakkoutWallace00, 
BaptisteDemassey04, Danna04, LePape04}). As in the case of local search,
there are many ways to combine CP and MIP, and we will just focus on two
particular examples.

A dynamic scheduling problem is solved in \cite{ElSakkoutWallace00}.
In this example, the linear solver includes only temporal constraints
(some of which have been added to the initial problem in order to
ensure the satisfaction of resource constraints) and the definition of
the optimization criterion as the total deviation of start times of
activities from the start times of the same activities in a reference
schedule. An interesting characteristic of this model is that the
optimal continuous solution of the linear sub-problem is guaranteed to
be integral; hence, either this solution satisfies all the resource
constraints and it is optimal, or it violates some resource constraint
which can be used to branch on the order of two conflicting
activities. CP is used to limit and select the explored branches.

\cite{Danna04} and \cite{LePape04} consider the case in which it is
not certain that an activity will use a given resource, either because
there are alternative resources, or because the activity can be left
unperformed against a certain cost (see
Section~\ref{SWNREFINED-MODEL-SEC}). We recall that an unperformed
activity will not require capacity, but will obey potential temporal
constraints, etc., and will also obey the calendar of the chosen
resource. \cite{Danna04, LePape04} do not consider resource calendars,
so an unperformed activity requires its normal processing time to be
completed. Note that to our knowledge no MIP approach exists that
handles resource calendars.

Also without resource calendars this problem is already challenging
for any optimization technique. MIP is a good candidate for
representing the cost function, but no good MIP model is known to
state that a resource can only perform one activity at a time.  CP
usually deals well with precedence and resource constraints, but
adding an upper bound on the optimization criterion does in general
not result in effective constraint propagation. LS operators
based on permuting activities are easy to design, but the impact of a
permutation on the total cost is hard to estimate.  In \cite{Danna04},
several cooperative optimization algorithms centered on a MIP model
have been proposed and compared with a pre-existing combination of CP
and LS:
\begin{itemize}
\item
The MIP algorithm relies on the default search strategy of CPLEX 9.0~\cite{Cplex90}.
\item
The IS+MIP algorithm consists in using CP to
construct an initial solution to the problem. This solution is then used
as a starting point for CPLEX.
\item
The IS+MIP+RINS algorithm is similar to IS+MIP but activates the
relaxation induced neighborhood search option of CPLEX \cite{DannaRothbergLePape05}.
Relaxation induced neighborhood search is a form of LS which relies
on the continuous relaxation to define a neighborhood of the current
solution: the integer variables that have the same values in the solution
of the continuous relaxation and in the best solution known so far are fixed to
these values and a sub-MIP on the remaining variables is solved
(with a limit on the number of nodes explored).
\item
The IS+MIP+RINS+GD algorithm adds the guided dives option of CPLEX
\cite{DannaRothbergLePape05} to the IS+MIP+RINS algorithm. When a variable
is selected for branching, the
``guided dives'' strategy will explore first the node in which this
variable is fixed to the value that it takes in the best solution known
so far.
\item
The IS+MIP+RINS+GD+MCORE algorithm adds to the IS+MIP+RINS+GD
algorithm another form of LS which defines a neighborhood
by heuristically reducing the values of ``big-M'' coefficients of the
MIP model.
\end{itemize}
These algorithms have been tested on 22 job shop instances from the
Manufacturing Scheduling Library (MaScLib)
\cite{NuijtenBousonville04}, with up to 260 activities. The results
have shown the interest of all the components that have been added to
the initial MIP algorithm.  They also show that on pure problems,
\index{hybrid methods}{hybrid algorithms} based on MIP can compete with state-of-the-art
techniques.

The generalization of these examples into a principled
approach is an important research issue for the forthcoming
years. In particular, MIP models are often
difficult to extend to the representation of additional constraints
such as \index{setup!time}{setup times} and \index{setup!cost}{costs}, \index{resource!calendar}{calendars}, etc.

\index{linear programming!mixed integer programming|)}

%%%%%%%%%%%%%%%%%%%%%%%%%%%%%%%%%%%%%%%%%%%%%%%%%%%%%%%%%%%%%%%%%%%%%
%%%%%%%%%%%%%%%%%%%%%%%%%%%%%%%%%%%%%%%%%%%%%%%%%%%%%%%%%%%%%%%%%%%%%


\section{Conclusions}\label{SConclusion}

In the introduction of this chapter we have seen that one of the key
factors of the success of \index{scheduling!constraint based}{Constraint-Based Scheduling} lies in the fact
that a powerful combination was found of the research fields of
\index{operations research}{Operations Research} (OR) and \index{artificial intelligence}{Artificial Intelligence} (AI). From OR its
efficient algorithms and its culture for searching for efficient
algorithms were used. From AI the general modeling and problem-solving
paradigm of CP and its culture for searching for natural ways of
modeling a problem in the needed real-life detail were used.

In this way Constraint-Based Scheduling preserves the general modeling
and problem-solving paradigm of CP while the integration of efficient
propagation algorithms improves the overall performance of the
approach. Efficient OR algorithms integrated in a CP approach allow
the user to benefit from the efficiency of OR techniques in a flexible
framework. Although the use of CP in planning is, due to the problem
complexity, less mature than its use in scheduling, Constraint-Based
Planning follows the same pattern as Constraint-Based Scheduling where
CP is used as a framework for integrating efficient special purpose
algorithms into a flexible and expressive paradigm. As in several
other areas of application, an important way to integrate efficient
algorithms in CP for scheduling and planning was found by
incorporating them inside \index{constraint!global}{global constraints}. Sections~\ref{SPropRct}
and~\ref{SPropOptCrit} pay attention to such constraint propagation.

Besides the powerful propagation, another strength was identified
namely the capacity to in a natural and flexible way model the
scheduling or planning problem at hand in the required real-life
detail. We want to stress that this capacity is becoming more and more
important. Indeed through the widespread adoption of ERP (Enterprise
Resource Planning) systems, more and more companies have access to the
data that allows them to capture the reality in the detail they
need. One of the reasons Advanced Planning and Scheduling systems
(APS's) are not as widely adopted as one would think following this
observation, is that these offerings often fail to model
reality in sufficient detail. This leads to the aforementioned
classical drawbacks of one being forced to discard degrees of
freedom and side constraints. It's especially on the side constraints
that APS's tend to be weak, thus leading to the system solving an
oversimplified problem resulting in producing impractical solutions
for the original problem. It is here that we believe
\index{planning!constraint based}{Constraint-Based Planning} and
\index{scheduling!constraint based}{Scheduling} have a great, largely
unused, potential.

Two other strengths identified in this chapter are i) a natural fit
of expressing scheduling specific heuristics using CP tree search, and
ii) a proven good potential of combining the CP approach with solution
techniques as Local Search, Large Neighborhood Search, and Mixed
Integer Programming. We have seen several examples of this in
Section~\ref{SHeuristics}. These strengths are thus about having the
flexibility in the approach to adapt the search such that the needed
performance to solve the problems is obtained. Although this has
indeed been a strength over the years, we want to stress that we
believe the field should pay increased attention to providing good
default search, \ie, a search procedure that works ``out-of-the-box''
at least for a certain class of problems. This is much like a lot of
the work done in the area of Mixed Integer Programming. That latter
work has led to a broadening of the audience that can use Mixed
Integer Programming to solve their problems. A similar effect should
be obtained for CP in general and \index{planning!constraint
based}{Constraint-Based Planning} and \index{scheduling!constraint
based}{Scheduling} in particular. This, combined with the natural way
of modeling problems present in CP, should open up CP for a much
broader use than today.

Another main research challenge is on doing planning and scheduling
under uncertainty. Uncertainty is inherent to planning and scheduling
environments and correctly dealing with it is of invaluable practical
importance. Two basic ways for dealing with uncertainty, together with
different combination of them, have been studied: reactive
(rescheduling) and proactive (robust scheduling). Lots of research has
been done starting many years ago but surprisingly few approaches
have been applied in practice. We feel the field is ripe to adopt
rescheduling and robust scheduling more broadly and believe CP can
play an important role there.

Further, more detailed, research challenges link back to the strengths
already mentioned. It remains a challenge to study industrial
properties in detail. Studies around breakable activities,
productivity profiles, continuous production and consumption,
unperformed activities, etc., are rare while there is a great need in
practice to correctly handle such properties. 

In Section~\ref{SPropOptCrit} constraint propagation methods related
to the minimization of the weighted number of late activities and to
the minimization of setup times and setup costs have been
presented. They drastically improve the behavior on problems involving
these criteria. However, there are many other interesting optimization
criteria. In particular, total tardiness is widely used in industry
but until now poor results are obtained on this problem. Constraint
propagation on such specific optimization criteria constitutes a very
interesting research area. Following the observation that users of
planning and scheduling applications often want to define
their own criteria, a possibly even more interesting research
challenge is to design generic lower-bounding techniques and
constraint propagation algorithms that could work for any criterion.

Finally, a research challenge in \index{planning!constraint based}{Constraint-Based Planning} is to still better
exploit the combination of AI and OR, \ie, to continue to follow the
same pattern as Constraint-Based Scheduling where CP is used as a
framework for integrating efficient special purpose algorithms into a
flexible and expressive paradigm. This will bring all the strengths of
Constraint-Based Scheduling mentioned in this chapter
to Constraint-Based Planning.

\index{scheduling|)}
\index{planning|)}

\bibliographystyle{plainnat}
\bibliography{sched}
%\bibliographystyle{plain}
%\bibliographystyle{abbrvnat}

\endinput

