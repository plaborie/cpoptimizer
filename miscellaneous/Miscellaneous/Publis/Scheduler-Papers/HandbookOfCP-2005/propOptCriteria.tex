%%%%%%%%%%%%%%%%%%%%%%%%%%%%%%%%%%%%%%%%%%%%%%%%%%%%%%%%%%%%%%%%%%%%%
%%%%%%%%%%%%%%%%%%%%%%%%%%%%%%%%%%%%%%%%%%%%%%%%%%%%%%%%%%%%%%%%%%%%%
\section{Propagation on Optimization Criteria}
(PhBap)

%%%%%%%%%%%%%%%%%%%%%%%%%%%%%%%%%%%%%%%%%%%%%%%%%%%%%%%%%%%%%%%%%%%%%
%%%%%%%%%%%%%%%%%%%%%%%%%%%%%%%%%%%%%%%%%%%%%%%%%%%%%%%%%%%%%%%%%%%%%
SET UP TIMES/COSTS, NUMBER OF LATE JOBS, EARLY/TARDY, ... ????
DO PAPER.

Much more complex.... 

In our model, a variable $\critvar$ represents the value taken by the
objective function.
\begin{equation} \label{OBJ-CT-BIS}
\critvar =  F(\evar(A_1), \ldots, \evar(A_n))
\end{equation}
Considering the objective constraint and the resource constraints
independently is not a problem when the objective function is a ``maximum'' manong sevral terms (\eg, 
maximum completion time or the maximum tardiness). 
Indeed, the upper
bound on $\critvar$ is directly propagated on the completion time of
each activity, \ie, deadlines are efficiently tightened. The situation
is much more complex for sum functions such as weighted sum of completion times 
$\sum w_i C_i$, weighted tardiness $\sum w_i
T_i$ or weighted number of late jobs  $\sum w_i U_i$. For these functions, the constraint
(\ref{OBJ-CT-BIS}) has to be taken into account at each step of the
search tree. An efficient constraint propagation technique must
consider the resource constraints and the objective constraint
simultaneously. 

We describe one example of what can be achieved on a simple the $\sum
w_i U_i$ criterion. XXXX ??? ON A SINGLE MACHINE ::::


%=====================================================================
\subsection{A Lower Bound of the Number of Late Activities}
%=====================================================================
Obtaining a good lower bound of the weighted number of late activities is
the first step of the constraint propagation process. This lower bound
is also a lower bound for the domain of the variable
$\critvar$. Relaxing non-preemption is a well-known technique to
obtain good lower bounds. Unfortunately, the preemptive problems
remain NP=-Hard. A ``relaxed
preemptive lower bound'', \ie, a slightly stronger relaxation than the
preemptive relaxation, can be used. As shown below, it can be computed
in $O(n^2 \log n)$ and is valid for all type of resource constraints
(either with one or several machines).

Let us recall a well-known result for the One-Machine Problem. Its
preemptive relaxation is polynomial and has some very interesting
properties: There exists a feasible preemptive schedule if and only if over any
interval $[t_1, t_2)$, the sum of the processing times of the
activities in $\{A_i : [t_1 \le \estm_i] \wedge [\letm_i \le t_2]\}$ is lower
than or equal to $t_2 - t_1$. It is well known that relevant values
for $t_1$ and $t_2$ are respectively the release dates and the
deadlines \cite{Carlier84}.

We introduce a decision variable $x_i$ per activity that equals $1$
when  the activity is on-time and 0 otherwise. Notice that if $\letm_i \le
\delta_i$, $A_i$ is on-time in any solution, \ie, $x_i = 1$. In such a
case we adjust the value of $\delta_i$ to $\letm_i$ (this has no
impact on solutions) so that due dates are always smaller than or
equal to deadlines. We also assume that there is a preemptive schedule
that meets all deadlines (if not, the resource constraint does
not hold and a backtrack occurs). The following Mixed Integer Program (MIP)
computes the minimum weighted number of late activities in the
preemptive case:
\begin{equation} \label{P-EQ}
\begin{array}{l}
\begin{displaystyle} \min \sum_1^n w_i(1-x_i) \end{displaystyle} \\
\text{u.c.~}
\left\{
\begin{array}{l}
\begin{displaystyle}
\forall t_1, \forall t_2 > t_1, \sum_{S(t_1,t_2)} p_i
+\sum_{P(t_1,t_2)} p_i x_i \le t_2 - t_1 \end{displaystyle} \\
\forall i \in \{1, \ldots, n\}, x_i \in \{0,1\}
\end{array}
\right.
\end{array}
\end{equation}
where $S(t_1, t_2)$ is the set of activities that are sure to execute
between $t_1$ and $t_2$ and where $P(t_1, t_2)$ is the set of
activities that are preferred to execute between $t_1$ and $t_2$.
\begin{eqnarray*} \label{S-AND-P-DEF-EQ}
S(t_1,t_2) &=& \{A_i: \estm_i \ge t_1 \wedge \letm_i \le t_2\} \\
P(t_1,t_2) &=& \{A_i: \estm_i \ge t_1 \wedge \letm_i > t_2\ \wedge \delta_i \le t_2\}
\end{eqnarray*}
Actually, it easy to see that the relevant values of $t_1$ and $t_2$
correspond respectively to 
\begin{itemize}
\item
the release dates,
\item
the due dates and the deadlines.
\end{itemize}
Hence, there are $O(n^2)$ constraints in the MIP. In the following,
the notation $(t_1, t_2)$ refers to the resource constraint over the
interval $[t_1, t_2)$. We now focus on the {\bf continuous relaxation}
of (\ref{P-EQ}) in which the following constraints are added : For any
activity $A_i$ such that $\estm_i + p_i > \delta_i$, \ie, for any late
activity, $x_i = 0$.
\begin{equation} \label{CP-EQ}
\begin{array}{l}
\begin{displaystyle} \min \sum_1^n w_i(1-x_i) \end{displaystyle} \\
\text{u.c.~}
\left\{
\begin{array}{l}
\forall t_1 \in \{\estm_i\}, 
\forall t_2 \in \{\letm_i\} \cup \{\delta_i\} \;\; (t_2 > t_1) \\
\begin{displaystyle}
\; \; \;  \sum_{S(t_1,t_2)} p_i +\sum_{P(t_1,t_2)} p_i x_i \le t_2 - t_1
\end{displaystyle} \\
\forall i, \estm_i + p_i > \delta_i \Rightarrow x_i=0 \\
\forall i \in \{1, \ldots, n\}, x_i \in [0,1]
\end{array}
\right.
\end{array}
\end{equation}
(\ref{CP-EQ}) can be solved in $O(n^2 \log n)$ steps. XXX


%=====================================================================
\subsection{Constraint Propagation} \label{CP-LATE-JOBS}
%=====================================================================
In this section, some deduction rules are presented. They determine
that some activities can, must or cannot end before their due date. Consider an
activity $A_u$ such that $\eetm_u \le \delta_u < \letm_u$; it can be
either late or on-time. Our objective is to compute efficiently a
lower bound of the weighted number of late activities if $A_u$ is
on-time (conversely if $A_u$ is late). If this lower bound is greater
than the maximal value in the domain of $\critvar$ then $A_u$ must be
late (conversely on-time). Algorithm \ref{N2LOGN-VIOLATION-ALGO} could
be used to compute such a lower bound. However, it would be called $n$
times, leading to a high overall complexity of $O(n^3 \log n)$. We
propose to use a slightly weaker lower bound that can be computed,
after some preprocessing, in linear time, for each activity. We will
see that the overall domain reduction scheme runs in $O(n^2)$.

First, {\em deadlines are relaxed} to a very large value (except for
the activities that have to be on-time), \ie, late activities can be scheduled
arbitrarily late. Once this is done, we compute the optimum $X$ of the
linear program obtained with the relaxed deadlines (the instance is
well-structured and so we can use Algorithm
\ref{N2LOGN-VIOLATION-ALGO}). It is easy to see that this program can
be rewritten as follows:
\begin{equation} \label{RCP-EQ}
\begin{array}{l}
\begin{displaystyle} \min \sum_1^n w_i(1-x_i) \end{displaystyle} \\
\text{u.c.~}
\left\{
\begin{array}{l}
\begin{displaystyle}
\forall \estm_j, \forall \delta_k > \estm_j, 
\sum_{A_i \in P(\estm_j, \delta_k)} p_i x_i \le \delta_k - \estm_j
\end{displaystyle} \\
\forall i, \estm_i + p_i > \delta_i \Rightarrow x_i=0 \\
\forall i, \letm_i \le \delta_i \Rightarrow x_i=1 \\
\forall i \in \{1, \ldots, n\}, x_i \in [0,1] \\
\end{array}
\right.
\end{array}
\end{equation}
The second and the third constraint simply take into account the fact
that some activities are known to be late (or on-time). Of course, the lower
bound provided by (\ref{RCP-EQ}) is weaker than the one provided by 
(\ref{CP-EQ}) (because deadlines have been relaxed). However, the
computation of $X$ is the basis of our domain reduction scheme. 
%'''''''''''''''''''''''''''''''''''''''''''''''''''''''''''''''''''''''
%'''''''''''''''''''''''''''''''''''''''''''''''''''''''''''''''''''''''
\subsubsection{Late Activity Detection}
%'''''''''''''''''''''''''''''''''''''''''''''''''''''''''''''''''''''''
%'''''''''''''''''''''''''''''''''''''''''''''''''''''''''''''''''''''''

Let (\ref{RCPo-EQ}) be the linear program (\ref{RCP-EQ}) to which the
constraint $x_u = 1$ has been added. 
\begin{equation} \label{RCPo-EQ}
\begin{array}{l}
\begin{displaystyle} \min \sum_1^n w_i(1-x_i) \end{displaystyle} \\
\text{u.c.~}
\left\{
\begin{array}{l}
\begin{displaystyle}
\forall \estm_j, \forall \delta_k > \estm_j, 
\sum_{A_i \in P(\estm_j, \delta_k)} p_i x_i \le \delta_k - \estm_j
\end{displaystyle} \\
\forall i, \estm_i + p_i > \delta_i \Rightarrow x_i=0 \\
\forall i, \letm_i \le \delta_i \Rightarrow x_i=1 \\
\forall i \in \{1, \ldots, n\}, x_i \in [0,1] \\
x_u = 1
\end{array}
\right.
\end{array}
\end{equation}
Assume that there is a feasible solution of (\ref{RCPo-EQ}) and let $\Xo$
be the optimal vector of (\ref{RCPo-EQ}) obtained by Algorithm 
\ref{N4-VIOLATION-ALGO}. Propositions \ref{SUMPIXOI-PROP} and
\ref{XOI-PROP} exhibit two relations that $X$ and $\Xo$ 
satisfy. These relations are used to compute a lower bound of the
weighted number of late activities.  
\begin{lemma} \label{SUMPIXOI-PROP}
$\sum p_i \Xo_i  \le  \sum p_i X_i$
\end{lemma}
\begin{lemma} \label{XOI-PROP}
$\forall i \neq  u, \Xo_i  \le  X_i$
\end{lemma}

\newpage
Thanks to Lemmas \ref{SUMPIXOI-PROP} and \ref{XOI-PROP}, we can
add constraints $\sum p_i \Xo_i  \le  \sum p_i X_i$ and $\forall i
\neq  u, x_i  \le  X_i$ to the linear program (\ref{RCPo-EQ}). Since we
are interested in a lower bound of (\ref{RCPo-EQ}), we can also relax
the resource constraints. As a consequence, we seek to solve the
following program (\ref{RRCPo-EQ}) that is solved in linear time by
Algorithm \ref{LATE-JOB-DETECT-ALGO}.
\begin{equation} \label{RRCPo-EQ}
\begin{array}{l}
\begin{displaystyle} \min \sum_1^n w_i(1-x_i) \end{displaystyle} \\
\text{u.c.~}
\left\{
\begin{array}{l}
\sum p_i x_i  \le  \sum p_i X_i \\
\forall i \neq u, x_i \le X_i \\
x_u = 1 \\
\forall i \in \{1, \ldots, n\}, x_i \in [0,1] \\
\end{array}
\right.
\end{array}
\end{equation}
\begin{center}

\end{center}
\begin{algorithm}
\caption{A linear time algorithm for late activity detection
\label{LATE-JOB-DETECT-ALGO}}
\begin{algorithmic}[1]
\FOR{$i \la 1$ to $n$}
  \STATE{$\Xo_i \la 0.0$}
\ENDFOR
\STATE $\Xo_u \la 1$
\STATE $\maxval \la \sum p_i X_i - p_u$
\FOR{$i \la 1$ to $n$ and $i \neq u$}
  \STATE{$\Xo_i \la \min(X_i, \maxval / p_i)$}
  \STATE{$\maxval = \maxval - p_i * \Xo_i$}
\ENDFOR
\end{algorithmic}
\end{algorithm}

\subsubsection{On-Time Activity Detection}

Let $A_u$ be an activity such that $\eetm_u \le \delta_u < \letm_u$.
$A_u$ can be either late or on-time. We want to compute a lower bound of the
number of late activities if it is late. Let $\Xl$ be the optimal vector
of the linear program (\ref{RCP-EQ}) to which the constraint $x_u = 0$ has
been added. We claim that $\sum  p_i \Xl_i  \le  \sum  p_i X_i$ and that
$\forall  i \neq  u, \Xl_i \ge X_i$ (proofs are similar to the proofs of
Lemmas \ref{SUMPIXOI-PROP} and \ref{XOI-PROP}). The same
mechanism than for the late activity detection then applies: The new
constraints are entered in the linear program while the resource
constraints are removed. The resulting linear program can be also
solved in linear time.

\subsubsection{Example}

Consider the instance of Figure
\ref{N2LOGN-VIOLATION-ALGO-ILLUSTRATION}. Deadlines are relaxed. The
optimal vector of (\ref{RCP-EQ}) is $X_1 = 1, X_2 = 5/7, X_3 = 3 /5, X_4 
= 0$. The lower  bound is then $(0 * 10) / 2 + (2 * 30) / 7 + (2 * 20)
/ 5 + (3 *  5) / 3 = 21.57$ Assume that the domain of $\critvar$ is
$[22, 26]$. Let us try to put $A_4$ on-time. The optimum of (\ref{RRCPo-EQ})
is $\Xo_1 = 1, \Xo_2 = 5/7, \Xo_3 = 0, \Xo_4 = 1$. Thus, the new lower
bound is $(0 * 10) / 2 + (2 * 30) / 7 + (5 * 20) / 5 + (0 *  5) / 3 =
28.57$. Given the domain of $\critvar$, we deduce that $A_4$ must be
late. 

