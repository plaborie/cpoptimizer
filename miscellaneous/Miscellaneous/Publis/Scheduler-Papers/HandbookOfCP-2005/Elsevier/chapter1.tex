\tracingcommands=2\tracingmacros=2
\chaptertitleauthors[This is the first chapter]
  {This is the first chapter and we use a long title for it}
  {Simon Pepping, Andy Deelen, Joyce Happee, Betsy Lightfoot, Hans
  Oosterom and George Burlage}
\authormark{Simon Pepping et al.}

This is the text of the first chapter.

\section{First section}

We start a new section. This section allows us to see the
effect of a section.
\tracingcommands=0\tracingmacros=0

\begin{eqnarray}
A&=&\sum_{n=0}^{\infty}A_n(n^2+1)\,.
\end{eqnarray}

\subsection{First subsection}

Some text.

\begin{equation}
A=\sum_{n=0}^{\infty}A_n(n^2+1)\,.
\end{equation}

\subsubsection{First subsubsection}

Some text.

\begin{displaymath}
A=\sum_{n=0}^{\infty}A_n(n^2+1)\,.
\end{displaymath}

\paragraph{First paragraph}

Some text.

$$
A=\sum_{n=0}^{\infty}A_n(n^2+1)\,.
$$

Some text.

\[
A=\sum_{n=0}^{\infty}A_n(n^2+1)\,.
\]

\section{This is the second section and again we use a long title for it}
\sectionmark{This is the second section}

One of the earliest successes of an ``abstract'' approach to practical
problems was in the discussion of the linear equation $Lf = g$ where
$L$ is an $n\times n$ matrix and $f$ and $g$ are $n$-vectors. By
regarding $L$ as a linear mapping on an $n$-dimensional vector space
reference to individual variables is avoided, and this leads to
simplification and conceptual clarification. However, the equations
which arise in applications are for the most part either differen-
tial or integral equations and cannot usually be reduced to the above
finite dimensional form; it is one of the principal tasks of
functional analysis to uncover the analogous simple algebraic
structure in these more difficult situations \citep{bib:1}.

    There are {\bfseries three} \textbf{major} divisions in a
    book\footnote{A book}: 
the front matter\index{front matter} or preliminaries\index{preliminaries}, 
the main matter\index{main matter} or text, 
and the back matter\index{back matter} or references. 
The main differences as
far as appearance goes is that in the front matter the folios\index{folio} are 
expressed as roman numerals and sectional divisions are not numbered. The 
folios\index{folio} are expressed as arabic numerals in the main and back matter. Sectional
divisions are numbered in the main matter but not in the back matter.
    See \citet{bib:2}.

    The front matter\index{front matter}\footnote{Front matter} consists of such elements as the title
of the book, a table of contents, and similar items. The first few pages
in the front matter are not paginated\index{pagination} and so do not have folios\index{folio}. The remainder
are paginated and the folios\index{folio} are usually expressed as roman numerals. Not all
books have all the elements described below \citep[e.g.][Ch. 2]{bib:3}.

    The first page is a recto \emph{half title}\index{half title page} 
page with no folio\index{folio}. 
The page is very simple and displays just the main title of the book --- 
no subtitle, author, or other information. One purported purpose of this
page is to protect the main title page.

    The first verso page, the back of the half-title page, may contain the 
series title, if the book is one in a series, a list of contributors, 
a frontispiece, or may be blank. The series title may instead be put on the 
half-title page or on the copyright page.

   The \emph{title page}\index{title page} is recto and contains the full 
title of the work, the names of the author(s) or editor(s), and often at the
bottom of the page the name of the publisher, together with the publisher's 
logo if it has one.

    The title page(s) may be laid out in a simple manner or can have various
fol-de-rols, depending on the impression the designer wants to give. In
any event the style of this page should give an indication of the style
used in the main body of the work.

    The verso of the title page is the copyright page\index{copyright page}.
This contains the copyright notice, the publishing/printing history, 
the country where printed, ISBN and/or CIP information. The page is usually 
typeset in a smaller font\index{font!change} than the normal text.

    Following the copyright page may come a dedication or an epigraph\index{epigraph}, 
on a recto page, with the following verso page blank.

    This essentially completes the unpaginated pages.

    The headings\index{heading} and textual forms for the paginated 
pages should be the same as those for the main matter, except that 
headings\index{heading} are usually unnumbered.

\def\toc{ToC}
\def\lof{LoF}
\def\lot{LoT}

    The first paginated page,
usually with roman numerals (e.g., this is folio i),
is recto with the Table of Contents (\toc). If the book contains 
figures\index{figure} (illustrations\index{illustration}) 
and/or tables\index{table}, the List of Figures (\lof) and/or List of Tables (\lot) come 
after the \toc, with no blank pages separating them. The \toc{} should contain
an entry for each following major element. If there is a \lot, say, this 
should be listed in the \toc. The main chapters\index{chapter} must be listed, of course, and
so should elements like a preface\index{preface}, bibliography\index{bibliography} or an index\index{index}.

    There may be a foreword\index{foreword} after the listings, with no blank
separator. A foreword is usually written by someone other than the author, 
preferably an eminent person, and is signed by the writer. The writer's
signature is often typeset in small caps after the end of the piece.

   A preface\index{preface} is normally written by the author, in which he
includes reasons why he wrote the work in the first place, and perhaps to 
provide some more personal comments than would be justified in the body. 
A preface starts on the page immediately following a foreword, or the lists.

   If any acknowledgements are required that have not already appeared in the
preface, these may come next in sequence.

   Following may be an introduction if this is not part of the main text. 
The last elements in the front material may be a list of abbreviations, list
of symbols, a chronology of events, a family tree, or other information of
a like sort depending on the particular work.

    There are {\bfseries three} \textbf{major} divisions in a book: 
the front matter\index{front matter} or preliminaries\index{preliminaries}, 
the main matter\index{main matter} or text, 
and the back matter\index{back matter} or references. 
The main differences as
far as appearance goes is that in the front matter the folios\index{folio} are 
expressed as roman numerals and sectional divisions are not numbered. The 
folios\index{folio} are expressed as arabic numerals in the main and back matter. Sectional
divisions are numbered in the main matter but not in the back matter.

    The front matter\index{front matter} consists of such elements as the title
of the book, a table of contents, and similar items. The first few pages
in the front matter are not paginated\index{pagination} and so do not have folios\index{folio}. The remainder
are paginated and the folios\index{folio} are usually expressed as roman numerals. Not all
books have all the elements described below.

    The first page is a recto \emph{half title}\index{half title page} 
page with no folio\index{folio}. 
The page is very simple and displays just the main title of the book --- 
no subtitle, author, or other information. One purported purpose of this
page is to protect the main title page.

    The first verso page, the back of the half-title page, may contain the 
series title, if the book is one in a series, a list of contributors, 
a frontispiece, or may be blank. The series title may instead be put on the 
half-title page or on the copyright page.

   The \emph{title page}\index{title page} is recto and contains the full 
title of the work, the names of the author(s) or editor(s), and often at the
bottom of the page the name of the publisher, together with the publisher's 
logo if it has one.

    The title page(s) may be laid out in a simple manner or can have various
fol-de-rols, depending on the impression the designer wants to give. In
any event the style of this page should give an indication of the style
used in the main body of the work.

    The verso of the title page is the copyright page\index{copyright page}.
This contains the copyright notice, the publishing/printing history, 
the country where printed, ISBN and/or CIP information. The page is usually 
typeset in a smaller font\index{font!change} than the normal text.

    Following the copyright page may come a dedication or an epigraph\index{epigraph}, 
on a recto page, with the following verso page blank.

    This essentially completes the unpaginated pages.

    The headings\index{heading} and textual forms for the paginated 
pages should be the same as those for the main matter, except that 
headings\index{heading} are usually unnumbered.

\def\toc{ToC}
\def\lof{LoF}
\def\lot{LoT}

    The first paginated page,
usually with roman numerals (e.g., this is folio i),
is recto with the Table of Contents (\toc). If the book contains 
figures\index{figure} (illustrations\index{illustration}) 
and/or tables\index{table}, the List of Figures (\lof) and/or List of Tables (\lot) come 
after the \toc, with no blank pages separating them. The \toc{} should contain
an entry for each following major element. If there is a \lot, say, this 
should be listed in the \toc. The main chapters\index{chapter} must be listed, of course, and
so should elements like a preface\index{preface}, bibliography\index{bibliography} or an index\index{index}.

    There may be a foreword\index{foreword} after the listings, with no blank
separator. A foreword is usually written by someone other than the author, 
preferably an eminent person, and is signed by the writer. The writer's
signature is often typeset in small caps after the end of the piece.

   A preface\index{preface} is normally written by the author, in which he
includes reasons why he wrote the work in the first place, and perhaps to 
provide some more personal comments than would be justified in the body. 
A preface starts on the page immediately following a foreword, or the lists.

   If any acknowledgements are required that have not already appeared in the
preface, these may come next in sequence.

   Following may be an introduction if this is not part of the main text. 
The last elements in the front material may be a list of abbreviations, list
of symbols, a chronology of events, a family tree, or other information of
a like sort depending on the particular work.

    Table~\ref{tab:front} summarises the potential elements in the front
matter.

\begin{table}
\centering
\caption{Front matter}\label{tab:front}
\begin{tabular}{llcc} \hline
Element                      & Page  & Paginated & Leaf \\ \hline
Half-title page              & recto & no        & 1 \\
Frontispiece, etc., or blank & verso & no        & 1 \\
Title page                   & recto & no        & 2 \\
Copyright page               & verso & no        & 2 \\
Dedication                   & recto & no        & 3 \\
Blank                        & verso & no        & 3 \\
Table of Contents            & recto & yes       & 3 or 4 \\
List of Figures     & recto or verso & yes       & 3 or 4 \\
List of Tables      & recto or verso & yes       & etc. \\
Foreword            & recto or verso & yes       & etc. \\
Preface             & recto or verso & yes       & etc. \\
Acknowledgements    & recto or verso & yes       & etc. \\
Introduction        & recto or verso & yes       & etc. \\
Abbreviations, etc  & recto or verso & yes       & etc. \\
\hline
\end{tabular}
\end{table}

    Note that the titles Foreword, Preface and Introduction are somewhat
interchangeable. In some books the title Introduction may be used for what
is described here as the preface, and similar changes may be made among the 
other terms and titles in other books. 

\section{This is another section}

Text

\section{This is another section}

Text

\section{This is another section}

Text

\section{This is another section}

Text

\section{This is another section}

Text

\section{This is another section}

Text

\section{This is another section}

Text

\section{This is another section}

Text

\section{This is another section}

Text

% \section*{Problems}
% \sectionmarknonum{Problems}
% \addcontentsline{toc}{section}{Problems}
% \begin{small}
\begin{Problems}

This would not be an academic book if it let you just read
on. Therefore we present some problems at the end of each chapter,
solving which will exercise your understanding, and challenge you to
think the theory through thoroughly. We are sure your will enjoy these
sections.

This would not be an academic book if it let you just read
on. Therefore we present some problems at the end of each chapter,
solving which will exercise your understanding, and challenge you to
think the theory through thoroughly. We are sure your will enjoy these
sections.

This would not be an academic book if it let you just read
on. Therefore we present some problems at the end of each chapter,
solving which will exercise your understanding, and challenge you to
think the theory through thoroughly. We are sure your will enjoy these
sections.

This would not be an academic book if it let you just read
on. Therefore we present some problems at the end of each chapter,
solving which will exercise your understanding, and challenge you to
think the theory through thoroughly. We are sure your will enjoy these
sections.

This would not be an academic book if it let you just read
on. Therefore we present some problems at the end of each chapter,
solving which will exercise your understanding, and challenge you to
think the theory through thoroughly. We are sure your will enjoy these
sections.

This would not be an academic book if it let you just read
on. Therefore we present some problems at the end of each chapter,
solving which will exercise your understanding, and challenge you to
think the theory through thoroughly. We are sure your will enjoy these
sections.

This would not be an academic book if it let you just read
on. Therefore we present some problems at the end of each chapter,
solving which will exercise your understanding, and challenge you to
think the theory through thoroughly. We are sure your will enjoy these
sections.

This would not be an academic book if it let you just read
on. Therefore we present some problems at the end of each chapter,
solving which will exercise your understanding, and challenge you to
think the theory through thoroughly. We are sure your will enjoy these
sections.

This would not be an academic book if it let you just read
on. Therefore we present some problems at the end of each chapter,
solving which will exercise your understanding, and challenge you to
think the theory through thoroughly. We are sure your will enjoy these
sections.

This would not be an academic book if it let you just read
on. Therefore we present some problems at the end of each chapter,
solving which will exercise your understanding, and challenge you to
think the theory through thoroughly. We are sure your will enjoy these
sections.

This would not be an academic book if it let you just read
on. Therefore we present some problems at the end of each chapter,
solving which will exercise your understanding, and challenge you to
think the theory through thoroughly. We are sure your will enjoy these
sections.

This would not be an academic book if it let you just read
on. Therefore we present some problems at the end of each chapter,
solving which will exercise your understanding, and challenge you to
think the theory through thoroughly. We are sure your will enjoy these
sections.
\end{Problems}

Some more text.

\begin{thebibliography}{}

% In numerical style the labels may be omitted; they have no effect
\bibitem[Pepping(2001)]{bib:1}
Pepping, S., 2001, Reference 1.
\bibitem[Pepping(2002)]{bib:2}
Pepping, S., 2002, Reference 2.
\bibitem[Pepping(2003)]{bib:3}
Pepping, S., 2003, Reference 3.
\bibitem[Pepping(2004)]{bib:4}
Pepping, S., 2004, Reference 4.

\end{thebibliography}

\endinput
