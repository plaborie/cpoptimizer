\chaptertitleauthors{Third chapter}{Simon Pepping}

This is the third chapter.

\section{First section}

Again we start a new section. This section allows us to see more of
the effect of a section.

\section{Second section}

Another section, to fill this chapter.

\begin{thm}
  This is a theorem.
\end{thm}

\begin{proof}
  This is the proof.
\end{proof}

\begin{defn}
  This is a definition.
\end{defn}

\begin{table}[htbp]
  \begin{center}
    \caption{A table}
    \label{tab:2}
    \begin{tabular}{ccc}
      \hline
      a&b&c\\
      \hline
      A&B&C\\
      D&E&F\\
      \hline
    \end{tabular}
  \end{center}
\end{table}

\begin{figure}[htbp]
  \begin{center}
    \vbox to 3pc{\vfill A figure\vfill}
    \caption{A figure}
    \label{fig:2}
  \end{center}
\end{figure}

    There are three major divisions in a book: 
the front matter\index{front matter} or preliminaries\index{preliminaries}, 
the main matter\index{main matter} or text, 
and the back matter\index{back matter} or references. 
The main differences as
far as appearance goes is that in the front matter the folios\index{folio} are 
expressed as roman numerals and sectional divisions are not numbered. The 
folios\index{folio} are expressed as arabic numerals in the main and back matter. Sectional
divisions are numbered in the main matter but not in the back matter.

    The front matter\index{front matter} consists of such elements as the title
of the book, a table of contents, and similar items. The first few pages
in the front matter are not paginated\index{pagination} and so do not have folios\index{folio}. The remainder
are paginated and the folios\index{folio} are usually expressed as roman numerals. Not all
books have all the elements described below.

    The first page is a recto \emph{half title}\index{half title page} 
page with no folio\index{folio}. 
The page is very simple and displays just the main title of the book --- 
no subtitle, author, or other information. One purported purpose of this
page is to protect the main title page.

    The first verso page, the back of the half-title page, may contain the 
series title, if the book is one in a series, a list of contributors, 
a frontispiece, or may be blank. The series title may instead be put on the 
half-title page or on the copyright page.

   The \emph{title page}\index{title page} is recto and contains the full 
title of the work, the names of the author(s) or editor(s), and often at the
bottom of the page the name of the publisher, together with the publisher's 
logo if it has one.

    The title page(s) may be laid out in a simple manner or can have various
fol-de-rols, depending on the impression the designer wants to give. In
any event the style of this page should give an indication of the style
used in the main body of the work.

    The verso of the title page is the copyright page\index{copyright page}.
This contains the copyright notice, the publishing/printing history, 
the country where printed, ISBN and/or CIP information. The page is usually 
typeset in a smaller font\index{font!change} than the normal text.

    Following the copyright page may come a dedication or an epigraph\index{epigraph}, 
on a recto page, with the following verso page blank.

    This essentially completes the unpaginated pages.
    
\endinput
