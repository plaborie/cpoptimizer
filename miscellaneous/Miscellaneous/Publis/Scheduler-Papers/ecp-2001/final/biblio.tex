
\begin{thebibliography}{}
%

\bibitem[1]{baptiste95}
P. Baptiste and C. LePape.
\newblock 1995.
\newblock A {T}heoretical and {E}xperimental {C}omparison of {C}onstraint
  {P}ropagation {T}echniques for {D}isjunctive {S}cheduling.
\newblock In {\em Proceedings of the Fourteenth International Joint Conference
on Artificial Intelligence}.

\bibitem[2]{beck97}
C. Beck, A. Davenport, E. Sitarski and M. Fox.
\newblock 1997.
\newblock Texture-based {H}euristics for {S}cheduling {R}evisited.
\newblock In {\em Proceedings AAAI-97}.

\bibitem[3]{carlier90}
J. Carlier and E. Pinson.
\newblock 1990.
\newblock A {P}ractical {U}se of {J}ackson's {P}reemptive {S}chedule for
  {S}olving the {J}ob-{S}hop {P}roblem.
\newblock {\em Annal of Operation Research} 26:269--287.

\bibitem[4]{cesta2000}
A. Cesta, A. Oddi and S. Smith. 
\newblock 2000.
\newblock A Constraint-Based Method for Project Scheduling with Time windows.
\newblock CMU RI Technical Report, February, 2000.

\bibitem[5]{cesta97}
A. Cesta and C. Stella.
\newblock 1997.
\newblock A Time and Resource Problem for Planning Architectures.
\newblock In {\em Proc. ECP-97}, pp. 117-129.

\bibitem[6]{chapman87}
D. Chapman.
\newblock 1987.
\newblock Planning for conjunctive goals.
\newblock {\em Artificial Intelligence} 32:333--377.

\bibitem[7]{drabble94}
B. Drabble and A. Tate.
\newblock 1994.
\newblock The use of optimistic and pessimistic resource profiles to inform
  search in an activity based planner.
\newblock In {\em Second International Conference on Artificial Intelligence
  Planning Systems},  243--248.

\bibitem[8]{erschler91}
J. Erschler, P. Lopez and C. Thuriot.
\newblock 1991.
\newblock Raisonnement temporel sous contraintes de ressources et probl\`emes
  d'ordonnancement.
\newblock {\em Revue d'Intelligence Artificielle} 5(3):7--32.

\bibitem[9]{erschler76}
J. Erschler.
\newblock 1976.
\newblock {\em Analyse sous contraintes et aide \`a la d\'ecision pour certains
  probl\`emes d'ordonnancement}.
\newblock Ph.D. Dissertation, Universit\'e Paul Sabatier.

\bibitem[10]{focacci00}
F. Focacci, P. Laborie and W. Nuijten.
\newblock 2000.
\newblock Solving scheduling problems with setup times and alternative
  resources.
\newblock In {\em Fifth International Conference on Artificial Intelligence
  Planning and Scheduling},  92--101.

\bibitem[11]{garcia96}
F. Garcia and P. Laborie.
\newblock 1996.
\newblock {\em New Directions in AI Planning}.
\newblock IOS Press, Amsterdam.
\newblock chapter Hierarchisation of the Search Space in Temporal Planning,
  217--232.

\bibitem[12]{scheduler51}
ILOG.
\newblock 2001.
\newblock {I}{L}{O}{G} {S}cheduler 5.1 {R}eference {M}anual.
\newblock http://www.ilog.com/.

\bibitem[13]{khambhampati96}
S. Khambhampati and X. Yang.
\newblock 1996.
\newblock On the role of disjunctive representations and constraint propagation
  in refinement planning.
\newblock In {\em KR96}.

\bibitem[14]{laborie95}
P. Laborie and M. Ghallab.
\newblock 1995.
\newblock Planning with sharable resource constraints.
\newblock In {\em Fourteenth IJCAI},  1643--1649.

\bibitem[15]{neumann99}
K. Neumann and C. Schwindt.
\newblock 1999.
\newblock Project scheduling with inventory constraints.
\newblock Technical Report WIOR-572, Institut f\"ur Wirtschaftstheorie und
  Operations Research. Universit\"at Karlsruhe.

\bibitem[16]{nuijten94}
W. Nuijten.
\newblock 1994.
\newblock {\em Time and Resource Constrained Scheduling: A Constraint
  Satisfaction Approach}.
\newblock Ph.D. Dissertation, Eindhoven University of Technology.

\bibitem[17]{pacciarelli99}
D. Pacciarelli and A. Mascis.
\newblock 1999.
\newblock Job-shop scheduling of perishable items.
\newblock In {\em INFORMS'99}.

\bibitem[18]{smith00}
D. Smith, J. Frank and A. Jonsson.
\newblock 2000.
\newblock Bridging the gap between planning and scheduling.
\newblock {\em Knowledge Engineering Review} 15(1).

\bibitem[19]{sourd00}
F. Sourd and W. Nuijten.
\newblock 2000.
\newblock Multiple-machine lower bounds for shop scheduling problems.
\newblock {\em INFORMS Journal of Computing} 4(12):341--352.


\end{thebibliography}
