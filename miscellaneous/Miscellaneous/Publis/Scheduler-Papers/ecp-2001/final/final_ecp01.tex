\documentclass{llncs}
\usepackage{epsfig}
\usepackage{color}
\linespread{0.9524}
\usepackage{makeidx}
\begin{document}
\title{Algorithms for Propagating Resource Constraints
       in AI Planning and Scheduling: Existing Approaches and New Results}

\author{Philippe Laborie}
\institute{ILOG, 9, rue de Verdun, BP 85,\\
	F-94253 Gentilly Cedex, France}	
\titlerunning{Algorithms for Propagating Resource Constraints} 
\authorrunning{Philippe Laborie} 


\maketitle 

\begin{abstract} This paper summarizes the main existing approaches to
propagate resource   constraints in  Constraint-Based   scheduling and
identifies some  of their limitations for using  them in an integrated
planning and scheduling framework. We then describe two new algorithms
to  propagate resource   constraints     on discrete  resources    and
reservoirs.   Unlike most of the   classical  work in scheduling,  our
algorithms focus on the precedence relations between activities rather
than on their absolute position in time.  They are efficient even when
the set of   activities is not   completely defined and when the  time
window   of  activities is large.    These   features explain  why our
algorithms    are particularly  suited   for  integrated  planning and
scheduling   approaches.   All  our algorithms  are   illustrated with
examples.  Encouraging   preliminary  results  are  reported  on  pure
scheduling problems.
\end{abstract}

\section{Introduction}

As underlined in \cite{smith00}, some tools are still missing to solve
problems that lie between pure  AI planning and pure scheduling. Until
now, the scheduling community  has focused on  the optimization of big
scheduling  problems involving a  well-defined  set of  activities and
resource constraints.  In contrast, AI planning  research - due to the
inherent  complexity of plan synthesis  - has focused on the selection
of    activities leaving aside  the  issues   of  optimization and the
handling of  time and complex resources.  From   the point of  view of
scheduling, mixed planning  and scheduling problems  have two original
characteristics. First, as the  set  of activities is  not  completely
known  beforehand  it is  better to   avoid  taking  strong scheduling
commitments during the search (e.g. instantiating or strongly reducing
the time  window of an activity). Secondly,  most of the partial plans
handled   by partial  order planners  (POP)   or by hierarchical  task
network  planners  (HTN) make  an     extensive usage of    precedence
constraints between activities.     And, surprisingly, until   now the
conjunction of precedence and resource constraints has not been deeply
investigated, even in the  scheduling field itself. Indeed, except for
the   special case  of unary    resources  (for  example in   job-shop
scheduling),    disjunctive  formulations   of   cumulative   resource
constraints are relatively  new techniques and   until now, they  were
mainly       used   for      search       control   and     heuristics
\cite{cesta2000,laborie95}.  This paper  proposes  some new constraint
propagation algorithms  that   strongly  exploit  the  conjunction  of
precedence and resource constraints and allow a natural implementation
of least-commitment   planning and  scheduling  approaches.  The first
section of the paper describes  our scheduling model.  The second  one
summarizes  the state-of-the-art scheduling propagation techniques and
explains why  most of   them are  not  satisfactory  for dealing  with
integrated planning and scheduling.   In the next section, we describe
the  basic structure -  precedence graphs - on  which our new proposed
algorithms rely.  Then,   we  present  two  original   techniques  for
propagating resource constraints:  the energy precedence algorithm and
the balance  algorithm.  Finally,  the   last  section  of  the  paper
describes  how  these propagation   algorithms can  be  embedded  in a
least-commitment search procedure  and gives some preliminary  results
on pure scheduling problems.

\section{Model and Notations}

\subsubsection{Partial schedule.} A {\bf partial schedule} corresponds
to the current scheduling information available at a given node in the
search tree. In a mixed planning and scheduling problem, it represents
all the temporal and resource information of a partial plan. A partial
schedule  is  composed of   activities, and temporal   constraints and
resource constraints. These concepts are detailed below.

\subsection{Activities.}   An   activity $A$  corresponds    to a time
interval $[start(A),  end(A))$  where $start(A)$ and $end(A)$  are the
decision  variables denoting the start and  end  time of the activity.
Conventionally, $start_{min}(A)$   denotes the current  earliest start
time, $start_{max}(A)$   the  latest start   time, $end_{min}(A)$  the
earliest end time, and $end_{max}(A)$ the latest  end time of activity
$A$.     The   duration   of     activity     $A$   is    a   variable
$dur(A)=end(A)-start(A)$. Depending on the  problem, the  duration may
be known in advance or may be a decision variable. In a mixed planning
and scheduling  problem,  a planning operator  is  composed of  one or
several activities.

\subsubsection{Temporal constraints.} A {\bf temporal constraint} is a
constraint  of the  form: $min  \le t_i-t_j  \le max$ where  $t_i$ and
$t_j$ are either some  variable representing the start  or end time of
an  activity or a   constant  and $min$  and   $max$ are two   integer
constants. Note  that simple precedence  between activities as well as
release dates and due dates are special cases of temporal constraints.

\subsubsection{Resources.} The  most  general  case of    resources we
consider in  this paper is the  reservoir resource.  A {\bf reservoir}
resource is a multi-capacity resource that can be consumed or produced
by the activities in the schedule.  A reservoir has an integer maximal
capacity and may have an initial level.  As an example of a reservoir,
you can think of a fuel tank. A {\bf discrete resource} is a reservoir
resource  that cannot be  produced. Discrete  resources are also often
called   cumulative   or   sharable    resources  in   the  scheduling
literature. A discrete resource has a known  maximal capacity that may
change over time. A discrete resource allows  for example to represent
a pool of  workers whose availability varies  over time. A {\bf  unary
resource} is a discrete resource with unit  capacity.  It imposes that
all the   activities requiring the  same  unary  resource  are totally
ordered. This is typically the case of a machine that can only process
one  job at  a time. Unary  resources are  the  simplest and the  most
studied resources in scheduling as well as in AI planning.

\subsubsection{Resource  constraints.}    
A {\bf  resource  constraint} defines how  a given   activity $A$ will
require and affect   the availability of  a  given resource   $R$.  It
consists  of a  tuple $<A,R,q,TE>$  where  $q$ is an  integer decision
variable describing  the quantity of  resource $R$ consumed (if $q<0$)
or produced (if $q>0$) by activity $A$ and $TE$ is  a time extent that
describes the time interval where the  availability of resource $R$ is
affected by the execution of activity $A$. For example:
\begin{itemize}
\item $<A,R_1,-1,FromStartToEnd>$ is a resource constraint that states
      that activity $A$ will require $1$ unit of resource $R_1$
      between its start time and its end time. 
\item $<A,  R_2, q=[2,3], AfterEnd>$   is  a resource constraint  that
      states  that activity  $A$  will  produce $2$   or  $3$ units of
      reservoir    $R_2$  at its end   time.    This will increase the
      availability of $R_2$ after the end time of $A$.
\item $<A, R_3, -4, AfterStart>$  is a resource constraint that states
      that activity $A$  will consume $4$  units of  resource $R_3$ at
      its  start  time. This  will decrease  the availability of $R_3$
      after the start time of $A$.
\end{itemize} 
Of course, the same activity $A$ may participate into several resource
constraints.  Note  that  the change of  resource availability  at the
start or  end time of  an activity is  considered to be instantaneous:
continuous changes are not handled.

\subsubsection{Close  Status of  a Resource.}   At  given node in  the
search,  a  resource  is said to  be   {\bf closed} if   no additional
resource constraint   on that resource will  be  added in  the partial
schedule when continuing in the  search tree.  In stratified  planning
and  scheduling approaches where the  planning phase is separated from
the scheduling one, all the resources can  be considered closed during
scheduling  as all the activities and   resource constraints have been
generated during the planning phase. Note also that in approaches that
interleave planning and scheduling and implement a hierarchical search
as in  \cite{garcia96},    resources belonging to  already   processed
abstraction levels can be considered closed.

\section{Existing Approaches} 

From the  point of view  of Constraint Programming, a partial schedule
is a  set of decision variables (start,   end, duration of activities,
required  quantities  of resource) and  a  set  of constraints between
these variables   (temporal  and resource  constraints).    A solution
schedule is an instantiation of all the decision variables so that all
the constraints are satisfied.   In  Constraint Programming, the  main
technique used   to    prune the search   space    is {\bf  constraint
propagation}.   It  consists in  removing from  the domain of possible
values of a  decision variable the  ones that will surely violate some
constraint. More  generally,  constraint propagation allows us  in the
current problem to find features shared by all the solutions reachable
from the  current search  node; these  features  may imply some domain
restriction or  some  additional constraints  that must  be satisfied.
Currently,  in constraint-based scheduling   there are two families of
algorithms to propagate  resource constraints: timetabling  approaches
and activity interaction techniques.

\subsection{Timetabling}

The first propagation technique, known as {\bf timetabling}, relies on
the computation for  every date $t$  of the minimal resource  usage at
this   date      by  the    current  activities    in     the schedule
\cite{lepape94}. This aggregated demand  profile is maintained  during
the search and it allows restricting the domains of  the start and end
times of activities  by removing  those  dates that  would necessarily
lead to an over-consumption  of the resource.  For simplicity  reason,
we describe this technique only on discrete resources and assuming all
the time extents  are $FromStartToEnd$.  Suppose  that an activity $A$
requires $q(A) \in [q_{min}(A),q_{max}(A)]$ units  of a given resource
$R$ and  is such that  $start_{max}(A)  < end_{min}(A)$, then  we know
surely  that  $A$ will at  least execute  between $start_{max}(A)$ and
$end_{min}(A)$ and thus, it will surely  require $q_{min}(A)$ units of
resource $R$ on this time interval. For  each resource $R$, a curve is
maintained that aggregates all these demands that is:

\vspace*{-2mm}
\[C_R(t) = \sum_{ \{<A,R,q,TE> / start_{max}(A) \le t < end_{min}(A)\}}
\hspace{-15mm} q_{min}(A) \] 

It's clear  that if there  exists  a date  $t$  such that  $C_R(t)$ is
strictly greater  than the maximal capacity of  the  resource $Q$, the
current schedule   cannot lead to   a  solution  and the search   must
backtrack. Furthermore, if  there   exists an activity  $B$  requiring
$q(B)$ units of resource $R$ and a date $t_0$ such that: $end_{min}(B)
\le t_0 < end_{max}(B)$ and $\forall t \in [t_0, end_{max}(B)), C_R(t)
+ q_{min}(B) > Q$ then, activity $B$ cannot end after date $t_0$ as it
would over-consume     the   resource.  Indeed,   remember    that, as
$end_{min}(B)  \le t_0$, $B$  is   never  taken  into account  in  the
aggregation on the time  interval $[t_0 , end_{max}(B)$).  Thus, $t_0$
is  a new valid upper bound  for $end(B)$. A  similar reasoning can be
applied  to find new lower  bounds on the  start time of activities as
well as  new  upper bounds on  the  quantity  of resource  required by
activities.   Moreover, this approach easily   extends to all types of
time extent and to reservoirs. The main advantage of this technique is
its relative simplicity and its  low algorithmic complexity. It is the
main technique used today     for scheduling discrete  resources   and
reservoirs. Unfortunately,  these  algorithms propagate  nothing until
the time windows of activities become so small that some dates $t$ are
necessarily  covered  by  some activity.   It  means  that unless some
strong commitments are made early in the search on the time windows of
activities,   these   approaches   are  not   able     to  efficiently
propagate. Furthermore, these  approaches do not  directly exploit the
existence of precedence constraints between activities.

\subsection{Activity Interactions}

The  second family   of algorithms is  based on   an analysis  of {\bf
activity interactions}.  Instead of considering what happens at a date
$t$,  it considers some  subsets  $\Omega$ of activities competing for
the same resource and performs  some propagation based on the position
of  activities   in  $\Omega$.  Some   classical activity  interaction
approaches are summarized below.

\subsubsection{Disjunctive Constraint.}  The  simplest example of such
an algorithm   is  the  disjunctive  constraint  on   unary  resources
\cite{erschler76}.  This  algorithm  analyzes  each pair of activities
$(A,B)$  requiring the same unary   resource and, whenever the current
time bounds of activities are so that $start_{max}(A) < end_{min}(B)$,
it deduces that as  activity $A$ necessarily  starts before the end of
activity $B$  is must  be  completely  executed before  $B$  and thus,
$end(A) \le start_{max}(B)$ and $start(B) \ge end_{min}(A)$. Actually,
the classical  disjunctive  constraint can be generalized  as follows:
whenever   the temporal constraints    are   so that  the   constraint
$start(A)<end(B)$  must hold, it  adds  the additional constraint that
$end(A) \le  start(B)$.    Note  that this  algorithm  is   the  exact
counterpart   in scheduling of   the disjunctive  constraint to handle
unsafe   causal      links   in    POCL   planners      proposed    in
\cite{khambhampati96}. Unfortunately,  such  a simple constraint  only
works in the restricted case of unary resources.

\subsubsection{Edge-Finding.}            Edge-finding       techniques
\cite{carlier90,nuijten94} are available for  both unary and  discrete
resources.    On a  unary    resource,  edge-finding techniques detect
situations where     a given activity $A$   cannot   execute after any
activity in a set  $\Omega$ because there  would not be enough time to
execute all  the activities in $\Omega  \cup  A$ between  the earliest
start time of activities in $\Omega \cup A$ and the latest end time of
activities in $\Omega  \cup A$. When  such a situation is detected, it
means that $A$ must execute before all  the activities in $\Omega$ and
it  allows  to compute  a  new valid upper  bound for  the end time of
$A$. More formally, let $\Omega$ be a subset of  activities on a unary
resource, and  $A  \notin \Omega$ another activity   on the same unary
resource. Most of  the edge-finding technique  can be captured by  the
rule $(1) \Rightarrow (2)$ where:
\begin{eqnarray*} 
(1) & end_{max}(\Omega \cup A) - start_{min}(\Omega) < dur(\Omega \cup A) \\
(2) & end(A) \leq \min_{\Omega' \subset \Omega}{(end_{max}(\Omega') - dur(\Omega'))}
\end{eqnarray*}

Similar rules allow  to detect  and  propagate the fact that  a  given
activity must end  after   all activities  in $\Omega$  ({\em  Last}),
cannot start  before all activities in  $\Omega$  ({\em Not First}) or
cannot   end   after all   activities        in $\Omega$ ({\em     Not
Last}). Furthermore,  edge-finding   techniques  can be    adapted  to
discrete resources by reasoning on the resource energy required by the
activities that  is, the product $duration  \times required~quantity$.
Most of the edge-finding algorithms can be implemented to propagate on
all  the activities  $A$  and all the   subsets $\Omega$ with a  total
complexity in $O(n^2)$.

\subsubsection{Energetic    Reasoning.}    As  for   the  edge-finding
techniques, energetic reasoning \cite{erschler91} analyzes the current
time-bounds of   activities in order to adjust   them by removing some
invalid values. A typical  example of energetic reasoning consists  in
finding  pairs  of activities  $A,B$ on   a  unary resource  such that
ordering activity $A$ before $B$ would lead  to a dead-end because the
unary  resource  would not    provide enough ``energy''   between  the
earliest start time of  $A$ and the latest  end time of $B$ to execute
$A$, $B$  and   all the  other  activities  that necessarily  needs to
execute on this time window. More formally, if $C$  is an activity and
$[t_1,t_2)$ a time window, the  energy necessarily required by $C$  on
the time window $[t_1,t_2)$ is:
\vspace*{-2mm}
\[W_C^{[t_1,t_2)} = min( end_{min}(C)-t_1, t_2 - start_{max}(C),
		        dur(C), t_2 - t_1)\]

Thus, as soon as  the condition below holds,  it means that $A$ cannot
be ordered before $B$ and thus, must be ordered after.
\begin{eqnarray*} 
\vspace*{-2mm}
end_{max}(B) - start_{min}(A) < 
dur(A) + dur(B) + \sum_{C \notin \{A,B\}}{W_C^{[start_{min}(A), end_{max}(B))} }
\vspace*{-2mm}
\end{eqnarray*}

It allows to update the earliest start time of  $A$ and the latest end
time of $B$.    Other adjustments   of  time bounds   using  energetic
reasoning can  be used, for example  to deduce that an activity cannot
start at its earliest start time or cannot end at its latest end time.
Furthermore,  energetic reasoning can easily  be  extended to discrete
resources.  A good starting point to learn more about edge-finding and
energetic reasoning on unary  resources is \cite{baptiste95} where the
authors describe and   compare several variants of   these techniques.
Although  these tools   (edge-finding,  energetic reasoning) are  very
efficient   in  pure scheduling  problems,   they suffer from the same
limitations  as timetabling  techniques.   Because they  consider  the
absolute position of  activities  in time  rather than their  relative
position, they will not propagate until the time windows of activities
have become small  enough and the propagation  may be very  limited in
case  the   current schedule   contains  many precedence  constraints.
Furthermore, these   tools   are  available for  unary    and discrete
resources only and are difficult to generalize to reservoirs.

The following sections of  this paper describes  two new techniques to
propagate discrete and reservoir    resources based on analyzing   the
relative position   of    activities  rather   than  their    absolute
position. These algorithms   fully exploit the  precedence constraints
between activities  and   propagate even  when  the time   windows  of
activities  are still very large    which is  typically  the case   in
least-commitment planners    and  schedulers.   Of course these    new
propagation algorithms can  be used in  cooperation with  the existing
techniques we just described  above. Both of  our algorithms are based
on the precedence graph structure described in the section below.

\section{Precedence Graph}

\subsection{Definitions} 
A {\bf resource  event} $x$  on a  given resource $R$  is a time-point
variable at which the availability of  the resource changes because of
an activity.  A  resource event corresponds to  the start or end point
of an activity. Let:
\begin{itemize}
\item $t(x)$ denote the time-point variable of event $x$. $t_{min}(x)$
      and  $t_{max}(x)$ will respectively  denote  the current minimal
      and maximal value in the domain of $t(x)$.
\item $q(x)$ denote the  relative change of resource  availability due
      to event  $x$ with the convention  that $q>0$ denotes a resource
      production and   $q<0$ a resource  consumption. $q_{min}(x)$ and
      $q_{max}(x)$  will respectively denote  the current  minimal and
      maximal value in the domain of $q(x)$.
\end{itemize}  
There is of course an evident mapping between the resource constraints
on a resource and the resource events.

A {\bf precedence   graph}  on a resource    $R$ is a directed   graph
$G_R=(V,E_{\le},E_{<})$ where $E_{<} \subset E_{\le}$ and:
\begin{itemize}
\item $V$ is the set of resource events on $R$ 
\item $E_{\le}  ={(x,y)}$ is the set   of precedence relations between
      events of the form $t(x) \le t(y)$.
\item $E_{<} ={(x,y)}$   is the  set of precedence   relations between
      events of the form $t(x) < t(y)$.
\end{itemize}

The precedence graph on  a  resource is  designed  to collect all  the
precedence  information    between  events  on     the resource. These
precedence information may come from: (1)  temporal constraints in the
initial statement  of  the problem,  (2)  temporal constraints between
activities   in the  same   planning operator,  (3) search   decisions
(e.g. causal   link,   promotion, demotion,  ordering  decisions    on
resources)  or (4) may  have been discovered by propagation algorithms
(e.g.  unsafe   causal    links  handling,    disjunctive  constraint,
edge-finding,    etc.)     or   simply   because    $t_{max}(x)    \le
t_{min}(y)$. When new events or new precedence relations are inserted,
the   precedence    graph  incrementally  maintains    its  transitive
closure. This leads to a worst-case complexity of $O(n^2)$ to maintain
the precedence graph. The precedence relations in the precedence graph
as well as  the  initial  temporal  constraints are propagated  by  an
arc-consistency algorithm. Given an event  $x$  in a precedence  graph
and assuming the  transitive closure has been  computed, we define the
following subsets of events:
\begin{itemize}
\item $S(x)$ is  the set of events simultaneous  with  $x$ that is the
      events $y$ such that $(x,y) \in E_{\le}$ and $(y,x) \in E_{\le}$
\item $B(x)$ is the set  of events before $x$  that is the events  $y$
      such that $(y,x) \in E_{<}$
\item $BS(x)$ is  the set  of events  before or  simultaneous with $x$
      that  is the events $y$  such that $(y,x)  \in E_{\le}$ , $(y,x)
      \notin E_{<}$ and $(x,y)\notin E_{\le}$
\item $A(x)$ is the  set of events after   $x$ that is the events  $y$
      such that $(x,y)\in E_{<}$
\item $AS(x)$ is the set of events after or simultaneous with $x$ that
      is the events  $y$ such  that that  $(x,y)\in  E_{\le}$ , $(x,y)
      \notin E_{<}$ and $(y,x)\notin E_{\le}$
\item $U(x)$ is the set of events unranked with respect to $x$ that is
      the events $y$  such that $(y,x)\notin E_{\le}$ and $(x,y)\notin
      E_{\le}$
\end{itemize}

Note that $\{S(x), B(x), BS(x), A(x),  AS(x),U(x)\}$ is a partition of
$V$.  An  example of  precedence  graph with an  illustration of these
subsets is  given in Figure  \ref{fig2} and corresponds to  a schedule
with          the        6        resource              constraints:\\
$<A_1,R,-2,FromStartToEnd>$,$<A_2,R,[-10,-5],AfterStart>$,\\
$<A_3,R,-1,AfterStart>$,  $<A_4,R,2,AfterEnd>$,  $<A_5,R,2,AfterEnd>$,
$<A_6,R,2,AfterEnd>$  and some  precedence  relations. The subsets are
relative  to the event   $x$  corresponding to  the start  of activity
$A_1$.

\begin{figure}[hbt]
\vspace*{-2mm}
\begin{center}
\input{figure2.pstex_t}
\caption{An Example of Precedence Graph} \label{fig2}
\end{center}
\vspace*{-10mm}
\end{figure}

\subsection{Implementation and Complexity}

As we will see in next  section, our propagation algorithms often need
to query  the precedence graph   about  the relative position  of  two
events on a  resource  so this  information needs to  be accessible in
$O(1)$ on  our structure. It  explains  why we  chose to implement the
precedence  graph as a matrix   that  stores the relative position  of
every pair of events. Furthermore, on our structure, the complexity of
traversing any  subset of events (e.g. $B(x)$  or $U(x)$)  is equal to
the  size of this subset. Note  that the precedence graph structure is
extensively used in  ILOG Scheduler  and is  not only useful  for  the
algorithms described in   this  paper. In  particular,  the precedence
graph    implementation  allows the user     to write his  own complex
constraints  that rely on this graph  as for example the one involving
alternative resources and transition times described in
\cite{focacci00}.

\section{New Propagation Algorithms}

\subsection{Energy Precedence Constraint}  

The    {\bf  energy precedence constraint}  is     defined on discrete
resources only. As it does not require that the resource be closed, it
can be used at  any time during the search.  For simplicity, we assume
that    all   the    resource constraints    have     a   time  extent
$FromStartToEnd$. Suppose that $Q$ denotes the maximal capacity of the
discrete resource over time.  If $x$ is a  resource event and $\Omega$
is a  subset of resource constraints  that  are constrained to execute
before  $x$, then the resource  must  provide enough energy to execute
all resource constraints in $\Omega$ between  the earliest start times
of activities of $\Omega$ and $t(x)$. More formally:

\vspace*{-2mm}
\[t_{min}(x)  \ge  \min_{<A,R,q,TE>\in \Omega} (start_{min}(A)) 
	     + \sum_{<A,R,q,TE>\in \Omega} \hspace{-5mm} (q_{min}(A) *
	       dur_{min}(A)) / Q \]
\vspace*{-2mm}

A very simple example of the  propagation performed by this constraint
is given in Figure \ref{fig3}. If we suppose that the maximal capacity
of the discrete  resource is $4$  and all activities must  start after
time $0$, then by considering  $\Omega=\{A_1, A_2, A_3, A_4\}$, we see
that          event  $x$   cannot         be  executed    before  time
$[0]+[(2*10)+(2*8)+(2*8)+(2*2)]/4 = 14$. Of course, a symmetrical rule
can be used  to  find  an upper  bound  on $t(x)$  by considering  the
subsets $\Omega$  of  resource  constraints  that must  execute  after
$x$. The   same idea as  the energy  precedence constraint  is used in
\cite{sourd00} to  adjust the  time-bounds of activities  on different
unary resources.

\begin{figure}[hbt]
\vspace*{-5mm}
\begin{center}
\input{figure3.pstex_t}
\caption{Example of Energy Precedence Propagation} \label{fig3}
\end{center}
\vspace*{-6mm}
\end{figure}

It's important to note that the energy precedence algorithm propagates
even when the time window of activities is  very loose (in the example
of Figure \ref{fig3}, the  latest end times  of activities may be very
large). This  is  an important  difference with  respect to  classical
energetic and edge-finding  techniques that would propagate nothing in
this case.  The propagation of the energy precedence constraint can be
performed for all the events $x$ on a resource and for all the subsets
$\Omega$ with  a total worst-case  time complexity of $O(n*(p+log(n))$
where $n$ is  the number  of the events  on  the resource and $p$  the
maximal number  of predecessors of a  given  event in the  graph ($p <
n$). Note that  when  the discrete resource   has a  maximal  capacity
profile that  varies over time, the   algorithm can take  into account
some fake  resource constraints with  instantiated start and end times
to accommodate the maximal capacity profile.

\subsection{Balance Constraint}

The {\bf balance constraint} is  defined on a reservoir resource. When
applied to  a reservoir, the  basic version of this algorithm requires
the reservoir to be closed.  When applied to  a discrete resource, the
resource may still be open.  The basic idea  of the balance constraint
is to compute, for each event $x$ in the precedence graph, a lower and
an upper bound on the reservoir level just before  and just after $x$.
The reader will     certainly  find some  similarities   between  this
constraint and the Modal  Truth Criterion on planning predicates first
introduced in  \cite{chapman87}.   Actually this is  not surprising as
the balance  constraint  can  be  considered as    a kind  of  MTC  on
reservoirs  that  detects only some necessary conditions\footnote{When
the reservoir is not closed, one can imagine extending our propagation
algorithm into  a real truth  criterion on reservoirs that would allow
justifying the insertion of new  reservoir producers or consumers into
the current schedule.  This   interesting extension clearly   worth to
study but is  out of the scope  of  this paper.}. Given an  event $x$,
using the graph we can compute  an upper bound  on the reservoir level
at   date  $t(x)-\epsilon$ just   before $x$  assuming   (1)  All  the
production events $y$  that {\bf may}  be executed strictly before $x$
are executed strictly before $x$ and  produce as much as possible that
is $q_{max}(y)$; (2) All the consumption events $y$ that {\bf need} to
be  executed strictly before $x$ are  executed strictly before $x$ and
consume  as little as possible that  is $q_{max}(y)$; and  (3) All the
consumption events that {\bf may}  execute simultaneously or after $x$
are executed simultaneously or after $x$. More formally, if $L_{init}$
is  the initial level   of the reservoir, $P$   the set of  production
events and $C$ the set of consumption  events, this upper bound can be
computed as follows:
\vspace*{-1mm}
\begin{eqnarray}
L^{<}_{max}(x) = L_{init} + \hspace{-10mm} \sum_{y \in P \cap (B(x) \cup BS(x) \cup
U(x))} \hspace{-14mm} q_{max}(y) + \sum_{y \in C \cap B(x)} \hspace{-4mm} q_{max}(y)
\end{eqnarray}
%\vspace*{-1mm}
Applying this  formula   to event  $x$ in  Figure   \ref{fig2} if with
suppose $L_{init}=2$ leads to $L^{<}_{max}(x)= 2 + [2+2+2] + [-5] =3$.
In a very similar way,  it is possible  to compute $L^{<}_{min}(x)$, a
lower bound  of the level just before  $x$; $L^{>}_{max}(x)$, an upper
bound of the level just after $x$ and  $L^{>}_{min}(x)$, a lower bound
of the level just  after $x$.  For each  of these bounds,  the balance
constraint is able  to discover four types  of information:  {\bf dead
ends}, new  bounds for  {\bf  resource usage variables} and  {\bf time
variables} and new {\bf precedence relations}. For symmetry reasons we
only describe the propagation based on $L^{<}_{max}(x)$.

\subsubsection{Discovering dead ends.} Whenever $L^{<}_{max}(x)<0$, we
know that the   level of the  reservoir will  surely  be negative just
before event $x$ so the search has reached a dead end.

\subsubsection{Discovering  new  bounds on  resource usage variables.}
Suppose there    exists a  consumption event   $y\in   B(x)$ such that
$q_{max}(y)-q_{min}(y)>L^{<}_{max}(x)$.   If   $y$ would    consume  a
quantity $q$  such that $q_{max}(y)-q>L^{<}_{max}(x)$  then, simply by
replacing $q_{max}(y)$ by $q(y)$ in formula (1), we see that the level
of the reservoir would be negative just before $x$.  Thus, we can find
a    better      lower        bound    on     $q(y)$      equal     to
$q_{max}(y)-L^{<}_{max}(x)$. In the example of Figure \ref{fig2}, this
propagation would restrict  the consumed quantity  at the beginning of
activity $A_2$ to $[-8,-5]$ as any value lower than $-8$ would lead to
a dead end.

\subsubsection{Discovering new bounds on time variables.} Formula (1)
can be rewritten as follows: 
\vspace*{-1mm}
\begin{eqnarray*}
L^{<}_{max}(x) = ( L_{init} + \hspace{-2mm} \sum_{y \in B(x)} \hspace{-2mm}
q_{max}(y)) + ( \hspace{-5mm}\sum_{y \in P \cap (BS(x) \cup U(x))}
\hspace{-10mm} q_{max}(y))
\end{eqnarray*}
If the first term of this equation is negative, it means that some
production events in $BS(x) \cup U(x)$ will have to be executed
strictly before $x$ in order to produce at least:
\vspace*{-1mm}
\begin{eqnarray*}
\Pi^{<}_{min}(x) = - L_{init} - \sum_{y \in B(x)}\hspace{-2mm} q_{max}(y)
\end{eqnarray*}
Let $P(x)$ denote the  set production events  in $BS(x) \cup U(x)$. We
suppose the events $(y_1,\cdots,y_i,\cdots,y_p)$ in $P(x)$ are ordered
by  increasing minimal  time $t_{min}(y)$. Let   $k$ be  the index  in
$[1,p]$ such that:
\vspace*{-1mm}
\begin{eqnarray*}
\sum_{i=1}^{k-1}q_{max}(y_i) < \Pi^{<}_{min}(x) \le \sum_{i=1}^{k}q_{max}(y_i)
\end{eqnarray*}
If event $x$ is executed at a date $t(x) \le t_{min}(y_k)$, not enough
producers will be  able to  execute strictly before   $x$ in order  to
ensure a positive level  just before $x$.  Thus, $t_{min}(y_k)+1$ is a
valid  lower  bound of $t(x)$. In   Figure \ref{fig2} if $L_{init}=2$,
$\Pi^{<}_{min}(x)=3$,  and  this propagation  will deduce  that $t(x)$
must  be strictly  greater than the   minimal between the earliest end
time of $A_5$ and the earliest end time of $A_6$.

\subsubsection{Discovering new precedence  relations.} There are cases
where we can  perform  an  even  stronger propagation. Suppose   there
exists a production event $y$ in $P(x)$ such that:
\vspace*{-1mm}
\begin{eqnarray*}
\sum_{z \in P(x) \cap (B(y) \cup BS(y) \cup U(y))} \hspace{-14mm}
q_{max}(z) < \Pi^{<}_{min}(x) 
\end{eqnarray*}
Then, if we had  $t(x)\le t(y)$, we would see  that again there is  no
way to produce $\Pi^{<}_{min}(x)$ before  event $x$ as the only events
that  could produce strictly  before event $x$  are the  ones in $P(x)
\cap (B(y) \cup BS(y)  \cup U(y))$. Thus,  we can deduce the necessary
precedence relation:  $t(y)<t(x)$.  For example  in Figure \ref{fig2},
the balance algorithm  would  discover that  $x$ needs to  be executed
strictly after the  end of $A_4$. Note  that a weaker version of  this
propagation has been proposed in \cite{cesta97}  that runs in $O(n^2)$
and does not analyze the  precedence relations  between the events  of
$P(x)$.   Like for  timetabling   approaches, one  can  show that  the
balance algorithm is  {\bf sound}, that is,  it will detect a dead end
on any fully  instantiated  schedule   that violates the     reservoir
resource  constraint.  In fact,  the  balance algorithm  does not even
need the  schedule  to be  fully  instantiated:  for example, it  will
detect a dead   end on any non-solution   schedule as soon  as all the
production events are ordered relatively to all the consumption events
on a  resource.  Furthermore, when  all events $x$  on a  reservoir of
capacity  $Q$ are so  that  $L^{<}_{max}(x)\le Q$, $L^{>}_{max}(x) \le
Q$, $L^{<}_{min}(x)\ge 0$, and $L^{>}_{min}(x)\ge  0$ - in that  case,
we say that event $x$ is {\bf safe} - then,  any order consistent with
the current precedence graph  satisfies  the reservoir constraint.  In
other  words, the reservoir is   solved.  This very important property
allows stopping the search on a reservoir when all the events are safe
and even if they are not completely ordered. Note also that, according
to the concepts introduced in \cite{laborie95}, the balance constraint
can be  seen as an algorithm that  implicitly  detects and solves some
deterministic  MCSs on the reservoir  while avoiding the combinatorial
explosion of enumerating  these  MCSs.  The balance algorithm   can be
executed for all the events $x$ with a global worst-case complexity in
$O(n^2)$ if the propagation that discovers new precedence relations is
not turned on, in $O(n^3)$ for a  full propagation. In practice, there
are many  ways to  shortcut  this worst  case  and  in particular,  we
noticed that the   algorithmic   cost of  the   extra-propagation that
discovers new    precedence    relations  was  negligible.   In    our
implementation,    at each node  of   the   search,  the full  balance
constraint is executed until a fix point is reached.

\section{First Results}

We implemented  a complete and  relatively  simple search procedure on
reservoirs that selects pairs  of unsafe events  $(x,y)$ and creates a
choice  point by adding either   the relation $t(x)<t(y)$ or  $t(y)\ge
t(x)$.  The heuristics  for  selecting which pair  of events  to order
relies   on the   bounds   on    reservoir  levels   $L^{<}_{max}(x)$,
$L^{>}_{max}(x)$, $L^{<}_{min}(x)$,  and $L^{>}_{min}(x)$ computed  by
the balance constraint. These levels can indeed  be considered as some
texture  measurements   \cite{beck97}  projected    on  the   schedule
events. Until now, few benchmarks  are available on problems involving
temporal constraints and  reservoirs. The only  one we are aware of is
\cite{neumann99}  where the authors  generate  300 project  scheduling
problems involving 5 reservoirs, min/max delays between activities and
minimization of makespan.  From these 300  problems, 12 hard instances
could not be  solved to optimality  by  their approach.  We tested our
algorithms  on these 12 open  problems\footnote{All the other problems
were easily solved using our approach.}. The results are summarized on
the table   below.   The size   of the   problem  is the    number  of
activities.  The   bounds are  the  best  lower  and  upper bounds  of
\cite{neumann99}.  The times  in the  table were  measured  on a HP-UX
9000/785 workstation. We can see that all of the 12 open problems have
been  closed in less   than  10  seconds  CPU time.  Furthermore,  our
approach produces highly parallel  schedules as the balance constraint
implements some sufficient conditions    for a partial  order  between
events to be a solution. 

\scriptsize
\vspace*{1mm}
\begin{center}
\begin{tabular}{|c|c|c|c|c|c|}\hline
{\bf $~$Problem$~$} & {\bf $~$Size$~$} &  {\bf {\it $~$Lower bound$~$}} & {\bf {\it $~$Upper bound$~$}} & {\bf $~$Optimal$~$} & {\bf $~$CPU Time (s)$~$} \\ \hline
\#10 &  50 & {\it 92} & {\it 93} & 92 & 0.28 \\ \hline
\#27 &  50 & {\it 85} & {\it $+\infty$} & 96 & 2.43 \\ \hline
\#82 &  50 & {\it 148} & {\it $+\infty$} & no solution & 0.05\\ \hline
\#6  & 100 & {\it 203} & {\it 223} & 211 & 0.97\\ \hline
\#12 & 100 & {\it 192} & {\it 197} & 197 & 0.72\\ \hline
\#20 & 100 & {\it 199} & {\it 217} & 199 & 0.46\\ \hline
\#30 & 100 & {\it 196} & {\it 218} & 204 & 2.11\\ \hline
\#41 & 100 & {\it 330} & {\it 364} & 337 & 0.62\\ \hline
\#43 & 100 & {\it 283} & {\it $+\infty$} & no solution & 7.65\\ \hline
\#54 & 100 & {\it 344} & {\it 360} & 344 & 0.46\\ \hline
\#58 & 100 & {\it 317} & {\it 326} & 317 & 0.49\\ \hline
\#69 & 100 & {\it 335} & {\it $+\infty$} & no solution & 1.96\\ \hline
\end{tabular}
\end{center}
\vspace*{1mm}
\normalsize

We  also  tested  the    energy    precedence constraint   on    unary
resources. For  this purpose, we wrote  a very simple least-commitment
search procedure\footnote{The C++ code  of  this search  procedure  is
available in  the  distribution  of  ILOG  Scheduler.}  based  on  the
precedence graph  that orders pairs of  activities on a unary resource
and aims at finding very good first  solutions. We benched this search
procedure  on 44 famous job-shop  problems (namely: {\em abz5-9}, {\em
ft6}, {\em ft10},  {\em   orb1-10},  {\em la1-30}) with   the   energy
precedence constraint as  well as the  disjunctive and the edge-finder
constraint. In average, the   makespan of the first solution  (without
using  any restart or randomization) produced  by our approach is only
7.4\%  greater than the optimal makespan  whereas the average distance
to optimal of  the best greedy  algorithms so far \cite{pacciarelli99}
is 9.3\% on the same problems.

\section{Conclusion and Future Work}

This paper describes   two  new algorithms for  propagating   resource
constraints  on discrete resources   and reservoirs.  These algorithms
strongly exploit the temporal  relations  in the partial schedule  and
are able to propagate even if the time windows of activities are still
very large. Furthermore, on discrete resource, they do not require the
resource to be closed.   These features explain why  they particularly
suit integrated approaches to planning  and scheduling.  Even from the
standpoint of pure scheduling, these  algorithms are powerful tools to
implement {\bf complete} and  {\bf efficient} search  procedures based
on  the relative position of activities.   An  additional advantage of
this approach is that  it  produces {\bf partially ordered  solutions}
instead  of fully instantiated ones. These  solutions are more robust.
All the algorithms described  in this paper  have been implemented and
are   available  in    the   current   version   of   ILOG   Scheduler
\cite{scheduler52}.  As far as AI   Planning is concerned, future work
will  mainly  consist in studying  the  integration  of our scheduling
framework into a HTN or a POP Planner  as well as improving our search
procedures.

%
% ---- Bibliography ----
%
\vspace*{-1mm}
\begin{thebibliography}{111}
\vspace*{-1mm}

\bibitem[1]{baptiste95}
P.~Baptiste and C.~Lepape.
\newblock A theoretical and experimental comparison of constraint
  propagation techniques for disjunctive scheduling.
\newblock In {\em Proceedings IJCAI-95}, 1995.

\bibitem[2]{beck97}
C.~Beck, A.~Davenport, E.~Sitarski, and M.~Fox.
\newblock Texture-based heuristics for scheduling revisited.
\newblock In {\em Proceedings AAAI-97}, 1997.

\bibitem[3]{carlier90}
J.~Carlier and E.~Pinson.
\newblock A practical use of {J}ackson's preemptive schedule for
  solving the job-shop problem.
\newblock {\em Annals of Operations Research}, 26:269--287, 1990.

\bibitem[4]{cesta97}
A.~Cesta and C.~Stella.
\newblock A time and resource problem for planning architectures.
\newblock In {\em Proceedings ECP-97}, 1997.

\bibitem[5]{cesta2000}
A.~Cesta, A.~Oddi, and S.~Smith.
\newblock A constrained-based method for project scheduling with time windows.
\newblock {\em Journal of Heuristics}, 8(1), 2002.
\newblock To appear.

\bibitem[6]{chapman87}
D.~Chapman.
\newblock Planning for conjunctive goals.
\newblock {\em Artificial Intelligence}, 32:333--377, 1987.

\bibitem[7]{erschler76}
J.~Erschler.
\newblock {\em Analyse sous contraintes et aide \`a la d\'ecision pour certains
  probl\`emes d'ordonnancement}.
\newblock PhD thesis, Universit\'e Paul Sabatier, 1976.

\bibitem[8]{erschler91}
J.~Erschler, P.~Lopez, and C.~Thuriot.
\newblock Raisonnement temporel sous contraintes de ressources et probl\`emes
  d'ordonnancement.
\newblock {\em Revue d'IA}, 5(3):7--32, 1991.

\bibitem[9]{focacci00}
F.~Focacci, P.~Laborie, and W.~Nuijten.
\newblock Solving scheduling problems with setup times and alternative
  resources.
\newblock In {\em Proceedings AIPS-00}, pages 92--101, 2000.

\bibitem[10]{garcia96}
F.~Garcia and P.~Laborie.
\newblock {\em New Directions in AI Planning}, chapter Hierarchisation of the
  Search Space in Temporal Planning, pages 217--232.
\newblock IOS Press, Amsterdam, 1996.

\bibitem[11]{scheduler52}
ILOG.
\newblock {I}{L}{O}{G} {S}cheduler 5.2 {R}eference {M}anual, 2001.
\newblock http://www.ilog.com/.

\bibitem[12]{khambhampati96}
S.~Khambhampati and X.~Yang.
\newblock On the role of disjunctive representations and constraint propagation
  in refinement planning.
\newblock In {\em Proceedings KR-96}, 1996.

\bibitem[13]{laborie95}
P.~Laborie and M.~Ghallab.
\newblock Planning with sharable resource constraints.
\newblock In {\em Proceedings IJCAI-95}, pages 1643--1649, 1995.

\bibitem[14]{lepape94}
C.~{Le Pape}.
\newblock Implementation of {R}esource {C}onstraints in {I}{L}{O}{G}
  {S}chedule: {A} {L}ibrary for the {D}evelopment of {C}onstraint-{B}ased
  {S}cheduling {S}ystems.
\newblock {\em Intelligent Systems Engineering}, 3(2):55--66, 1994.

\bibitem[15]{neumann99}
K.~Neumann and C.~Schwindt.
\newblock Project scheduling with inventory constraints.
\newblock Technical Report WIOR-572, Universit\"at Karlsruhe, 1999.

\bibitem[16]{nuijten94}
W.~Nuijten.
\newblock {\em Time and resource constrained scheduling: A constraint
  satisfaction approach}.
\newblock PhD thesis, Eindhoven University of Technology, 1994.

\bibitem[17]{pacciarelli99}
D.~Pacciarelli and A.~Mascis.
\newblock Job-shop scheduling of perishable items.
\newblock In {\em Proceedings INFORMS-99}, 1999.

\bibitem[18]{smith00}
D.E. Smith, J.~Frank, and A.K. Jonsson.
\newblock Bridging the gap between planning and scheduling.
\newblock {\em Knowledge Engineering Review}, 15(1), 2000.

\bibitem[19]{sourd00}
F.~Sourd and W.~Nuijten.
\newblock Multiple-machine lower bounds for shop scheduling problems.
\newblock {\em INFORMS Journal of Computing}, 4(12):341--352, 2000.

\end{thebibliography}

\end{document}

