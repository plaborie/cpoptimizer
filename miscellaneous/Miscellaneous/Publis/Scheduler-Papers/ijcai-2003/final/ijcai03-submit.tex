%%%% ijcai03-submit.tex

\typeout{IJCAI-03 Submission Instructions for Authors}

% This is the instructions for authors for IJCAI-03.
\documentclass{article}
% The file ijcai03.sty is the style file for IJCAI-03. 
\usepackage{ijcai03}
% Use the postscript times font!
\usepackage{times}

% Following comment is from ijcai97-submit.tex:
% The preparation of these files was supported by Schlumberger Palo Alto
% Research, AT\&T Bell Laboratories, and Morgan Kaufmann Publishers.
% Shirley Jowell, of Morgan Kaufmann Publishers, and Peter F.
% Patel-Schneider, of AT\&T Bell Laboratories collaborated on their
% preparation. 

% These instructions can be modified and used in other conferences as long
% as credit to the authors and supporting agencies is retained, this notice
% is not changed, and further modification or reuse is not restricted.
% Neither Shirley Jowell nor Peter F. Patel-Schneider can be listed as
% contacts for providing assistance without their prior permission.

% To use for other conferences, change references to files and the
% conference appropriate and use other authors, contacts, publishers, and
% organizations.
% Also change the deadline and address for returning papers and the length and
% page charge instructions.
% Put where the files are available in the appropriate places.

\title{IJCAI-03 Format Instructions for Submissions}
\author{Content Areas:  common sense reasoning, knowledge
representation}

\begin{document}

\maketitle

\begin{abstract}
  The {\it IJCAI-03 Proceedings} will be printed from electronic
  manuscripts submitted by the authors. We require submissions
  in the final format, except in extenuating circumstances.  
  This file includes the
  style instructions for submissions.  Authors should also be sure to
  consult the Call for Papers.
\end{abstract}

\section{Introduction}

For the submission of papers to the reviewing process, we require both
a hardcopy {\em and} the electronic version of the manuscript.
The electronic version must be a PDF ({\em Portable Document Format})
file formatted for 8-1/2" x 11" paper. No electronic submissions
formatted for A4 paper will be accepted for review.

The hardcopy should be printed at 300 dpi or better using a laser
printer adhering to the formatting instructions specified below.  {\em
  Submissions that deviate from these instructions will {\bf not} be
  reviewed.}  However, authors from countries in which access to
word-processing systems is limited may submit unformatted papers.  The
body of their submissions must be at most 6200 words, including
footnotes, figure captions, tables, appendices, and bibliography.
Each half-page of figures will be counted as 500 words.

\subsection{Submission Title Page}

Each hardcopy of a submitted paper must have a title page, separate from the
body of the paper, that includes the title of the paper, the names and
addresses of all authors, a short abstract of less than 200 words, a list of
content areas from those given on the Call for Papers, 
and acknowledgments
(if any).  In addition, it must contain the following declaration:
\begin{quote}
``This paper has not already been accepted by and is not currently
under review for a journal or another conference, nor will it be
submitted for such during IJCAI's review period.''
\end{quote}
Please do not staple the title page to the body of the paper.  Because
this title page is only included on the submission, and not on final
accepted papers, the \LaTeX{} package does not produce it; you must
produce it separately. 

\subsection{Word Processing Software}

As detailed below, IJCAI has prepared and made available a
set of \LaTeX{} macros  for use in formatting your
paper. If you are using some other word processing software
(such as Word, WordPerfect, etc.), please follow the format
instructions given below and ensure that your final paper looks as much
like this sample as possible.

\section{Style and Format}


\LaTeX{} and Bib\TeX{} style files that implement these instructions
can be retrieved electronically.
(See Appendix~\ref{latex} for instructions on how to obtain these files.)

\subsection{Layout}

Print manuscripts two columns to a page, in the manner in which these
instructions are printed.  The exact dimensions for pages are:
\begin{itemize}
\item left and right margins: $.75''$
\item column width: $3.375''$
\item gap between columns: $.25''$
\item top margin---first page: $1.375'''$
\item top margin---other pages: $.75''$
\item bottom margin: $1.25''$
\item column height---first page: $6.625''$
\item column height---other pages: $9''$
\end{itemize}

All measurements assume an {\bf $8$-$1/2 \times 11''$} page size.  For
A4 hardcopies assume the $8$-$1/2 \times 11''$ paper size, i.e.,
use the given top and left margins, column width, height, and gap and
modify the bottom and right margins as necessary.


\subsection{Format of Electronic Manuscript}

For the production of the electronic manuscript you must use Adobe's
{\em Portable Document Format} (PDF). This format can be generated, for
instance, on Unix systems using {\tt ps2pdf} (in this case your 
postscript file from which you generate PDF should be in {\tt letter} 
format; where this is not a default, letter format can be obtained 
from a dvi file by using {\tt dvips} with the option {\tt -t letter}).
Under Windows, Adobe's
{\tt Distiller} can be used.  For reasons of uniformity, Adobe's {\em Times
Roman} font should be used (note that this font is about 10\% denser
than \LaTeX's default {\em Computer Modern} font). In \LaTeX2e{} this is accomplished by putting
\begin{quote} 
\mbox{\tt $\backslash$usepackage\{times\}} \\
\mbox{\tt $\backslash$usepackage\{latexsyms\}}
\end{quote}
in the preamble.

Additionally, it is of utmost importance to specify the American {\bf
  letter} format (corresponding to {\em $8$-$1/2 \times 11''$}) when
formatting the paper.  Otherwise the paper is not printable on North
American printers. When working with {\tt dvips}, for instance, one
should specify {\tt -t letter}.


\subsection{Title and Author Information}

Center the title on the entire width of the page in a 14-point bold
font.  Because IJCAI is using blind reviewing, authors should omit
their names and affiliations from their submissions. 
Below the title a  list of content
areas for the paper should appear.  Similarly, credit to a sponsoring
agency should appear only on the Submission Title Page; in their final
form, accepted papers may include this information on the first page.


\subsection{Abstract}

Place the abstract at the beginning of the first column $3.0''$ from the
top of the page, unless that does not leave enough room for the title and
author information.  Use a slightly smaller width than in the body of the
paper.  Head the abstract with ``Abstract'' centered above the body of the
abstract in a 12-point bold font.  The body of the abstract should be in
the same font as the body of the paper.

The abstract should be a concise, one-paragraph summary
describing the general thesis and conclusion of your
paper. A reader should be able to learn the purpose of the
paper and the reason for its importance from the abstract. The
abstract should be no more than 200 words long.

\subsection{Text}

The main body of the text immediately follows the abstract.  Use
10-point type in a clear, readable font with 1-point leading (10 on
11). 

Indent when starting a new paragraph, except after major headings.

\subsection{Headings and Sections}

When necessary, headings should be used to separate major sections of your
paper.
(These instructions use many headings to demonstrate their
appearance---your paper should have fewer headings.)

\subsubsection{Section Headings}

Print section headings in 12-point bold type in the style shown in these
instructions.  Leave a blank space of approximately 10 points above and 4
points below section headings.  Number sections with arabic numerals.

\subsubsection{Subsection Headings}

Print subsection headings in 11-point bold type.  Leave a blank space of
approximately 8 points above and 3 points below subsection headings.
Number subsections with the section number and the subsection number (in
arabic numerals) separated by a period.

\subsubsection{Subsubsection Headings}

Print subsubsection headings in 10-point bold type.  Leave a blank space of
approximately 6 points above subsubsection headings.  Do not number
subsubsections.

\subsubsection{Special Sections}

In the final version of your paper, you may include an
acknowledgments section, including  acknowledgments of help from
colleagues, financial support, and permission to publish.  However,
please omit this from your submission in order to facilitate blind
reviewing. 

Any appendices directly follow the text and look like sections, except
that they are numbered with capital letters instead of arabic
numerals.

The references section is headed ``References,'' printed in the same
style as a section heading, but without a number.  A sample list of
references is given at the end of these instructions.  Use a
consistent format for references, such as provided by Bib\TeX{}.  The
reference list should not include unpublished work.  Also, when
referring to your own work in the text, use the third person, rather
than the first person, again, to facilitate blind reviewing.  Say,
``Previously, Gottlob has shown that . . .''
rather than, ``In my previous work, I showed that. . ." 

\subsection{Citations}

Citations within the text should include the author's last name and
the year of publication, for example \cite{cheeseman:probability}.
Append lowercase letters to the year in cases of ambiguity.
Treat multiple authors as in the following examples:
\cite{abelson-et-al:scheme} (for more than two authors) and
\cite{brachman-schmolze:kl-one} (for two authors).
If the author portion of a citation is obvious, omit it,
e.g., Levesque \shortcite{levesque:belief}.
Collapse multiple citations as follows:
\cite{levesque:functional-foundations,haugeland:mind-design}.%
\nocite{abelson-et-al:scheme}%
\nocite{brachman-schmolze:kl-one}%
\nocite{cheeseman:probability}%
\nocite{haugeland:mind-design}%
\nocite{lenat:heuristics}%
\nocite{levesque:functional-foundations}%
\nocite{levesque:belief}

\subsection{Footnotes}

Place footnotes at the bottom of the page in a 9-point font.  Refer to them
with superscript numbers.\footnote{This is how your footnotes should
appear.} Separate them from the text by a short line.\footnote{Note the
line separating these footnotes from the text.}
Avoid footnotes as much as possible; they interrupt the
flow of the text. 

\section{Illustrations}

\subsection{General Instructions}

Place illustrations (figures, drawings, tables, and photographs) throughout
the paper at the places where they are first discussed, rather than at the
end of the paper.  If placed at the bottom or top of a page, illustrations
may run across both columns.  

Whenever possible, illustrations should be rendered electronically or
scanned and placed directly in your document pages. All illustrations should
be in black and white since color illustrations may cause problems. 
If you cannot merge illustrations directly into your file, securely attach
them to the master form with glue stick, spray adhesive, rubber cement, or
white tape.
%Do not use transparent tape as the printing process blurs
%copy under transparent tape.

Number illustrations sequentially.  Use references of the following form:
Figure 1, Table 2, etc.  Place illustration numbers and captions under
illustrations.  Leave a margin of 1/4-inch around the area covered by the
illustration and caption.  Use 9-point type for captions, labels, and
other text in illustrations.

Do not use line-printer printouts or screen-dumps for
figures---they will be illegible when printed.
Avoid screens or pattern fills as they tend to reproduce poorly.

\subsection{Photographs}

As mentioned above, whenever possible, illustrations should be
rendered electronically or scanned and placed directly in your
document pages.  If you need to include photographs, use only
glossy black and white photographs.  Color photographs do not
reproduce well.  (Red will reproduce as black, for example.)
Photographs incur extra expense, so please use them judiciously.  {\em
  Do not attach photographs to the pages}---leave sufficient space for
them and write figure numbers in the space.  Label photographs on the
back with the figure number, and enclose them in a separate envelope.

\section{Length of Papers}

Submissions must not be more than {\em six\/} (6) pages.  All
illustrations and references must be included in the 6-page allowance.
{\em Papers that exceed this limit will not be reviewed.}\footnote{Each
accepted paper will be allowed six pages in the proceedings; up to two
additional pages may be purchased at a price of \$250 per page.}

\section*{Acknowledgments}

The preparation of these instructions and the \LaTeX{} and Bib\TeX{} files
that implement them was supported by Schlumberger Palo Alto Research, AT\&T
Bell Laboratories, and Morgan Kaufmann Publishers.

\appendix

\section{Using \LaTeX{}}\label{latex}

A \LaTeX{} style file for version 2e of \LaTeX{} that implements these
instructions has been prepared, as has a Bib\TeX{} style file for version
0.99c of Bib\TeX{} ({\em not version 0.98i}) that implements the citation and
reference styles here. 

%The relevant files are available from the IJCAI server via the World-Wide Web,
%anonymous ftp, and email.  As these files may be changed to fix bugs, you
%should ensure that you are using the most recent versions.

The relevant files are {\tt ijcai03.sty} and \\ {\tt ijcai03-submit.tex}
and the bib-tex file {\tt named.bst} 
wich are currently available at the \\
{\tt www.ijcai-03.org} portal 
at the ``Documents'' page.
The  file {\tt ijcai03-submit.tex} 
contains the \LaTeX{} source of the present document
which
may serve as a formatting sample. 

Note that the ijcai03.sty file is the same as the ijcai01.sty file 
used for IJCAI'01. Moreover, the required document layout is the same 
as for IJCAI'01 except that this year tracking numbers will not 
be used. 

Further information on using these styles for the preparation of papers for
IJCAI-03 can be obtained by contacting pcchair03@ijcai.org.

%%%

%% This section was initially prepared using BibTeX.  The .bbl file was
%% placed here later
%\bibliography{publications}
%\bibliographystyle{named}
%% The file named.bst is a bibliography style file for BibTeX 0.99c
\begin{thebibliography}{}

\bibitem[\protect\citeauthoryear{Abelson \bgroup \em et al.\egroup
  }{1985}]{abelson-et-al:scheme}
Harold Abelson, Gerald~Jay Sussman, and Julie Sussman.
\newblock {\em Structure and Interpretation of Computer Programs}.
\newblock MIT Press, Cambridge, Massachusetts, 1985.

\bibitem[\protect\citeauthoryear{Brachman and
  Schmolze}{1985}]{brachman-schmolze:kl-one}
Ronald~J. Brachman and James~G. Schmolze.
\newblock An overview of the {KL-ONE} knowledge representation system.
\newblock {\em Cognitive Science}, 9(2):171--216, April--June 1985.

\bibitem[\protect\citeauthoryear{Cheeseman}{1985}]{cheeseman:probability}
Peter Cheeseman.
\newblock In defense of probability.
\newblock In {\em Proceedings of the Ninth International Joint Conference on
  Artificial Intelligence}, pages 1002--1009, Los Angeles, California, August
  1985. International Joint Committee on Artificial Intelligence.

\bibitem[\protect\citeauthoryear{Haugeland}{1981}]{haugeland:mind-design}
John Haugeland, editor.
\newblock {\em Mind Design}.
\newblock Bradford Books, Montgomery, Vermont, 1981.

\bibitem[\protect\citeauthoryear{Lenat}{1981}]{lenat:heuristics}
Douglas~B. Lenat.
\newblock The nature of heuristics.
\newblock Technical Report CIS-12 (SSL-81-1), Xerox Palo Alto Research Centers,
  April 1981.

\bibitem[\protect\citeauthoryear{Levesque}{1984a}]{levesque:functional-foundat%
ions}
Hector~J. Levesque.
\newblock Foundations of a functional approach to knowledge representation.
\newblock {\em Artificial Intelligence}, 23(2):155--212, July 1984.

\bibitem[\protect\citeauthoryear{Levesque}{1984b}]{levesque:belief}
Hector~J. Levesque.
\newblock A logic of implicit and explicit belief.
\newblock In {\em Proceedings of the Fourth National Conference on Artificial
  Intelligence}, pages 198--202, Austin, Texas, August 1984. American
  Association for Artificial Intelligence.

\end{thebibliography}

\end{document}







