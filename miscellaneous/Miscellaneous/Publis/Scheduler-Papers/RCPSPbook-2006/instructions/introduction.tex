\PresentationOfAuthors{Introduction written by~}
\ChapterAuthor[%
Introduction]{%
Introduction}{%
Roger \Name{Rousseau}\and Christian \Name{Scheen}}
\label{chap-introduction}

This document describes how to use \latex{} in order to typeset books
(namely, monographs or edited collections) according to the
\Hermes{}~(\HSP{}) guidelines and instructions to authors, with the \oh{}
package.
Many examples are given (\latex{} source files templates, syntax of user
commands and environments,~etc.\@).

The \oh{} package consists of the following files:
\begin{itemize}
 \item\index{ouvrage-hermes.cls@\class{ouvrage-hermes}}
 the \class{ouvrage-hermes} \latex{} document class file.
 It shall be used instead of the \class{book} standard \latex{} document
 class file;
 \item\index{ouvrage-hermes.bst@\bstyle{ouvrage-hermes}}
 the \bstyle{ouvrage-hermes} \bibtex{} bibliography style file.
 It shall drive the \bibtex{} program in order to produce appropriate
 alphanumeric bibliographies;
 \item\index{ouvrage-hermes.ist@\istyle{ouvrage-hermes}}
 the \Makeindex{}\index{Makeindex script@\Makeindex{} (script)} Bourne shell
 script.
 It can be used instead of the standard \mkndx{} program, but this mechanism
 is completely optional; if it is actually used, it will automatically
 produce the desired index layout and it will properly sort out
 (French,~etc.\@) words with diacritical signs (in alphabetical order).
 This script uses two filters written in the AWK~language (namely the
 \preawk{}\index{makeindex-pre.awk@\preawk{}} pre-processor and the
 \postawk{}\index{makeindex-post.awk@\postawk{}} post-processor).
 It also uses the \istyle{ouvrage-hermes} \mkndx{} index style file; if one
 does not use the \Makeindex{} script, this index style file shall drive the
 standard \mkndx{} program that comes with every \tex{} system distribution;
 \item\index{Makefile script@\Makefile{} (script)}
 the \Makefile{} script.
 It can drive the \make{} utility, but this mechanism is also completely
 optional; if it is actually used, it will automatically keep the final
 document (in PostScript format) current, based on differences in the
 modification times of the \tex{} source files that the PostScript target is
 dependent on.
\end{itemize}

Though the \oh{} package scrupulously respects each and every typesetting
guideline designed by \Hermes{}, it does its utmost to achieve this aim
without sacrificing compatibility with standard \latex{} system commands and
environments.
One should therefore use those standard structures in the usual way, with the
usual syntax and semantics (some have indeed been internally redefined in
order to comply with specific \hermes{} guidelines, but this is transparent
from the user's point of view).

However, there are few cases where compliance could not be achieved through
the redefinition of existing standard \latex{} commands and environments.
In that case, we provide the user with a new structure whose name always has
an initial capital letter (this distinguishes new structures from standard
ones).
Here is the complete list of these new commands and environments (see also
Table~\ref{tab:new_commands}):
\begin{itemize}
 \item\index{ChapterAuthor@\cmd{ChapterAuthor}}
 the \cmd{ChapterAuthor} command.
 It is used in the case of edited collections (joint works) and starts a new
 chapter that has been written by specific authors;
 \item\index{Name@\cmd{Name}}
 the \cmd{Name} command.
 It mentions the name of an author and typesets it in small capitals font
 shape;
 \item\index{PresentationOfAuthors@\cmd{PresentationOfAuthors}}
 the \cmd{PresentationOfAuthors} command.
 It introduces specifically the names of the authors of an unnumbered chapter
 (such as a foreword or an introduction);
 \item\index{Figure@\env{Figure}}\index{Table@\env{Table}}
 the \env{Figure} and \env{Table} environments.
 These are functionally similar to their standard counterparts (namely, the
 \env{figure} and \env{table} standard \latex{} environments), but the
 caption and the label are now specified as the mandatory argument of the new
 environments.
 This mechanism controls the caption positioning;
 \item\index{GenericRemark@\cmd{GenericRemark}}
 the \cmd{Remark},\index{Remark@\cmd{Remark}}
 \cmd{Example},\index{Example@\cmd{Example}}
 and \cmd{Note}\index{Note@\cmd{Note}} commands.
 These typeset a \emph{short} text as a comment, an example, or a note.
 Actually, any similar inset may be designed with the \cmd{GenericRemark}
 command;
 \item
 the \env{Remarks},\index{Remarks@\env{Remarks}}
 \env{Examples},\index{Examples@\env{Examples}}
 and \env{Notes}\index{Notes@\env{Notes}} environments.
 These typeset a \emph{longer} text (made up of two or more paragraphs) as a
 comment, an example, or a note;
 \item\index{GenericDefinition@\cmd{GenericDefinition}}
 the \cmd{Definition},\index{Definition@\cmd{Definition}}
 \cmd{Theorem},\index{Theorem@\cmd{Theorem}}
 and \cmd{Lemma}\index{Lemma@\cmd{Lemma}} commands.
 These typeset text as a definition, a theorem, or a lemma.
 Actually, any similar inset may be designed with the \cmd{GenericDefinition}
 command;
 \item\index{Publisher@\cmd{Publisher}}
 the \cmd{Publisher} command.
 It typesets (at the end of the book) a note intended for the \hermes{}
 office.
\end{itemize}

We have also introduced some supplementary commands that will certainly be
seldom used, but that can be a help to \latex{} novices (see also
Table~\ref{tab:new_commands}):
\begin{itemize}
 \item the \cmd{DelayNewPage}\index{DelayNewPage@\cmd{DelayNewPage}} and
 \cmd{ForceNewPage}\index{ForceNewPage@\cmd{ForceNewPage}} commands.
 These control \tex{}'s page-breaking mechanism in the remaining problematic
 cases where \tex{}'s decision is not satisfactory;
 \item\index{EqnCont@\cmd{EqnCont}}
 the \cmd{EqnCont} command.
 Within the \env{eqnarray} environment, it breaks a long equation into two
 lines and helps to stay in the type area;
 \item\index{AddToContents@\cmd{AddToContents}}
 the \cmd{AddToContents} command.
 It introduces some specific text in the table of contents;
 \item the \cmd{CropMarksOn}\index{CropMarksOn@\cmd{CropMarksOn}} and
 \cmd{CropMarksOff}\index{CropMarksOff@\cmd{CropMarksOff}} commands.
 These respectively activate and disable crop marks on output pages;
 \item the \cmd{LargeFnRule}\index{LargeFnRule@\cmd{LargeFnRule}} and
 \cmd{SmallFnRule}\index{SmallFnRule@\cmd{SmallFnRule}} commands.
 These respectively produce long (namely,~12\,cm) and short (namely,~25\,mm)
 footnote rules.
 These commands shall be used whenever a footnote exceptionally extends over
 two or more pages.
\end{itemize}

\begin{Table}[!htbp]{%
New structures introduced by the \oh{} package\label{tab:new_commands}}
\setlength{\tabcolsep}{4pt}
\renewcommand{\arraystretch}{1.2}
 \begin{tabular}{|l|l|l|}
  \hline
  \multicolumn{1}{|c|}{\textbf{Command or environment}}
  & \multicolumn{1}{c|}{\textbf{Description}}
  & \multicolumn{1}{c|}{\textbf{Scope}} \\ \hline
  \cmd{ChapterAuthor[]\{\}\{\}}
  \index{ChapterAuthor@\cmd{ChapterAuthor}}
  & start a chapter with specific authors & collections \\
  \cmd{Name\{\}}
  \index{Name@\cmd{Name}}
  & typeset a surname in small capitals & all books \\
  \cmd{PresentationOfAuthors\{\}}
  \index{PresentationOfAuthors@\cmd{PresentationOfAuthors}}
  & introduce author full names & collections \\
  \env{Figure[]\{\}}
  \index{Figure@\env{Figure}}
  & control typesetting of floating figures & all books \\
  \env{Table[]\{\}}
  \index{Table@\env{Table}}
  & control typesetting of floating tables & all books \\
  \cmd{Remark\{\}}
  \index{Remark@\cmd{Remark}}
  & typeset a small comment (single paragraph) & all books \\
  \cmd{Example\{\}}
  \index{Example@\cmd{Example}}
  & typeset a small example (single paragraph) & all books \\
  \cmd{Note\{\}}
  \index{Note@\cmd{Note}}
  & typeset a small note (single paragraph) & all books \\
  \cmd{GenericRemark\{\}\{\}}
  \index{GenericRemark@\cmd{GenericRemark}}
  & define a new ``comment-like'' inset & all books \\
  \env{Remarks}
  \index{Remarks@\env{Remarks}}
  & typeset a longer comment & all books \\
  \env{Examples}
  \index{Examples@\env{Examples}}
  & typeset a longer example & all books \\
  \env{Notes}
  \index{Notes@\env{Notes}}
  & typeset a longer note & all books \\
  \cmd{Definition\{\}}
  \index{Definition@\cmd{Definition}}
  & typeset a definition (italic shape) & all books \\
  \cmd{Theorem\{\}}
  \index{Theorem@\cmd{Theorem}}
  & typeset a theorem (italic shape) & all books \\
  \cmd{Lemma\{\}}
  \index{Lemma@\cmd{Lemma}}
  & typeset a lemma (italic shape) & all books \\
  \cmd{GenericDefinition\{\}\{\}}
  \index{GenericDefinition@\cmd{GenericDefinition}}
  & define a new ``definition-like'' inset & all books \\
  \cmd{Publisher\{\}\{\}\{\}}
  \index{Publisher@\cmd{Publisher}}
  & produce a note for the \hermes{} office & optional \\ \hline
  \cmd{DelayNewPage\{\}}
  \index{DelayNewPage@\cmd{DelayNewPage}}
  & control \tex{}'s page breaks & optional \\
  \cmd{ForceNewPage}
  \index{ForceNewPage@\cmd{ForceNewPage}}
  & control \tex{}'s page breaks & optional \\
  \cmd{EqnCont}
  \index{EqnCont@\cmd{EqnCont}}
  & break a long equation into two lines & optional \\
  \cmd{AddToContents\{\}\{\}\{\}}
  \index{AddToContents@\cmd{AddToContents}}
  & introduce text in the table of contents & optional \\
  \cmd{CropMarksOn}
  \index{CropMarksOn@\cmd{CropMarksOn}}
  & activate crop marks on output pages & optional \\
  \cmd{CropMarksOff}
  \index{CropMarksOff@\cmd{CropMarksOff}}
  & disable crop marks on output pages & optional \\
  \cmd{LargeFnRule}
  \index{LargeFnRule@\cmd{LargeFnRule}}
  & make long footnote rules & all books \\
  \cmd{SmallFnRule}
  \index{SmallFnRule@\cmd{SmallFnRule}}
  & make short footnote rules & all books \\ \hline
 \end{tabular}
\end{Table}

The \oh{} package shall be used in order to typeset either a monograph
or an edited collection (see section~\ref{type-ouvrage} on
page~\pageref{type-ouvrage} for the definition of these terms).
Though this user's guide is basically laid out as an edited collection,
it does exhibit some features that are only encountered in the case of
monographs; its general layout is somewhat hybrid.
For this very reason, authors should definitely \emph{not} use the \latex{}
source files of this user's guide as the starting point of the \latex{}
source files of any actual book.
Instead, the \dir{Ouvrage-Hermes} top-level directory holds appropriate
templates of master \latex{} source files.
On the other hand, local excerpts from this user's guide may provide one with
examples of good typographic practice.

Chapter~\ref{chap-struct} describes structure commands.
These commands shall be used in order to lay out any actual book; they
structure the whole document and all its components (\latex{} preamble, front
matter, main matter, and back matter; these great subdivisions may then
contain parts, chapters, sections, paragraphs, and all other components down
to the smallest units).
Chapter~\ref{chap-autres-commandes} describes new commands and environments
that are introduced by the \oh{} package through its \class{ouvrage-hermes}
\latex{} document class file.
Excerpts from the \Hermes{} guidelines and instructions to authors are given
wherever appropriate.
The final bibliography and the index have been produced respectively by the
\bibtex{} program (with the mandatory \bstyle{ouvrage-hermes} style file) and
by the suggested \Makeindex{} Bourne shell script (with the mandatory
\istyle{ouvrage-hermes} style file and the optional \awk{} utility and
scripts).

%%% Local Variables: 
%%% mode: latex
%%% TeX-master: "users-guide.ltx"
%%% End: 
