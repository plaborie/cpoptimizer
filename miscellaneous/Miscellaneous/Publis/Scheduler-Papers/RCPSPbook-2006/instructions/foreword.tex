\chapter*{Foreword}
\addtocontents{toc}{\protect\contentsline{chapter}{Foreword}{\thepage}}
\addtocontents{toc}{\protect\TextInToc{0}{0pt}{%
 Roger \Name{Rousseau}, Christian \Name{Scheen}}}
\addtocontents{toc}{\protect\addvspace{5pt plus 1pt minus 1pt}}

The aim of this document is to describe how to use \latex{} in order to
typeset books (namely, monographs or edited collections; this terminology
will be defined later on in this document) according to the
\Hermes{}~(\HSP{}) guidelines and instructions to authors.
The \oh{} package contains a \latex{} document class file, a \bibtex{}
alphanumeric bibliography style file, and a \mkndx{} index style file.
Index typesetting is made easier through a basic Bourne shell script and two
associated \awk{} (or \gawk{}) filters.
Of course, the use of this indexing mechanism is completely optional;
everything is possible (but sometimes more involved) through the standard
\mkndx{} program that comes with every \tex{} system distribution.
Finally, the \oh{} package contains a \Makefile{} script for the \make{}
utility; its use is also optional and makes it easy to keep the main target
(namely, the final document in PostScript format) current, based on
differences in the modification times of the \tex{} source files that the
PostScript target is dependent on.

Let us stress that users of the \oh{} package should definitely \emph{not}
use the \latex{} source files of this user's guide as the starting point of
the \latex{} source files of any actual book (this user's guide is a very
special case that is laid out as an edited collection, but that also
possesses some monograph features).
The \oh{} package provides users with appropriate templates instead; see the
\file{hsp-monograph.ltx} and \file{hsp-treatise.ltx} files in the
\dir{Ouvrage-Hermes} top-level directory for details.

\latex{} itself is Leslie Lamport's generic typesetting system~\cite{Lam94}
that uses Donald E.\@~Knuth's \tex{} formatting system~\cite{Knu86} as its
underlying engine.
The main idea behind \latex{} is to let the user concentrate on the layout
and the structure of the document rather than on formatting details.
With this aim in mind, the \latex{} system adds an abstraction layer onto the
plain \tex{} commands; the user is provided with high-level commands that
make document typesetting easier.
The \latex{} format contains a precompiled image of all these \latex{}
commands, the \tex{} font metrics~(``.tfm'') information for preloaded fonts,
and a set of word-breaking hyphenation patterns for each language that one
might want to use.
A number of excellent books on \latex{} and related tools and techniques are
available.
In addition to the books by Knuth~\cite{Knu86} and Lamport~\cite{Lam94}, let
us refer to the introduction guide by \hbox{Helmut} Kopka and Patrick
W.\@~Daly~\cite{KD04}, to the editions of the \latex{} ``companion'' by Frank
Mittelbach \emph{et~al.\@}~\cite{GMS94, MGBCRDS04}, to the \latex{} graphics
``companion'' by Michel Goossens \emph{et~al.\@}~\cite{GRM97}, and to the
\latex{} Web ``companion'' by Michel Goossens \emph{et~al.\@}~\cite{GRGMS99}.
French-speaking users are moreover referred to the introduction guides by
Christian Rolland~\cite{Rol99} and Bernard Desgraupes~\cite{Des03}.

\section*{Typographic conventions}
\addtocontents{toc}{\protect\contentsline{section}{%
 Typographic conventions}{\thepage}}
\index{typographic convention}

We use the standard typographic conventions throughout the document (see for
instance~\cite[p.\@~\hbox{11-13}]{MGBCRDS04}):
\begin{itemize}
 \item programs (such as \mkndx{}, \xindy{}, \make{}, \awk{}, and \gawk{})
 and drivers (such as \dvips{} and \pstopdf{}) are typeset in sans serif text
 font;
 \item scripts (such as \Makefile{} and \Makeindex{}) and so-called filters
 (such as \preawk{} and \postawk{}) are typeset in monospaced typewriter text
 font;
 \item \LaTeX{}, \bibtex{}, and \mkndx{} classes and style files (such as
 \class{book}, \style{chapterbib}, \bstyle{ouvrage-hermes}, and
 \istyle{ouvrage-hermes}) are typeset in sans serif text font;
 \item \LaTeX{} commands and environments (such as \cmd{makeindex} and
 \env{flushright}) are typeset in monospaced typewriter text font.
 Besides, environments are typeset in slanted shape font;
 \item place-holders (also known as meta-variables) are typeset in italic
 shape font between ``$\langle$''~and~``$\rangle$'' angle brackets (for
 instance, \meta{text}).
\end{itemize}


\vspace{11pt}

\begin{flushright}
 Roger \Name{Rousseau}\\
 Christian \Name{Scheen}
\end{flushright}

%%% Local Variables: 
%%% mode: latex
%%% TeX-master: "users-guide.ltx"
%%% End: 
