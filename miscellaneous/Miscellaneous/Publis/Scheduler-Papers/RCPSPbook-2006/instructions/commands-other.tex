\ChapterAuthor[%
Other Specific Commands]{%
Other Specific Commands}{%
Roger \Name{Rousseau}\and Christian \Name{Scheen}}
\label{chap-autres-commandes}

This chapter describes the specific commands defined by the \oh{} package.
These new commands have been introduced in order to comply with special
\Hermes{} guidelines.
Of course, other standard \latex{} commands may be used in the usual way,
with the usual syntax and semantics.

\section{Floating figures and tables}

In order to comply with the \hermes{} guidelines, one should use the new
\env{Figure}\index{Figure@\env{Figure}} and
\env{Table}\index{Table@\env{Table}} environments, not the standard \latex{}
\env{figure}\index{figure@\env{figure}} and
\env{table}\index{table@\env{table}} environments; the syntax is slightly
different.

\subsection{An environment for figures}
\label{ssec:figenv}

All figures (namely, those produced directly by \latex{} commands and those
included with the \style{epsfig}\index{epsfig.sty@\style{epsfig}} or
\style{graphicx}\index{graphicx.sty@\style{graphicx}} style file) shall be
typeset with the new \env{Figure}\index{Figure@\env{Figure}} environment.

\GenericRemark{Syntax}{%
The \env{Figure} syntax is as follows:}
\begin{quote}
\index{Figure@\env{Figure}}
\index{label@\cmd{label}}
\index{includegraphics@\cmd{includegraphics}}
\index{epsfig@\cmd{epsfig}}
 \cmd{begin\{Figure\}[\meta{place}]\{\meta{text}\cmd{label\{\meta{ref}\}}\}}
 \\
 \hspace*{5mm}
 [\cmd{includegraphics[...]\{...\}} or \cmd{epsfig\{...\}} commands]
 \\
 \cmd{end\{Figure\}}
\end{quote}
\pagebreak

The arguments are as follows:
\begin{itemize}
 \item\index{float placement specifier}
 \meta{place} is the float placement specifier (``\verb+h+''~means \emph{here
 if possible}, ``\verb+t+''~means \emph{top of page}, ``\verb+b+''~means
 \emph{bottom of page}, and ``\verb+p+''~means \emph{page of floats});
 \item\index{caption text}
 \meta{text} is the complete text of the caption (if desired, one can use
 the~\bsbs{}\index{backslashbackslash@\bsbs{}} line-breaking command at
 appropriate places within \meta{text});
 \item\index{label internal key}
 \meta{ref} is the label internal key, for cross-references only.
 The \cmd{label}\index{label@\cmd{label}} command shall be stated as part of
 the mandatory argument of the opening declaration.
\end{itemize}

There exists a number of commands that are able to crop, shrink or enlarge,
and include (encapsulated) PostScript figure files:
\begin{itemize}
 \item\index{epsfig.sty@\style{epsfig}}
 \cmd{epsfig}\index{epsfig@\cmd{epsfig}} (from the \style{epsfig} style file)
 is described in~\cite[p.\@~317-320]{GMS94}, but may be considered obsolete;
 \item\index{graphicx.sty@\style{graphicx}}
 \cmd{includegraphics}\index{includegraphics@\cmd{includegraphics}} (from the
 \style{graphicx} style file) is described in~\cite[p.\@~613-628]{MGBCRDS04};
 it is strongly recommended.
\end{itemize}

\Example{%
The following code produces Figure~\ref{figA}.
It modifies the final height of the included figure; its width is computed
automatically:}
\begin{quote}
 \begin{verbatim}
\begin{Figure}[!htbp]{<text>\label{figA}}
 \includegraphics[height=175pt]{12.example-figure.eps}
\end{Figure}
 \end{verbatim}
\end{quote}

\begin{Figure}[!htbp]{%
Short captions (width~$w$ such that $w < \mbox{36\,cm}$) are automatically
centered on one, two, or three lines; the standard line-breaking command
(namely,~\textup{\bsbs{}}) is available, but its use is optional.
Long captions (width~$w$ such that $w \geq \mbox{36\,cm}$), like the present
one, are automatically justified.\label{figA}}
\includegraphics[height=175pt]{12.example-figure.eps}
\end{Figure}

Users of the \style{epsfig}\index{epsfig.sty@\style{epsfig}} style file
might also want to use our \cmd{EpsFigure}\index{EpsFigure@\cmd{EpsFigure}}
command.
It adds two arguments to the aforementioned\index{Figure@\env{Figure}}
\env{Figure} environment: the first argument is the final height of the
included figure; the second argument is the (EPS) file name.
The following command produces the same result as the preceding code:
\begin{quote}
\cmd{EpsFigure[!htbp]\{\meta{text}\cmd{label\{figA\}}\}\%}\\
\hspace*{5mm}\texttt{\{175pt\}\{12.example-figure.eps\}}
\end{quote}
\indent
The final width of the included figure is computed automatically.
If the height argument is empty, the original dimensions are used.

\subsection{An environment for tables}

Tables are typeset, as usual, with the\index{tabular@\env{tabular}}
\env{tabular}, \env{tabular*},\index{tabular*@\env{tabular*}}
\env{tabularx},\index{tabularx@\env{tabularx}} and\index{array@\env{array}}
\env{array} \latex{} standard environments.
Table structures shall then be placed inside our\index{Table@\env{Table}}
\env{Table} environment: it takes care of the required font size and sets
the line thickness to~0.25\,pt.

\GenericRemark{Syntax}{%
The \env{Table} syntax is as follows:}
\begin{quote}
\index{Table@\env{Table}}
\index{tabular@\env{tabular}}
 \cmd{begin\{Table\}[\meta{place}]\{\meta{text}\cmd{label\{\meta{ref}\}}\}}
 \\
 \hspace*{5mm}
 [table construction (with \env{tabular},~etc.\@ environments)]
 \\
 \cmd{end\{Table\}}
\end{quote}
\noindent
where the float placement specifier \meta{place}, the complete text of the
caption \meta{text}, and the label internal key \meta{ref} are used as in the
case of a figure (see paragraph~\ref{ssec:figenv}).

\section{Special insets}

\subsection{Comments, examples, notes, and similar insets}

Comments, examples, and notes shall be typeset with our
\cmd{Remark\{\meta{text}\}},\index{Remark@\cmd{Remark}}
\cmd{Example\{\meta{text}\}},\index{Example@\cmd{Example}} and
\cmd{Note\{\meta{text}\}}\index{Note@\cmd{Note}} commands; their mandatory
argument \meta{text} holds the text of the comment, example, or note.
Other similar insets shall use the\index{GenericRemark@\cmd{GenericRemark}}
\cmd{GenericRemark\{\meta{keyword}\}\{\meta{text}\}} command.

\GenericRemark{Syntax}{%
In order to typeset this ``syntax'' inset, the following command
with appropriate \meta{text} argument shall be used:}
\begin{quote}
\index{GenericRemark@\cmd{GenericRemark}}
 \cmd{GenericRemark\{Syntax\}\{\meta{text}\}}
\end{quote}
\indent
All these insets may contain several paragraphs; the vertical skip between
paragraphs is adjusted automatically.
Users who would rather typeset \emph{environments} (and not \emph{commands})
may use the \env{Remarks},\index{Remarks@\env{Remarks}}
\env{Examples},\index{Examples@\env{Examples}} and
\env{Notes}\index{Notes@\env{Notes}} environments (paying special attention
to the final letter~``\texttt{\textit{s}}'').

\GenericRemark{Syntax}{%
The \env{Remarks} syntax is as follows:}
\begin{quote}
\index{Remarks@\env{Remarks}}
 \cmd{begin\{Remarks\}}\\
 \hspace*{5mm}\meta{paragraphs separated by blank lines}\\
 \cmd{end\{Remarks\}}
\end{quote}

\Remark{%
Command names (\cmd{Remark} and \cmd{GenericRemark}) and environment names
(\env{Remarks}) have been choosen so as to be unambiguous and compatible
with the classical notion of a \emph{commented code} in the \tex{}/\latex{}
sense.}
\index{Remark@\cmd{Remark}}
\index{GenericRemark@\cmd{GenericRemark}}
\index{Remarks@\env{Remarks}}

\subsection{Definitions, theorems, lemmas, and similar insets}

\newcommand{\Axiom}[1]{\GenericDefinition{Axiom}{#1}}
\newcounter{statement}[chapter]
\renewcommand{\thestatement}{\thechapter.\arabic{statement}}
\newcommand{\NumberedStatement}[1]{%
 \refstepcounter{statement}%
 \GenericDefinition{Statement~\thestatement{}}{#1}}

Definitions, theorems, and lemmas shall be typeset with
\cmd{Definition\{\meta{text}\}},\index{Definition@\cmd{Definition}}
\cmd{Theorem\{\meta{text}\}},\index{Theorem@\cmd{Theorem}} and
\cmd{Lemma\{\meta{text}\}}\index{Lemma@\cmd{Lemma}} commands; their mandatory
argument \meta{text} holds the text of the definition, theorem, or lemma (the
text is automatically typeset in italic shape).
These insets are generally rather short; for this reason, environment-based
counterparts are not provided.
Other similar insets shall\index{GenericDefinition@\cmd{GenericDefinition}}
use the \cmd{GenericDefinition\{\meta{keyword}\}\{\meta{text}\}} command.

\Axiom{%
In order to typeset this ``axiom'' inset, the following command
with appropriate \meta{text} argument shall be used:}
\begin{quote}
\index{GenericDefinition@\cmd{GenericDefinition}}
 \cmd{GenericDefinition\{Axiom\}\{\meta{text}\}}
\end{quote}
\indent
If one wishes to typeset several similar insets (for instance, several
axioms), one should define a new command (for instance, the \cmd{Axiom}
command) in the preamble of the master \latex{} source file (next, any
\cmd{Axiom\{\meta{text}\}} command will produce the desired effect); for
instance:
\begin{quote}
\index{GenericDefinition@\cmd{GenericDefinition}}
 \cmd{newcommand\{\cmd{Axiom}\}[1]\{\cmd{GenericDefinition\{Axiom\}\{\#1\}}\}}
\end{quote}
\indent
If one wishes to equip definitions, theorems, or lemmas with an automatic
numbering, one should proceed according to the following example in the
preamble of the master \latex{} source file:
\begin{quote}
\index{GenericDefinition@\cmd{GenericDefinition}}
 \begin{verbatim}
\newcounter{statement}[chapter]
\renewcommand{\thestatement}{\thechapter.\arabic{statement}}
\newcommand{\NumberedStatement}[1]{%
 \refstepcounter{statement}%
 \GenericDefinition{Statement~\thestatement{}}{#1}}
 \end{verbatim}
\vspace{-11pt}
\end{quote}

\index{label@\cmd{label}}\index{ref@\cmd{ref}}
\NumberedStatement{\label{statement1}%
The cross-references mechanism is available within numbered insets owing to
the usual \latex{} commands, namely \textup{\cmd{label\{\meta{ref}\}}} and
\textup{\cmd{ref\{\meta{ref}\}}}.}

\NumberedStatement{\label{statement2}%
Statement~\ref{statement2} proves Statement~\ref{statement1}.}

\subsection{Quotes}

The standard \latex{} environment for extended quotes, namely
\env{quote},\index{quote@\env{quote}} has been redefined; it automatically
produces appropriate vertical skips, but its use remains standard.

\section{The optional form for the \hermes{} office}

The \cmd{Publisher}\index{Publisher@\cmd{Publisher}} command is optional;
it should be placed at the end of the book (after the index), where it will
bring together some information (phone \meta{phone}, fax \meta{fax}, and
email \meta{email} of the book director, and various typesetting data)
intended for the \hermes{} office.

\GenericRemark{Syntax}{%
The \cmd{Publisher} syntax is as follows:}
\begin{quote}
\index{Publisher@\cmd{Publisher}}
 \cmd{Publisher\{\meta{phone}\}\{\meta{fax}\}\{\meta{email}\}}
\end{quote}

\section{Exponents, indices, and numberings}
\index{exponent!}
\index{index!}
\index{numbering}

Mathematical exponents\index{exponent!mathematical} and mathematical
indices\index{index!mathematical} are typeset, of course, in mathematical
mode.
For instance, the Ricci curvature tensor is written as $R^{\beta}{}_{\delta}
\equiv R^{\mu \beta}{}_{\mu \delta}$ with the following code:
\begin{quote}
 \begin{verbatim}
$R^{\beta}{}_{\delta} \equiv R^{\mu \beta}{}_{\mu \delta}$
 \end{verbatim}
\vspace{-11pt}
\end{quote}
\indent
Textual exponents\index{exponent!textual} shall be typeset with the
\cmd{up}\index{up@\cmd{up}} command and textual indices\index{index!textual}
shall be typeset with the \cmd{down}\index{down@\cmd{down}} command; examples
are as follows:
\begin{itemize}
 \item\index{up@\cmd{up}}
 the code ``\texttt{1}\cmd{up\{st\}}'' produces ``1\up{st}'',
 \item\index{down@\cmd{down}}
 the code ``\texttt{H}\cmd{down\{2\}}\texttt{SO}\cmd{down\{4\}}'' produces
 ``H\down{2}SO\down{4}''.
\end{itemize}

Numberings shall be typeset with various commands that are defined by the
\style{babel}\index{babel.sty@\style{babel}} style file\footnote{The
\class{ouvrage-hermes} document class automatically loads the \style{babel}
style file with the required style options.}.
\pagebreak

Some examples are as follows:
\begin{itemize}
 \item\index{no@\cmd{no}}
 the code ``\cmd{no\{\}\~{}2}'' produces ``\no{}~2'',
 \item\index{No@\cmd{No}}
 the code ``\cmd{No\{\}\~{}2}'' produces ``\No{}~2'',
 \item\index{ier@\cmd{ier}}
 the code ``\texttt{1}\cmd{ier\{\}}'' produces ``1\ier{}'',
 \item\index{ieme@\cmd{ieme}}
 the code ``\texttt{2}\cmd{ieme\{\}}'' produces ``2\ieme{}'',
 \item etc.
\end{itemize}

\section{Specific control commands}

\subsection{Controlling the page-breaking mechanism}
\index{page break}

Under rare problematic circumstances, it might be useful to enlarge specific
pages by one or two lines and/or to force page breaks in order to coerce the
\tex{} typesetting engine into producing the desired presentation; \latex{}
commands \cmd{enlargethispage}\index{enlargethispage@\cmd{enlargethispage}}
and \cmd{pagebreak}\index{pagebreak@\cmd{pagebreak}} would come in very handy
in such a case.

The \class{ouvrage-hermes} document class introduces two more commands that
are arguably easier to use:
\begin{itemize}
 \item\index{DelayNewPage@\cmd{DelayNewPage}}
 the \cmd{DelayNewPage\{\meta{number}\}} command enlarges the current page
 by a number of lines equal to its mandatory argument \meta{number}.
 Sensible values are \meta{number}~$=$~1 and \meta{number}~$=$~2;
 \item\index{ForceNewPage@\cmd{ForceNewPage}}
 the \cmd{ForceNewPage} command forces a page break at its location in the
 source code.
 It does not indent text at the start of the next page.
\end{itemize}

\subsection{Controlling displayed mathematics}
\index{displayed mathematics}

Numbered formulas are typeset by the usual\index{equation@\env{equation}}
\env{equation} and\index{eqnarray@\env{eqnarray}} \env{eqnarray} \latex{}
environments.
They automatically produce the required type of numbering (see the example
of the Painlev�~\textsc{vi} equation~[\ref{eqn:P-VI}]).

Many long formulas need at least one possible break point and an associated
line-breaking decision.
The \cmd{EqnCont}\index{EqnCont@\cmd{EqnCont}} basic command may be used
at any appropriate break point, within the\index{eqnarray@\env{eqnarray}}
\env{eqnarray} environment:
\begin{eqnarray}
 \frac{\mathrm{d}^2 u}{\mathrm{d} x^2}
 & = &      \frac{1}{2}
            \left[ \frac{1}{u} + \frac{1}{u - 1} + \frac{1}{u - x} \right]
            \left( \frac{\mathrm{d} u}{\mathrm{d} x} \right)^2
          - \left[ \frac{1}{x} + \frac{1}{x - 1} + \frac{1}{u - x} \right]
            \frac{\mathrm{d} u}{\mathrm{d} x} \EqnCont
 &   & {} + \frac{u (u - 1) (u - x)}{x^2 (x - 1)^2}
            \left[   \alpha
                   + \frac{\beta x}{u^2}
                   + \frac{\gamma (x - 1)}{(u - 1)^2}
                   + \frac{\delta x (x - 1)}{(u - x)^2}
            \right] \label{eqn:P-VI}
\end{eqnarray}
\pagebreak

\indent
This equation corresponds to the following code (it uses the \cmd{EqnCont}
command\index{EqnCont@\cmd{EqnCont}} in order to split the right-hand side
and it aligns both sides with ampersands\index{ampersand}):
\begin{quote}
 \begin{footnotesize}
  \begin{verbatim}
\begin{eqnarray}
 \frac{\mathrm{d}^2 u}{\mathrm{d} x^2}
 & = &      \frac{1}{2}
            \left[ \frac{1}{u} + \frac{1}{u - 1} + \frac{1}{u - x} \right]
            \left( \frac{\mathrm{d} u}{\mathrm{d} x} \right)^2
          - \left[ \frac{1}{x} + \frac{1}{x - 1} + \frac{1}{u - x} \right]
            \frac{\mathrm{d} u}{\mathrm{d} x} \EqnCont
 &   & {} + \frac{u (u - 1) (u - x)}{x^2 (x - 1)^2}
            \left[   \alpha
                   + \frac{\beta x}{u^2}
                   + \frac{\gamma (x - 1)}{(u - 1)^2}
                   + \frac{\delta x (x - 1)}{(u - x)^2}
            \right] \label{eqn:P-VI}
\end{eqnarray}
  \end{verbatim}
 \end{footnotesize}
\vspace{-11pt}
\end{quote}

\subsection{Controlling the table of contents}
\index{table of contents}

The table of contents is produced automatically.
However, it is sometimes necessary to add further information into the
table of contents.
Any direct hand-editing of the table of contents would of course be
ill-advised: \latex{} refreshes the \texttt{\meta{master}.toc} file each
time it is run.
The solution is to use the new\index{AddToContents@\cmd{AddToContents}}
\cmd{AddToContents} command inside the source file itself: this ensures
that these adjustments are automatically taken into account.

\GenericRemark{Syntax}{%
The \cmd{AddToContents} syntax is as follows:}
\begin{quote}
\index{AddToContents@\cmd{AddToContents}}
 \cmd{AddToContents\{\meta{level}\}\{\meta{indent}\}\{\meta{text}\}}
\end{quote}
\noindent
where:
\begin{itemize}
 \item \meta{level} is the numerical level of the corresponding sectioning
 command (see Table~\ref{tab:Subtitles} on page~\pageref{tab:Subtitles} for
 details, especially column~2);
 \item \meta{indent} is a dimension corresponding to the total indentation
 from the left \hbox{margin};
 \item \meta{text} is the complete text that should appear in the table of
 contents.
 Fragile commands must be protected with the\index{protect@\cmd{protect}}
 \cmd{protect} command.
\end{itemize}

\Example{%
Here is an example for the \cmd{AddToContents} command:}
\begin{quote}
\index{AddToContents@\cmd{AddToContents}}
 \begin{verbatim}
\AddToContents{2}{10mm}{Further information}
 \end{verbatim}
\vspace{-11pt}
\end{quote}

\subsection{Controlling overruns}
\index{overrun}

The generated \texttt{\meta{master}.log} transcript file provides users with
some valuable information dealing with the typesetting process.
One important information deals with underfull\index{underfull box} and
overfull\index{overfull box} horizontal\index{hbox@\cmd{hbox}} (\cmd{hbox})
and vertical\index{vbox@\cmd{vbox}} (\cmd{vbox}) boxes.\index{box}
Underfull boxes are characterized by a ``badness'' number; overfull boxes are
much more serious and are associated with the value of the overrun in points
(1\,pt~$=$~$\frac{1}{72.27}$\,in~$\approx$~0.351\,mm).\index{point (unit)}
It is possible to visualize the overruns:
\begin{itemize}
 \item\index{cropmarks class option@\texttt{cropmarks} (class option)}
 the \texttt{cropmarks} class option (see section~\ref{opt-cropmarks}) draws
 the page limits and the text borders; this allows one to assess overruns.
 This class option acts at the level of the whole book;
 \item the \cmd{CropMarksOn}\index{CropMarksOn@\cmd{CropMarksOn}} and
 \cmd{CropMarksOff}\index{CropMarksOff@\cmd{CropMarksOff}} commands
 respectively activate and disable the crop marks plotting.
 These commands act at the local (page) level.
\end{itemize}

\subsection{Controlling footnote rules}
\index{footnote rule}
\LargeFnRule

As far as footnote rules are concerned, the \Hermes{} guidelines are as
follows:
\begin{itemize}
 \item at the \emph{start} of footnotes, the width of the footnote rule is
 equal to~25\,mm (default value);
 \item whenever a footnote extends over more than one single page, the width
 of the \emph{continuation} footnote rule is equal to~120\,mm (exceptional
 value).
 Of course, the setting must be back to normal~(25\,mm) as soon as the long
 footnote is over.
\end{itemize}

The \class{ouvrage-hermes} document class sets the default width of footnote
rules to~25\,mm.
The \cmd{LargeFnRule}\index{LargeFnRule@\cmd{LargeFnRule}} command shall be
used whenever a footnote extends over more than one single page; this command
must be located in the page where the footnote \emph{continues} (not where
the footnote \emph{starts}\footnote{In order to illustrate the point being
made here, the \cmd{LargeFnRule} command has been used on the current page
even though this is not necessary (the footnote starts and ends on the same
page).}).
Moreover, the \cmd{SmallFnRule}\index{SmallFnRule@\cmd{SmallFnRule}} command
shall be used as soon as the long footnote is over (on the next page); this
command brings the setting back to normal.


%%% Local Variables: 
%%% mode: latex
%%% TeX-master: "users-guide.ltx"
%%% End: 
