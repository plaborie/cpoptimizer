\documentclass[runningheads]{llncs}
\usepackage[numbers]{natbib}
\usepackage{amsmath}
\usepackage{amssymb}
\usepackage{pstricks}
\usepackage{pst-node}
\usepackage{float}
\usepackage{mdwlist}
\usepackage{enumerate}
\usepackage{soul}    % For text highlighting using \hl
\usepackage{listings}
\usepackage{bold-extra} % Support for bold typewriter

\def\cplx{ {\mathcal O} }

\DeclareMathOperator{\abs}{abs}
\DeclareMathOperator{\est}{est}
\DeclareMathOperator{\lct}{lct}
\DeclareMathOperator{\dur}{p}
\DeclareMathOperator{\capacity}{c}
\DeclareMathOperator{\energy}{e}
\DeclareMathOperator{\present}{present}
\DeclareMathOperator{\true}{true}
\DeclareMathOperator{\false}{false}
\DeclareMathOperator{\ect}{ect}
\DeclareMathOperator{\ECT}{Ect}
\DeclareMathOperator{\preECT}{preEct}
\DeclareMathOperator{\envelope}{Env}
\DeclareMathOperator{\extenv}{ExtEnv}
\DeclareMathOperator{\env1}{Env1}
\DeclareMathOperator{\fail}{fail}
\DeclareMathOperator{\LCut}{LCut}
\DeclareMathOperator{\lson}{left}
\DeclareMathOperator{\rson}{right}
\DeclareMathOperator{\Leaves}{Leaves}
\DeclareMathOperator{\Left}{Left}
\DeclareMathOperator{\Right}{Right}
\DeclareMathOperator{\rest}{rest}
\DeclareMathOperator{\before}{EndsBefore}
\DeclareMathOperator{\maxx}{maxX}

\newcommand{\capmax}{\overline{\capacity}}
\newcommand{\durmax}{\overline{\dur}}
\newcommand{\enermax}{\overline{\energy}}
\newcommand{\envmax}{\overline{\envelope}}

\newcommand{\lenergy}{\energy^\Lambda}
\newcommand{\lenvelope}{\envelope^\Lambda}

\newcommand{\TODO}{\hl{TODO}}
\newcommand{\todo}[1]{\hl{TODO: #1}}
\newcommand{\azaroven} {\; \& \;}

\def\eg{e.g.,\ }
\def\Eg{E.g.,\ }
\def\ie{i.e.,\ }
\def\Ie{I.e.,\ }

\newgray{barva1}{0.85}
\newgray{barva2}{0.80}
\newgray{barva3}{0.90}
\newgray{barva4}{0.70}

\newcommand{\sectionref}[1]{Section~\ref{#1}}
\newcommand{\tableref}[1]{Table~\ref{#1}}
\newcommand{\figureref}[1]{Figure~\ref{#1}}
\newcommand{\algref}[1]{Algorithm~\ref{#1}}
\newcommand{\propertyref}[1]{Property~\ref{#1}}
\newcommand{\lemmaref}[1]{Lemma~\ref{#1}}
\newcommand{\observationref}[1]{Observation~\ref{#1}}
\newcommand{\propositionref}[1]{Proposition~\ref{#1}}
\newcommand{\definitionref}[1]{Definition~\ref{#1}}

\renewcommand\bibsection{\section*{References}}

\lstloadlanguages{Pascal}
\lstset{
  language=Pascal,
  escapechar=\#,
  extendedchars,
  morekeywords={fail, continue, require, break},
  morecomment=[l]//,
  texcl,
  numbers=left,
  basicstyle=\rmfamily,
  keywordstyle=\ttfamily\bfseries,
  commentstyle=\itshape,
  stringstyle=\rmfamily,
  identifierstyle=\ttfamily,
  frame=htb
}
\renewcommand\lstlistingname{Algorithm}
\newcommand\skiplstline{\global\advance\c@lstnumber\@ne}
% To not count algorithms inside chapters even though \thechapter is defined by llncs:
\catcode`\@=11 % Viz texbook naruby, str. 21
\renewcommand\thelstlisting{\@arabic\c@lstlisting}
\catcode`\@=12

\newcommand{\keywords}[1]{\par\addvspace\baselineskip
\noindent {\bf Keywords:} \enspace\ignorespaces#1}


\begin{document}
\bibliographystyle{plainnat} 

\title{Max Energy Filtering Algorithm for Discrete Cumulative Resources}
\author{Petr Vil\'{\i}m}
\institute{ILOG, an IBM Company; 9, rue de Verdun, BP 85\\
           F-94253 Gentilly Cedex, France\\
           \email{petr\_vilim@cz.ibm.com}
           }

\maketitle

\begin{abstract}
In scheduling using constraint programming we usually reason only about possible start times and end times of
activities and remove those which are recognized as unfeasible. However often in practice there are more
variables in play: variable durations of activities and variable resource capacity requirements. This paper
presents a new algorithm for filtering maximum durations and maximum capacity requirements for discrete
cumulative resources. It is also able to handle optional interval variables introduced in IBM ILOG CP Optimizer
2.0. Time complexity of the algorithm is $\cplx(n \log n)$. The algorithm is based on never published
algorithm by Wim Nuijten and a on slightly modified e-feasibility checking algorithm by Armin Wolf and Gunnar
Schrader. The later algorithm is also described in the paper.
\keywords{Constraint Programming, Scheduling, Discrete Cumulative Resource, Propagation}
\end{abstract}

\section{Introduction}

Nowadays, constraint based scheduling engines like IBM ILOG CP Optimizer \cite{ILOG2008} allows to describe and
solve very complex scheduling problems involving a variety of different constraints. This paper is focused
on one of them -- discrete cumulative resource for the case when durations and/or individual capacity
requirements are not fixed. Traditionally we reason only about minimum start times and maximum end times using
algorithms like Edge Finding \cite{Mercier2005CumulativeEF}, Not-First/Not-Last
\cite{Schutt2005CumulativeNFNL} or Energetic Reasoning \cite{CBS}. This paper provides an algorithm for
filtering of maximum activity durations and maximum capacity requirements.

The algorithm is not completely new. Although it was never published, Wim Nuijten implemented a similar
algorithm for ILOG Scheduler several years ago. The old version of the algorithm has time complexity
$\cplx(n^2)$, this paper presents a faster version with time complexity $\cplx(n \log n)$.

\medskip

To demonstrate the problem on a simple example, lets consider the following subproblem: there is a pool of
$10$ workers (\ie a discrete capacity resource with maximum capacity $C=10$) who perform different tasks.
Among these tasks there is a task to produce one particular product~$P$. How many units of the product $P$ is
produced depends on how many workers are assigned to the tasks (\ie how much capacity of the resource is used)
and for how long (\ie what is the duration of the task):
\[
  \text{nbP} = \text{workers} \times \text{duration}
\]
If we do not produce at least $500$ units of product~$P$ then we will have to buy the rest for the following
cost (deduced from an initial budget):
\[
  \text{cost} = \max(0, 500 - \text{nbP}) \times 1\$
\]

In this example, the budget and production of product~$P$ are tightly connected:
\begin{enumerate*}
\item If we see that no more than $200$\$ can be invested into the purchase of product~$P$ (because the rest of
the budget is needed for other things) then we need to allocate workers to produce at least $300$ units of
product~$P$.
\item On the other hand if we see that there is no way to produce more than $100$ units of product~$P$
(because the workers are needed for other tasks) we can immediately allocate $400$\$ from the budget to
buy remaining products~$P$. This is a critical propagation especially if the budget is short.
\end{enumerate*}
Both propagations above are very important for speeding up the search by better pruning the search tree.
However for the propagation $2$ it is necessary to be able to compute the maximum possible production of
product~$P$. And this is the topic of the paper.

\medskip

The algorithm presented in the paper is also useful if there are optional activities -- activities which may
or may not be present in the solution (for example alternatives between several resources). In this case the
algorithm can detect that there is no way to process an optional activity and therefore it cannot be present
in the solution (and, in case of an alternative, another alternative must be chosen), see
\cite{Laborie2008ConditionalTI, ILOG2008}.

\section{Notation}

Let us formalize the problem. There is a set~$T$ of $n=|T|$ non-preemptive non-optional activities. For the
first part of the paper we assume that none of the activities in~$T$ is optional, that is, all activities in~$T$
are necessarily present in the solution. After we present Max Energy algorithm for non-optional activities we
will show how to use it for optional activities.

Each activity $i \in T$ is described by the following attributes:
\begin{itemize*}
\item the earliest possible starting time $\est_i \in \mathbb N$,
\item the latest possible completion time $\lct_i \in \mathbb N$,
\item the minimum processing time (duration) $\dur_i \in \mathbb N$,
\item the maximum processing time (duration) $\durmax_i \in \mathbb N$.
\end{itemize*}
Moreover, each activity $i \in T$ consumes during its processing some capacity of a resource. The capacity
consumption during the whole processing of the activity is constant, however it may not be known in advance.
In this case there is a range of possible capacity consumption:
\begin{itemize*}
\item the minimum required capacity $\capacity_i \in \mathbb N$,
\item the maximum required capacity $\capmax_i \in \mathbb N$.
\end{itemize*}
The resource can process several activities at the same time, however at any time the total used capacity
cannot exceed the maximum resource capacity~$C$. For an example see \figureref{fig:actexample}.

\begin{figure}[htb]
\centering
\psset{unit=0.9cm, arrowsize=5pt}
\begin{pspicture}(-5, -1.5)(6.3, 1.5)
\psline[linewidth=1.3pt](-4.8, -0.7)(-5, -0.7)(-5, 0.7)(-4.8, 0.7)
\rput[t](-5, -0.8){$\est_i = 0$}
\psline[linewidth=1.3pt](4.8, -0.7)(5, -0.7)(5, 0.7)(4.8, 0.7)
\rput[t](5, -0.8){$\lct_i = 20$}
\psframe[fillstyle=vlines, hatchcolor=barva2](-2.5, -0.5)(2.5, 0.5)
\psframe[fillstyle=solid, fillcolor=barva2](-1, -0.25)(1, 0.25)
\rput(0,0){$i$}
\psline{->}(2.5, 0)(5,0)
\psline{->}(-2.5, 0)(-5, 0)
\psline(-5.3, -1.5)(6.3, -1.5)
\psline(-5.3,  1.5)(6.3,  1.5)
\psset{linewidth=0.25pt, arrowsize=3pt}
\psline{<->}(-1, -0.6)(1, -0.6)
\rput[t](0, -0.7){$\dur_i = 4$}
\psline{<->}(-2.5, 0.6)(2.5, 0.6)
\rput[b](0, 0.7){$\durmax_i = 10$}
\psline{<->}(-2.4, -0.5)(-2.4, 0.5)
\rput[l](-2.3, 0){$\capmax_i = 2$}
\psline{<->}(1.1, -0.25)(1.1, 0.25)
\rput[l](1.2, 0){$\capacity_i = 1$}
\psline{<->}(6, -1.5)(6, 1.5)
\rput[l](6.1, 0){$C=6$}
\psset{linewidth=0.15pt}
\psline(-5, -1.35)(-5, -1.65)     \rput[t](-5, -1.8){\scriptsize $0$}
\psline(-4.5, -1.4)(-4.5, -1.6)
\psline(-4.0, -1.4)(-4.0, -1.6)
\psline(-3.5, -1.4)(-3.5, -1.6)
\psline(-3.0, -1.4)(-3.0, -1.6)
\psline(-2.5, -1.35)(-2.5, -1.65) \rput[t](-2.5, -1.8){\scriptsize $5$}
\psline(-2.0, -1.4)(-2.0, -1.6)
\psline(-1.5, -1.4)(-1.5, -1.6)
\psline(-1.0, -1.4)(-1.0, -1.6)
\psline(-0.5, -1.4)(-0.5, -1.6)
\psline( 0.0, -1.35)( 0.0, -1.65) \rput[t](-0, -1.8){\scriptsize $10$}
\psline( 0.5, -1.4)( 0.5, -1.6)
\psline( 1.0, -1.4)( 1.0, -1.6)
\psline( 1.5, -1.4)( 1.5, -1.6)
\psline( 2.0, -1.4)( 2.0, -1.6)
\psline( 2.5, -1.35)( 2.5, -1.65) \rput[t](2.5, -1.8){\scriptsize $15$}
\psline( 3.0, -1.4)( 3.0, -1.6)
\psline( 3.5, -1.4)( 3.5, -1.6)
\psline( 4.0, -1.4)( 4.0, -1.6)
\psline( 4.5, -1.4)( 4.5, -1.6)
\psline( 5.0, -1.35)( 5.0, -1.65) \rput[t](5, -1.8){\scriptsize $20$}
\psline( 5.5, -1.4)( 5.5, -1.6)
\end{pspicture}
\caption{An example of an activity~$i$ on a resource with capacity $C=6$.}
\label{fig:actexample}
\end{figure}

Another way to characterize an activity is its energy. Informally, energy of an activity is:
\[
\text{energy} = \text{processingTime} \times \text{capacity}
\]
Because processing time and/or required capacity may be unbound, we characterize the energy of a task~$i$
by two numbers: minimum energy $\energy_i = \capacity_i \dur_i$ and maximum energy $\enermax_i = \capmax_i
\durmax_i$. The presented algorithm modifies maximum energy $\enermax_i$ and this way also maximum capacity
$\capmax_i$ and maximum processing time $\durmax_i$: 
\begin{align}
%  \energy_i    &:= \max\left\{\capacity_i \dur_i,\; \energy_i \right\}  \\
%  \enermax_i   &:= \min\left\{\capmax_i \durmax_i,\; \enermax_i\right\} \\
%  \capacity_i  &:= \max\left\{ \left\lceil  \energy_i  / \durmax_i \right\rceil,\; \capacity_i \right\} \\
  \capmax_i    &:= \min\left\{ \left\lfloor \enermax_i / \dur_i \right\rfloor,\; \capmax_i \right\} 
    \label{rule:capmax} \\
%  \dur_i       &:= \max\left\{ \left\lceil \energy_i / \capmax_i \right\rceil,\; \dur_i \right\}  \\
  \durmax_i    &:= \min\left\{ \left\lfloor \enermax_i / \capacity_i \right\rfloor,\; \durmax_i \right\}
    \label{rule:durmax}
\end{align}

%\begin{tabular}{ll}
%{\bf Symbol} & {\bf Description} \\
%\hline
%$T$            & Set of activities on the resource.\\
%$n$            & Number of activities on the resource: $n = |T|$.\\
%$C$            & Capacity of the resource.\\
%%$\present_i$ & Domain of execution status, subset of $\{\true, \false\}$.\\
%%$R$            & Set of regular activities on the resource: 
%%                 $R = \left\{i,\; i \in T \azaroven \present_i=\left\{\true\right\}\right\}$ \\
%%$O$            & Set of optional activities on the resource: 
%%                 $O = \left\{i,\; i \in T \azaroven \present_i=\left\{\true,\false\right\}\right\}$ \\
%$\est_i$       & Earliest starting time (startMin) of the activity~$i$.\\
%$\lct_i$       & Latest completion time (endMax) of the activity~$i$. \\
%$\dur_i$       & Minimum duration (processing time) of the activity~$i$. \\
%$\capacity_i$  & Minimum capacity demanded by the activity~$i$ during the processing.\\
%$\energy_i$    & Minimum energy demanded by the activity~$i$.\\
%$\durmax_i$    & Maximum duration (processing time) of the activity~$i$.\\
%$\capmax_i$    & Maximum capacity demand by the activity~$i$ during the processing.\\
%$\enermax_i$   & Maximum energy demanded by the activity~$i$.\\
%%$k$            & Number of distinct capacity demands: $k = \left|\left\{ c_i,\; i \in T\right\}\right|$.\\
%\hline
%\end{tabular}

\subsection{Earliest Completion Time, Energy Envelope}

For the following algorithms we need a way to quickly estimate the earliest completion time of any set of
activities $\Theta \subseteq T$. If~$\Theta$ contains only one activity~$i$ then the computation of the
earliest completion time is simple:
\[
\ect_i = \est_i + \dur_i
\]
However in the general case it is much more complicated. Therefore we are looking for a good lower bound,
traditionally defined as:
\[
\preECT\left(\Theta\right) = \est_\Theta + \left\lceil \frac{\energy_\Theta}{C} \right\rceil
\]
where:
\begin{align*}
  \est_\Theta    &= \min_{i \in \Theta} \{\est_i\} \\
  \energy_\Theta &= \sum_{i \in \Theta} \energy_i
\end{align*}

Note that $\preECT$ is only a lower bound. For example:
\begin{enumerate}[(i)]
  \item \label{ect:nosubset} If we closely inspect all subsets $\Omega \subseteq \Theta$ we can achieve
    better estimation of earliest completion time of the set~$\Theta$. For an example
    \figureref{fig:ectexample2}.
  %\item \label{ect:start} We assume that all activities in~$\Omega$ can actually start at $\est_\Omega$. When
  %  this assumption is wrong then the estimation of earliest completion time is too low. For an example see
  %  \figureref{fig:ectexample3}.
  \item \label{ect:elastic} Since we take into account only total energy of the set~$\Theta$, we assume that
    all activities in~$\Theta$ are fully ``elastic''. For example for $\Theta=\{i\}$ from
    \figureref{fig:actexample} $\preECT(\Theta) = 1$ even though activity~$i$ cannot end before $\ect_i=4$.
    \footnote{ That is also the reason why earliest start time of an activity~$i$ is denoted by lower-case
    letters $\ect_i$ but earliest completion time of a set of activities~$\Theta$ is denoted by
    $\ECT(\Theta)$ with capital E.}
\end{enumerate}

\begin{figure}[htb]
\centering
\psset{unit=0.8cm, arrowsize=5pt}
\begin{pspicture}(-5, -0.7)(8, 0.7)
\psline[linewidth=1.3pt](-4.8, -0.7)(-5, -0.7)(-5, 0.7)(-4.8, 0.7)
\rput[b](-5, 0.8){$\est_a = 0$}
\psframe[fillstyle=solid, fillcolor=barva1](-3.5, -0.5)(-2.5, 0.5)
\rput(-3,0){$a$}
\psline[linewidth=1.3pt](-1.2, -0.7)(-1, -0.7)(-1, 0.7)(-1.2, 0.7)
\rput[b](-1, 0.8){$\lct_a = 8$}
\psline{->}(-2.5, 0)(-1, 0)
\psline{<-}(-5, 0)(-3.5, 0)
\psframe[fillstyle=solid, fillcolor=barva2](3, -0.5)(4, 0.5)
\rput(3.5, 0){$b$}
\psline{<-}(2, 0)(3, 0)
\psline[linewidth=1.3pt](2.2, -0.7)(2, -0.7)(2, 0.7)(2.2, 0.7)
\rput[b](2, 0.8){$\est_b = 14$}
\psline{->}(4, 0)(5,0)
\psline[linewidth=1.3pt](4.8, -0.7)(5, -0.7)(5, 0.7)(4.8, 0.7)
\rput[b](5, 0.8){$\lct_b = 20$}
\psline(-5.3, -0.5)(6.3, -0.5)
\psline(-5.3,  0.5)(6.3,  0.5)
\psset{linewidth=0.25pt, arrowsize=3pt}
\psline{<->}(6, -0.5)(6, 0.5)
\rput[l](6.1, 0){$C=c_a=c_b=2$}
\psset{linewidth=0.15pt}
\psline(-5, -0.35)(-5, -0.65)     \rput[t](-5, -0.8){\scriptsize $0$}
\psline(-4.5, -0.4)(-4.5, -0.6)
\psline(-4.0, -0.4)(-4.0, -0.6)
\psline(-3.5, -0.4)(-3.5, -0.6)
\psline(-3.0, -0.4)(-3.0, -0.6)
\psline(-2.5, -0.35)(-2.5, -0.65) \rput[t](-2.5, -0.8){\scriptsize $5$}
\psline(-2.0, -0.4)(-2.0, -0.6)
\psline(-1.5, -0.4)(-1.5, -0.6)
\psline(-1.0, -0.4)(-1.0, -0.6)
\psline(-0.5, -0.4)(-0.5, -0.6)
\psline( 0.0, -0.35)( 0.0, -0.65) \rput[t](-0, -0.8){\scriptsize $10$}
\psline( 0.5, -0.4)( 0.5, -0.6)
\psline( 1.0, -0.4)( 1.0, -0.6)
\psline( 1.5, -0.4)( 1.5, -0.6)
\psline( 2.0, -0.4)( 2.0, -0.6)
\psline( 2.5, -0.35)( 2.5, -0.65) \rput[t](2.5, -0.8){\scriptsize $15$}
\psline( 3.0, -0.4)( 3.0, -0.6)
\psline( 3.5, -0.4)( 3.5, -0.6)
\psline( 4.0, -0.4)( 4.0, -0.6)
\psline( 4.5, -0.4)( 4.5, -0.6)
\psline( 5.0, -0.35)( 5.0, -0.65) \rput[t](5, -0.8){\scriptsize $20$}
\psline( 5.5, -0.4)( 5.5, -0.6)
\end{pspicture}
\caption{An example: $\preECT\left(\left\{a, b\right\}\right) = 4$ even though $\{a, b\}$ cannot end before $\ect_b=16$.}
\label{fig:ectexample2}
\end{figure}

In this paper we will address only issue (\ref{ect:nosubset}) by defining better estimation of earliest
completion time:
\[
\ECT(\Theta) = \max_{\Omega \subseteq \Theta} \left\{ \preECT\left(\Omega\right) \right\}
\]
What is algebraically equivalent to:
\[
\ECT(\Theta)
= \max_{\Omega \subseteq \Theta} \left\{ \est_\Omega + \left\lceil \frac{\energy_\Omega}{C} \right\rceil \right\} \\
= \left\lceil \frac{\max_{\Omega \subseteq \Theta} \{C\est_\Omega+\energy_\Omega\}}{C} \right\rceil
\]
Lets call the numerator of the last fraction {\em energy envelope} of the set~$\Theta$:
\begin{gather}
\envelope(\Theta) = \max_{\Omega \subseteq \Theta} \{C\est_\Omega+\energy_\Omega\} 
\label{envdef}
\end{gather}
Hence:
\[
\ECT(\Theta) = \left\lceil \frac{\envelope(\Theta)}{C} \right\rceil
\]
For an example of $\ECT(\Theta)$ and $\envelope(\Theta)$ see \figureref{fig:theexample}.

\begin{figure}[htb]
\centering
\psset{unit=0.75cm, arrowsize=5pt}
\begin{pspicture}(0, -0.4)(15, 2.5)
\psframe[fillstyle=solid, fillcolor=barva3](2, 0)(5, 1)
\psframe[fillstyle=solid, fillcolor=barva3](2, 1)(5, 2)
\psframe[fillstyle=solid, fillcolor=barva3](8, 0)(10, 2)
\psframe[fillstyle=solid, fillcolor=barva3](10, 0)(12, 2)
\psframe[fillstyle=hlines, hatchcolor=barva4](0, 0)(12,2)
\rput(3.5, 1.5){$a$}
\rput(3.5, 0.5){$b$}
\rput(9, 1){$c$}
\rput(11, 1){$d$}
\psline[linewidth=1.3pt](2.2, -0.2)(2, -0.2)(2, 2.2)(2.2, 2.2)
\rput[b](2, 2.3){\scriptsize $\est_a = \est_b = 2$}
\psline[linewidth=1.3pt](4.8, -0.2)(5, -0.2)(5, 2.2)(4.8, 2.2)
\rput[b](5, 2.3){\scriptsize $\lct_a = \lct_b = 5$}
\psline[linewidth=1.3pt](8.2, -0.2)(8, -0.2)(8, 2.2)(8.2, 2.2)
\rput[b](8, 2.3){\scriptsize $\est_c = \est_d = 8$}
\psline[linewidth=1.3pt](12.8, -0.2)(13, -0.2)(13, 2.2)(12.8, 2.2)
\rput[b](13, 2.3){\scriptsize $\lct_a = \lct_b = 13$}
\psline{->}(12, 1)(13, 1)
\psline(-0.5, 0)(15, 0)
\psline(-0.5, 2)(15, 2)
\psset{linewidth=0.25pt, arrowsize=3pt}
\psline{<->}(14, 0)(14, 2)
\rput[l](14.1, 1){\scriptsize $C = 2$}
\rput(6.5, 1){\scriptsize $\envelope(\Theta) = 24$}
\psset{linewidth=0.15pt}
\psline(0, -0.15)(0, 0.15) \rput[t](0, -0.3){\scriptsize $0$}
\psline(1, -0.1) (1, 0.1)
\psline(2, -0.1) (2, 0.1)
\psline(3, -0.1) (3, 0.1)
\psline(4, -0.1) (4, 0.1)
\psline(5, -0.15)(5, 0.15) \rput[t](5, -0.3){\scriptsize $5$}
\psline(6, -0.1) (6, 0.1)
\psline(7, -0.1) (7, 0.1)
\psline(8, -0.1) (8, 0.1)
\psline(9, -0.1) (9, 0.1)
\psline(10, -0.15)(10, 0.15) \rput[t](10, -0.3){\scriptsize $10$}
\psline(11, -0.1) (11, 0.1)
\psline(12, -0.1) (12, 0.1)
\psline(13, -0.1) (13, 0.1)
\psline(14, -0.1) (14, 0.1)
\end{pspicture}
\caption{Example of a set $\Theta = \{a, b, c, d\}$ with earliest completion time $\ECT(\Theta) = 12$ and
energy envelope $\envelope(\Theta)=24$. Maximum envelope is achieved by the set $\Omega = \{c, d\}$. Energy
envelope is depicted by gray lines.} \label{fig:theexample}
\end{figure}

The reason we defined energy envelope is that it is simpler to use in the algorithms than the earliest
completion time.
%However it needs a little mental practice to switch from earliest completion time to energy
%envelope.

\section{Overload, E-Feasibility}

This section provides a variation of the e-feasibility checking algorithm by Armin Wolf and Gunnar Schrader
\cite{Wolf2005CumulativeOverload}. This algorithm is the basis of the Max Energy algorithm presented later in
this paper.

\smallskip

Traditionally, we define an {\em overload} as a situation when a subset of activities $\Omega \subseteq T$
requires more resource energy than what is available between earliest possible start and latest possible end
time of the set~$\Omega$ (see for example \cite{CBS}). If there is overload then no solution exists:
\begin{gather}
\forall \Omega \subseteq T:\quad
\left(
\energy_\Omega > C(\lct_\Omega - \est_\Omega) \quad\Rightarrow\quad \fail
\right)
\tag{OL}\label{OL}
\end{gather}
where:
\[
\lct_\Omega = \max\left\{ \lct_i,\; i \in \Omega \right\}
\]
If there is no overload then we say that the problem is {\em e-feasible}.

It would take too much time to check all subsets $\Omega \subseteq T$. Fortunately there is a faster way:

\begin{proposition}
\label{prop:feasible}
The problem is e-feasible if and only if
\[
\forall j \in T:\quad
\envelope\left(\LCut\left(T, j\right)\right) \le C\lct_j
\]
where $\LCut(T, j)$ is a {\em left cut of~$T$ by activity~$j$}:
\[
\LCut(T, j) = \left\{k,\; k \in T \azaroven \lct_k \le \lct_j \right\}
\]
\end{proposition}
\begin{proof}
We will prove the equivalence by proving both implications:
\begin{enumerate*}
\item If rule (\ref{OL}) detects overload then there is a set~$\Omega$ such that $C \lct_\Omega <
C\est_\Omega + \energy_\Omega$. In this case we define $j \in \Omega$ to be activity from set~$\Omega$ such that $\lct_j =
\lct_\Omega$ (if there are more activities with this property, we can choose arbitrarily). Thanks to the
definition of~$j$ it holds that $\Omega \subseteq \LCut(T, j)$ and therefore:
\[
  C \lct_j = C \lct_\Omega < C \est_\Omega + \energy_\Omega
    \stackrel{(\ref{envdef})}{\le} 
  \envelope\left(\LCut\left(T, j\right)\right)
\]
Therefore the second rule also detects overload.
\item If $\envelope\left(\LCut\left(T, j\right)\right) > C\lct_j$ then by (\ref{envdef}) there is a set
$\Omega \subseteq \LCut(T, j)$ such that $C \est_\Omega + \energy_\Omega = \envelope\left(\LCut\left(T,
j\right)\right)$. And for this set~$\Omega$:
\[
  C \lct_\Omega \le C \lct_j < \envelope\left(\LCut\left(T, j\right)\right) = C \est_\Omega + \energy_\Omega
\]
And therefore rule (\ref{OL}) also detects overload.
\qed
\end{enumerate*}
\end{proof}

The key idea of the algorithm is to organize set $\LCut(T, j) = \Theta$ in a balanced binary tree, which we
call $\Theta$-tree (it is an extension of~$\Theta$-tree structure for unary resources described for example in
\cite{Vilim2007PhD}). Activities are represented by leaf nodes\footnote{This is the main difference from the
algorithm in \cite{Wolf2005CumulativeOverload}, that algorithm represents activities also in the internal
nodes of the tree.} and sorted by $\est_i$ from left to right. Each node~$v$ of
the tree holds the following values:
\begin{align}
  \energy_v    &= \energy_{\Leaves(v)} \\
  \envelope_v  &= \envelope\left(\Leaves\left(v\right)\right) \label{ENVdef}
\end{align}
Where $\Leaves(v)$ is a set of all activities represented by leaves of the subtree rooted in~$v$.
\figureref{fig:ttree} shows a $\Theta$-tree from an example from \figureref{fig:theexample}. Notice that the
energy envelope of the represented set~$\Theta$ is equivalent to the value $\envelope$ of the root node.

\newcommand{\lnode}[5]{
  \rput(#1){\rnode{#2}{\psframebox[framearc=0.4]{\parbox{1.25cm}{
    \vspace{-0.35cm}
    \begin{align*}
      \est_{#2} &= #3 \\
      \energy_{#2} &= #4 \\
      \envelope &= #5
    \end{align*}
    \vspace{-0.4cm}
  }}}}
}

\newcommand{\inode}[4]{
  \rput(#1){\rnode{#2}{\psframebox[framearc=0.4]{\parbox{1.25cm}{
    \vspace{-0.35cm}
    \begin{align*}
      \energy &= #3 \\
      \envelope &= #4
    \end{align*}
    \vspace{-0.4cm}
  }}}}
}

\begin{figure}[htb]
  \psset{xunit=1.6cm, yunit=2.2cm}
  \centering
  \begin{pspicture}(1, 0.7)(7, 3)
    \inode{4, 2.6}{x}{14}{24}       % Koren
    \inode{2, 1.9}{y}{6}{10}        % Prva vrstva vlevo
    \inode{6, 1.9}{z}{8}{24}        % Druha vrstva vpravo
    \lnode{1, 1}{a}{2}{3}{7}        % Prvy list
    \lnode{3, 1}{b}{2}{3}{7}        % Druhy list
    \lnode{5, 1}{c}{8}{4}{20}       % Treti list
    \lnode{7, 1}{d}{8}{4}{20}       % Ctvrty list
    \ncline{x}{y}
    \ncline{x}{z}
    \ncline{y}{a}
    \ncline{y}{b}
    \ncline{z}{c}
    \ncline{z}{d}
  \end{pspicture}
  \caption{An example of a $\Theta$-tree for
  $\Theta = \left\{a, b, c, d\right\}$ from \figureref{fig:theexample}.}
  \label{fig:ttree}
\end{figure}

For a leaf node~$v$ representing an activity $i \in T$ the values in the tree are set to:
\begin{align*}
  \energy_v   &= \energy_i \\
  \envelope_v &= \envelope\left(\left\{i\right\}\right) = C \est_i + \energy_i
\end{align*}
For internal nodes~$v$ these values can be computed recursively from their children nodes $\lson(v)$ and
$\rson(v)$:

\begin{proposition}
For an internal node~$v$, values $\energy_v$ and $\envelope_v$ can be computed by the following recursive
formula:
\begin{align}
  \energy_v   &= \energy_{\lson(v)} + \energy_{\rson(v)} \label{recurs:energy} \\
  \envelope_v &= \max \left\{ \envelope_{\lson(v)} + \energy_{\rson(v)},\; \envelope_{\rson(v)} \right\}
  \label{recurs:envelope}
\end{align}
\end{proposition}
\begin{proof}
Formula (\ref{recurs:energy}) is trivial, we will prove only formula (\ref{recurs:envelope}).
From the definition (\ref{ENVdef}), the value $\envelope_v$ is:
\[
  \envelope_v    = \envelope\left(\Leaves\left(v\right)\right) = 
               \max\left\{C\est_\Omega + \energy_\Omega,\; 
                           \Omega \subseteq \Leaves(v) \right\}
\]
With respect to the node~$v$ we will split the sets~$\Omega$ into the
following two categories:
\begin{enumerate*}
\item $\Left(v) \cap \Omega = \emptyset$, \ie $\Omega \subseteq
\Right(v)$. Clearly:
\[
  \max\left\{C\est_\Omega + \energy_\Omega,\; \Omega \subseteq \Right(v)\right\} 
  = \envelope\left(\Right(v)\right) = \envelope_{\rson(v)}
\]
\item $\Left(v) \cap \Omega \neq \emptyset$. Then $\est_{\Omega} =
\est_{\Omega \cap \Left(v)}$ because leaf nodes are sorted by $\est_i$. Let
$S$ be the set of all possible~$\Omega$ considered in this part of the
proof:
\[
  S = \{\Omega,\; \Omega \subseteq \Theta \azaroven \Omega \cap
  \Left(v) \neq \emptyset \}
\]
Then:
\begin{align*}
  &\hspace{-1cm}\max\left\{C\est_\Omega + \energy_\Omega,\; \Omega \in S\right\} = \\
  &= \max\left\{C\est_{\Omega \cap \Left(v)} + \energy_{\Omega \cap \Left(v)} + \energy_{\Omega \cap \Right(v)},\; \Omega \in S\right\} = \\
  &= \max\left\{C\est_{\Omega \cap \Left(v)} + \energy_{\Omega \cap \Left(v)},\; \Omega \in S\right\} + \energy_{\Right(v)} = \\
  &= \envelope_{\lson(v)} + \energy_{\rson(v)}
\end{align*}
We used the fact that the maximum is achieved only by such a set $\Omega$ for which $\Right(v) \subsetneq
\Omega$. We also used the fact that $\Omega \cap \Left(v)$ enumerates all possible subsets of $\Left(v)$ and
therefore:
\[
\max\left\{C\est_{\Omega \cap \Left(v)} + \energy_{\Omega \cap \Left(v)},\; \Omega \in S\right\} =
\envelope_{\lson(v)}
\]
\end{enumerate*}
Combining the results of parts 1 and 2 together we see that formula (\ref{recurs:envelope}) is correct.\qed
\end{proof}

Thanks to formulas (\ref{recurs:energy}) and (\ref{recurs:envelope}), computation of values $\energy_v$ and
$\envelope_v$ can be integrated within usual operations with balanced binary trees without changing their time
complexity, see \tableref{tab:complex1}.

\begin{table}[htb]
\centering
  \begin{tabular}{@{\hspace{5pt}}l@{\hspace{5pt}}@{\hspace{5pt}}l@{\hspace{5pt}}}
  \hline
  {\bf Operation} & {\bf Time Complexity} \\ \hline
  $\Theta := \emptyset$ & $\cplx(1)$ \\ 
  $\Theta := \Theta \cup \{i\}$ & $\cplx(\log n)$ \\ 
  $\Theta := \Theta \setminus \{i\}$ & $\cplx(\log n)$ \\ 
  $\envelope(\Theta)$ & $\cplx(1)$ \\ 
  \hline
\end{tabular}
\caption{Worst-case time complexities of operations on $\Theta$-tree.}
\label{tab:complex1}
\end{table}

\medskip

The idea of the overload checking algorithm follows. We will iterate over all left cuts $\LCut(T, j)$ by
non-decreasing $\lct_j$. The cuts will be represented by $\Theta$-tree what allows to quickly recompute
$\envelope\left(\LCut\left(T, j\right)\right)$ each time when~$j$ is changed. For each set $\Theta = \LCut(T,
j)$ we check e-feasibility using \propositionref{prop:feasible}. The resulting \algref{alg:overload} has
worst-case time complexity $\cplx(n \log n)$. 

\begin{lstlisting}[
  float=htb,
  caption={Overload Checking in $\cplx(n \log n)$},
  label=alg:overload
]
#$\Theta$# := #$\emptyset$#;
for #$j \in T$ in non-decreasing order of $\lct_j$# do begin
  #$\Theta$# := #$\Theta \cup \{j\}$#;
  if #$\envelope(\Theta) > C\lct_j$# then
    fail; {#No solution exists#}
end;
\end{lstlisting}


\section{Max Energy Propagation}

In this section we will extend the algorithm for overload detection to compute maximum energy of each activity
$i \in T$. The idea of the propagation is to protect possible overload caused by increase of some energy
demand $\energy_i$.

Consider for example situation on \figureref{fig:theexample}. In this example, minimum required energy of
activity~$c$ is $\energy_c=4$. Maximum required energy $\enermax_c$ is not depicted on the figure, but lets
say that $\enermax_c = 10$. However considering also activity~$d$ (which requires at least $\energy_d = 4$)
the maximum feasible energy for activity~$c$ is $6$, otherwise there would be an overload for $\Omega=\{c,
d\}$. Therefore we can update $\enermax_c := 6$, and according to formula (\ref{rule:durmax}) $\durmax_c :=
3$. What we just described on the example is the goal of the presented algorithm: for each activity $i \in T$,
compute maximum feasible energy $\enermax_i$ such that if $\energy_i$ is increased above $\enermax_i$ then
there will be an overload.

\medskip

In \propositionref{prop:feasible} we have learned that the resource is e-feasible iff:
\[
\forall j \in T: \;\; \envelope\left(\LCut\left(T, j\right)\right) \le C\lct_j
\]
In other words we can assign to each set $\LCut\left(T, j\right)$ a maximum feasible envelope $\envmax$:
\begin{gather}
\envmax\left(\LCut\left(T,j\right)\right) := C \lct_j
\label{rootmax}
\end{gather}
The idea is to propagate maximum feasible envelope from the set $\LCut(T, j)$ into all its members and this
way find maximum feasible energy of all activities.

Lets have have a look on the $\Theta$-tree representing a particular set $\Theta = \LCut(T,j)$. For overload
checking we compute recursively in each node the following values by formulas (\ref{recurs:energy}) and
(\ref{recurs:envelope}):
\begin{align}
  \energy_v   &= \energy_{\lson(v)} + \energy_{\rson(v)} \tag{\ref{recurs:energy}} \\
  \envelope_v &= \max \left\{ \envelope_{\lson(v)} + \energy_{\rson(v)},\; \envelope_{\rson(v)} \right\}
  \tag{\ref{recurs:envelope}}
\end{align}
The idea is to extend the tree by adding two more attributes into each node of the tree:
\begin{itemize}
  \item maximum feasible energy envelope $\envmax_v$ of the set $\Leaves(v)$,
  \item and maximum feasible energy $\enermax_v$ for the set $\Leaves(v)$.
\end{itemize}
The additional attributes can be also computed recursively, this time from root down to the leaves. It starts at
the root node~$r$ (see (\ref{rootmax})):
\begin{align}
  \envmax_r  &:= C \lct_j     \label{rootmaxenv} \\
  \enermax_r &:= \infty       \label{rootmaxener}
\end{align}
The recursive rules to propagate these values down the tree are:
\begin{align}
  \envmax_{\rson(v)}  &:= \envmax_v
    \label{recurs:renvmax} \\
  \envmax_{\lson(v)}  &:= \envmax_v - \energy_{\rson(v)}
    \label{recurs:lenvmax} \\
  \enermax_{\rson(v)} &:= \min \left\{ \envmax_v - \envelope_{\lson(v)}, \enermax_v - \energy_{\lson(v)} \right\} 
    \label{recurs:renermax} \\
  \enermax_{\lson(v)} &:= \enermax_v - \energy_{\rson(v)}
    \label{recurs:lenermax}
\end{align}
For an example of computation of $\enermax$ and $\envmax$ see \figureref{fig:maxtree}.

\newcommand{\lmaxnode}[7]{
  \rput(#1){\rnode{#2}{\psframebox[framearc=0.4]{\parbox{1.25cm}{
    \vspace{-0.35cm}
    \begin{align*}
      \est_{#2} &= #3 \\
      \energy_{#2} &= #4 \\
      \envelope &= #5 \\
      \enermax &= #6 \\
      \envmax &= #7
    \end{align*}
    \vspace{-0.4cm}
  }}}}
}

\newcommand{\imaxnode}[6]{
  \rput(#1){\rnode{#2}{\psframebox[framearc=0.4]{\parbox{1.25cm}{
    \vspace{-0.35cm}
    \begin{align*}
      \energy &= #3 \\
      \envelope &= #4 \\
      \enermax &= #5 \\
      \envmax &= #6
    \end{align*}
    \vspace{-0.4cm}
  }}}}
}

\begin{figure}[htb]
  \psset{xunit=1.6cm, yunit=3.5cm}
  \centering
  \begin{pspicture}(1, 0.5)(7, 3)
    \imaxnode{4, 2.6}{x}{14}{24}{\infty}{26}       % Koren
    \imaxnode{2, 1.9}{y}{6}{10}{\infty}{18}        % Prva vrstva vlevo
    \imaxnode{6, 1.9}{z}{8}{24}{16}{26}            % Druha vrstva vpravo
    \lmaxnode{1, 1}{a}{2}{3}{7}{\infty}{15}        % Prvy list
    \lmaxnode{3, 1}{b}{2}{3}{7}{11}{18}            % Druhy list
    \lmaxnode{5, 1}{c}{8}{4}{20}{12}{22}           % Treti list
    \lmaxnode{7, 1}{d}{8}{4}{20}{6}{26}            % Ctvrty list
    \ncline{x}{y}
    \ncline{x}{z}
    \ncline{y}{a}
    \ncline{y}{b}
    \ncline{z}{c}
    \ncline{z}{d}
  \end{pspicture}
  \caption{Computation of $\envmax$ and $\enermax$ for $\Theta=\{a, b, c, d\}$ from \figureref{fig:theexample}.
  Notice that from the nodes representing activities~$c$ and~$d$ we can conclude that $\enermax_c \le 6$ and
  $\enermax_d \le 6$. In case of activity~$d$ because of $\enermax$ in the node, in case of activity~$c$
  because of $\envmax$ in the node as will be described by (\ref{prop:enermax}).}
  \label{fig:maxtree}
\end{figure}

Formal proof of the recursive rules (\ref{recurs:renvmax}) -- (\ref{recurs:lenermax}) will follow. But let us
first explain for example construction of formula~(\ref{recurs:renermax}). If $\energy_{\rson(v)}$ is
increased then it will cause also an increase of $\energy_v$ by formula~(\ref{recurs:energy}). However maximum
feasible value of $\energy_v$ is $\enermax_v$ and therefore the maximum feasible value of $\energy_{\rson(v)}$
has to fulfill the following formula:
\[
\enermax_{\rson(v)} \le \enermax_v - \energy_{\lson(v)}
\]
Similarly, increase of $\energy_{\rson(v)}$ can lead to the increase of $\envelope_v$ by
formula~(\ref{recurs:envelope}) but it cannot exceed the maximum feasible value $\envmax_v$. Therefore:
\[
\enermax_{\rson(v)} \le \envmax_v - \envelope_{\lson(v)}
\]
Combining the these two formulas we get rule~(\ref{recurs:renermax}). The remaining three rules
(\ref{recurs:renvmax}), (\ref{recurs:lenvmax}) and (\ref{recurs:lenermax}) are constructed in a similar way.

\medskip

Let us formally proof correctness of the rules (\ref{recurs:renvmax}) -- (\ref{recurs:lenermax}). We start
with the following lemma:

\begin{lemma}
For a node~$w$ and its parent node~$v$ in a $\Theta$-tree: if one of the values $\envmax_w$ and $\enermax_w$
are not respected (that is $\envelope_w > \envmax_w$ or $\energy_w > \enermax_w$) then at least one of the
values $\envmax_v$, $\enermax_v$ is not respected too.
\end{lemma}
\begin{proof}
We will split the proof into two parts depending on whether~$w$ is left or right son of node~$v$:
\begin{enumerate*}
\item Case $w = \lson(v)$. If $\enermax_w$ is not respected then:
\begin{align*}
  \energy_w &> \enermax_w 
    \stackrel{(\ref{recurs:lenermax})}{=} 
  \enermax_v - \energy_{\rson(v)} \\
  \energy_w + \energy_{\rson(v)} &> \enermax_v \\
  \energy_v &> \enermax_v \quad\quad\text{by (\ref{recurs:energy})}
\end{align*}
Therefore if $\enermax_v$ is not respected then $\enermax_v$ is not respected too.

Similarly if $\envmax_w$ is not respected then:
\begin{align*}
  \envelope_w &> \envmax_w
    \stackrel{(\ref{recurs:lenvmax})}{=}
  \envmax_v - \energy_{\rson(v)} \\
  \envelope_w + \energy_{\rson(v)} &> \envmax_v \\
  \max\{\envelope_w + \energy_{\rson(v)},\; \envelope_{\rson(v)}\} &> \envmax_v \\
  \envelope_v &> \envmax_v \quad\quad\text{by (\ref{recurs:envelope})}
\end{align*}
So if $\envmax_w$ is not respected then $\envmax_v$ is not respected too.

\item Case $w = \rson(v)$. If $\enermax_w$ is not respected then:
\[
  \energy_w > \enermax_w 
    \stackrel{(\ref{recurs:renermax})}{=} 
  \min\{ \envmax_v - \envelope_{\lson(v)},\; \enermax_v - \energy_{\lson(v)}\} \\
\]
Therefore 
\begin{enumerate*}
\item Either:
  \begin{align*}
    \energy_w &>  \envmax_v - \envelope_{\lson(v)} \\
    \envelope_{\lson(v)} + \energy_w &> \envmax_v \\
    \envelope_v &> \envmax_v \quad\quad\text{by (\ref{recurs:envelope})}
  \end{align*}
  And so $\envmax_v$ is not respected.
\item Or:
  \begin{align*}
    \energy_w &> \enermax_v - \energy_{\lson(v)} \\
    \energy_{\lson(v)} + \energy_w &> \enermax_v \\
    \energy_v &> \enermax_v \quad\quad\text{by (\ref{recurs:energy})}
  \end{align*}
  And so $\enermax_v$ is not respected.
\end{enumerate*}

\medskip

Finally if $\envmax_w$ is not respected then:
\[
\envelope_w > \envmax_w 
    \stackrel{(\ref{recurs:renvmax})}{=} 
    \envmax_v
\]
Therefore
\[
  \envelope_v 
    \stackrel{(\ref{recurs:envelope})}{=}
  \max\{\envelope_{\lson(v)} + \energy_w,\; \envelope_w\} \ge \envelope_w > \envmax_v
\]
And thus $\envmax_v$ is not respected.
\qed
\end{enumerate*}
\end{proof}

\medskip

A consequence of this lemma is:

\begin{proposition}
Let $i \in \LCut(T, j)$ and let~$v$ be a leaf node representing activity~$i$ in $\Theta$-tree for
$\LCut(T,j)$. If $\energy_i > \min \left\{ \enermax_v,\; \envmax_v - C\est_i \right\}$ then there is an
overload and therefore the problem is unfeasible.
\end{proposition}
\begin{proof}
If $\energy_i > \min \left\{ \enermax_v,\; \envmax_v - C\est_i \right\}$ then it means that either
$\enermax_v$ or $\envmax_v$ in the node~$v$ is not respected. By the previous lemma it means that at least one
of these values is not respected also in parent node of~$v$. And so we continue this way to the root node~$r$
and prove that $\enermax_r$ or $\envmax_r$ is not respected.

However for root node~$r$, $\enermax_r = \infty$ by (\ref{rootmaxener}) therefore $\enermax_r$ has to be
respected. The conclusion is that $\envmax_r$ is not respected and therefore:
\begin{align*}
  \envelope_r &> \envmax_r
    \stackrel{(\ref{rootmaxenv})}{=} C \lct_j \\
  \envelope\left(\LCut\left(T, j\right)\right) &> C \lct_j
\end{align*}
So there is overload by \propositionref{prop:feasible}. \qed
\end{proof}

The proposition above gives as an upper bound for maximum energy available for each activity $i \in \LCut(T,j)$:
\begin{gather}
\enermax_i \le \min \left\{ \enermax_v,\; \envmax_v - C\est_i \right\}
\label{prop:enermax}
\end{gather}
Notice that for example on Figures~\ref{fig:theexample}~and~\ref{fig:maxtree} the formula (\ref{prop:enermax}) gives
$\enermax_c = 6$ and $\enermax_d = 6$ and therefore by (\ref{rule:durmax}) $\durmax_c = 3$ and $\durmax_d=3$.

\medskip

The basic idea of the algorithm follows: we iterate over all activities $j \in T$ and for each $j$ we build
$\Theta$-tree representing $\LCut(T, j)$ by adding new nodes into the $\Theta$-tree from the previous
iteration. In each $\Theta$-tree we propagate the maximum energy envelope $C \lct_j$ from the root to leave
nodes and assign maximum energies to activities $i \in \LCut(T, j)$ according to formula (\ref{prop:enermax}).
First version of the algorithm with time complexity $\cplx(n^2)$ is provided by \algref{alg:enermax1}. Note
that this is not the $\cplx(n^2)$ algorithm by Wim Nuijten, for better understanding we start with
$\cplx(n^2)$ algorithm and then speed it up to $\cplx(n \log n)$.

The algorithm uses two procedures:
\begin{itemize*}
  \item {\tt push\_down($v$)} pushes the values $\envmax_v$ and $\enermax_v$ from the node~$v$ down the tree
    using the rules~(\ref{recurs:renvmax})~--~(\ref{recurs:lenermax}). 
  \item {\tt set\_energy\_max($i$)} sets maximum energy $\enermax_i$ of the activity~$i$ using
    formula~(\ref{prop:enermax}).
\end{itemize*}

\begin{lstlisting}[
  float=htb,
  caption={Maximum energy propagation in $\cplx(n^2)$},
  label=alg:enermax1
]
#$\Theta$# := #$\emptyset$#;
for #$j \in T$ in non-decreasing order of $\lct_j$# do begin
  #$\Theta$# := #$\Theta \cup \{j\}$#;
  if #$\envelope(\Theta) > C\lct_j$# then
    fail; {#No solution exists#}
  #$\envmax_{\Theta}$# := #$C\lct_j$#;
  for #nodes~$v$ in $\Theta$-tree in non-decreasing order of their depth# do
    push_down(#$v$#);
  for #$i \in \Theta$# do
    set_energy_max(#$i$#);
end;
\end{lstlisting}

Time complexity of this algorithm is $\cplx(n^2)$ because the inner cycles on lines 7 -- 8 and 9 -- 10 have
time complexity $\cplx(n)$. In the following we will show how to improve the time complexity from $\cplx(n^2)$
to $\cplx(n \log n)$. 

\medskip

The key observation is that it is not necessary to push values $\enermax_v$ and $\envmax_v$ down to leaves
immediately. The values $\enermax_v$ and $\envmax_v$ stays valid until new node is inserted into the subtree
of~$v$. Therefore it is possible to postpone {\tt push\_down($v$)} until new node is inserted somewhere under
the node~$v$.

Current procedure {\tt push\_down($v$)} simply overwrites values $\enermax$ and $\envmax$ in children nodes of
$v$. However that is no longer possible in the new algorithm because children nodes may contain information
which was not pushed down yet. Therefore it is necessary create new procedure {\tt push\_down2($v$)}
which implements modified rules~(\ref{recurs:renvmax})~--~(\ref{recurs:lenermax}):
\begin{align}
  \envmax_{\rson(v)}  &:= \min \left\{ \envmax_v,\; \envmax_{\rson(v)} \right\} \\
  \envmax_{\lson(v)}  &:= \min \left\{ \envmax_v - \energy_{\rson(v)},\; \envmax_{\lson(v)} \right\} \\
  \enermax_{\rson(v)} &:= \min \left\{ \envmax_v - \envelope_{\lson(v)},\; 
                                        \enermax_v - \energy_{\lson(v)},\;
                                        \enermax_{\rson(v)}
                               \right\} \\
  \enermax_{\lson(v)} &:= \min \left\{ \enermax_v - \energy_{\rson(v)},\; \enermax_{\lson(v)} \right\}
\end{align}
The values $\enermax_v$ and $\envmax_v$ must be initialized to:
\begin{align*}
  \enermax_v &:= \infty\\
  \envmax_v &:= \infty
\end{align*}

The postponed calls of {\tt push\_down2($v$)} must be executed just before new node is added into the tree. The
most suitable place is to replace $\Theta := \Theta \cup \{j\}$ on line 3 by procedure {\tt add($\Theta,j$)}
which will make necessary postponed calls.

For simplicity, lets assume that the shape of the tree is fixed and no re-balancing occurs during the addition
of new node into the tree\footnote{This can be achieved for example by computing perfectly balanced tree of
all activities in advance.}. The procedure {\tt add($\Theta,j$)} has following steps:
\begin{enumerate}
  \item Find a node~$w$ under which new node will be inserted.
  \item Call {\tt push\_down2($v$)} on all nodes~$v$ on the path from the root to~$w$, and then reset these
    nodes to:
    \begin{align*}
      \enermax_v &:= \infty\\
      \envmax_v &:= \infty
    \end{align*}
  \item Add new node in the tree, fill the leaf representing~$j$ by data about the activity~$j$.
  \item Recompute values $\energy_v$ and $\envelope_v$ on the path from this leaf to the root.
\end{enumerate}

The resulting \algref{alg:enermax2} has worst case time complexity $\cplx(n \log n)$.

\begin{lstlisting}[
  float=htb,
  caption={Maximum energy propagation in $\cplx(n \log n)$},
  label=alg:enermax2
]
#$\Theta$# := #$\emptyset$#;
for #$j \in T$ in non-decreasing order of $\lct_j$# do begin
  add(#$\Theta$#, #$j$#);
  if #$\envelope(\Theta) > C\lct_j$# then
    fail; {#No solution exists#}
  #$\envmax_{\Theta}$# := #$C\lct_j$#;
end;
for #nodes~$v$ in $\Theta$-tree in non-decreasing order of their depth# do
  push_down2(#$v$#);
for #$i \in T$# do
  set_energy_max(#$i$#);
\end{lstlisting}

\section{Optional Activities}

As mentioned in the introduction, IBM ILOG CP Optimizer 2.0 \cite{ILOG2008} introduce a new variable type designed
for scheduling -- interval variable. The difference between activity (as we used this word in this paper) and
an interval variable is that an interval variable by itself does not require any resource. An activity is
created when an interval variable is constrained to require a resource. One interval variable may require more
than one resource, in this case the interval variable is associated with several activities and all of them
share their start times and end times.

Interval variable in IBM ILOG CP Optimizer has very important aspect: it can be declared as optional, that is, it
may or may not be present in the solution. In this case all its activities are also optional.

Typical use of optional interval variables are alternative between two different actions. This can be easily
modeled by two optional interval variables and a constraint that exactly one of them is present in the
solution. Both optional intervals can require some discrete cumulative resource during their execution,
however the interval which is not present in the solution does not affect any resource. For more details about
optional interval variables see \cite{Laborie2008ConditionalTI, ILOG2008}.

From the point of view of the resource, there are some optional activities which may or may not be executed on
the resource, this is yet to be decided. During the search an optional activity may become:
\vspace{-1mm}
\begin{enumerate}[A)]
\item preset if:
\begin{enumerate}[1)]
\item we decide to execute it as a search decision,
\item or if we proved (by propagation) that no other alternative is possible,
\end{enumerate}
\item absent if:
\begin{enumerate}[1)]
\item we decided to not to execute it by a search decision,
\item or if we proved (by propagation) that execution of this activity is not possible.
\end{enumerate}
\end{enumerate}
\vspace{-1mm}
The algorithm Max Energy presented in this paper can  do the propagation B2) above. This is very important
propagation because usually any propagation on optionality status has a big impact on other variables.

\smallskip

How to use \algref{alg:enermax2} with optional activities? First observe that from the point of view of
discrete cumulative resource, there is no difference between absent activity and activity with zero energy
(that is, activity with zero duration or zero capacity requirement). Zero-energy activity can be processed
anytime even though the resource is already full.

The idea is to use \algref{alg:enermax2} directly without any modification, but on modified input data. If an
activity is optional then (just for the algorithm) its minimum energy is set to zero. This way, optional activity
cannot influence any other activity, however upper bound $\enermax_i$ for energy computed by the algorithm is
valid also for optional activity~$i$. Furthermore if $\enermax_i < \energy_i$ we can deduce that the
activity~$i$ cannot be present in the solution.

\section{Conclusions and Further Work}

This paper presents a new Max Energy propagation algorithm which updates maximum durations and maximum
capacity requirements on discrete cumulative resource with optional activities. The algorithm has better time
complexity ($\cplx(n \log n)$ versus $\cplx(n^2)$) than old never published algorithm by Wim Nuijten.
Experiments show that the new algorithm begin to be faster than the old one for $n$ around $10$. The
algorithm is used by IBM ILOG CP Optimizer \cite{ILOG2008} since version 2.0.

%The cumulative $\Theta$-tree structure described in this paper can be also used by other filtering
%algorithms for discrete cumulative resource. A paper describing Edge Finding algorithm for discrete cumulative
%resource with time complexity $\cplx(kn \log n)$ is under preparation.

\bibliography{citations}
\end{document}

% vim: textwidth=110
