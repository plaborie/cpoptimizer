\documentclass{article}
\usepackage{ijcai05}

\begin{document}
%\pagestyle{headings}

\title{Commentary on:\linebreak "Multi-Objective Scheduling for the Cluster II Constellation" \linebreak by Mark D. Johnston and Mark Giuliano}

\author{Philippe Laborie\\
IBM, 9 rue de Verdun, 94250 Gentilly, France\\
\texttt{laborie@fr.ibm.com}}

\maketitle

\section{Brief Overview}

The paper describes a multi-objective scheduling application for the Cluster II constellation. It is a quite complete overview of the application including the architecture, user interaction, scheduling and visualization aspects with some focus on the visualization elements for analyzing the objective value trade-offs. 

\section{Comments and Questions}

In spite of the high level of description, the paper gives a good idea of the application and of what is going on inside.

It clearly fits well the workshop topics and in fact, both the architecture and the visualization elements are clearly relevant for many multi-objective scheduling applications beside the space domain.

First, I have a couple of minor questions on the problem itself:

{\vspace*{1mm}}{\bf \underline{Question 1:}} {\em Can the cluster serve two opportunities at the same time? }

{\vspace*{1mm}}{\bf \underline{Question 2:}} {\em The way spacecraft preferences are aggregated into a single criterion value for a given opportunity is a bit unclear. I could not see how you get the figure 5.88 from the example given in the paper. Could you elaborate a little bit? }

{\vspace*{1mm}}The section on the representation of the scheduling problem could give more details in particular about the encoding that is used. This section only describes in some detail how the opportunity selection is encoded (through a floating point variable whose value represents a feasible subset of opportunities) but not much is said about (a) the encoding of the subset of spacecrafts participating to an opportunity and (b) the dates at which opportunities will be scheduled apart from {\em "The residual value is used to determine which spacecraft combination to select from among all those allowed, while the minimum timing for the opportunity is used for evaluating the schedule"}.

{\vspace*{1mm}}{\bf \underline{Question 3:}} {\em What is the encoding for the subset of spacecrafts participating to an opportunity? Is it encoded in a similar way as the subsets of opportunities, that is, in this case, for each opportunity, 15 possible values representing the possible allocations? Are the preferences on the possible spacecraft allocations also encoded here? How? }

{\vspace*{1mm}}{\bf \underline{Question 4:}} {\em Concerning the dates at which opportunities are scheduled, you say that the minimum timing for the opportunity is used: does it mean there is no actual decision on the dates in case there are windows of flexibility? Is it not in contradiction with some preferences (collision, spacing) being directly related with particular start and end time values of opportunities? }

{\vspace*{1mm}}The user interface seems efficient to analyze the possible compromises in the solution space, particularly the notion of "brushed" histograms that really looks appealing in the context of multi-objective optimization. 

{\vspace*{1mm}}{\bf \underline{Question 5:}} {\em As the application has been utilized for some time now, it would be interesting to have the users feedback on this type of visualization. In particular, the histograms and the X-Y plots are partially redundant for analyzing the trade-offs between different objectives. What is a typical scenario for a user trying to analyze these trade-offs? }

{\vspace*{1mm}}{\bf \underline{Question 6:}} {\em Did you investigate the manipulation of 3D plots in addition to 2D plots in order to compare a subset of 3 objectives? }

The strong point of the approach is that it provides different levels for analyzing the solution set. Starting from a statistical level (the histograms) that provides a global view on the objective interaction to a pairwise comparison of objectives (the X-Y plots) where the complete set of solutions is visible and finally down to a detailed view of individual solutions (the solution panel). This three-level approach is really interesting for any multi-objective scheduling applications involving more than two criteria.

\end{document}

