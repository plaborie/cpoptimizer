% !TEX spellcheck = en-US
%\def\year{2018}\relax
\documentclass[letterpaper]{article} 
\setcounter{secnumdepth}{2}  
\usepackage{amsmath}

%% CP OPTIMIZER OPERATORS
\DeclareMathOperator{\minimize}{minimize}
\DeclareMathOperator{\cpoEndOf}                 {endOf}
\DeclareMathOperator{\cpoLengthOf}              {lengthOf}
\DeclareMathOperator{\cpoNoOverlap}             {noOverlap}
\DeclareMathOperator{\cpoEndBeforeStart}        {endBeforeStart}
\DeclareMathOperator{\cpoStartBeforeStart}      {startBeforeStart}
\DeclareMathOperator{\cpoPresenceOf}            {presenceOf}
\DeclareMathOperator{\cpoAlternative}           {alternative}
\DeclareMathOperator{\cpoSpan}                  {span}
\DeclareMathOperator{\cpoPulse}                 {pulse}
\DeclareMathOperator{\cpoInterval}              {interval}
\DeclareMathOperator{\cpoSize}                  {size}
\DeclareMathOperator{\cpoOptional}              {optional}

\begin{document}

\section{CP Optimizer model}

\setcounter{equation}{0}
\begin{alignat}{2}
&\min \quad w_1 \cdot \sum \limits_{j \in J} \max(0,\cpoEndOf(a_j)-\overline{\omega}_j) & \label{eq1}\\
&\quad \quad + w_2 \cdot \sum_{j \in J} \sum_{e \in (E \setminus E^{Pr}_j)} \cpoPresenceOf(a^{Em}_{ej}) & \label{eq2}\\
&\quad \quad + w_3 \cdot \sum_{p \in P} \sum_{e \in E} \big( 0 < \sum_{j \in J_p} \cpoPresenceOf(a^{Em}_{ej}) \big) & \label{eq3}\\
&\quad \quad + w_4 \cdot \sum_{p \in P} \cpoLengthOf(b_p) & \label{eq4}\\
&\cpoSpan(b_p, [a_j]_{j \in J_p}) \ & \forall p \in P \label{eq5}\\
&\cpoEndBeforeStart(a_k,a_j) \quad & \forall k \in P_j \label{eq6}\\
&\cpoAlternative(a_j, [a_{ij}]_{ i \in M_j} , 1) & \forall j \in J \label{eq7}\\
&\cpoAlternative(a_{ij}, [a^{EmM}_{eij}]_{e \in E_j} , r^{Em}_{ij}) & \forall j \in J, i \in M_j \label{eq8}\\
&\cpoAlternative(a^{Em}_{ej}, [a^{EmM}_{eij}]_{i \in M_j} , 1) & \forall j \in J, e \in E_j \label{eq9}\\
&\cpoNoOverlap([a^{EmM}_{eij}]_{j \in J, i \in M_j : e \in E_j}) &  \forall e \in E  \label{eq10}\\
&\cpoNoOverlap([a^{Em}_{ej}]_{j \in J : e \in E_j}) &  \forall e \in E  \label{eq11}\\
&\cpoPresenceOf(a^{Em}_{ej})==\cpoPresenceOf(a^{Em}_{ek}) & \forall j \in J, k \in L_i, e \in E \label{eq12}\\
&\cpoAlternative(a_{j}, [a^{Wb}_{bj}]_{b \in B_j} , r^{Wb}_{j}) & \forall j \in J \label{eq13}\\
&\cpoNoOverlap([a^{Wb}_{bj}]_{j \in J : b \in B_j}) &  \forall b \in B  \label{eq14}\\
&\cpoAlternative(a_{j}, [a^{Eq}_{gdj}]_{d \in G_{gj}} , r^{Eq}_{gj}) & \forall j \in J, g \in G^* \label{eq15}\\
&\cpoNoOverlap([a^{Eq}_{gdj}]_{j \in J : d \in G_{gj}}) &  \forall g \in G^*, d \in G_{g}  \label{eq16}\\
&\cpoInterval\ b_p & \forall p \in P \label{eq17}\\
&\cpoInterval\ a_j \subset [\alpha_j, \omega_j) & \forall j \in J \label{eq18}\\
&\cpoInterval\ a_{ij}\ \cpoOptional\ \cpoSize\ d_{ij} \ & \forall j \in J, i \in M_j \label{eq19}\\
&\cpoInterval\ a^{EmM}_{eij}\ \cpoOptional\  & \forall j \in J, i \in M_j, e \in E_j \label{eq20}\\
&\cpoInterval\ a^{Em}_{ej}\ \cpoOptional\  & \forall j \in J, e \in E_j \label{eq21}\\
&\cpoInterval\ a^{Wb}_{bj}\ \cpoOptional\  & \forall j \in J, b \in B_j \label{eq22}\\
&\cpoInterval\ a^{Eq}_{gdj}\ \cpoOptional\  & \forall j \in J, g \in G^*, d \in G_{gj} \label{eq23}
\end{alignat}

Decision variables are defined in equations \ref{eq17}-\ref{eq23}:
\begin{itemize}
\item In Eq. \ref{eq17}, $b_p$ represent a sub-project $p$, it starts at the start time of the project and ends at its end time. 
\item In Eq. \ref{eq18}, $a_j$ represents a job j. A project $b_p$ spans all its jobs (constraint in Eq. \ref{eq5}).
\item Eq. \ref{eq19} defines a set of optional interval variables $a_{ij}$ for each job $j$ and each possible model $i$. Interval variable $a_{ij}$ will be present if and only if mode $i$ is selected for job $j$. The duration of jobs is mode-dependent that is why each optional interval variable $a_{ij}$ is defined its own size $d_{ij}$. Constraint \ref{eq7} specifies that among the different modes for job $j$ one and only one will be selected.
\item For a given job $j$ executed in a given mode $i$, several employee allocations are possible. Eq. \ref{eq20} defines one optional interval $a^{EmM}_{eij}$ for each possible allocated employee perform job $j$ in mode $i$. Constraint \ref{eq8} specifies that exactly $r^{Em}_{ij}$ employees will have to be selected (intervals will be present) for each interval $a_{ij}$. The $r^{Em}_{ij}$ selected intervals will start and end at the same value as the master interval variable $a_{ij}$.
\item Eq. \ref{eq21} defines a set of optional interval variables $a^{Em}_{ej}$ for each job $j$ and each possible employee $e$. Constraint \ref{eq9} states that if an employee is selected for a given job $j$ then it must be selected by one of the possible modes. This type of multi-level alternative constraints are common in CP Optimizer models.
\item Intervals defined in Eq. \ref{eq22} represent the possible allocation of a workbench to a job $j$. The selection of $r^{Wb}_{j}$ for a job $j$ is formulated by constraint \ref{eq13}.
\item Similarly, for each group of equipments, intervals defined in Eq. \ref{eq23} represent the possible allocation of an equipment from a given group to a job $j$. The selection of $r^{Eq}_{gj}$ for a job $j$ is formulated by constraint \ref{eq15}.
\end{itemize}

The unary resource constraints (employees, workbenches, equipments) are formulated as no-overlap constraints in Eq. \ref{eq11},\ref{eq14},\ref{eq16}. Equation \ref{eq10} is a redundant constraint.

Precedence constraints between jobs are modeled as $\cpoEndBeforeStart$ constraints in Eq. \ref{eq6}.

Linked jobs are formulated in Eq. \ref{eq12}. In practice, this constraint can be posted only on the subset of common employees between job $j$ and job $k$. A constraint $!\cpoPresenceOf(a^{Em}_{ej})$ or $!\cpoPresenceOf(a^{Em}_{ek})$ should be posted on the employees not in $E_j \cap E_k$.

The objective function is similar to the one in the paper, except that the project span can directly be formulated as the length of the project interval variables $b_p$.

Note that the same redundant constraints as in the paper can be formulated using cumulative functions in the CP Optimizer model. For instance a constraint similar to (20) would be:

\setcounter{equation}{0}
\begin{alignat}{2}
&\sum_{j \in J, i \in M_j} \cpoPulse(a_{ij}, r^{Em}_{ij}) \leq |E| & 
\end{alignat}

\end{document}
