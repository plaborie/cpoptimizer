\documentclass[]{article}

\usepackage{framed}
\usepackage{color}
\usepackage{array} %don't know why this is needed; get csname error otherwise

\definecolor{shadecolor}{gray}{0.85}
\definecolor{shadecolor2}{gray}{0.75}

\renewenvironment{shaded}{%
  \def\FrameCommand{\fboxsep=\FrameSep \colorbox{shadecolor}}%
  \MakeFramed{\advance\hsize-\width \FrameRestore\FrameRestore}}%
 {\endMakeFramed}

% automatically number warnings and errors throughout document
\newcounter{warnno}
\newenvironment{warnings}
{
\begin{flushleft}
\begin{tabular}{>{(W\refstepcounter{warnno}\thewarnno)}rp{10cm}}
}
{
\end{tabular}
\end{flushleft}
}
\newcounter{errno}
\newenvironment{errors}
{
\begin{flushleft}
\begin{tabular}{>{(E\refstepcounter{errno}\theerrno)}rp{10cm}}
}
{
\end{tabular}
\end{flushleft}
}


\title{SmartDeliveries Engine Documentation}
\author{}
\date{}
\begin{document}
\maketitle

\pagestyle{plain}

\begin{table}[h]
	\centering
		\begin{tabular}{rcl}
			Executable & : & {\tt cpo\_optimod} \\
      Version    & : & {\tt 1.0.0}\\
      Date       & : & {\tt 10/12/2014}\\
      Built on   & : & IBM ILOG CPLEX Optimization Studio {\tt 12.6.1.0}
		\end{tabular}
\end{table}


\section{SmartDeliveries engine state}

The state of the SmartDeliveries engine consists of 5 elements:
\begin{enumerate}
\item A travel time tensor
\item A visit duration specification
\item A forbidden times specification 
\item A precedence constraints specification
\item The terminal visit specification (if any)
\end{enumerate}

The specification of each of these elements is described in the following subsections.

\subsection{Travel time tensor}

Let $n$ denote the number of visits, $m$ the number of time steps and $d$ the duration of a time step. The time-dependent travel time tensor is defined by the 3 above integers followed by $n^2\,m$ integers defining the travel times:

\begin{shaded}
\begin{verbatim}
TravelTimeTensor n m d x[0] x[1] x[2] ...
\end{verbatim}
\end{shaded}

The travel time from a visit of index $0 \leq i < n$ to a visit of index $0 \leq j < n$ when starting at $0 \leq t < m \, d$ is given by:
\[ TT(i,j,t) = x[(j + i \, n) \, m + \lceil t / d \rceil ] \]

\underline{Important remarks:}
\begin{itemize}
\item The travel time tensor implicitly defines the set of visits of the route: a route must visit {\bf all the $n$ visits} that define the tensor. 
\item The travel time tensor implicitly defines the initial (depot) visit of the route which is the {\bf visit of index $0$}.
\end{itemize}

\underline{Remarks:}
\begin{itemize} 
\item Integer $n$ should be positive otherwise a warning (W\ref{w17}) is generated and the specification of durations is ignored.
\item Integer $m$ should be positive otherwise a warning (W\ref{w18}) is generated and the specification of durations is ignored.
\item Integer $d$ should be positive otherwise a warning (W\ref{w19}) is generated and the specification of durations is ignored.
\item If more than $n^2\,m$ travel times are provided, only the first $n^2\,m$ ones will be considered. 
\item If less than $n^2\,m$ travel times are provided, the behavior is unspecified.
\item Each of the travel times {\tt x} should be non-negative otherwise a warning (W\ref{w20}) is generated the travel time is considered to be $0$.
\end{itemize}

\subsection{Visit duration specification}

The duration of visits is specified by passing a vector of integer durations.

\begin{shaded}
\begin{verbatim}
Durations n d[0] d[1] d[2] ... d[n-1]
\end{verbatim}
\end{shaded}

\underline{Remarks:}
\begin{itemize}
\item Integer $n$ should be non-negative otherwise a warning (W\ref{w0}) is generated and the specification of durations is ignored.
\item If more than $n$ durations are provided, only the first $n$ ones will be considered. 
\item If less than $n$ durations are provided, the behavior is unspecified.
\item Integers $d[i]$ should be strictly positive integers. In case a null or negative duration is provided, a warning (W\ref{w1}) will be generated and the duration will be supposed to be $1$ (the smallest non-zero duration)
\item The specification of a distance vector completely replaces the current vector in case one was previously specified.
\item At solve time, if the size $n$ of the tensor is larger than the size of the duration vector, by default the duration of the remaining visits with no specified duration will be supposed to be $1$ (the smallest non-zero duration) and a warning (W\ref{w3}) will be generated.
\item At solve time, if no duration vector was specified, all durations are supposed to be $1$ and a warning (W\ref{w4}) will be generated.
\item At solve time, if a duration was specified for a visit $i$ that is larger than the size $n$ of the tensor so that $i$ is not a visit of the route, a warning (W\ref{w16}) will be generated.
\end{itemize}

\subsection{Forbidden times specification}

A set of forbidden time windows can be specified for any visit. The syntax is as follows:

\begin{shaded}
\begin{verbatim}
ForbiddenTimes i f a[0] b[0] a[1] b[1] ... a[f-1] b[f-1]
\end{verbatim}
\end{shaded}

It means that visit $i$ has $f$ forbidden time windows $[a[0],b[0])$, $[a[1],b[1])$, ..., $[a[f-1],b[f-1])$ meaning that the visit cannot overlap any of the time windows $[a[j],b[j])$ for $0 \leq j < f$.

\underline{Remarks:}
\begin{itemize}
\item Visit index $i$ must be a non-negative integer otherwise a warning (W\ref{w5}) will be generated and the specification is ignored.
\item Number of time-windows $f$ must be a non-negative integer otherwise a warning (W\ref{w6}) will be generated and the specification is ignored.
\item If more than $f$ time-windows are provided, only the first $f$ ones will be considered. 
\item If less than $f$ time-windows are provided, the behavior is unspecified.
\item Each of the time-windows $a[j],b[j]$ must be consistent that is such that: $0 \leq a[j] < b[j]$ otherwise a warning (W\ref{w7}) will be generated and the inconsistent time-window will be ignored.
\item If a visit $i$ was not specified any forbidden time windows it behaves as if it was explicitly specified 0 time-windows using {\tt ForbiddenTimes i 0}. No warning is generated in this case.
\item The specification of a set of forbidden time windows for a visit $i$ completely replaces the current set of forbidden time windows in case one was previously specified.
\item At solve time, if a forbidden time specification holds on a visit $i$ that is larger than the size $n$ of the tensor so that $i$ is not a visit of the route, a warning (W\ref{w16}) will be generated.
\end{itemize}

\subsection{Precedence constraints specification}

A set of precedence constraints between visits can be specified. The syntax is as follows:

\begin{shaded}
\begin{verbatim}
Precedences q p[0] s[0] p[1] s[1] ... p[q-1] s[q-1]
\end{verbatim}
\end{shaded}

Each precedence constraint $0 \leq i < q$ defines the predecessor visit $p[i]$ and the successor visit $s[i]$.

\underline{Remarks:}
\begin{itemize}
\item Number of precedences $q$ must be a non-negative integer otherwise a warning (W\ref{w9}) will be generated and the specification is ignored.
\item If more than $q$ precedences are provided, only the first $q$ ones will be considered. 
\item If less than $q$ precedences are provided, the behavior is unspecified.
\item Indexes of visits in each precedence constraint $p[i] \rightarrow s[i]$ must be non-negative integers otherwise a warning (W\ref{w10}) will be generated and the precedence constraint will be ignored.
\item Precedence constraints $p[i] \rightarrow s[i]$ must hold on different visits. If $p[i]=s[i]$ a warning (W\ref{w11}) will be generated and the precedence constraint will be ignored.
\item If no precedence constraints were specified, it behaves as if it was explicitly specified 0 precedences using {\tt Precedences 0}. No warning is generated in this case.
\item The specification of a set of precedence constraints completely replaces the current set of precedence constraints in case one was previously specified.
\item At solve time, if a precedence constraint holds on a visit $i$ that is larger than the size $n$ of the tensor so that $i$ is not a visit of the route, a warning (W\ref{w16}) will be generated.

\end{itemize}

\subsection{Terminal visit specification}

One of the visit in the tensor can be specified to be the terminal visit. For instance, often, the last index in the tensor will be a duplicate of the depot (index 0) and will be specified to be the terminal visit.

\begin{shaded}
\begin{verbatim}
TerminalVisit i
\end{verbatim}
\end{shaded}

It is possible to remove the terminal visit specification by using:

\begin{shaded}
\begin{verbatim}
ClearTerminalVisit
\end{verbatim}
\end{shaded}

In this case, the route can end at any visit. 

\underline{Remarks:}
\begin{itemize}
\item Index $i$ of the terminal visit should be a non-negative integer otherwise a warning (W\ref{w12}) will be generated and the specification is ignored.
\item The specification of a terminal visit replaces the current terminal visit specification so, by construction, at most one terminal visit can be specified.
\item If no terminal visit was specified, it behaves as if it was explicitly specified {\tt ClearTerminalVisit}. No warning is generated in this case.
\item At solve time, if a terminal visit $i$ was specified and if the size $n$ of the tensor is smaller than $i$ so that $i$ is not a visit of the route, a warning (W\ref{w16}) will be generated and no terminal visit will be considered.
\end{itemize}

\subsection{State reset}

The current state of the engine can be reset to the original state where nothing was defined using:

\begin{shaded}
\begin{verbatim}
Reset
\end{verbatim}
\end{shaded}


\subsection{Verbosity}

The verbosity level of the engine can be controlled by a parameter with levels {\tt l} in 0..9. 

\begin{shaded}
\begin{verbatim}
Verbosity l
\end{verbatim}
\end{shaded}

Here are some specifications of the verbosity levels (they will possibly be refined later):
\begin{itemize}
\item Default (if no verbosity level was specified) is level 0
\item If specified level is negative, it is considered as 0
\item If specified level is larger than 9, it is considered as being 9
\item At level 0, no warning is generated and during solve, no information about the search is generated on the {\tt stderr} stream. Still, errors are send on the {\tt stderr} stream.
\item At level 2 and higher, all warnings are send on the {\tt stderr} stream but no information about the search is generated.
\item At level 4 and higher, some information about the search can be generated on the {\tt stderr} stream.
\end{itemize}

\subsection{Version}

The {\tt Version} instruction displays the version of the executable on the {\tt stderr} stream.

\begin{shaded}
\begin{verbatim}
Version
\end{verbatim}
\end{shaded}

\noindent For instance:

\begin{verbatim}
Executable  :  cpo_optimod
   Version  :  1.0.0
      Date  :  10/12/2014
  Built on  :  IBM ILOG CPLEX Optimization Studio 12.6.1.0
\end{verbatim}


\section{Problem resolution and output}

The instruction to solve the TDTSP problem defined by the current state is specified by a time limit $tl$ and a Boolean telling whether or not all solutions visited by the search should be send on the standard output (value 1) or only the best solution found at the time-limit (value 0, this is the default):

\begin{shaded}
\begin{verbatim}
Solve tl 0|1
\end{verbatim}
\end{shaded}

The engine extracts the TDTSP problem defined by the current state and will start a search using a time limit {\tt tl}. If the Boolean is 1  all the solutions found by the engine will be sent to the standard output otherwise, only the best solution found at the time limit will be sent. After the time-limit is reached the engine will be available to read new input.

If no travel time tensor has been specified in the current state an error (E\ref{e1}) will be generated.

Instruction {\tt Quit} ends the process:

\begin{shaded}
\begin{verbatim}
Quit
\end{verbatim}
\end{shaded}

\renewenvironment{shaded}{%
  \def\FrameCommand{\fboxsep=\FrameSep \colorbox{shadecolor2}}%
  \MakeFramed{\advance\hsize-\width \FrameRestore\FrameRestore}}%
 {\endMakeFramed}

Solutions of a TDTSP instance consists of a permutation of the $n$ visits. Additionally, it specifies the end time $t$ of the route.
The keyword {\tt Optimal} is appended to the message in case the engine proved the optimality of the returned solution.

\begin{shaded}
\begin{verbatim}
Solution n v[0] v[1] ... v[n-1] t [Optimal]
\end{verbatim}
\end{shaded}

When the Boolean to display all solution is 1, intermediary solutions are displayed using the keywork {\tt CurrentSolution} instead of {\tt Solution}:

\begin{shaded}
\begin{verbatim}
CurrentSolution n v[0] v[1] ... v[n-1] t
\end{verbatim}
\end{shaded}

If the problem is shown to be infeasible, the output is: 

\begin{shaded}
\begin{verbatim}
Infeasible
\end{verbatim}
\end{shaded}

If no solution was found at the time-limit, the output is: 

\begin{shaded}
\begin{verbatim}
NoSolutionFound
\end{verbatim}
\end{shaded}


If an error occurred during the solve the output is: 

\begin{shaded}
\begin{verbatim}
Error
\end{verbatim}
\end{shaded}

\section{Warnings and errors}

Warnings and errors are send on the {\tt std::err} output. Here is a list of errors:

\begin{errors}
\label{e1}  & Cannot solve: no travel time tensor was defined\\
\label{e2}  & Engine exception: {\tt EXCEPTION} \\
\end{errors}

Here is a list of warnings:

\begin{warnings}
\label{w17} & Non-positive number of visits, ignoring tensor specification\\
\label{w18} & Non-positive number of time steps, ignoring tensor specification\\
\label{w19} & Non-positive step duration, ignoring tensor specification\\
\label{w20} & Negative travel time {\tt (<i>,<j>)} at time-step {\tt <k>}, using 0\\
\label{w0}  & Negative number of visits, ignoring duration specification\\
\label{w1}  & Non-positive duration replaced by duration 1 for visit {\tt <i>}\\
%\label{w2}  & Duration for visits of index larger than {\tt <n>} will be ignored\\ 
\label{w5}  & Negative visit index  {\tt <i>}, ignoring time-window specification\\
\label{w6}  & Negative number of time-windows for visit {\tt <i>}, ignoring specification\\
\label{w7}  & Ignoring inconsistent time-window {\tt <a>..<b>} for visit {\tt <i>}\\
%\label{w8}  & Forbidden time-windows with index greater than {\tt <f>} will be ignored for visit {\tt <i>}\\
\label{w9}  & Negative number of precedences {\tt <q>}, ignoring specification\\
\label{w10} & Ignoring precedence {\tt <i>..<j>} using negative visit index\\
\label{w11} & Ignoring inconsistent precedence  {\tt <i>..<i>}\\
\label{w12} & Negative visit index {\tt <i>}, ignoring terminal visit specification\\
\label{w3}  & Non-specified duration for visits {\tt <i>..<j>}, using duration 1\\
\label{w4}  & Non-specified duration vector, using duration 1 for all visits\\
\label{w16} & Ignoring specifications on visit {\tt <i>} who is outside of the route\\
\end{warnings}


\end{document}

