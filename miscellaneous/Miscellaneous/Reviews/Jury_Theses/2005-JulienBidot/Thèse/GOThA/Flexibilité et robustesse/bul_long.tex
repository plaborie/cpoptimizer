% Version longue en vue publication dans bulletin ROADEF printemps 2002

% Derniere revision 31 mars 2002 : ajout Christian Artigues
%                               juin 2002 : insertion partie metriques Francis et Mireille
%                                               quelques corrections par ES sur parties 5 et 6
% Version 14/06 : correction lors de la r�union GOThA

% Ajout de biblio. M. A. Aloulou et de la liste des participants. 25 juin


\documentclass[12pt]{article}

\usepackage[french]{babel}
\usepackage[latin1]{inputenc}

\def\noi{\noindent}
\def\ordo{ordon\-nan\-ce\-ment}
\def\ordos{ordon\-nan\-ce\-ments}


\begin{document}

\bibliographystyle{alpha}

\nocite{billaut96,davenport,kouvBook,kouvFlow,mehtaTbar,bean,leon,wugraph,mahadev,stankovic,vincke1,kacem1}

\begin{center}{\huge\bf Flexibilit\'e et Robustesse en Ordonnancement}\end{center}
\begin{center}{\bf GOThA, groupe flexibilit\'e}\end{center}


%\begin{center}Adresse postale\end{center}

% \begin{center}{\tt email@email}\end{center}


% \begin{center}{\bf R\'esum\'e}\end{center}

{\small
 Cet article est le fruit de deux ann\'ees de r\'eflexions et de partage d'exp\'e\-rien\-ces
 au sein du
groupe de travail {\em flexibilit\'e\/}, qui s'est r\'eguli\`erement r\'euni durant
les journ\'ees du Groupe de recherche sur l'Ordonnancement Th\'eorique et Appliqu\'e (GOThA). Les
membres du groupe partagent une pr\'eoc\-cu\-pation commune, comment r\'eagir aux
al\'eas lors de la r\'esolution d'un probl\`eme d'\ordo\,; ils viennent cependant
d'horizons applicatifs tr\`es diff\'erents. Un des objectifs du groupe est
 de proposer un cadre
 de r\'ef\'erence qui soit
acceptable par les diff\'erents intervenants concern\'es~: probl\'e\-ma\-tique,
d\'efinitions, mise en \oe uvre de la flexibilit\'e.
}

\section{Introduction}

\label{intro}
Souvent les donn\'ees initiales n\'ecessaires \`a la construction d'un
\ordo\ sont incertaines avant la phase d'ex\'ecution (d'une production, d'un projet, d'un
programme parall\`ele). Pourtant une premi\`ere \'etude est possible (et souhaitable\,!) avant cette phase d'ex\'ecution.
 Que faire dans cette r\'egion un peu floue entre d\'eterminisme et temps r\'eel?
La solution consiste \`a introduire de la flexibilit\'e dans le processus m\^eme de
construction de l'\ordo.


Nous nous pla\c cons ici strictement dans la probl\'ematique de
l'\ordo, m\^eme si la notion ou le d\'esir de flexibilit\'e peuvent
d\'ependre d'autres niveaux de d\'ecision (par exemple, en \ordo\ de
production, de la planification amont). La cause des al\'eas est peu
\'evoqu\'ee ici. Ceci facilite la construction d'un cadre commun pour les
 diff\'erentes applications~:
pilotage d'atelier, gestion de projet, parall\'elisation d'applications
informatiques.

Dans la suite, on consid\`ere un probl\`eme d'\ordo\ de t\^aches soumises \`a des
contraintes sur les dur\'ees d'ex\'ecution, de communication (ou de
transfert), des contraintes de pr\'ec\'edence, des contraintes de ressources.
 La performance d'un \ordo\ peut \^etre \'evalu\'ee par un crit\`ere
fix\'e. Le
probl\`eme a une version {\bf statique} o\`u tout est suppos\'e fix\'e, et une version
{\bf dynamique} o\`u des al\'eas de natures diverses viennent perturber le probl\`eme
statique. Dans ce dernier cas, il est n\'ecessaire de prendre des d\'ecisions au moment
m\^eme de l'ex\'ecution des t\^aches. Informellement, la {\bf flexibilit\'e} est la libert\'e
dont on dispose durant cette phase d'ex\'ecution. Nous utiliserons le terme
de {\bf robustesse} pour caract\'eriser la performance d'un algorithme (ou
plut\^ot d'un processus complet de construction d'un \ordo) en pr\'esence
d'al\'eas.

A noter que les termes pr\'edictif / r\'eactif sont employ\'es couramment
 en automatique \`a la place de statique / dynamique, que nous avons pr\'ef\'er\'es
par souci de g\'en\'eralit\'e.

 \section{Flexibilit\'e}

\label{flex}
Suivant le contexte,
il est possible d'introduire  la flexibilit\'e de diff\'erentes mani\`eres.
On peut l'exprimer comme l'existence
 de modifications possibles d'un
\ordo\ calcul\'e en statique, et entra\^\i nant une perte de performance acceptable.
Une seconde mani\`ere (tr\`es proche) de l'exprimer est de d\'efinir pour un
\ordo\ statique un voisinage de solutions pouvant \^etre
accept\'ees \`a
l'ex\'ecution.
Une troisi\`eme mani\`ere (voir section suivante) consiste \`a consid\'erer
que durant l'\'etape statique une famille d'\ordos\ est propos\'ee,
sans en privil\'egier forc\'ement un seul.

 On distingue plusieurs types (ou degr\'es) de flexibilit\'e~:
\begin{enumerate}
\item flexibilit\'e sur les {\bf temps}. Seules les
dates effectives de d\'ebut et de fin des t\^aches peuvent varier.

\item flexibilit\'e sur les {\bf ordres}. Les ordres
relatifs d'ex\'ecution (s\'equences) d'un ensemble de t\^aches peuvent \^etre modifi\'es
durant l'ex\'ecution, ce qui implique la flexibilit\'e sur les temps. Cet
ensemble de t\^aches doit utiliser une m\^eme ressource qui oblige \`a les
ex\'ecuter s\'equentiellement. Les ressources utilis\'ees par chaque t\^ache restent
donc inchang\'ees.

\item flexibilit\'e sur les {\bf ressources}.
Il est possible de changer l'affectation des ressources aux  t\^aches. La flexibilit\'e sur 
les ressources n�cessite d'accepter la flexibilit� sur les temps et les ordres.

\item flexibilit\'e sur les {\bf modes
d'ex\'ecution}~: suivant le contexte \`a l'ex\'ecution, on peut d\'ecider, au prix
peut-\^etre d'une d\'egradation de performances, de changer le mode
d'ex\'ecution d'une t\^ache (ou plusieurs). En pr\'esence d'al\'eas, on permettra ainsi la pr\'eemption, la
duplication, le recouvrement des ex\'ecutions par les communications, le
changement de gamme, voire
la d\'elo\-ca\-li\-sa\-tion.
\end{enumerate}

La flexibilit\'e sur les temps est
en quelque sorte le degr\'e z\'ero de la flexibilit\'e car indispensable
d\`es qu'un al\'ea doit \^etre pris en compte. En particulier les \'etudes de sensibilit\'e
 existantes supposent
toutes au moins cette flexibilit\'e sur les temps. Fr\'equemment, elle est
implicite dans la d\'efinition d'un \ordo.

Pour chaque type de flexibilit\'e, il est utile de d\'efinir un outil
quantitatif de mesure, un {\bf indicateur de flexibilit\'e}.
 Cette mesure doit porter sur un ensemble
d'\ordos\ d\'etermin\'es par un algorithme statique (voir section
suivante). On pourra mesurer par exemple, outre la cardinalit\'e de l'ensemble, la distance maximale entre deux
\ordos\ en termes de diff\'erences de dates de fin, nombre d'\'echanges
de t\^aches, nombre de changements de ressources, etc.

\section{Prise en compte des al\'eas\,: mise en \oe uvre}

\label{processus}

Le processus global
d'ordon\-nan\-ce\-ment en pr\'esence d'al\'eas peut se d\'ecom\-po\-ser comme
suit~:
\begin{itemize}
\item[\'etape 1] D\'efinition du {\bf probl\`eme statique}.
 Cette d\'efinition comprend,
en plus des sp\'ecifications classiques en \ordo\ d\'eterministe, la
sp\'ecification des al\'eas possibles. La notion de {\bf qualit\'e}
 d'un \ordo\
doit \^etre \'egalement pr\'ecis\'ee \`a ce niveau (crit\`ere de performance).
\item[\'etape 2] Calcul d'un ensemble de solutions (famille
d'\ordos\ r\'eali\-sa\-bles) par un {\bf algorithme statique} (ou
algorithme d\'eterministe ) $\alpha$.
\item[\'etape 3] Lors de l'ex\'ecution, calcul d'une solution unique
(l'\ordo\ r\'ealis\'e) issue de
cet ensemble par un {\bf algorithme dynamique} $\delta$.
\end{itemize}

Le processus complet
n\'ecessite donc le choix de deux algorithmes statique et dynamique.
Fr\'equemment il n'est que partiellement explicit\'e, l'un des deux
algorithmes \'etant trivial.


\medskip
Un probl\`eme voisin de celui de la mise en \oe uvre consiste \`a \'etudier la
difficult\'e du probl\`eme (statique) de calcul d'un nouvel \ordo\ en
supposant les al\'eas connus\,: est-il plus facile de partir du premier
\ordo\ statique que de recalculer \`a partir du d\'ebut un nouvel
\ordo\ dans les nouvelles conditions?\\
Dans l'affirmative, on parle de {\em post-optimisation\/} lorsqu'on recherche le nouvel optimum,
 ou de {\em r\'epa\-ra\-tion\/} si on cherche simplement une solution r\'ealisable (voir \cite{chien00}).
  On peut \'egalement d\'efinir alors la
stabilit\'e d'une solution (est-elle plus ou moins loin d'une nouvelle
solution optimale?), voir la section suivante pour une d\'efinition plus formelle. 


Outre un int\'er\^et th\'eorique (ce type d'\'etudes a \'et\'e
effectu\'e pour des probl\`emes comme le voyageur de commerce ou l'arbre de
poids min, mais pas en \ordo), cette probl\'ematique intervient naturellement quand
on cherche \`a \'evaluer l'efficacit\'e de l'algorithme  dynamique de l'\'etape 3.

Les premiers r\'esultats \cite{Picouleau} montrent que
pour plusieurs probl\`emes NP-difficiles (probl\`emes \`a $1$, $2$ ou $m$
machines avec minimisation du makespan), lorsqu'on introduit une perturbation
m\^eme minimale (incr\'ementation ou d\'ecr\'ementation d'une date de
disponibilit\'e, d'\'ech\'eance ou d'une dur\'ee d'ex\'ecution), le probl\`eme de
post-optimisation est lui aussi NP-difficile.



\section{ M\'etriques pour la robustesse}

\label{metriques}
L'\'etude de la robustesse du processus global
d'\ordo\
 est la recherche de garanties de performances sur
l'\ordo\ r\'ealis\'e.

Cette section propose des outils
quantitatifs de mesure de la robustesse.

\subsection{Notations}

Les d\'efinitions suivantes sont
utilis\'ees. La performance d'un
\ordo\ est \'evalu\'ee selon un crit\`ere de performance quelconque $Z$.
\begin{itemize}
\item[${\cal P}$] un probl\`eme statique, avec la description des
al\'eas potentiels (ce qui peut correspondre \`a une infinit\'e
d'instances du probl\`eme d\'eterministe)
\item[$\Sigma^{\alpha}$] un ensemble de solutions (\ordos) obtenus par
$\alpha$, l'algorithme statique.
\item[$S^{\alpha}_{\delta}$] un \ordo\ r\'ealis\'e obtenu par
$(\alpha,\delta)$, l'algorithme statique et l'algorithme
dynamique.
\item[${\cal I}$] une instance r\'ealis\'ee de ${\cal P}$ (conditions effectives \`a
l'ex\'ecution) appel\'ee aussi sc\'enario
\item[$z_{{\cal I}}(S)$] performance d'un \ordo\ $S$ r\'ealis\'e
sur ${\cal I}$
\item[$z^*_{{\cal I}}$] performance d'un \ordo\ optimal sur ${\cal I}$
\end{itemize}

\subsection{Types de robustesses et de m\'etriques}

Dans l'\'etude de la robustesse d'un processus global, $\alpha$,
$\delta$, ${\cal P}$ sont fix\'es. La mesure de robustesse au pire
du processus global est donn\'ee par~:

$$R_1=\max_{{\cal I} \in {\cal P}} d (z^*_{{\cal I}}, z_{{\cal
I}}(S^{\alpha}_{\delta}) )$$ o\`u $d$ est une diff\'erence, une
d\'eviation, ou un rapport, entre deux performances. Un
ordonnancement est d'autant plus robuste que sa mesure de
robustesse $R_1$ est petite. La mesure de robustesse en moyenne
suppose connue la distribution des diff\'erentes instances
r\'ealis\'ees de ${\cal P}$, et s'exprime alors comme une
esp\'erance~:
$$\bar{R}_1=E_{{\cal I} \in {\cal P}} \left( d (z^*_{{\cal I}},z_{{\cal I}}(S^{\alpha}_{\delta}) ) \right)$$

\bigskip
La robustesse ``au pire'' est une garantie absolue de performance,
alors que la robustesse en moyenne ne garantit pas de performance
pour chaque r\'ealisation. La d\'efinition ci-dessus est tr\`es
g\'en\'erale (quels que soient $\alpha$ et $\delta$).

\bigskip
On peut aussi limiter l'\'etude \`a la robustesse de l'algorithme statique
$\alpha$. Dans ce cas on recherche une garantie
    de performance v\'erifi\'ee pour tout algorithme dynamique (ou pour une
    famille raisonnable d'algorithmes dynamiques). La mesure de la robustesse au pire
devient alors~:
$$R_2=\max_{{\cal I} \in {\cal P}} \max_{S \in
\Sigma^{\alpha}}
 d \left(z^*_{{\cal I}}, z_{{\cal I}}(S) \right)$$
La mesure de robustesse en moyenne $\bar{R}_2$ peut \^etre adapt\'ee de la m\^eme mani\`ere.

D'autres m\'etriques propos\'ees sont\,:
\begin{itemize}
    \item la distance \`a une performance pr\'evue
(n\'egoci\'ee) $\tilde z({\cal P})$. La mesure de la robustesse
s'\'ecrit~:

$$R_3=\max_{{\cal I} \in {\cal P}} d (\tilde z({\cal P}),z_{{\cal I}}(S^{\alpha}_{\delta}) )$$

\item la mesure du niveau de service. Il s'agit de maximiser la probabilit\'e que $z$ soit en-dessous d'un certain seuil fix\'e $T({\cal P})$\,:

$$R_4 = P(z(S^{\alpha}_{\delta}) \le T({\cal P}))$$

\item  la mesure de la stabilit\'e d'un \ordo\ statique.
On cherche \`a minimiser l'\'ecart entre un \ordo\ privil\'egi\'e
calcul\'e en statique $\tilde S \in \Sigma^{\alpha}$,
 et celui r\'ealis\'e par $\delta$, $S^{\alpha}_{\delta}$~:
$$R_5 = \max_{{\cal I} \in {\cal P}} d'(\tilde S,S^{\alpha}_{\delta})$$
$d'$ mesure l'\'ecart entre deux \ordos\ (similarit\'e), elle est donc utilisable
pour la mesure de flexibilit\'e introduite section \ref{flex}.


Dans le cas particulier o\`u $d' \in \{0,1\}$ on retrouve la
notion de stabilit\'e d'un \ordo\ issue de l'automatique~: un
\ordo\ reste stable s'il est inchang\'e en pr\'esence des al\'eas.
Enfin si la distance de similarit\'e calcule un \'ecart entre les
performances des deux ordonnancements, on retrouve la notion de
stabilit\'e de l'algorithme ou du processus global (voir r\'ef\'erences  \cite{sotskov97,sotskov98})~:
$$R_6 = \max_{{\cal I} \in {\cal P}} d(z_{\cal I}(\tilde S),z_{\cal I}(S^{\alpha}_{\delta}))$$

Dans ce sens, une \'etude de stabilit\'e s'inscrit donc dans le
cadre plus g\'en\'eral de l'analyse de sensibilit\'e,
historiquement issue de la programmation lin\'eaire\,: quelle est
la d\'egradation de performance d'une solution quand les conditions
initiales varient ? Les \'etudes r\'ecentes de sensibilit\'e en
ordonnancement sont assez nombreuses (voir par exemple \cite{hall}).
Cela revient dans notre cadre \`a
n'accepter qu'une flexibilit\'e temporelle, l'algorithme dynamique
se contentant de d\'ecaler les dates d'ex\'ecution des t\^aches.
\end{itemize}


\section{Exemples de probl\`emes trait\'es par des \\ \'equipes du groupe}

\subsection{Gestion des retards dans l'approvisionnement d'une ligne d'assemblage}
% Marc Sevaut Valenciennes

Dans l'industrie automobile, la plupart des sous-traitants sont
install\'es \`a proximit\'e des usines d'assemblage. Ils d\'elivrent
les pi\`eces pr\^etes \`a monter plusieurs fois par jour selon les
directives de l'usine d'assemblage (bas\'ees sur la \emph{car
 sequence}). Malheureusement, il arrive que les sous-traitants ne
puissent livrer \`a temps.

On se propose alors de calculer un ordonnancement des pi\`eces \`a
assembler sachant que les dates de d\'ebut au plus t\^ot sont sujettes
\`a modification. L'objectif ici est  la
minimisation du nombre pond\'er\'e de t\^aches en retard.

Une approche par algorithme g\'en\'etique est propos\'ee. A
chaque it\'eration, on \'evalue le crit\`ere pour une
 s\'erie d'instances dont  les
dates de disponibilit\'e ont \'et\'e modifi\'ees al\'eatoirement. Le r\'esultat est un
ordonnancement plus robuste qui peut absorber des variations des
dates de disponibilit\'e (voir \cite{inp-sorensen-01,inp-sevaux-02}).


\subsection{Recherche de solutions robustes et flexibles pour la gestion d'un parc de machines polyvalentes}
% Mireille Jacomino Andre Rossi Grenoble

Nous consid\'erons un atelier de photolithographie pour la
fabrication de  semi-conducteurs. Cet atelier est compos\'e de
machines polyvalentes. Ces machines doivent \^etre configur\'ees
avant la phase de production, ce qui a un co\^ut que l'on souhaite
minimiser. Le second objectif  est la minimisation du temps total
de production. Pour ce type d'atelier, le volume de production
demand\'e varie beaucoup au cours du temps car les perturbations
issues des ateliers en amont sont amplifi\'ees par l'aspect
r\'eentrant de cet atelier.


Nous nous int\'eressons ici \`a l'\'elaboration de solutions dites robustes,
 c'est-\`a-dire qui garantissent un
niveau de performance pour un ensemble de jeux de donn\'ees
possibles. L'exigence sur l'optimalit\'e de la solution est alors
rel\^ach\'ee. On a mod\'elis\'e le probl\`eme de r\'epartition des
t\^aches sur les machines et de la configuration de ces machines
sous la forme d'un programme lin\'eaire \`a variables mixtes
bi-crit\`eres. Des r\'esultats de complexit\'e ont \'et\'e
d\'emontr\'es, une m\'ethode exacte pour d\'eterminer une
configuration robuste est en cours de d\'eveloppement, ainsi que
des heuristiques (voir \cite{rossi01,rossi02}).

\subsection{Gestion des indisponibilit\'es des machines dans des ateliers de production}
% Marie Laure Espinouse Troyes

Les probl\`emes classiques d'ordonnancement consid\`erent que les machines sont
toujours disponibles. Or, dans l'industrie ceci n'est que tr\`es rarement le
cas. Les machines peuvent conna\^\i tre des p\'eriodes d'indisponibilit\'e, ces
p\'eriodes correspondant par exemple \`a des p\'eriodes de pannes, \`a des p\'eriodes
bloqu\'ees pour pouvoir r\'ealiser des commandes urgentes ou encore \`a des
p\'eriodes de maintenance pr\'eventive. Plusieurs chercheurs du groupe
flexibilit\'e se sont int\'eress\'es \`a la prise en compte de ces p\'eriodes
d'indisponibilit\'es en ordonnancement dans des ateliers \`a une machine, m
machines ou
encore dans des ateliers de type flowshop (voir par exemple \cite{espinouse}),
 openshop et jobshop. Plusieurs
r\'esultats de complexit\'e ont \'et\'e d\'emontr\'es, des m\'ethodes de r\'esolutions
exactes ou approch\'ees ont \'et\'e propos\'ees. Mais la plupart de ces travaux
 (voir les revues \cite{lee97, sanlaville98})
consid\`erent
 que les caract\'eristiques (dur\'ee,
date de d\'ebut) de ces p\'eriodes sont parfaitement connues en d\'ebut
d'ordonnancement. Or, par exemple, lorsque ces p\'eriodes d'indisponibilit\'e
correspondent \`a des p\'eriodes de maintenance pr\'eventive, souvent seule une
dur\'ee approximative peut \^etre donn\'ee. Notre objectif est donc d'\'etudier
 la prise en compte d'al\'eas tels qu'une
p\'eriode d'indisponibilit\'e pouvant \^etre plus longue que pr\'evue et ainsi
\'etudier la robustesse des algorithmes propos\'es auparavant (voir aussi le travail de \cite{guo}
sur le flowshop).



\subsection{Gestion en temps r\'eel d'une machine soumise \`a des al\'eas}
% Mohamed Aloulou Antony Vignier Nancy

  Nous consid\'erons ici des ateliers compo\-s\'es d'une seule machine. Les cri\-t\`eres de performance sont la du\-r\'ee
totale d'une production et la somme des retards pon\-d\'e\-r\'es suivant
l'importance de chaque produit fini.

 Les al\'eas pris en compte sont \`a la fois les retards de livraison (donc retard
 de la date de disponibilit\'e), les retards lors de l'ex\'ecution d'une t\^ache, et
 les pannes de la machine (voir \cite{aloulou01}).

Au lieu de donner un unique ordonnancement \`a l'atelier, nous fournissons
un ensemble d'ordonnancements qui suivent une structure donn\'ee afin de
passer d'un ordonnancement \`a un autre facilement. La structure choisie est
un graphe partiel non cyclique contenant les
pr\'ec\'edences imp\'eratives ainsi que les pr\'ec\'edences rajout\'ees par l'\ordo\ statique.
 La flexibilit\'e choisie est donc une
flexibilit\'e sur les ordres. Elle est mesur\'ee par le nombre d'\ordos\ respectant ce graphe.
Les ordonnancements propos\'es sont semi-actifs
au sens o\`u une machine peut rester oisive en attendant une op\'eration
prioritaire. Chaque fois
qu'une d\'ecision doit \^etre prise, suite \`a l'arriv\'ee en temps r\'eel d'un al\'ea
quelconque, un ensemble d'alternatives compatibles avec les contraintes du
probl\`eme est donn\'e au d\'ecideur. Celui-ci peut choisir l'action qui lui
convient selon ses pr\'ef\'erences, l'\'etat de l'atelier et \'eventuellement des
contraintes non mod\'elis\'ees (voir \cite{aloulourairo}).




\subsection{Gestion de l'incertitude sur les temps de communication entre ordinateurs en r\'eseaux}
% Eric sanlaville Clermont, Aziz Moukrim Compiegne, Frederic Guinand Le Havre
L'utilisation efficace des ordinateurs parall\`eles ou des r\'eseaux
d'ordinateurs n\'ecessite
de d\'ecouper les applications en modules ex\'ecut\'es sur diff\'erents
processeurs. Ces modules sont li\'es par des contraintes
de pr\'ec\'edence. Mais leur temps d'ex\'ecution
ainsi que les dur\'ees de communication entre ces modules sont tr\`es
difficiles \`a
\'evaluer exactement. En particulier les dur\'ees de communication d\'ependent \`a
la fois\,: des modules qui communiquent,
des processeurs qui les ex\'ecutent, mais aussi de l'\'etat pr\'esent du r\'eseau
de communication.

L'objectif consiste \`a minimiser la dur\'ee totale d'une application. Pour
des mod\`eles simples (2 machines, nombre illimit\'e de machines,\ldots)
nous menons des \'etudes de  sensibilit\'e d'algorithmes existants optimaux
dans le cas statique (voir \cite{guinand}). Pour des mod\`eles plus complexes,
nous proposons de conserver une flexibilit\'e sur les ordres lors de
l'ex\'ecution (changer l'affectation des t\^aches  au moment de l'ex\'ecution
 est en effet
souvent impossible, et en tout cas trop co\^uteux). Par rapport
\`a une approche purement en ligne, nous ajoutons des
 contraintes issues de l'\'etude statique. L'algorithme dynamique choisit donc
la
t\^ache \`a ex\'ecuter parmi celles disponibles en respectant ces contraintes
additionnelles (voir \cite{moukrim}).

\subsection{Recherche et maximisation d'un niveau de service en ordonnancement stochastique}
%Stephane Dauzere Peres Nantes

Nous nous int\'eressons aux probl\`emes d'ordonnancement dans lesquels les
dur\'ees op\'eratoires sont connues de mani\`ere probabiliste, \`a travers
par exemple une densit\'e de probabilit\'e (continue ou discr\`ete). La
plupart des travaux dans ce contexte ont pour objectif de minimiser la
valeur moyenne (esp\'erance math\'ematique) d'un crit\`ere classique, le
makespan habituellement.

Nous trouvons plus pertinent de consid\'erer un niveau de service, comme
en gestion des stocks par exemple. En particulier, nous consid\'erons la
probabilit\'e que le makespan soit inf\'erieur ou \'egal \`a une valeur fix\'ee
(par exemple la dur\'ee d'une p\'eriode pendant laquelle les jobs doivent
\^etre termin\'es). On peut facilement montrer sur un exemple qu'un
ordonnancement $O1$ peut dominer un autre ordonnancement $O2$ sur la
valeur moyenne du makespan, alors que $O1$ aura un moins bon niveau de
service que $O2$.

Le premier probl\`eme, d\'ej\`a complexe, consiste \`a calculer le niveau
de service pour un ordonnancement donn\'e (exemple\,: quelle est la
probabilit\'e que le makespan soit inf\'erieur ou \'egal \`a 100 ?). Dans un
deuxi\`eme temps, l'objectif serait de d\'eterminer l'ordonnancement qui
maximise le niveau de service (voir r�f�rences \cite{Iida,cSchmidt}).

\subsection{Insertion d'une t\^ache impr\'evue en gestion de projets sous contraintes de ressources}
% Christian Artigues Avignon

Le probl\`eme de d\'etermination des dates de d\'ebut des
t\^aches
d'un projet
en res\-pec\-tant les contraintes de succession entre t\^aches et en
assurant une dur\'ee de r\'ealisation minimale du projet devient
tr\`es difficile lorsque les t\^aches utilisent des ressources
disponibles en quantit\'es limit\'ees (contraintes cumulatives).
De nombreuses m\'ethodes exactes et heuristiques ont \'et\'e
d\'evelopp\'ees pour r\'esoudre dans un contexte statique ce probl\`eme, connu sous le nom de RCPSP
 ({\it Resource-Constrained Project
  Scheduling Problem}).
Nous consid\'e\-rons le probl\`eme de la mise en
\oe{u}vre d'une solution pr\'ecalcul\'ee (\`a l'aide d'un algorithme
statique $\alpha$ quelconque) dans un contexte dynamique caract\'eris\'e
par la possibilit\'e d'arriv\'ee en cours de r\'ealisation
d'une t\^ache impr\'evue au d\'epart (voir \cite{artigues99}).

Dans le but de proposer un algorithme dynamique rapide et de ne pas
trop perturber la solution pr\'ecalcul\'ee, nous utilisons dans \cite{artigues02} un mod\`ele
de flots pour repr\'esenter  cette solution et nous imposons
\`a l'algorithme dynamique de conserver dans un premier temps
le s\'equencement relatif des
t\^aches d\'ej\`a pr\'esentes. Sous cette derni\`ere contrainte, nous
 pouvons ins\'erer la t\^ache en un temps polynomial
tout en minimisant la dur\'ee totale. Dans un second temps,
l'algorithme
applique une m\'ethode s\'erielle classique pour rendre actif l'ordonnancement obtenu dans la premi\`ere
phase.

Une \'etude exp\'erimentale
effectu\'ee sur 600 instances \`a 120 t\^aches et 4 ressources
a vis\'e \`a comparer notre algorithme ($\delta$) avec un
algorithme de r\'eordonnancement classique ($\alpha$) de m\^eme complexit\'e
temporelle et bas\'e sur une r\`egle de priorit\'e. Pour chacune des
instances, 27 insertions d'une t\^ache g\'en\'er\'ee al\'eatoirement ont
\'et\'e r\'ealis\'ees, la solution initiale \'etant obtenue par $\alpha$.
La comparaison des performances
$z_{{\cal I}}(S^{\alpha}_{\alpha})$ et
$z_{{\cal I}}(S^{\alpha}_{\delta})$ pour chaque insertion montre que
pour ces instances l'insertion par $\delta$ est g\'en\'eralement plus robuste que
le r\'eordonnancement par $\alpha$.\\

 % \subsection{Gestion des ateliers de type job-shop flexible}

 %  (Equipe COMPIL, EC-LILLE Imed Kacem)

 % Partie retiree car hors du cadre du groupe


\section{ Extensions et Conclusions}
 Notre effort a port\'e sur l'\'elaboration d'un cadre de r\'ef\'erences \`a l'int\'erieur duquel chacun puisse
ins\'erer sa propre probl\'ematique. Mais aussi large qu'il soit, le cadre propos\'e exclut, ou s'adapte
mal \`a, certains
probl\`emes int\'e\-res\-sants.   C'est le cas de l'\ordo\ cyclique, non trait\'e ici, ou des perturbations sur le crit�re lui-m\^eme.


\label{extensions}



%\subsection{\ordo\ cyclique}

\subsection{Perturbations sur le crit\`ere de performances}

Dans certaines applications, c'est la fa\c con de mesurer la performance
d'un \ordo\ qui peut changer \`a l'ex\'ecution. C'est en particulier
le cas pour des crit\`eres associant aux t\^aches des pond\'erations (somme pond\'er\'ee
 des dates de fin, des retards, \ldots). Ceci n'introduit pas
n\'ecessairement une flexibilit\'e sur l'\ordo, qui peut rester
fig\'e. Il est n\'eanmoins possible de mesurer la robustesse d'un
\ordo\ de mani\`ere similaire, si on introduit des hypoth\`eses sur la
perturbation possible de $Z$~: robustesse au pire et en moyenne, seuil de
performance et niveau de service, mesure de stabilit\'e \ldots

On peut cependant concevoir que l'on puisse \`a l'ex\'e\-cu\-tion modifier
l'\ordo\ pour prendre en compte ces modifications de $Z$. On
retrouve alors la n\'e\-ces\-sit\'e de fixer le degr\'e de flexi\-bi\-li\-t\'e acceptable.

\subsection{Approche commune � d'autres secteurs de la R. O.}
 Bien s\^ur, l'\'etude de la flexibilit\'e et de la robustesse n'est pas limit\'ee au seul domaine de l'\ordo.
Bien des domaines de  la Recherche Op\'erationnelle sont concern\'es,  principalement
ceux incluant une dimension temporelle  comme la planification ou la logistique
 (par exemple les probl\`emes de tourn\'ees de v\'ehicules, allocation dynamique de fr\'equences).
 Il sera donc int\'eressant dans l'avenir de
 confronter les diff\'erentes approches.


\bibliography{bul_long}

\bigskip

\centerline{\bf Liste des membres du groupe flexibilit\'e du GOThA}


 \begin{verbatim}
 Amine MAHJOUB, gilco Grenoble, <mahjoub@gilco.inpg.fr>,
 Andr� ROSSI, lag Grenoble, <rossi@lag.ensieg.inpg.fr>,
 Antoine JOUGLET, heudiasyc Compi�gne, <antoine.jouglet@hds.utc.fr>,
 Antony VIGNIER, loria Nancy, <avignier@loria.fr>,
 Aziz MOUKRIM, heudiasyc Compi�gne, <moukrim@utc.fr>,
 Bernard PENZ, gilco Grenoble, <Bernard.Penz@gilco.inpg.fr>,
 Carl ESSWEIN, li Tours, <carl.esswein@etu.univ-tours.fr>,
 Christian ARTIGUES, lia Avignon,
                           <christian.artigues@lia.univ-avignon.fr>,
 Christophe PICOULEAU, lip6 Paris, <christophe.picouleau@cnam.fr>,
 Christophe THOMAS, gilco Grenoble, <Christophe.THOMAS@gilco.inpg.fr>,
 Claude YUGMA, gilco Grenoble, <yugma@gilco.inpg.fr>,
 Cyril BRIAND, laas Toulouse, <briand@laas.fr>,
 Denis TRYSTRAM, id imag Grenoble, <Denis.Trystram@imag.fr>,
 Emmanuel NERON, li Tours, <neron@univ-tours.fr>
 Emmanuel PODER, limos Clermont, <poder@math.univ-bpclermont.fr>,
 Eric SANLAVILLE (coordinateur), limos Clermont,
                            <Eric.Sanlaville@math.univ-bpclermont.fr>
 Francis SOURD, lip6 Paris, <Francis.Sourd@lip6.fr>,
 Fr�d�ric GUINAND, lih Le Havre,<guinand@ecrins.univ-lehavre.fr>,
 Imed KACEM, lail Lille, <imed.Kacem@ec-lille.fr>,
 Jean-Charles BILLAUT, li Tours, <billaut@rabelais.univ-tours.fr>,
 Julien BIDOT, ilog Paris, <jbidot@ilog.fr>,
 Marc SEVAUX, lamih Valenciennes,
                            <Marc.Sevaux@univ-valenciennes.fr>,
 Marie-Claude PORTMANN, loria Nancy, <portmann@mines.u-nancy.fr>,
 Marie-Laure ESPINOUSE, losi Troyes,
                            <Marie_Laure.Espinouse@univ-troyes.fr>,
 Mireille JACOMINO, lag Grenoble, <jacomino@lag.ensieg.inpg.fr>,
 Mohamed Ali ALOULOU, loria Nancy, <mohamed-ali.aloulou@loria.fr>,
 Pierre LOPEZ, laas Toulouse, <lopez@laas.fr>,
 Safia KEDAD-SIDHOUM, lip6 Paris, <Safia.Kedad-Sidhoum@lip6.fr>,
 St�phane DAUZERE-PERES, irccyn Nantes, <dauze@emn.fr>,
 Yann LE QUERE, lamih Valenciennes et sncf,
                            <yann.lequere@univ-valenciennes.fr>
 \end{verbatim}





\end{document}